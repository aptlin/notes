%%% Local Variables:
%%% mode: latex
%%% TeX-master: t
%%% End:
\documentclass{article}
\usepackage{fullpage}
\usepackage{hyperref}
\usepackage{graphicx}
\usepackage{color}
\definecolor{mygrey}{gray}{0.90}
\pagenumbering{gobble}
\raggedbottom
\raggedright
\setlength{\tabcolsep}{0in}
\addtolength{\oddsidemargin}{-0.5in}
\addtolength{\evensidemargin}{-0.5in}
\addtolength{\textwidth}{1.0in}
\addtolength{\topmargin}{-0.5in}
\addtolength{\textheight}{1.0in}
\date{}
\title{Notes on Real Analysis}
\hypersetup{
  pdfkeywords={},
  pdfsubject={},
  pdfcreator={Emacs 24.4.1 (Org mode 8.2.10)}}
\begin{document}

\maketitle

\section{Foundations}
\label{sec-1}
\subsection{Postulates}
\label{sec-1-1}
\subsubsection{Numbers}
\label{sec-1-1-1}
\begin{enumerate}
\item \textbf{Real Numbers as a Field}
\label{sec-1-1-1-1}

\begin{enumerate}
\item \textbf{Associativity}

$\forall a,b,c \in \Re: a+(b+c)=(a+b)+c$

\emph{Exercise 1}:

Prove that the sums of an arbitrary number of equivalent variables in an immutable sequence are equal up to the placement of parentheses.

\emph{Exercise 2}:

Let the immutable sequence written in such a form that there are no two elements not parenthesised be called a \emph{nested} sequence.

For example, $((a+b)+c)+d$ and $(a+b)+(c+d)$ are both nested sequences.

How many different nested sequences can be written from a sequence of n letters?

\item \textbf{Commutativity of Addition}

$\forall \ a, b \in \Re: a+b=b+a$

\item \textbf{Commutativity of Multiplication}

$\forall \ a, b \in \Re: a \times b=b \times a$

\item \textbf{Existence of an Additive Identity}

$\exists \ 0\in \Re\ \forall a \in \Re: a+0=a$

\item \textbf{Existence of a Multiplicative Identity}

$\exists \ 1\in\Re\ \forall a \in \Re: a\times 1=a$

\item \textbf{Existence of an Additive Inverse}

$\forall \ a\in\Re\ \exists -a: a+(-a)=0$

\item \textbf{Existence of a Multiplicative Inverse}

$\forall \ a\in\Re\ \exists\ a^{-1}: a\times a^{-1}=1$

\item \textbf{Distributivity}

$\forall \ a,b,c \in \Re: a\times(b+c)=a\times b+a\times c$
\end{enumerate}
\item \textbf{Real Numbers as an Ordered Field}
\label{sec-1-1-1-2}

Let $P$ be the set of positive numbers.

Let the binary operator $>$ be defined so that $\forall \ a,b \in \Re: a > b \iff a-b\in P$.

Similarly,  $\forall \ a,b \in \Re: a < b \iff b-a\in P$.

\begin{enumerate}
\setcounter{enumi}{8}
\item \textbf{Trichotomy Law}

$\forall \ a \in \Re$ one and only one of the following holds:

\begin{itemize}
\item $a=0$

\item $a \in P$

\item $a \not\in P$
\end{itemize}

\item 
\end{enumerate}
\end{enumerate}
% Emacs 24.4.1 (Org mode 8.2.10)
\end{document}