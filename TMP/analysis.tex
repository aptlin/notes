% Intended LaTeX compiler: pdflatex
%%% Local Variables:
%%% mode: latex
%%% TeX-master: t
%%% End:
\documentclass[11pt]{scrartcl}
\usepackage[customauthor]{sdll}
\date{}
\title{Notes on Real Analysis}
\hypersetup{
 pdfauthor={},
 pdftitle={Notes on Real Analysis},
 pdfkeywords={},
 pdfsubject={},
 pdflang={English}}
\begin{document}

\maketitle

\section{Foundations}
\label{sec:org1f3f2b7}
\subsection{Postulates}
\label{sec:org497cd41}
\subsubsection{Numbers}
\label{sec:orgb2d9ab9}
\begin{enumerate}
\item \textbf{Real Numbers as a Field}
\label{sec:orgd43ad42}

\begin{enumerate}
\item \textbf{Associativity}

\(\forall a,b,c \in \RR: a+(b+c)=(a+b)+c\)

\emph{Exercise 1}:

Prove that the sums of an arbitrary number of equivalent variables in an immutable sequence are equal up to the placement of parentheses.

\emph{Exercise 2}:

Let the immutable sequence written in such a form that there are no two elements not parenthesised be called a \emph{nested} sequence.

For example, \(((a+b)+c)+d\) and \((a+b)+(c+d)\) are both nested sequences.

How many different nested sequences can be written from a sequence of n letters?

\item \textbf{Commutativity of Addition}

\(\forall \ a, b \in \RR: a+b=b+a\)

\item \textbf{Commutativity of Multiplication}

\(\forall \ a, b \in \RR: a \times b=b \times a\)

\item \textbf{Existence of an Additive Identity}

\(\exists \ 0\in \RR\ \forall a \in \RR: a+0=a\)

\item \textbf{Existence of a Multiplicative Identity}

\(\exists \ 1\in\RR\ \forall a \in \RR: a\times 1=a\)

\item \textbf{Existence of an Additive Inverse}

\(\forall \ a\in\RR\ \exists -a: a+(-a)=0\)

\item \textbf{Existence of a Multiplicative Inverse}

\(\forall \ a\in\RR\ \exists\ a^{-1}: a\times a^{-1}=1\)

\item \textbf{Distributivity}

\(\forall \ a,b,c \in \RR: a\times(b+c)=a\times b+a\times c\)
\end{enumerate}
\item \textbf{Real Numbers as an Ordered Field}
\label{sec:orgd6ca6e2}

Let \(P\) be the set of positive numbers.

Let the binary operator \(>\) be defined so that \(\forall \ a,b \in \RR: a > b \iff a-b\in P\).

Similarly,  \(\forall \ a,b \in \RR: a < b \iff b-a\in P\).

\begin{enumerate}
\setcounter{enumi}{8}
\item \textbf{Trichotomy Law}

\(\forall \ a \in \RR\) one and only one of the following holds:

\begin{itemize}
\item \(a=0\)

\item \(a \in P\)

\item \(a \not\in P\)
\end{itemize}

\item 
\end{enumerate}
\end{enumerate}
\end{document}