% Intended LaTeX compiler: pdflatex
%%% Local Variables:
%%% mode: latex
%%% TeX-master: t
%%% End:
\documentclass[11pt]{scrartcl}
\usepackage[beaue, nofancy]{sdll}
\author{aleph}
\date{\today}
\title{}
\hypersetup{
 pdfauthor={aleph},
 pdftitle={},
 pdfkeywords={},
 pdfsubject={},
 pdfcreator={Emacs 24.4.1 (Org mode 9.0)}, 
 pdflang={English}}
\begin{document}

\tableofcontents

\section{Preliminaries}
\label{sec:org37e08a1}

\subsection{Set Theory}
\label{sec:org88f89af}

\begin{itemize}
\item \(\NN\), \(\QQ\)
\item Operations:
\begin{itemize}
\item union
\item intersection
\item symmetric difference
\item difference
\end{itemize}
\item \href{https://ocw.mit.edu/courses/mathematics/18-703-modern-algebra-spring-2013/lecture-notes/MIT18_703S13_pra_l_3.pdf}{Equivalence relations and equivalence classes}
\end{itemize}

\subsection{Logic}
\label{sec:org03f9ff7}

\begin{itemize}
\item Quantifiers

\begin{itemize}
\item presentation style: quantifiers first and assertions last
\item quantifier order is crucial
\end{itemize}

\item Operators

\begin{itemize}
\item inclusivity of $\cor$
\end{itemize}
\end{itemize}

\subsection{Cuts}
\label{sec:org8a10d3a}
\begin{theorem}
\(\sqrt 2\) is irrational.
\end{theorem}
\begin{definition}
  A \textbf{cut} in $\QQ$ is a pair of subsets of $A, B$ of $\QQ$ such
  that:
  \begin{enumerate}[label=(\alph*)]
  \item
    $A\cup B = \QQ, A \neq \emptyset, B \neq \emptyset, A\cap B =
    \emptyset$.
  \item If $a\in A$ and $b\in B$, then $a<b$.
  \item $A$ contains no largest element.
  \end{enumerate}
\end{definition}
\begin{definition}
  A \textbf{real number} is a cut in $\QQ$.
\end{definition}

\begin{definition}
  The cut $x=A\vert B$ is \textbf{less than or equal} to the cut
  $y=C\vert D$, if $A\su C$.
\end{definition}
\begin{definition}
  $M\in \RR$ is an \textbf{upper bound} for a set $S\su \RR$ if each
  $s\in S$ satisfies $s\leq M$. Thus, we say the set $S$ is
  \textbf{bounded above} by $M$.\vv

  An upper bound for $S$ that is less than all other upper bounds for
  S is a \textbf{least upper bound} for $S$.
\end{definition}
\begin{theorem}[\textbf{Least Upper Bound Property}]
  The set $\RR$, constructed by the means of Dedekind cuts, is \textbf{complete}:

  \begin{center}
    If $S$ is a non-empty subset of $\RR$ and is bounded above,

    then in $\RR$ there exists a least upper bound for $S$.
  \end{center}
\end{theorem}

\begin{proof}
  Let \(\SC\in\RR\) be any non-empty collection of cuts. Suppose
  \(\SC\) is bounded above by some \(X|Y\).\vv

  Define two sets as follows:

  \begin{align}
    C  &= \set{a\in\QQ; \exists(A|B\in\SC):a\in A}\\
    C' &= \QQ\setminus C
  \end{align}

  We claim that \(C|C'\) is a cut, checking three conditions.\vv

  \begin{enumerate}[label=(\alph*)]
  \item $C\cup C' = \QQ$, \(C \neq \emptyset\), since \(\SC\) is not
    empty by definition, \( C' \neq \emptyset\), since \(\SC\) is
    bounded above, and \(C\cap C' = \emptyset\) by definition of \(C\)
    and \(C'\).
  \item If $c\in C$ and $c'\in C'$, then $c<c'$, since, for all
    \(d\in C'\), \(d\not\in C\).
  \item $C$ contains no largest element, since any \(A\) in
    \(A|B\) of \(\SC\) contains no greatest element.
  \end{enumerate}

  Note that, for all \(A\) in \(A|B\) of \(\SC\), \(A\su C\), and
  hence \(C|C'\) is an upper bound for \(\SC\).

  Let \(D|D'\) be any upper bound for \(\SC\). Therefore, for all
  \(A|B \in \SC\), \(A\su D\), and hence \(C\su D\), giving
  \(C|C'\leq D|D'\). Thus, of all upper bounds for \(\SC\), \(C|C'\)
  is the least.
  
\end{proof}

\begin{theorem}
  The set $\RR$ of all cuts in $\QQ$ is a complete ordered field that
  contains $\QQ$ as an ordered subfield.
\end{theorem}
\begin{theorem}[Triangle Inequality]
  $\forall(x,y\in\RR): |a+b| \leq |a| + |b|$
\end{theorem}

\begin{definition}
  Let $a_1, a_2, a_3, \dots = (a_n)$, $n\in\NN$, be a sequence of real numbers.\vv

  The sequence $(a_n)$ \textbf{converges to a limit} $b\in\RR$ as
  $n\to\infty$ provided that for each $\epsilon > 0$ there exists
  $N\in \NN$ such that for all $n\geq N$:
  \[|a_n-b| < \epsilon \]
\end{definition}

\begin{definition}[Cauchy Condition]
  $\forall(\epsilon<0)\exists(N\in\NN): n,m \geq N \ra |a_n-a_m|<\epsilon $.
\end{definition}
\begin{theorem}
  $\RR$ is \textbf{complete} with respect to Cauchy sequences, that
  is, if $(a_n)$ is a sequence of real numbers obeying a Cauchy
  condition, then it converges to a limit in $\RR$.
\end{theorem}

\begin{proof}
  Let \(A\) be the set of real numbers comprising the sequence \((a_{n})\),

  \[A = \set{x\in \RR; \exists n\in \NN: a_{n} = x}.   \]

  Since \(A\) obeys the Cauchy condition, then for \(\epsilon = 1\)
  there exists an integer \(N_{1}\) such that for all
  \(n,m\geq N_{1}\), \(\abs{a_{n}-a_{m}}<1\). Then, for each \(n\geq N_{1}\),
  
  \begin{equation}
    \label{eq:1}
    \abs{a_{n}-  a_{N_{1}}}<1.
  \end{equation}
 
  Therefore, for \(n\geq N_{1}\), \(a_{n}\in(a_{N_{1}}-1, a_{N_{1}}+1)\).\vv

  For the finite set \(B=\set{a_{1},a_{2},\dots,a_{N_{1}}+1}\) choose
  \(M=\max\{\abs{\min B}, \abs{\max B}\}\).\vv

  By definition of \(M\) and from the equation \ref{eq:1} it follows
  that all the elements of \(A\) are in \([-M,M]\), and so \(A\) is
  bounded.

  Consider now the set

  \[S = \set{s\in [-M, M]: \exists{\text{\ infinitely many
          \(n\in\NN\), for which \(a_{n} \geq s \)\ }}}\].

  Since \(-M\) is in \(S\), \(S\) is not empty. Moreover, \(S\) is
  bounded above by \(M\). Therefore, by the Least Upper Bound property
  for \(\RR\), there exists \(b\in\RR\) such that \(b=\sup S\). We
  claim that the sequence \(a_{n}\) converges to \(b\).\vv

  Suppose some \(\epsilon>0\) is given. Since \((a_{n})\) satisfies
  the Cauchy condition,

  \[\exists N_{2}: m,n \geq N_{2} \ra \abs{a_{m}-a_{n}}<\frac{\epsilon}{2}\]

  Since \(b\) is the least upper bound, \(b+\frac{\epsilon}{2}\) is
  not in \(S\). Thus, terms in \((a_{n})\) are greater or equal to
  \(b+\frac{\epsilon}{2}\) only finitely many times. Therefore, there
  exists \(N_{3}\) such that

  \[n\geq N_{3} \ra a_{n}\leq b + \frac{\epsilon}{2}.\]  \vv

  Since \(b\) is the least upper bound, note also that
  \(b-\frac{\epsilon}{2}\) is not an upper upper bound. Since real
  numbers are dense, there exists \(s\in S\) such that
  \(s> b- \frac{\epsilon}{2}\), which implies that there exists
  \(N\geq N_{3}\) such that \(a_{N}>b-\frac{\epsilon}{2}\).

  Since \(N\geq N_{3}\),

  \[a_{N}\in(b-\frac{\epsilon}{2}, b+\frac{\epsilon}{2}].\]

  
  
\end{proof}
\begin{theorem}[Cauchy Convergence Criterion for sequences]
  A sequence $(a_n)$ in $\RR$ converges if and only if
  \[ \forall(\epsilon > 0)\exists(N\in\NN): n,m \geq N \ra |a_n-a_m|<\epsilon \]
\end{theorem}
\begin{definition}
  Let $a<b$ be given in $\RR$. Define the \textbf{intervals} $(a,b)$
  and $[a,b]$ as follows:

  \begin{align}
    (a,b) &= \{x\in\RR: a<x<b \}\\
    [a,b] &= \{x\in\RR: a\leq x\leq b \}
  \end{align}
\end{definition}

\begin{theorem}
Every interval $(a,b)$ contains both rational and irrational numbers.
\end{theorem}
\begin{lemma}
  $\RR$ has the \textbf{Archimedean property}: for each $x\in \RR$
  there is an integer $n$ that is greater than $x$.
\end{lemma}
\begin{theorem}[$\epsilon$-principle]
  If $a,b$ are real numbers and if, for each $\epsilon > 0$,
  $a \leq b + \epsilon$, then $a\leq b$.\\ If $x, y$ are real numbers and, for each $\epsilon>0$,
  $|x-y|\leq \epsilon$, then $x=y$.
\end{theorem}



\subsection{Euclidean Space}
\label{subsec:eusp}
\begin{definition}
  Given sets $A$ and $B$, the \textbf{Cartesian product} of $A$ and
  $B$ is the set $A\times B$ of all ordered pairs $(a,b)$ such that
  $a\in A$ and $b\in B$.\\
  The Cartesian product of $\RR$ with itself $m$ times is denoted as
  $\RR^m$.
\end{definition}
\begin{definition}
  The \textbf{dot product} of $x = (x_1,x_2,\dots, x_m),y=(y_1,y_2,\dots, y_m) \in \RR^m$ is defined as
  \[ \inp{x}{y} = \sum_{i=1}^mx_iy_i  \]
\end{definition}
\begin{lemma}
  The dot product operation is bilinear, symmetric, and positive
  definite, i.e., for any $x,y,z\in\RR^m$ and any $c\in\RR$,
  \begin{align}
    \inp{x}{y+cz} &= \inp{x}{y} + c \inp{x}{z}\\
    \inp{x}{y} &= \inp{y}{x}\\
    \inp{x}{x} \geq 0, &\text{\ and\ }\inp{x}{x} = 0 \lra x=\bm{0}
  \end{align}
\end{lemma}
\begin{definition}

  The \textbf{length} or \textbf{magnitude} of a vector $x\in\RR^m$ is
  defined to be
  \[ \abs{x} = \sqrt{\inp{x}{x}}  \]
\end{definition}
\begin{theorem}[Cauchy-Schwarz Inequality]
For all $x,y\in \RR^m, \inp{x}{y} \leq \abs{x}|y|$.
\end{theorem}
\begin{corollary}
  For all $x,y\in\RR^m$,
  \[ |x+y| \leq \abs{x} + |y| \]
\end{corollary}
\begin{definition}
  The \textbf{Euclidean distance} between $x,y\in\RR^m$ is defined as
  the length of their difference.
  \[ |x-y| = \sqrt{\inp{x-y}{x-y}} \]
\end{definition}
\begin{definition}
  The $j$th coordinate of the point $(x_1,\dots, x_m)$ is the number
  $x_j$ appearing in the $j$th position.
\end{definition}

\begin{definition}
  The $j$th coordinate axis is the set of $x\in\RR^m$ which $k$th
  coordinates are zero for all $k\neq j$.
\end{definition}

\begin{definition}
  The \textbf{integer lattice} is the set $\ZZ^m \su \RR^m$ of ordered
  $m$-tuples of integers.
\end{definition}


\begin{definition}
  The \textbf{first orthant} of $\RR^m$ is the set of points
  $x\in \RR^m$ all of which coordinates are nonnegative.
\end{definition}

\begin{definition}
  A \textbf{box} is a Cartesian product of intervals in $\RR^m$
  \[ [a_1,b_1]\times\dots\times[a_m,b_m]\]
\end{definition}
\begin{definition}
  The \textbf{unit cube} in $\RR^m$ is the box
  $[0,1]^m = [0,1]\times\dots\times[0,1]$.
\end{definition}

\begin{definition}
  The \textbf{unit ball} in $\RR^m$ is the set
  \[ B^m = \set{ x\in\RR^m ; \abs{x} \leq 1 }. \]
\end{definition}
\begin{definition}
  The \textbf{unit sphere} in $\RR^m$ is the set
  \[ S^{m-1} = \set{ x\in\RR^m ; \abs{x} = 1 }. \]
\end{definition}

\section{Continuity}
\label{sec:continuity}
\begin{definition}
  The function \(f:[a,b]\to\RR\) is \textbf{continuous} if for each
  \(\epsilon > 0\) and each \(x\in[a,b]\) there is a \(\delta>0\) such that
  \[t\in [a,b] \text{\ and\ } |t-x|< \delta \ra \abs{f (t) -f(x)}<\epsilon\]
\end{definition}

\subsection{Three Hard Theorems}
\label{subsec:tht}
\begin{theorem}
  \label{thm:bounded}
  If \(f\) is a continuous function on \([a,b]\) and \(f(a)<0<f(b)\), then its
  values form a bounded subset of \(\RR\). Thus, there exist
  \(m,M\in\RR\) such that for all \(x\in[a,b]\), \(m\leq f(x)\leq M\).
\end{theorem}
\begin{proof}
  For \(x \in [a,b]\), let
  \[V_{x}=\set{y\in\RR ; \exists (t\in[a,x]):y = f(t)}.\]

  Set \[X=\set{x\in[a,b];V_{x}\text{\ is a bounded set of\ }\RR }.\]

  We prove now that \(b\) is in \(X\).\vv

  Since \(a\in X\), \(X\) is not empty. Note also that \(b\) is an
  upper bound for \(X\).

  Thus, there exists in \(\RR\) a least upper bound \(c \leq b\) for
  \(X\). \vv

  Since \(f\) is continuous, consder the neighbourhood of \(c\) for
  \(\epsilon = 1\). By definition of continuity, there exists a
  \(\delta > 0\) such that \(\abs{x-c}<\delta\) implies
  \(\abs{f(x) - f(c)}<1\) Since \(c\) is the least upper bound for
  \(X\), there is some \(x\in X\) in the interval \([c-\delta, c]\)
  (otherwise \(c-\delta\) is a smaller upper bound for \(X\)). \vv

  With \(t\) varying from \(a\) to \(c\), \(t\) is first mapped to
  \(f(t)\in V_{x}\), and then \(f(t)\) varies in the bounded set
  \(J = (f(c)-1, f(c)+1)\). \vv

  The union of two bounded set is a bounded set, and thus \(V_{c}\)
  is bounded. Therefore, \(c\in X\). \vv

  If \(c<b\), then by continuity, for some \(t>c\), \(f(t)\) still
  varies in the bounded set \(J\), which contradicts the fact that
  that \(c\) is an upper bound for \(X\). Thus, \(c=b\), \(b \in X\),
  and the values of \(f\) form a bounded subset of \(\RR\).
\end{proof}
\begin{theorem}
  If \(f\) is a continuous function on \([a,b]\), then there exist
  some numbers \(x_{0}, x_{1}\) in \([a,b]\) such that
  \(f(x_{0})\leq f(x)\leq f(x_{1})\) for all \(x\) in \([a,b]\).
\end{theorem}
\begin{proof}
  Let \(M=\sup \set{f(t); t\in[a,b]}\). By theorem \ref{thm:bounded},
  \(M\) exists. \vv

  Let \(X=\set{x\in[a,b];\sup\{V_{x}\} < M}\), where
  \[V_{x}=\set{y\in\RR ; \exists (t\in[a,x]):y = f(t)}.\]
  We first prove that \(f\) achieves a maximum on \([a,b]\)].
  \begin{case*}[\textbf{1}]
    \(f(a) = M\)
  \end{case*}
  Thus, \(f\) takes on a maximum at \(a\).
  \begin{case*}[2]
    \(f(a)< M\)
  \end{case*}
  Thus, \(X\) is not empty and \(\sup\{X\}\) exists. Suppose  \(\sup\{X\}=c\). \vv

  If \(f(c)<M\), choose \(\epsilon >0\) such that \(\epsilon < M-f(c)\). By continuity,
  there exists a \(\delta>0\) such that \(\abs{t-c}<\delta\) implies
  \(\abs{f(t)-f(c)}<\epsilon\). Thus, \(\sup\{V_{c}\}<M\). \vv

  If \(c<b\), then there exists a point \(t>c)\) at which
  \(\sup\{(V_{c}\} < M\), which contradicts the fact that \(c\) is an
  upper bound of such points. \vv

  Thus, \(c=b\), and hence \(M<M\), which is a
  contradiction. Therefore, \(f(c) = M\), so \(f\) achieves a maximum
  at \(c\).

  

  
\end{proof}
\begin{theorem}
  A continuous function defined on an interval \([a,b]\) achieves all
  intermediate values: if \(f(a)=\alpha, f(b)=\beta \), and \(\gamma\)
  is given such that \(\alpha \leq \gamma \leq \beta\) or
  \(\beta \leq \gamma \leq \alpha\) , then there exists some
  \(c\in[a,b]\) such that \(f(c) = \gamma\).
\end{theorem}




\end{document}
%%% Local Variables:
%%% mode: latex
%%% TeX-master: t
%%% End:
