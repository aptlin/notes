% -*- coding: utf-8; -*-
%%% Local Variables:
%%% mode: latex
%%% TeX-engine: xetex
%%% TeX-master: t
%%% End:
\documentclass[11pt]{scrartcl}
\usepackage[fancy, beaue, pset, anon]{sdll}
\pSet{\nt{PHY152}{III}{Buoyancy}}
\usepackage{lineno}
% ----------------------------------------------------------------------
% Page setup
% ----------------------------------------------------------------------

\pagenumbering{gobble}

% ----------------------------------------------------------------------
% Custom commands
% ----------------------------------------------------------------------

% alignment

\newcommand*{\LongestHence}{$\Rightarrow$}% function name
\newcommand*{\LongestName}{$f_o(-x)+f_e(-x)$}% function name
\newcommand*{\LongestValue}{$(-a)x +(-a)(-y)$}% function value
\newcommand*{\LongestText}{\defi}%

\newlength{\LargestHenceSize}%
\newlength{\LargestNameSize}%
\newlength{\LargestValueSize}%
\newlength{\LargestTextSize}%

\settowidth{\LargestHenceSize}{\LongestHence}%
\settowidth{\LargestNameSize}{\LongestName}%
\settowidth{\LargestValueSize}{\LongestValue}%
\settowidth{\LargestTextSize}{\LongestText}%

% Choose alignment of the various elements here: [r], [l] or [c]

\newcommand*{\mbh}[1]{{\makebox[\LargestHenceSize][r]{\ensuremath{#1}}}}%
\newcommand*{\mbn}[1]{{\makebox[\LargestNameSize][r]{\ensuremath{#1}}}}%
\newcommand*{\mbv}[1]{\ensuremath{\makebox[\LargestValueSize][r]{\ensuremath{#1}}}}%
\newcommand*{\mbt}[1]{\makebox[\LargestTextSize][l]{#1}}%

\newcommand{\R}[1]{\label{#1}\linelabel{#1}}
\newcommand{\lr}[1]{line~\lineref{#1}}

% ----------------------------------------------------------------------
% Launch!
% ----------------------------------------------------------------------

\begin{document}

\section*{Buoyancy}
\label{sec:buoy}


Suppose a solid object is immersed in a fluid, with the height \(d_{t}\) to
the top of the object and \(d_{b}\) to the bottom.

There are two forces resulting from the atmospheric pressure and
pressure of the column of water acting on the solid.

Note that \(P_{b} > P_{t}\) and thus \((P_{b} - P_{t})A > 0\). Why? Ask Archimedes!

Consider the following, where \(A\) is the cross-section area of the solid:

\[(P_{b} - P_{t})A  = (P_{atm} + \rho_{f}gd_{b})A - (P_{atm} + \rho_{f}gd_{t})A= \rho_{f}g(d_{b}-d_{t})A > 0\]

Note that \(\rho_{f}g(d_{b}-d_{t})A = \rho_{f}g V_{\text{solid}}\), and
thus \(\textbf{F} = \rho_{f}V_{\text{solid}}\textbf{g}\)

The magnitude of the buoyant force, acting upwards on the floating
object, determines whether it sinks, floats or goes up. However, the
buyoancy force acts does not act on the centre of mass but on the
\textbf{centre of buoyancy}, so for the stable equilibrium of a
floating object to occur, the centre of buyoancy must be above the
centre of mass. Mind the torque otherwise!


\end{document}