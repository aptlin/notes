% -*- coding: utf-8; -*-
%%% Local Variables:
%%% mode: latex
%%% TeX-engine: xetex
%%% TeX-master: t
%%% End:
\documentclass[11pt]{scrartcl}
\usepackage[fancy, beaue, pset, anon]{masty}
\pSet{\nt{CSC240}{IX}{Induction III}}
\usepackage{lineno}
% ----------------------------------------------------------------------
% Page setup
% ----------------------------------------------------------------------

\pagenumbering{gobble}

% ----------------------------------------------------------------------
% Custom commands
% ----------------------------------------------------------------------

% alignment

\newcommand*{\LongestHence}{$\Rightarrow$}% function name
\newcommand*{\LongestName}{$f_o(-x)+f_e(-x)$}% function name
\newcommand*{\LongestValue}{$(-a)x +(-a)(-y)$}% function value
\newcommand*{\LongestText}{\defi}%

\newlength{\LargestHenceSize}%
\newlength{\LargestNameSize}%
\newlength{\LargestValueSize}%
\newlength{\LargestTextSize}%

\settowidth{\LargestHenceSize}{\LongestHence}%
\settowidth{\LargestNameSize}{\LongestName}%
\settowidth{\LargestValueSize}{\LongestValue}%
\settowidth{\LargestTextSize}{\LongestText}%

% Choose alignment of the various elements here: [r], [l] or [c]

\newcommand*{\mbh}[1]{{\makebox[\LargestHenceSize][r]{\ensuremath{#1}}}}%
\newcommand*{\mbn}[1]{{\makebox[\LargestNameSize][r]{\ensuremath{#1}}}}%
\newcommand*{\mbv}[1]{\ensuremath{\makebox[\LargestValueSize][r]{\ensuremath{#1}}}}%
\newcommand*{\mbt}[1]{\makebox[\LargestTextSize][l]{#1}}%

\newcommand{\R}[1]{\label{#1}\linelabel{#1}}
\newcommand{\lr}[1]{line~\lineref{#1}}

% ----------------------------------------------------------------------
% Launch!
% ----------------------------------------------------------------------

\begin{document}

\section{More on Induction}

\begin{theorem}
Every integer greater than $1$ is a product of primes.
\end{theorem}
\begin{proof}
  For $n\in\NN$, let $p(n)=$"n is a product of primes".

  Let $n\in\NN$ be arbitrary.

  Suppose $n>1$ and $\forall i\in \NN.[1<i<n \THEN P(i)]$.

  If $n$ is prime, then $n$ is a product of primes, and thus $P(n)$.

  Otherwise, there are positive integeres $1<k<n$, $1< m < n$ such that $n = k\* m$.

  By the inductive hypothesis, $k$ and $m$ are products of primes: $P(k)$ and $P(m)$.

  Thus, $n = k\*m$ is a product of primes.

  Therefore, for all $n\in\NN$, and hence $(n>1)\THEN p(n)$.
\end{proof}

\section{Recursively Defined Sets}

Recursive definitions have 2 parts:
\begin{itemize}
\item a base case, that does not depend on anything else
\item a constructor case, that depends on previous cases.
\end{itemize}

\begin{example}
  \hfill
  
\begin{enumerate}
\item\label{item:1} Let $\set{0, 1}^{*}$ be a set of all finite strings of bits.

  Base Case: $\lambda$, empty string of length 0, is in $\set{0, 1}^{*}$.

  Constructor Case: If $s\in\set{0, 1}^{*}$, then $s0$ and $s1$ are in $\set{0, 1}^{*}$.

  Similarly, for any set $\Sigma$, $\Sigma^{*}$ is the of all finite
  length strings of letters from $\Sigma$.
\item Let $B$ be a set of finite strings of matched brackets.

  Base Case: $\lambda$, empty string of zero matched brackets, is in $B$.

  Constructor Case: if $p, q\in B$, then $p[q] \in B$.
\item Let $S$ be a set of syntactically correct formulas of propositional logic.

  Base Case: propositional variables are in $S$.

  Constructor Case: if $p, q \in S$, then $\NOT p \in S$, $p \AND q \in S$, $p \OR q \in S, \dots$
\item Let $M$ be a set of syntactically correct monotone formulas of propositional logic.

  Base Case: propositional variables are in M

  Constructor Case: if $p, q \in M$, then $\NOT p \in M$, $p \AND q \in M$, $p \OR q \in M, \dots$
  
  Note that $M$ is hte smallest set of formulas containing all the propositional variables and closed under $\AND$ and $\OR$.
\end{enumerate}
\end{example}

\section{Structural Induction}
Structural induction can be used to prove properties about recursively defined sets.

To prove $\forall s \in S. p(s)$, where $p: S\to \set{T, F}$ is a predicate, prove the following:
\begin{itemize}
\item $p(s)$ for all base cases $s$ given by the definition of $S$.
\item $p(s)$ for the constructor cases $s$ given by the definition of $S$, assuming $p$ is true for the components of $s$.
\end{itemize}
\begin{example}
For all $f\in M$, let $V(f)$ be the number of occurrences of propositional variables in $f$ and let $B(f)$ be the number of occurrences of binary connectives in $f$.

Let $p(f)="N(f) = B(f) + 1"$.

Base Case: If $f$ is a propositional variable, then $N(f) = 1 +B(f) = 1$, so $p(f)$ is true.

Constructor Cases: Consider $f = f'\OR f''$.

Assume $p(f')$ and $p(f'')$. Then $N(f) = N(f') + N(f'')$ and $B(f) = B(f') +B(f'') + 1$.

By inductive hypothesis, $N(f) = (B(f') + 1) + (B(f'') + 1) = (N(f')+N(f'')+1) + 1 = N(f) +1$. So $p(f)$.

Similarly, if $f = f' \AND f''$, then $P(f)$ is true.

By structural induction, $\forall f\in M. p(f)$.
\end{example}
\begin{example}

  $\NN$ can be defined recursively.

  Base Case: $0\in \NN$.

  Constructor Case: if $n\in \NN$, then $n+1\in \NN$.

  $\forall m \in \NN. \forall n \in \NN.p(m, n)$

  Let $m\in \NN$ be arbitrary.

  Prove that $\forall n\in\NN. p(m, n)$ by induction or generalisation.

  $\forall m\in\NN.\forall n\in\NN.p(m, n)$ by generalisation.

  Otherwise, define $\NN\times \NN$ recursively as follows:
  
\begin{enumerate}
\item\label{item:2} Base Case: $(0, 0) \in \NN\times\NN$.
\item Constructor Case: If $(m,n)\in \NN\times\NN$, then $(m+1, n), (m, n+1)\in \NN$.
\end{enumerate}

Then assume $p(i, j)$ for all $(i, j)$, where $i\leq m$ or $j\leq n$, and either $i< m$ or $j <n$. Then prove $p(m, n)$.

\end{example}




\end{document}