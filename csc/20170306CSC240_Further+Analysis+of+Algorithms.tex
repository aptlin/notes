% -*- coding: utf-8; -*-
%%% Local Variables:
%%% mode: latex
%%% TeX-engine: xetex
%%% TeX-master: t
%%% End:
\documentclass[11pt]{scrartcl}
\usepackage[fancy, beaue, pset, anon]{masty}
\pSet{\nt{CSC240}{}{Further Analysis of Algorithms}}
\usepackage{lineno}
% ----------------------------------------------------------------------
% Page setup
% ----------------------------------------------------------------------

\pagenumbering{gobble}

% ----------------------------------------------------------------------
% Custom commands
% ----------------------------------------------------------------------

% alignment

\newcommand*{\LongestHence}{$\Rightarrow$}% function name
\newcommand*{\LongestName}{$f_o(-x)+f_e(-x)$}% function name
\newcommand*{\LongestValue}{$(-a)x +(-a)(-y)$}% function value
\newcommand*{\LongestText}{\defi}%

\newlength{\LargestHenceSize}%
\newlength{\LargestNameSize}%
\newlength{\LargestValueSize}%
\newlength{\LargestTextSize}%

\settowidth{\LargestHenceSize}{\LongestHence}%
\settowidth{\LargestNameSize}{\LongestName}%
\settowidth{\LargestValueSize}{\LongestValue}%
\settowidth{\LargestTextSize}{\LongestText}%

% Choose alignment of the various elements here: [r], [l] or [c]

\newcommand*{\mbh}[1]{{\makebox[\LargestHenceSize][r]{\ensuremath{#1}}}}%
\newcommand*{\mbn}[1]{{\makebox[\LargestNameSize][r]{\ensuremath{#1}}}}%
\newcommand*{\mbv}[1]{\ensuremath{\makebox[\LargestValueSize][r]{\ensuremath{#1}}}}%
\newcommand*{\mbt}[1]{\makebox[\LargestTextSize][l]{#1}}%

\newcommand{\R}[1]{\label{#1}\linelabel{#1}}
\newcommand{\lr}[1]{line~\lineref{#1}}

% ----------------------------------------------------------------------
% Launch!
% ----------------------------------------------------------------------
\usepackage{program}
\begin{document}

\section{Further Analysis of Algorithms}
Consider the following function:

function square(n)

\begin{program}
  \IF n = 1
  \THEN |return | n
  \ELSE |return| (n+n-1 +|square|(n-1))
  \FI
\end{program}

Define $T:\ZZ^+\to \NN$, where $T(n)=$\textquote{the number of
  arithmetic operations performed by the square$(n)$}.

Then $T(n) = \begin{cases}
  0, n =1\\
  4+T(n-1) , n > 1
\end{cases}$.

Now, consider the mergesort algorithm:

MERGESORT($A, n$)
\begin{program}
  \IF n =1 \THEN |return|
  |divide A into 2 subarrays |A'| and |A''| of size| \ceiling{\frac{n}{2}}| and |\floor{\nicefrac{n}{2}}
  |MERGESORT|(A', \ceiling{\nicefrac{n}{2}})
  |MERGESORT|(A'', \floor{\nicefrac{n}{2}})
  \FI
  A\gets |MERGE|(A', A'')
  |return|
\end{program}

For $n\in\ZZ^{+}$, let $M(n)=$\textquote{the worst case time
  complexity of MERGESORT($A, n$) over all arrays $A$ of size $n$}.

Then $M(n) = \begin{cases}
  c, n = 1\\
  M(\ceiling{\nicefrac{n}{2}}) + M(\floor{\nicefrac{n}{2}}))+dn, n > 1
\end{cases}$, where $c, d$ are constants.

Consider now the binary search algorithm.

BINSEARCH$(A, f, l, x)$

\begin{program}
  \IF f = l \THEN
  \IF A[f]= x \THEN |return| f
  \ELSE |return | 0
  \FI
  \FI
  m \gets \floor{ \nicefrac{f+l}{2} }
  \IF A[m] \geq x \THEN
  |return| | BINSEARCH|(A, f, m, x)
  \ELSE |return| |BINSEARCH|(A, m+1, l, x)
\end{program}


Define $B:\ZZ^+\to \NN$, where $B(n)$ denotes the wors case number of
comparisons with $x$ performed by BINSEARCH$(A, f, l, x)$, where $n = l s f +1$ over all choices of $A, f, l, x$.

Then $B(n) = \begin{cases}
  1, n = 1\\
  1+\max\{B(\ceiling{\nicefrac{n}{2}}), B(\floor{\nicefrac{n}{2}})\}, n > 1
\end{cases}$
\section{Methods of Solving Recurrences}

\subsection{Guess and Verify}
\begin{itemize}
\item genereate a table of values
\item look for a pattern
\item guess a solution
\item prove it is a solution by induction
\end{itemize}
\subsection{Repeated Substitution / Plug \& Chug}
To find a closed form for $M(n)$ from above, consider the special case where $n$ is a power of $2$.

Then $M(n) = \begin{cases}
  c, n=1\\
  2M(\frac{n}{2})+dn, n>1
\end{cases}$.

Therefore,

\begin{align}
  M(n)&= 2M(\nicefrac{n}{2}) + dn\\
      &=2[2M(\nicefrac{n}{4}+d \nicefrac{n}{2}] + dn\\
      &= 4M(\nicefrac{n}{4})+ 2 dn\\
      &= 2^iM(\nicefrac{n}{2^{i}})+ i dn.
\end{align}

For $k\in\NN$, let $Q(k) =$\textquote{$M(2^k) = c2^k+dk2^k$}.

Prove that $\forall k\in\NN.Q(k)$.

\begin{theorem}
  $M(n) \in O(n\log n)$.
\end{theorem}
\begin{proof}
  \hfill

  To prove this, we show that $M(n)$ is a nondecreasing function
  $M(n+k)\geq M(n)$ for all $n\in \ZZ^+$, $k\in \NN$.

  Let $n\in\ZZ^{+}$ be arbitrary. Let $2^k$ be the smallest power of
  two that is greater than or equal to $n$, i.e.
  $k = \ceiling{\log_2 n}$.

  Since $M$ is nondecreasing, then $n \leq 2^k \leq 2n$.

  Thus, $M(n) \leq M(2^k) = c2^k+dk2^k < c2n + d2n\log(2n) = 2cn + 2dn(\log_2n+1)$.

  Hence, $\forall n\in\ZZ^+.M(n) \leq 2cn+2dn\log_2n+2dn$, and so $M(n)\in O(n\log n)$.
\end{proof}

\begin{lemma}
  $\forall m\in\ZZ^+. \forall n\in\ZZ^+.[m\leq n \THEN M(m) \leq M(n)]$.
\end{lemma}
\begin{proof}
  \hfill

  For $n\in\ZZ^+$, let $R(n) =$\textquote{$\forall.m\in\ZZ^+.m\leq n \THEN M(m) \leq M(n)$}.
  
  Let $n\in\ZZ^{+}$ be arbitrary and suppose that $R(n')$ is true for all $n'\in\ZZ^{+}$ such that $n' < n$.

  Then $M(1) = c < 2c +2d = M(2) $. Therefore, $R(1)$ and $R(2)$ are true.

  Now consider $n> 2$. Then $1\leq \floor{\nicefrac{n}{2}} \leq \ceiling{\nicefrac{n}{2}\leq n-1 < n}$.

  Note that $R(\floor{\nicefrac{n}{2}})$, $R(\ceiling{\nicefrac{n}{2}})$
  and $R(n-1)$ are all true by inductive hypothesis.

  Let $M\in\ZZ^{+}$ be arbitrary. Suppose $m \leq n$.

  If $m = n$, then $M(m) = M(n)$ by substitution.

  Suppose then $m = n-1$.

  \textbf{Exercise}: continue the proof. And look up the Master Theorem.
\end{proof}

\end{document}