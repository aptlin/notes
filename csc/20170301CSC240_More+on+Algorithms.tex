% -*- coding: utf-8; -*-
%%% Local Variables:
%%% mode: latex
%%% TeX-engine: xetex
%%% TeX-master: t
%%% End:
\documentclass[11pt]{scrartcl}
\usepackage[fancy, beaue, pset, anon]{masty}
\pSet{\nt{CSC240}{}{Analysis of Algorithms}}
\usepackage{lineno}
% ----------------------------------------------------------------------
% Page setup
% ----------------------------------------------------------------------

\pagenumbering{gobble}

% ----------------------------------------------------------------------
% Custom commands
% ----------------------------------------------------------------------

% alignment

\newcommand*{\LongestHence}{$\Rightarrow$}% function name
\newcommand*{\LongestName}{$f_o(-x)+f_e(-x)$}% function name
\newcommand*{\LongestValue}{$(-a)x +(-a)(-y)$}% function value
\newcommand*{\LongestText}{\defi}%

\newlength{\LargestHenceSize}%
\newlength{\LargestNameSize}%
\newlength{\LargestValueSize}%
\newlength{\LargestTextSize}%

\settowidth{\LargestHenceSize}{\LongestHence}%
\settowidth{\LargestNameSize}{\LongestName}%
\settowidth{\LargestValueSize}{\LongestValue}%
\settowidth{\LargestTextSize}{\LongestText}%

% Choose alignment of the various elements here: [r], [l] or [c]

\newcommand*{\mbh}[1]{{\makebox[\LargestHenceSize][r]{\ensuremath{#1}}}}%
\newcommand*{\mbn}[1]{{\makebox[\LargestNameSize][r]{\ensuremath{#1}}}}%
\newcommand*{\mbv}[1]{\ensuremath{\makebox[\LargestValueSize][r]{\ensuremath{#1}}}}%
\newcommand*{\mbt}[1]{\makebox[\LargestTextSize][l]{#1}}%

\newcommand{\R}[1]{\label{#1}\linelabel{#1}}
\newcommand{\lr}[1]{line~\lineref{#1}}

% ----------------------------------------------------------------------
% Launch!
% ----------------------------------------------------------------------

\begin{document}

\section{Iterative Algorithms}

An \textbf{iterative algorithm} is encoded without loops and with
procedure or function calls taking constant time.

If a loop is executed $O(f(n))$ times and each iteration takes
$O(g(n))$ time, then the entire loop takes the time of $O(f(n)g(n))$.

If an if-then-else statement has corresponding complexities of
$O(h(n))$, $O(f(n))$, and $O(g(n))$, then the complexity of the
statement is $O(\max\set{h(n), f(n), g(n)})$.

For procedural or functional calls, if P, with the size of input $m$,
has a running time $T(m)\in O(f(m))$,then a call to $P$ with an input
of size $g(n)$ takes the time of $O(f(g(n))$.

Let $T_{IS}:\NN \to \NN$ be such that $T_{IS}(n)$ is the maximum
number of comparisons and assignments taken by insertion sort on
arrays $A$ of length $n$.

An Insertion Sort algorithm on the list $A$ is as follows:

\begin{flagderiv}
  \step{}{i \gets 1}{}
  \assume{}{\text{while } i \leq \text{length}(A) \text{ do}}{}
  \assume{}{\text{while } j> 1 \text{ and } A[j]< B[j-1] \text{ do} }{}
  \step{}{B[j] \gets B[j-1]}{}
  \step{}{j \gets j-1}{}
  \step{}{B[j] \gets A[i]}{}
  \done
  \step{}{i \gets i+1}{}
  \conclude{}{\text{return}(B)}{}
\end{flagderiv}

\begin{lemma}
$T_{IS}(n) \leq 2n^2+4n+2$
\end{lemma}
\begin{proof}
  \hfill

  Let $n\in\NN$ be arbitrary.

  Let $A$ be an arbitrary array of length $n$.

  From the code, there are $n$ complete iterations of the outer while loop.

  Each iteration of the outer while loop consists of $4$ steps (on
  lines 2, 3, 7, 8) plus an execution of the inner while loop.

  For each complete iteration of the inner while loop at most 3 steps
  are performed (on lines $4, 5, 6$).

  During the $i$th iteration of the outer loop, there are at most
  $i-1$ complete iterations of the inner while loop.

  The final (incomplete) iteration of the inner while loop takes at
  most $2$ steps.

  Therefore, the number of steps taken by the inner while loop is at
  most $4(i-1)+2$.

  Thus, the total number of steps taken by the $n$ complete iterations of the outer while loop is at most

  \begin{equation*}
\sum_{i=1}^n[(4i-1)+2 +4] = 2n^2+4n.
  \end{equation*}

  The final incomplete iteration of the outer while loop takes 1 step and there is 1 step before the outer while loop.

  Hence $T_{IS}(n) \leq 2n^2 + 4n + 2$ by generalisation.
\end{proof}
% \begin{lemma}

% \end{lemma}
% \begin{proof}
%   \hfill

% Let $n\in \NN$ be arbitrary. Consider $A= [n, n-1, \cdots, 1]$ of length $n$.
% \end{proof}
\begin{claim*}
  For $i\in\set{1, \cdots, n}$, $4i+1$ steps are performed during the
  iteration $i$ of the outer while loop on input $A$. After the
  iteration is completed, $B$ contains the elements
  $\set{n- i + 1, \cdots, n}$.
\end{claim*}
\begin{proof}
  \hfill

  Let $P(i) =$\textquote{$4i+1$ steps are performed during the iteration
    $i$ and afterwards $\set{n-i+1, \cdots, n} \suq B$}.

  Base Case:
  $i=1$.

  5 steps are performed in iteration 1 (on lines 2. 3, 4, 7 and 8).

  By line 7,  $B$ contains $\set{n}$. Hence, $P(1)$ holds.

  Let $i< n$ and assume $P(i)$.

  During the iteration $i+1$ of the outer while loop there are $4$ steps in addition to the inner loop (on lines 2, 3, 7, 8).

  There are $i$ complete iterations of the inner while loop.

  Since $A[i+1] = n-i$, which is smaller than all the elements in $b$,
  each iteration of the inner loop takes $4$ steps (on lines 4, 5,
  6). In the final incomplete iteration of the inner while loop, 1
  step is performed and $j=1$, so $A[i+1] = n-i$ is written into
  $B[i]$, while all the other elements of $B$ have been shifter right.

  Thus, after the iteration $\set{n\i, \cdots, n} \suq B$ and $4i+5$
  steps are performed during the iteration $i+1$. Thus, $P(i+1)$ is
  true.

  The final iteration of the outer while loop takes 1 step on line 4,
  while there is also 1 step taken before the outer loop on line 1.

  Thus $t_{IS}(A) = 2+ \sum_{i=1}^n(4i+1)$, so
  $T_{IS}(n) \geq 2+\sum_{i=1}^n(4i+1)$, and thus
  $T_{IS}(n) \leq 2n^2+3n+2$.

  

\end{proof}


\end{document}