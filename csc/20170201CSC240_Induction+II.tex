% -*- coding: utf-8; -*-
%%% Local Variables:
%%% mode: latex
%%% TeX-engine: xetex
%%% TeX-master: t
%%% End:
\documentclass[11pt]{scrartcl}
\usepackage[fancy, beaue, pset, anon]{masty}
\pSet{\nt{CSC240}{VIII}{More on Induction}}
\usepackage{lineno}
% ----------------------------------------------------------------------
% Page setup
% ----------------------------------------------------------------------

\pagenumbering{gobble}

% ----------------------------------------------------------------------
% Custom commands
% ----------------------------------------------------------------------

% alignment

\newcommand*{\LongestHence}{$\Rightarrow$}% function name
\newcommand*{\LongestName}{$f_o(-x)+f_e(-x)$}% function name
\newcommand*{\LongestValue}{$(-a)x +(-a)(-y)$}% function value
\newcommand*{\LongestText}{\defi}%

\newlength{\LargestHenceSize}%
\newlength{\LargestNameSize}%
\newlength{\LargestValueSize}%
\newlength{\LargestTextSize}%

\settowidth{\LargestHenceSize}{\LongestHence}%
\settowidth{\LargestNameSize}{\LongestName}%
\settowidth{\LargestValueSize}{\LongestValue}%
\settowidth{\LargestTextSize}{\LongestText}%

% Choose alignment of the various elements here: [r], [l] or [c]

\newcommand*{\mbh}[1]{{\makebox[\LargestHenceSize][r]{\ensuremath{#1}}}}%
\newcommand*{\mbn}[1]{{\makebox[\LargestNameSize][r]{\ensuremath{#1}}}}%
\newcommand*{\mbv}[1]{\ensuremath{\makebox[\LargestValueSize][r]{\ensuremath{#1}}}}%
\newcommand*{\mbt}[1]{\makebox[\LargestTextSize][l]{#1}}%

\newcommand{\R}[1]{\label{#1}\linelabel{#1}}
\newcommand{\lr}[1]{line~\lineref{#1}}

% ----------------------------------------------------------------------
% Launch!
% ----------------------------------------------------------------------

\begin{document}

\begin{problem*}
Prove $q(n)$ is true for all even natural numbers.
\end{problem*}

\begin{soln}
  Let $p(k) = q(2k)$.

  $\forall k\in \NN. p(k)$

  means the same as

  $\forall k\in \NN. q(2k)$, which is the same as

  $\forall n \in \NN.(\text{$n$ is even $\THEN$ $q(n)$}$.

  Base Case:

  $p(0) = q(0)$

  Induction Step:

  $p(k) \THEN p(k+1)$,

  which is the same as

  $q(2k) \THEN q(2k+2)$.

  It is sufficient to prove

  $q(0) \AND \forall n\in\NN(q(n)\THEN q(n+2))$ .
\end{soln}

\begin{theorem}
  For all $n\in\ZZ^{+}$ and all $a_1, \dots, a_n\in \RR^+$,

  \begin{equation*}
    (\prod_{i=1}^na_i)^{\frac{1}{n}}\leq \frac{\sum_{i=1}^na_i}{n}
  \end{equation*}
\end{theorem}

\begin{proof}
  We prove $\forall n \in \ZZ^+.P(n)$.

  Base Case:

  $n = 2$

  Let $a_1, a_2 \in \RR^+$ be arbitrary.

  Then $0\leq (a_1-a_2)^2 = a_1^2-2a_1a_2+a_2^2$.

  Hence, $a_1^2+a_2^2\geq 2a_1a_{2}$.

  Thus, 
  \begin{equation*}
    (\frac{a_1+a_2}{2})^2= \frac{a_1^2+a_2^2+2a_1a_2}{4} \geq a_1a_2
  \end{equation*}

  Hence, $P(2)$ is true by generalisation.

  Induction Step:

  Let $n\in\ZZ^{+}$ be arbitrary and suppose $n\geq 2$.

  Assume $P(n)$.

  Let $a_1, \dots, a_{n-1}\in\RR^{+}$ be arbitrary.

  Let $b_i=a_i$ for $i=1, \dots, n-1$.

  Let $b_n = \frac{a_1+a_2+\dots+a_{n-1}}{n-1}$.

  By specialisation of $p(n)$,

  \begin{align}
    b_1\cdots b_{n-1}b_n\leq (\frac{b_1+\cdots+b_n}{n})^n & = (\frac{b_1+\cdots+b_n}{n})^n \\
                                                          & =  (\frac{a_1+\cdots+a_{n-1}+b_n}{n})^n \\
                                                          &  =  (\frac{(n-1)b_n+b_n}{n})^n \\
                                                          &  =  b_n^n 
  \end{align}

  Therefore, $b_1b_2\cdots b_{n-1} \leq b_n^{n-1}$.

  Hence, $P(n-1)$ is true by generalisation.

  Let $a_1, \dots, a_n\in\RR^+$ be arbitrary.

  Let $b_1=\frac{a_1+\cdots+a_n}{n}$ and $b_2=\frac{a_{n+1}+\cdots+a_{2n}}{n}$.

  By specialisation of $P(n)$,
  
  \begin{equation*}
    \prod_{i=1}^na_i\leq (\frac{1}{n}\sum_{i=1}^na_i)^n
  \end{equation*}
  and
  
  \begin{equation*}
    \prod_{i=n+1}^{2n}a_i\leq (\frac{1}{n}\sum_{i=n}^{2n}a_i)^n
  \end{equation*}

  and by specialisation of $P(2)$,

  \begin{equation*}
    b_1b_2\leq (\frac{b_1+b_2}{2})^2
  \end{equation*}

  Hence

  \begin{equation*}
  \prod_{i=1}^{2n}a_i \leq (\frac{\sum_{i=1}^na_i}{n})(\frac{\sum_{i=n+1}^{2n}a_i}{n})^n=(b_1b_2)^n\leq(\frac{b_1+b_2}{n})^{2n}.
\end{equation*}

Note that $(\frac{b_1+b_2}{n})^{2n} = (\frac{1}{2n}\sum_{i=1}^{2n}a_i)^{2n}$.

By generalisation, $P(2n)$ is true.

$\forall n\in\NN[(n\geq 2 \AND P(n))\THEN P(2n)]$.

Therefore, by induction,

$\forall n\in\ZZ^+.P(n)$
\end{proof}
\subsection{Induction in Finite Sets}

\begin{problem*}
Prove $\forall i\in\set{0, \dots, n}.P(i)$.
\end{problem*}

\begin{soln}
  Base Case:

  $p(0)$

  Induction Step:

  Let $i\in\set{0, \dots, n-1}$ be arbitrary.

  Assume $p(i)$.

  $\vdots$

  $p(i+1)$.

  $\forall i\in \set{0, \dots, n-1}.[p(i) \THEN p(i+1)]$

  $\forall i\in\set{0, \dots, n} p(i)$ by induction
  
\end{soln}

\subsection{Strong Induction}

To prove $\forall i\in \NN. p(i)$ it suffices to prove that

\[\forall i\in\NN.\forall j\in\NN.[\big((j<i)\THEN p(j)\big)\THEN P(i)]\]
  
The only difference of the strong induction from the weak induction is
$p(0), \dots, p(i-1)$.

A template proof follows.
\begin{proof}
Let $i\in\NN$ be arbitrary.

Assume $\forall j\in\NN.(j<i \THEN p(i)$.

\textit{$\dots$ various cases, including the base case $\dots$}

$p(i)$

$\forall i\in \NN[\forall j\in\NN.(j <i)\THEN p(j)]\THEN p(i)$ by direct proof and generalization.

$\forall i\in\NN.p(i)$ by strong induction
\end{proof}

\begin{theorem}
  For all $n\geq 4$, exactly a sum of $n$ can be exchanged in coins
  with nomination 2 and 5\$ bills.
\end{theorem}
\begin{proof}
  Let $p(n) = \exists f\in\NN.\exists g\in\NN.(n = 2f+ 5g)$ for all $n\in\NN$.

  Let $n\in\NN$ be arbitrary.

  Suppose $n\geq 4$ and $\forall j\in \NN.(4\leq j < n \THEN p(j))$.

  If $n=4$, then $n = 2\* 2 +0\*5$.

  If $n=5$, then $n = 0\*2 + 1\*5$.

  If $n\leq 6$, then $4\leq n-2 <n$. Then $P(n-2)$ is true by specialisation.
\end{proof}





\end{document}