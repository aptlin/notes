% -*- coding: utf-8; -*-
%%% Local Variables:
%%% mode: latex
%%% TeX-engine: xetex
%%% TeX-master: t
%%% End:
\documentclass[11pt]{scrartcl}
\usepackage[fancy, beaue, pset, anon]{masty}
\pSet{\nt{CSC240}{}{Analysis of Algorithms}}
\usepackage{lineno}
% ----------------------------------------------------------------------
% Page setup
% ----------------------------------------------------------------------

\pagenumbering{gobble}

% ----------------------------------------------------------------------
% Custom commands
% ----------------------------------------------------------------------

% alignment

\newcommand*{\LongestHence}{$\Rightarrow$}% function name
\newcommand*{\LongestName}{$f_o(-x)+f_e(-x)$}% function name
\newcommand*{\LongestValue}{$(-a)x +(-a)(-y)$}% function value
\newcommand*{\LongestText}{\defi}%

\newlength{\LargestHenceSize}%
\newlength{\LargestNameSize}%
\newlength{\LargestValueSize}%
\newlength{\LargestTextSize}%

\settowidth{\LargestHenceSize}{\LongestHence}%
\settowidth{\LargestNameSize}{\LongestName}%
\settowidth{\LargestValueSize}{\LongestValue}%
\settowidth{\LargestTextSize}{\LongestText}%

% Choose alignment of the various elements here: [r], [l] or [c]

\newcommand*{\mbh}[1]{{\makebox[\LargestHenceSize][r]{\ensuremath{#1}}}}%
\newcommand*{\mbn}[1]{{\makebox[\LargestNameSize][r]{\ensuremath{#1}}}}%
\newcommand*{\mbv}[1]{\ensuremath{\makebox[\LargestValueSize][r]{\ensuremath{#1}}}}%
\newcommand*{\mbt}[1]{\makebox[\LargestTextSize][l]{#1}}%

\newcommand{\R}[1]{\label{#1}\linelabel{#1}}
\newcommand{\lr}[1]{line~\lineref{#1}}

% ----------------------------------------------------------------------
% Launch!
% ----------------------------------------------------------------------

\begin{document}

\section{Analysis of Algorithms}

Let $\SF$ denote the set of all functions from $\NN$ to $\RR^{+}$.

For any $f\in\SF$, let
\[O(f) = \set{g\in\SF; \exists c \in\RR^+.\exists b\in\NN.\forall
    n\in\NN.(n \geq b \THEN g(n) \leq c f(n)}.\]

For an algorithm $A$, let $t(I)$ be the number of steps the algorithm
$A$ takes to halt on input $I$.

And what is a step?

Pick 1 or 2 operations such that the total number of operations
performed by $A$ is the same as the number of these opoerations
performed by $A$, to within a constant factor.

\subsection{Properties of $O$ Notation}

\begin{itemize}
\item For any $c\in\RR^+\cup{0}$, $c f(n) \in O(f(n))$ and
  $f(n) \in O(c f(n))$.
\item If $\lim_{n\to \infty} \frac{h(n)}{g(n)} = 0$, then
  $g(n) + h(n) \in O(g(n))$
\item If $f(n)\in O(g(n))$ and $g(n) \in O(h(n))$, then $f(n)\in (h(n))$
\item If $f_1\in O(g_1)$ and $f_2\in O(g_2)$, then $f_1+f_2 \in O(g_1+g_2)$.
\item $\max \set{f, g} \in O(f+g)$
\item $f+g \in O(\max\set{f, g})$
\item If $f_1\in O(g_1)$ and $f_2\in O(g_2)$, then $f_1f_2\in O(g_1g_2)$
\item Leh $a+b$ be constant. If $a< b$,then $n^a\in O(n^{b}$
\item If $1 < a < b$, then $a^n\in O(b^n)$, but $b^n not\in O(a^n)$
\item For all $a, b > 1$, $\log_a(n)\in O(\log_b(n))$.
\end{itemize}

\begin{example}

  Consider the following algorithm $LS(L, x)$ such that, if $x$ occurs
  in $L$, the algorithm returns an index of $L$ at which $x$
  occurs. Otherwise, return $0$. Let $L$ be  an array with the index of $1$:

  i <- 1
  
  while $i\leq$ length(L) do

  if $L[i] = x$

  then return $i$

  i < i+1

  end while

  return 0.

  Now, count the number of comparisons with $x$. Suppose that each
  iteration of the loop performs $O(1)$ steps (assuming LENGTH takes
  $O(1)$, and outside the loop $O(1)$ steps are performed.

  Now, we can express the complexity as a function of the input size:
  \begin{align}
    T_A: \NN \to \NN\\
    T_A(n)=\max\set{t_A(I); \text{size}(I) = n},
  \end{align}
  which gives the worst case time complexity of an algorithm $A$. For $LS$, $\text{size}((L, x)) = \text{length}(L)$
\end{example}
\begin{example}
  Now we can estimate the average cas time complexity. Define $T'_A:\NN\to\RR^+\cup\set{0}$, where $T_A'(n) = \EE[t_{A}]$, where the expectation is taken over a probability space of all inputs of size $n$. If all inputs of size $n$ are equally likely,  then $T'_A(n) = \frac{\sum \set{t_A(I); \text{size}(I) =n}}{\#\set{I; \text{size}(I) = n }}$


\end{example}
\end{document}