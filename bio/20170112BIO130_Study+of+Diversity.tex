% -*- coding: utf-8; -*-
%%% Local Variables:
%%% mode: latex
%%% TeX-engine: xetex
%%% TeX-master: t
%%% End:
\documentclass[11pt]{scrartcl}
\usepackage[fancy, beaue, pset, anon]{sdll}
\pSet{\nt{BIO130}{III}{Study of Diversity}}
\usepackage{lineno}
% ----------------------------------------------------------------------
% Page setup
% ----------------------------------------------------------------------

\pagenumbering{gobble}

% ----------------------------------------------------------------------
% Custom commands
% ----------------------------------------------------------------------

% alignment

\newcommand*{\LongestHence}{$\Rightarrow$}% function name
\newcommand*{\LongestName}{$f_o(-x)+f_e(-x)$}% function name
\newcommand*{\LongestValue}{$(-a)x +(-a)(-y)$}% function value
\newcommand*{\LongestText}{\defi}%

\newlength{\LargestHenceSize}%
\newlength{\LargestNameSize}%
\newlength{\LargestValueSize}%
\newlength{\LargestTextSize}%

\settowidth{\LargestHenceSize}{\LongestHence}%
\settowidth{\LargestNameSize}{\LongestName}%
\settowidth{\LargestValueSize}{\LongestValue}%
\settowidth{\LargestTextSize}{\LongestText}%

% Choose alignment of the various elements here: [r], [l] or [c]

\newcommand*{\mbh}[1]{{\makebox[\LargestHenceSize][r]{\ensuremath{#1}}}}%
\newcommand*{\mbn}[1]{{\makebox[\LargestNameSize][r]{\ensuremath{#1}}}}%
\newcommand*{\mbv}[1]{\ensuremath{\makebox[\LargestValueSize][r]{\ensuremath{#1}}}}%
\newcommand*{\mbt}[1]{\makebox[\LargestTextSize][l]{#1}}%

\newcommand{\R}[1]{\label{#1}\linelabel{#1}}
\newcommand{\lr}[1]{line~\lineref{#1}}

% ----------------------------------------------------------------------
% Launch!
% ----------------------------------------------------------------------

\begin{document}

\section{Tree of Life}

\begin{itemize}
\item Prokaryotes: Eubacteria

  \begin{description}

  \item[e.g.] Cyanobacteria are capable of photosynthesis.

  \end{description}

\item Prokaryotes: Archaea

  DNA replication, transcription, translation are similar to these of
  eucarya, and hence archaea are closer to eucarya than to eubacteria.

  \begin{description}

  \item[e.g.] Methanococcus: predominantly live in hydrothermal vents on the bottom of the ocean

  \item[e.g.] Methanobacterium: anaerobic, low pH conditions

  \end{description}


\item Eucarya

  \begin{description}

  \item[e.g.] animals

  \end{description}
\end{itemize}

\section{Study of Diversity}

\begin{enumerate}
\item\label{item:1} XIX and early XX: use large collections and study
  each species independently
\item\label{item:2} Late XX and XXI: use model organisms
\end{enumerate}

\begin{definition}
  Model organisms are species that have been studied intensively over
  a long period of time and thus serving as models for deriving
  fundamental biological principles.
\end{definition}

General attributes:

\begin{itemize}
\item Rapid development with short life cycles

\item Small size

\item Readily available

\item Tractable -- easy to manipulate and modify

\item Have understandable genomics
\end{itemize}

\begin{description}

\item[e.g.] elephant is a horrible model organism!
\item[e.g.] eukaryotes: zebra fish

\end{description}

\begin{ques*}

How does the diversity can be explained?

\end{ques*}

\begin{answer*}

  Central dogma:

  DNA  $\xrightarrow{\text{transcription}}$  RNA  $\xrightarrow{\text{translation}}$  protein

\end{answer*}
\subsection{Central Dogma}
There are different kinds of RNA:

\begin{itemize}
\item mRNA -- produces protein by translation
\item tRNA -- transport AAs and for protein synthesis
\item rRNA -- part of the ribosome
\end{itemize}

There are other kinds of RNAs, some have nothing to do with making
proteins.

\subsection{Elaborated Central Dogma}
\begin{table*}[h]
  \centering

  \begin{tabulary}{1.0\textwidth}{C R C L L}
    \multicolumn{5}{l}{  \textbf{genome}: complete set of DNA sequences in a cell or organism}\\ \cmidrule{1-5}
    \multicolumn{5}{l}{ \textbf{transcriptome}: complete set of RNA sequences in a cell or organism}\\ \cmidrule{2-4}
    \multicolumn{5}{l}{ \textbf{proteome}: complete set of protein sequences in a cell or organism}\\ \cmidrule{3-4}
    &                                      &     &                                    & \\ \cmidrule{3-4}
    DNA                                                                                                                                                                             & $\xrightarrow{\text{transcription}}$ & RNA & $\xrightarrow{\text{translation}}$ & Protein                            \\
    \shortstack[l]{organisation                                                                                                                                      \\replication} &                                      &     &                                    & \shortstack[l]{\textbf{interactome} \\ a complete set of interactions\\ in a cell or
                                                                                                                        organism}                                    \\
                                                                                                                                                                                    &                                      &     &                                    & \shortstack[l]{\textbf{metabolome} \\a complete set of small\\ molecule metabolites\\
                                                                                                                                                                                      in a cell or organism}\\
                                                                                                                                                                                    \toprule
                                                                                                                      \multicolumn{5}{l}{  \textbf{phenome}: a complete set of phenotypes}
                                                                                                                    \end{tabulary}
\end{table*}

\section{Review}

DNA, RNA and proteins are linear chains of
information. The order of aminoacids contains the information.

Information in nucleic acid sequence is translated into an aminoacid
sequence via a genetic code which is essentially universal among all
species.

\subsection{Eukaryotes}
Nuclear agenda:
\begin{enumerate}
\item\label{item:3} Transcription

\item\label{item:4} RNA processing

\item\label{item:5} Nuclear export

\item\label{item:6} Translation

\item\label{item:7} Protein folding and modification
\end{enumerate}

\subsection{Nucleic Acids}

\begin{enumerate}

\item\label{item:8} The genetic material in a cell is a blueprint for an organsim.

\item\label{item:9} DNA is a deoxyribonucleic acid

\item\label{item:10} Ribonucleic acid
\end{enumerate}

Three parts of a nucleic acid:

\begin{enumerate}
\item\label{item:11} Pentose sugar -- scaffold for a base

\item\label{item:12} Nitrogenous base -- varies

\item\label{item:13} Phosphate group -- backbone, can be 1P, 2P's or 3P's
\end{enumerate}

\subsubsection{Bases}

\begin{itemize}
\item Purines -- guanine and adenine (Al Gore stings - Pu!)

\item Pyrimidines -- cytosine, thymine and uracil (U C The Pyramides)
\end{itemize}

\subsubsection{DNA vs RNA}

\begin{enumerate}
\item\label{item:14} DNA: G, C, A, T, RNA: G, C, A, U

\item\label{item:15} DNA: Deoxyribose, RNA: ribose
\end{enumerate}

\subsubsection{Nucleic Acid Nomenclature}

\begin{enumerate}
\item\label{item:16} Nucleoside monophosphate: sugar + base + 1P

\item\label{item:17} Nucleoside diphosphate: sugar + base + 2P


\item\label{item:18} Nucleoside triphosphate: sugar + base + 3P

  Nucleoside: base + sugar
  Nucleotide: base + sugar + at least one P
\end{enumerate}
\begin{remark}\hfill\\
  Adenoside: nucleoside\\
  AMP: nucleotide, nucleoside monophosphate\\
  ADP: nucleotide, nucleoside diphosphate\\
  ATP: nucleotide, nucleoside triphosphate
\end{remark}

\section{Molecular Interactions and Introduction to Proteins}

In nature, structure and function are closely intertwined.

Structure of DNA makes it stable and allows it to function as a
hereditary material.

\subsection{Molecular Interactions}
 Interactions between individual molecules usually mediateed by
noncovalent attactions (within a molecule, covalent nature of
interactions is prevalent, cf Karp's p.33-42)

\begin{enumerate}
\item\label{item:20} Electrostatic attractions

\item\label{item:19} Hydrogen bonds

\item\label{item:21} van der Waals attractions

\item\label{item:22} Hydrophobic interactions the tendency of nonpolar
  molecules to aggregate to minimize interaction with surrounding
  polar molecules
\end{enumerate}
\subsection{Properties of Nucleic Acids}

\begin{enumerate}
\item\label{item:23} DNA is synthesised from deoxyribonucleoside
  triphosphates, otherwise known as dNTPs

\item\label{item:24} RNA is synthesised from ribonucleoside
  triphosphates, also known as NTPs

\item\label{item:25} Nucleotides are linked by phosphodiester bonds
  (links 1' carbon of one nucleotide to 3' carbon of the other
  nucleotide)
\end{enumerate}

Base pairing holds the DNA double helix together with the H-bonding:
\begin{enumerate}
\item\label{item:27} A - T: 2 H-bonds

\item\label{item:28} G - C: 3 H-bonds
\end{enumerate}

However, H-bonds are not the only interactions holding DNA together --
van der Waals attractions (in general, between the bases) and
hydrophobic interactions (in general, between the bases) also play a
role.

Between two strands the interactions are noncovalent -- strictly
speaking, DNA consists of two molecules. However, some scientists
consider DNA as a single molecule composed of two strands. Both are
fine.


DNA of one complete turn have on average 10.5 base pairs.

Gaps on either side are not the same -- there is a major groove on one
side, and a minor groove on the other.

5' end is close to a phosphate (-PO$_{4}$), 3' is close to OH.

The strands in a double helix are antiparallel, with DNΑ backbone
having a negative charge. Base stacking contributes significantly to
the stability of the helix.

At high enough temperatures (called melting temperature) or at high
pH, the noncovalent bonds are broken, which denaturates DNA. If cooled
or if pH is lowered, the bases will recombine.

Covalent bonds can also be broken above $\dgc{100}$, but this situation
is not a focus of biology.

The sequence of the two strands is complementary, and the strands can
be unzipped. This is important for DNA replication (including PCR) and
RNΑ synthesis.

G-C rich DNA is more stable at high temperatures.

\subsection{Introduction to Proteins}

\begin{enumerate}
\item\label{item:29} Primary (sequence)
  \begin{description}

  \item[e.g.] AA sequence

  \end{description}


\item\label{item:30} Secondary (local folding)

  \begin{description}

  \item[e.g.] a helix, $\beta$ sheet

  \end{description}


\item\label{item:31} tertiary (long-range folding)

  \begin{description}

  \item[e.g.] 3D structure

  \end{description}


\item\label{item:32} Quaternary (multimeric organization)

  \begin{description}

  \item[e.g.] multiple polypeptide chains

  \end{description}


\item\label{item:33} Multiprotein complexes

  \begin{description}

  \item[e.g.] molecular machines

  \end{description}
\end{enumerate}

\subsection{Protein Structure}

Proteins are composed of amino acids.

The amino acid side-chain, or $R$ group, is variable and determines
the type of amino acid.

4 main categories
\begin{itemize}
\item polar charged
\item polar uncharged
\item nonpolar
\item those with unique properties
\end{itemize}

$\alpha$ carbon is attached to the side chain, amino and carboxyl group.

Cysteine looks different in reduced conditions (found in cytosol) and
oxidized conditions (lumen of organellesd). A disulphide bond often
acts as a base to stabilise the protein structure.

\begin{definition}
  The \textbf{genetic code} is the set of rules specifying the
  correspondence between nucleotide triplets (codons) in DNA or RNA
  and amino acids in proteins.

  It is degenerate (multiple options for some amino acids) and almost
  universal.
\end{definition}

\begin{ques*}

What is the minimum number of mutational steps between amino acids?

\end{ques*}

\begin{answer*}

The number of mutations between codons for different amino acids.

\end{answer*}

\begin{example}

How many mutational steps required to get from a codon for proline to one for cysteine, at a minimum?

\end{example}

\begin{answer*}

Two, CCG $\to$ UGC

\end{answer*}

Groups of aminoacids with similar properties tend to be clustered in
the genetic code table. Codons of aminoacids with similar properties
tend to have fewer mutational steps between them.  One random mutation
in a codon is less likely to result in a dramatic change in amino acid
properties than two random mutations.
\end{document}