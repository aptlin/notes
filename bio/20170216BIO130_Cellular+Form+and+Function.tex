% -*- coding: utf-8; -*-
%%% Local Variables:
%%% mode: latex
%%% TeX-engine: xetex
%%% TeX-master: t
%%% End:
\documentclass[11pt]{scrartcl}
\usepackage[fancy, beaue, pset, anon]{masty}
\pSet{\nt{BIO130}{}{Cellular Form and Function}}
\usepackage{lineno}
% ----------------------------------------------------------------------
% Page setup
% ----------------------------------------------------------------------

\pagenumbering{gobble}

% ----------------------------------------------------------------------
% Custom commands
% ----------------------------------------------------------------------

% alignment

\newcommand*{\LongestHence}{$\Rightarrow$}% function name
\newcommand*{\LongestName}{$f_o(-x)+f_e(-x)$}% function name
\newcommand*{\LongestValue}{$(-a)x +(-a)(-y)$}% function value
\newcommand*{\LongestText}{\defi}%

\newlength{\LargestHenceSize}%
\newlength{\LargestNameSize}%
\newlength{\LargestValueSize}%
\newlength{\LargestTextSize}%

\settowidth{\LargestHenceSize}{\LongestHence}%
\settowidth{\LargestNameSize}{\LongestName}%
\settowidth{\LargestValueSize}{\LongestValue}%
\settowidth{\LargestTextSize}{\LongestText}%

% Choose alignment of the various elements here: [r], [l] or [c]

\newcommand*{\mbh}[1]{{\makebox[\LargestHenceSize][r]{\ensuremath{#1}}}}%
\newcommand*{\mbn}[1]{{\makebox[\LargestNameSize][r]{\ensuremath{#1}}}}%
\newcommand*{\mbv}[1]{\ensuremath{\makebox[\LargestValueSize][r]{\ensuremath{#1}}}}%
\newcommand*{\mbt}[1]{\makebox[\LargestTextSize][l]{#1}}%

\newcommand{\R}[1]{\label{#1}\linelabel{#1}}
\newcommand{\lr}[1]{line~\lineref{#1}}

% ----------------------------------------------------------------------
% Launch!
% ----------------------------------------------------------------------

\begin{document}

\section{Cellular Form and Function}

\subsection{Saccharomyces cerevisiae}

Saccharomyces cerevisiae is an example of a single-celled eukaryotic organism. Note that, in comparison, there are much more diversity in the type of a cell in humans -- more than 200 types of cells make humans work!

\subsection{Typical Animal Cell}

\textbf{Lysosome} consists of remnants of the degradation of cellular
parts that are not longer needed. Note that lysosome is animal cell specific.

\textbf{Extracellular matrix} is the specialised material outside of a cell.

\subsection{Typical Plant Cell}

Typical plant cells are not found in animal cells.

\textbf{Cell wall} is a tough protective outer coat.

\textbf{Vacuoles} are of two types, one functioning like an animal lysosome, while the other are used for storage.

\subsection{Membrane Structure}

\textbf{Cytoplasm} contains the contents of the cell outside of the nucleus.

\textbf{Cytosol} is the aqueous part of the cytoplasm, without the membrane-bound cells.

\textbf{Lumen} is the interior of organelles.

Many cellular functions occur at membranes: ompartmentalization, scaffolding for biochemical activities, etc.

\subsection{Membrane Bilayers}

Singer and Nicolson (1972) describes membrane bilayers with a fluid mosaic model. Thus, there is one membrane, but with two layers or leaflets.

The lipids of the membrane have an ampipathic property, thus
possessing hydrophilic or polar head groups and hydrophobic tails.

\subsection{Membrane Lipids}

There are many kind of lipids. One kind of a lipid is a phospholipid. One example of a phospholypid is phosphoglyceride, and there are many phosphoglycerides.

Membrane lipids have the polar head group interacting with water, two hydrophobic hydrocarbon tails interacting with other hydrophobic tails, which in an aqueous environment phospholipids spontaneously self-associate into a bilayer.

Lipid bilayers, called liposomes, can be obtained artificially. These are used to study lipid and membrane protein properties, while also providing a promising method of drug delivery.

\subsection{Phosphoglycerides}

Phosphoglycerides have three main groups (phosphate group, glycerol group, and two hydrophobic fatty acid chains).

Hydrocarbon tails have the carbon skeleton length of approximately 14-24, and can be saturated or unsaturated.

\subsection{Steroids}

Steriods are another kind of membrane lipids.

In animals, all steroids are mainly cholesterol. In plants, there are
also plant steroids. Therey decrease the mobility of phospholipid
tails, while making plasma membrane less permeable to polar molecules.

There can be up to 1:1 ratio of cholesterol and phospholipids.

\subsection{Cell Membranes are Fluid}

Phospholipids rapidly diffuse laterally within each leaflet.

Phospholipids rarely move from one leaflet to another on their own.

They move in the membrane by the way of flexion (free tail wobble),
rotation and lateral shift, going around other tails.

A membrane can be deformed without causing damage.

For example, laser tweezers can be used to deform it without breaking.

At lower temperatures lipid bilayers become rigid (more
gel-like).

Presence of cis-double bonds allows the membrane to remain
fluid at lower temperatures.

Shorter hydrocarbon tails increase fluidity at lower temperatures,
since the lipid tails interact less.

Enzymes in the cell membrane flip lipids from one leaflet to another,
with the phospholipid translocators catalysing the rapid
flip-flop. This is needed, because phospholipids are synthesized in
cytosolic leaflet of endoplasmic reticulum.

The nucleus has two membranes: an inner and an outer membrane.

In mitochondria, there are also two layers.

\subsection{Assymetry of the Lipid Bilayer}

Glycolipids and glycoproteins, found on the exoplasmic leaflet, are formed by adding sugar groups. Phospholipids, in turn, are assymetrically distributed, and some can bind cytosolic proteins.
\end{document}