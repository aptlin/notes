%%% Local Variables:
%%% mode: latex
%%% TeX-master: t
%%% End:

\documentclass[11pt]{scrartcl}
\usepackage[fancy, beaue, pset, anon]{sdll}
\pSet{\nt{BIO130}{I}{Intro to Cells, Diversity and Nucleic Acids}}
\usepackage{lineno}
% ----------------------------------------------------------------------
% Page setup
% ----------------------------------------------------------------------

\pagenumbering{gobble}

% ----------------------------------------------------------------------
% Custom commands
% ----------------------------------------------------------------------

% alignment

\newcommand*{\LongestHence}{$\Rightarrow$}% function name
\newcommand*{\LongestName}{$f_o(-x)+f_e(-x)$}% function name
\newcommand*{\LongestValue}{$(-a)x +(-a)(-y)$}% function value
\newcommand*{\LongestText}{\defi}%

\newlength{\LargestHenceSize}%
\newlength{\LargestNameSize}%
\newlength{\LargestValueSize}%
\newlength{\LargestTextSize}%

\settowidth{\LargestHenceSize}{\LongestHence}%
\settowidth{\LargestNameSize}{\LongestName}%
\settowidth{\LargestValueSize}{\LongestValue}%
\settowidth{\LargestTextSize}{\LongestText}%

% Choose alignment of the various elements here: [r], [l] or [c]

\newcommand*{\mbh}[1]{{\makebox[\LargestHenceSize][r]{\ensuremath{#1}}}}%
\newcommand*{\mbn}[1]{{\makebox[\LargestNameSize][r]{\ensuremath{#1}}}}%
\newcommand*{\mbv}[1]{\ensuremath{\makebox[\LargestValueSize][r]{\ensuremath{#1}}}}%
\newcommand*{\mbt}[1]{\makebox[\LargestTextSize][l]{#1}}%

\newcommand{\R}[1]{\label{#1}\linelabel{#1}}
\newcommand{\lr}[1]{line~\lineref{#1}}

% ----------------------------------------------------------------------
% Launch!
% ----------------------------------------------------------------------

\begin{document}

% ----------------------------------------------------------------------
% Body
% ----------------------------------------------------------------------

\section{Introduction to Cells, Diversity and Nucleic Acids}
\label{sec:intro}

\subsection{Prokaryotic and Eukariotic Cells}
\label{subsec:preu}
\subsubsection{Prokaryotic Cells}
\begin{definition*}
  Prokaryotic cells:
  \begin{itemize}
  \item No Nuclei
  \item Single-celled (mostly)
  \end{itemize}
\end{definition*}
\begin{description}

\item[e.g.] eubacteria (inc. bacteria), archaea (more extreme environments)

\end{description}

\begin{description}

\item[Capsule:]  a polysacchiride layer for protection from engulfment by
  eukariotic cells (optional)

\item[Cell wall:] tough protective outer coat (optional)

\item[DNA in a nucleoid:] compact structure of DNA and proteins

\item[Ribosome:]  making protein

\item[Bacterial flagellum:] locomotion (optional)

\item[Pilus:] sexual conjugation, locomotion (optional)

\item[Cytoplasm:] fluid inside the cell

\end{description}

\subsubsection{Eukariotic Cells}
\begin{definition*}
  Eukariotic cells:
  \begin{itemize}
  \item Nuclei
  \item Single-celled or multicellular
  \end{itemize}  
  
\end{definition*}
\begin{description}

\item[e.g.] plants, fungi, animals

\end{description}

Typically 1000 times larger in volume \(\ra\) have cytoskeleton
\begin{description}

\item[Mitochondria:] power cell (produces ATP)

\begin{ques*}

  What is the difference between nucleoid and nucleus?

\end{ques*}

  When comparing plants to animal cells, plants have a cell wall
  (animals don't), chloroplasts (animals don't), vacuoles (animals don't).

\item[Cell wall:] tough protective coat

\item[Chloroplasts:] photosynthesis

\item[Vacuoles:] Type I -- storage, Type II -- animal lysosome in degradation

\item[Endosymbiotic Theory:] Did eukariotes evolve from prokaryotes?

\end{description}


\subsection{Origins of Eukaryotes}
\label{subsec:euk}

Not all prokaryotes have non-enclosed internal membranes (some do),
but most eukaryotes do, most of them enclosed.

Prokaryotes originally could not use oxygen (did not have mitochondria).

Prokaryotes with internal membrane could engulf particles.

Ancient symbiosis was formed: ATP from mitochondria (free living
aerobic prokaryotes able to use oxygen to generate ATP) for protection
from prokaryotes.

Chloroplasts are also thought to evolve the same way.

Mitochondria and chloroplasts have their own DNA, RNA, ribosomes and
proteins. In addition, they would have a double membrane (two
phospholipid biolayers). Even today there are carnivorous
single-celled eukaryotes (eg Didinium, Neutrophils).

\end{document}