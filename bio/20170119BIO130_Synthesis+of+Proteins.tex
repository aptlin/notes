% -*- coding: utf-8; -*-
%%% Local Variables:
%%% mode: latex
%%% TeX-engine: xetex
%%% TeX-master: t
%%% End:
\documentclass[11pt]{scrartcl}
\usepackage[fancy, beaue, pset, anon]{sdll}
\pSet{\nt{BIO130}{II}{Synthesis of Proteins}}
\usepackage{lineno}
% ----------------------------------------------------------------------
% Page setup
% ----------------------------------------------------------------------

\pagenumbering{gobble}

% ----------------------------------------------------------------------
% Custom commands
% ----------------------------------------------------------------------

% alignment

\newcommand*{\LongestHence}{$\Rightarrow$}% function name
\newcommand*{\LongestName}{$f_o(-x)+f_e(-x)$}% function name
\newcommand*{\LongestValue}{$(-a)x +(-a)(-y)$}% function value
\newcommand*{\LongestText}{\defi}%

\newlength{\LargestHenceSize}%
\newlength{\LargestNameSize}%
\newlength{\LargestValueSize}%
\newlength{\LargestTextSize}%

\settowidth{\LargestHenceSize}{\LongestHence}%
\settowidth{\LargestNameSize}{\LongestName}%
\settowidth{\LargestValueSize}{\LongestValue}%
\settowidth{\LargestTextSize}{\LongestText}%

% Choose alignment of the various elements here: [r], [l] or [c]

\newcommand*{\mbh}[1]{{\makebox[\LargestHenceSize][r]{\ensuremath{#1}}}}%
\newcommand*{\mbn}[1]{{\makebox[\LargestNameSize][r]{\ensuremath{#1}}}}%
\newcommand*{\mbv}[1]{\ensuremath{\makebox[\LargestValueSize][r]{\ensuremath{#1}}}}%
\newcommand*{\mbt}[1]{\makebox[\LargestTextSize][l]{#1}}%

\newcommand{\R}[1]{\label{#1}\linelabel{#1}}
\newcommand{\lr}[1]{line~\lineref{#1}}

% ----------------------------------------------------------------------
% Launch!
% ----------------------------------------------------------------------

\begin{document}

\section{Synthesis of Proteins}

\begin{definition}
  \textbf{Condensation reaction} involves the removal of a molecule of
  water from the reagents.
\end{definition}

If $n$ aminoacids are stitched together, $n-1$ peptide bonds form in a
\textbf{polypeptide chain} with two ends, one called
\textit{C-terminus} close to a carbonyl carbon and the other
\textit{N-terminus} close to an amide nitrogen, with the alkyls
attached to a main chain/peptide bond. The N-terminus is to the left
by convention.

In the carbonyl bond, a carbonyl carbon is bonded to an amide
nitrogen.

The order of aminoacids is very important.

\begin{description}

\item[e.g.] Leu-Enkephalin, a pentapeptide of Tyr-Gly-Gly-Phe-Leu, is
  a natural opioid peptide which modulates the perception of pain.
  Leu-Phe-Gly-Gly-Tyr does not have any effects.

\end{description}

Hydrogen bonds consists of a polypeptide backbone, with hydrogen bonds
as stabilisors forming between the carbonyl-oxygen and amide
hydrogen. The bonds are 4 aminoacids apart and within the same
strand.. Alkyl groups do not increase the stability of the helix.

\begin{ques*}
Why does an alpha helix has a secondary structure?
\end{ques*}

\subsection{$\beta$ sheet}
\begin{remark}
  Oxygen is coded as red.
  Carbon is coded as black.
  Hydrogen is coded as white.
  Nitrogen is coded as black.
  Alkyl groups are coded as purple.
\end{remark}

The $\beta$ sheet also has a backbone with hydrogen bonds as
stabilisors forming between the amide hydrogen and carbonyl oxygen.

Typically there is a hydrogen bonding between aminoacids in different
strands (usually in the same polypeptide chain).

\subsection{Specialised Alpha Helix: Coiled Coil}

Although the alkyl groups do not stabilise the helix, if two stripes
of side chains in the helix are such that one of them is hydrophobic
and the other is hydrophilic, then a coil will be coiled, giving it
different biochemical and biophysical properties in different
basis. This coiled coil is called \textit{amphipathic}, has a
supersecondary structure.

\subsection{Tertiary Structure: Rhodopsin}

Phodopsin has a three-dimensional structure held together by hydrophobic
interactions, non-covalent bonds and covalent disulfide bonds. In this
case, a single polypeptide chain makes up the proteins.

\subsection{Protein Domains}
The domains are often specialized for different functions:
\begin{itemize}
\item Structural and functional unit
\item Independent folding
\item Independent/semi-independent function
\item Important in the evolution of proteins
\end{itemize}

Note that one polypeptide chain may have multiple domains.

\subsection{Quaternary Structure: Hemoglobin}

\begin{itemize}
\item Hemoglobin protein formed from different subunits: 2$\alpha$, 2$\beta$
\item Each subunit is a separate polypeptide
\item Sickle cell anaemia is caused by a mutation in the $\beta$ subunit
\end{itemize}

Haemoglobin:

\begin{itemize}
\item Transports O$_{2}$ from lungs to tissues

\item Heterozygotes fro the sickle cell anaemia mutation in the
  $\beta$ globin gene are partially protected against malaria.
\item Frequency of the sickle cell allele has reached highest levels in Africa and India

\item A related molecule, myoglobin, is
\end{itemize}

\subsection{Multiprotein Complexes}
\begin{itemize}
\item 1 polypeptide in 1 functional unit
\item Many proteins in one structure/molecular machine
\end{itemize}

Specific data is needed to differentiate between quaternary structures
and multiprotein complexes.

\subsection{Proteomics}

\begin{definition}
  \textbf{Proteomics } is a study of methods to separate, isolate and analyze proteins.

\end{definition}

One of the methods involves two-dimensional gel electrophoresis
followed by mass spectrometry.

Gel electrophoresis separates the proteins and allows to choose a
method for their isolation.

An unknown protein can be treated with trypsin, which breaks the
protein into peptide fragments and then analyze them by mass
spectrometry.

\section{Genes and Chromosomes}

Genome is the entirety of an organism's hereditary information.

Almost all genetic information is stored in DNA, but some viruses have
the genetic information stored in RNA.

\subsection{Human genome}

23 maternal chromosomes are maternal, 23 chromosomes are
paternal. There are 3 billion base pairs per genome, with 25 000 genes
spread across 23 chromosomes.

Genomes can come in all sizes. Bacteriophages, for example, have $48 000$ base pairs.

\textit{E.coli}, a standard prokaryote, has 4.6 billion base pairs,
while mitochondria have $16000$ base pairs. Over the evolutionary
periods, some of the mitochondrial DNA has moved to the nuclear DNA.

Eucalyptus globulus have chloroplast DNA with $160000$ base pairs.

\subsection{Comparing genome sizes}

There are about $21000$ genes encoding proteins, while having about
three billions of base pairs.

Taegleria fowleri, a brain-eating amoeba, has 670 billion base pairs.

About $50%$ of the genome is repetitive DNA, genes are at $20%$, only
$1.5%$ of the genome encodes proteins,

Non-repetitive DNA that is neither in introns nor codons are not
transcribed, but they can help to determine how much and when to
transcribe.

Thousands to millions of sequences are repeated. Transposons are
regions that can cut themselves out of the DNA, make a copy and then
get inserted back. There are also remnants of virus infections
(SINEs), as well as long interspersed nuclear elements (LINEs,
typically greater than 500 base pairs).

\subsection{Packaging  of DNA in the Cell}

In a non-packaged state, even the small prokaryotic genome would occupy a considerable portion of the cell volume, which is even a greater probelem in eukaryotes.

In prokaryotes, the DNA forms the prokaryotic nucleoid.

Topoisomerases are enzymes that wind and unwind DNA.

There are 6 billion base pairs per cell, 2 meters of DNA per cell,
located in the nucleus 6 $\mu$m wide, which is a geometric equivalent
of packing 40 km of fine thread into a tennis ball.
\subsection{Structure of the Nucleus}
All DNA is in the nucleus.

There are two envelopes of phospholib

Nuclear pore is used for movement in and out of the nucleus.

Chromatin structures hold DNA together in the nucleus.

Nucleoplasm is a fluid in the nucleus.

Nucleolus is the place where a ribosome begins to be built.

The eukaryotic genome is packed into cells using chromosomes.

In this way, a karyotype can be constructed which pairs chromosomes in
numerical order, which is an excellent diagnostic aid.

Chromosome painting hybridization can be perfomed in order to
distinguish a variety of chromosomes (get the probes, mark them with
fluorescent labels and heat the solution with the chromosomes up).

FISH (fluorescence in situ hybridization) can also be administered to
test for the presence of a particular DNA sequence (heat the probe,
heat the sample and cool it down).

Each chromosome contains a single, long, linear DNA molecule and
associated proteins (clled chromatin). Chromatin is tightly packaged
and dynamic.

\begin{ques*}

What is a chromatid?

\end{ques*}

There are 8 different proteins forming core histones. The DNA is
wrapped around $1.6-1.8$ times in a nucleosome core particle, with
about 146 base pairs in the loop. Linker DNA connects histones
(proteins rich in lysine and arginine) and thus constitutes a
nucleosome if taken together with a nucleosome core particle (consider
an analogy of beads on a string).

A linker histone clips the DNA to a histone.

Positive charge of histones neutralizes a negative charge of DNA.

DNA is not always completely loose or tightly packages and changes its
topology depending on the conditions.

Cohesin may be involved in forming chromatin loops, which can also be
attached to a nuclear scaffold to enable increased compactification.

To sum up, each DNA molecule is packaged into a mitotic chromosome
that is $10000$ shorter than its extended length. Note that packaging
and unpackaging requires ATP.

\end{document}