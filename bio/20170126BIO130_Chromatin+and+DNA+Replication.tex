% -*- coding: utf-8; -*-
%%% Local Variables:
%%% mode: latex
%%% TeX-engine: xetex
%%% TeX-master: t
%%% End:
\documentclass[11pt]{scrartcl}
\usepackage[fancy, beaue, pset, anon]{masty}
\pSet{\nt{BIO130}{III}{Chromatin and DNA Replication}}
\usepackage{lineno}
% ----------------------------------------------------------------------
% Page setup
% ----------------------------------------------------------------------

\pagenumbering{gobble}

% ----------------------------------------------------------------------
% Custom commands
% ----------------------------------------------------------------------

% alignment

\newcommand*{\LongestHence}{$\Rightarrow$}% function name
\newcommand*{\LongestName}{$f_o(-x)+f_e(-x)$}% function name
\newcommand*{\LongestValue}{$(-a)x +(-a)(-y)$}% function value
\newcommand*{\LongestText}{\defi}%

\newlength{\LargestHenceSize}%
\newlength{\LargestNameSize}%
\newlength{\LargestValueSize}%
\newlength{\LargestTextSize}%

\settowidth{\LargestHenceSize}{\LongestHence}%
\settowidth{\LargestNameSize}{\LongestName}%
\settowidth{\LargestValueSize}{\LongestValue}%
\settowidth{\LargestTextSize}{\LongestText}%

% Choose alignment of the various elements here: [r], [l] or [c]

\newcommand*{\mbh}[1]{{\makebox[\LargestHenceSize][r]{\ensuremath{#1}}}}%
\newcommand*{\mbn}[1]{{\makebox[\LargestNameSize][r]{\ensuremath{#1}}}}%
\newcommand*{\mbv}[1]{\ensuremath{\makebox[\LargestValueSize][r]{\ensuremath{#1}}}}%
\newcommand*{\mbt}[1]{\makebox[\LargestTextSize][l]{#1}}%

\newcommand{\R}[1]{\label{#1}\linelabel{#1}}
\newcommand{\lr}[1]{line~\lineref{#1}}

% ----------------------------------------------------------------------
% Launch!
% ----------------------------------------------------------------------

\begin{document}

\section{Chromatin}

\subsection{Heterochromatin}
\begin{itemize}
\item Highly condensed chromatin

\begin{itemize}
\item Meiotic and mitotic chromosomes (no tangles, no breaks)
\item Centromeres (central portion of a chromosome) and telomeres (the
  ends of linear chromosomes)
\item One $X$ chromosome in human females (Barr body, which is
  not transcribed)
\end{itemize}
\item Heterochromatic regions of interphase chromosomes are areas
  where gene expression is suppressed. Since the DNΑ is very
  condensed, the information cannot be accessed. Some regions in
  during the interphase are thus compact, some are not so much.
\end{itemize}

\subsection{Euchromatin}
\begin{itemize}
\item Relatively non-condensed chromatin
\item Euchromatic regions of intrephase chromosomes regions where
  genes tend to be expressed.
\item The reversible switching from euchromatic to heterochromatic
  regions is modulated by covalent modification of histones, the
  presence of chromatin remodelling complexes and RNA polymerase
  (transcription) complexes.
\end{itemize}

There are several implications of existence of condensed chromatin for gene expression:

\begin{itemize}
\item the interphase chromosomes are in discrete regions of the nucleus
\item The expressing gene within the chromatin can be re-oriented. If
  there is a need to express a gene, transcription can happen by
  decondensing particular regions of a chromosome. If the chromatin
  should not be expressed, everything stays condensed.

  For example, chromatin can be remodelled to form loops in order to
  access to DNA.
\end{itemize}

\subsection{Transcription Factories}

\textbf{Transcription factories} are regions of the nucleus that have
lots of RNA polymerase, lots of substrates, lots of materials for
transcription.
\begin{ques*}

  What are the advantages to the cell of being able to package DNA
  into a heterochromatics state?

\end{ques*}

\subsection{DNA Replication}

DNA replication is important.

\begin{ques*}

Is DNA repllication conservative or semiconservative?

\end{ques*}

Before the cell division, a cell is a mother cell (or parental
cell). Afterwards, they are daughter cells.

In the conservative model, when the parental cell divides, one of the
cells have the same strands, the other has all the new strands.

In the semiconservative model, one of the strands in each daughter
cell is old, the other one is new.

Only the semiconservative method has been observed.

\begin{ques*}

What is the direction of the DNA replication?

\end{ques*}
Note the following:
\begin{itemize}
\item DNA is antiparallel.
\item New DNA is synthesised from five' to three'
\item The template is read 3' to 5'.
\end{itemize}
This implies that three modes of behaviour are possible:
\begin{itemize}
\item Unidirectional growth of single strands from two starting
  points (for example, linear viruses use this technique)
\item Unidirectional growth of two strands from one starting
  point. The leading strand is in the same direction as the
  separation. The other strand is called a lagging strang. (for
  example some plasmids use predominantly this method).

  The fragments produced are called \textbf{Okazaki fragments}.
\item Bidirectional growth from one starting point.

  There are two replication forks, each leading a leading and a
  lagging strand. (for example, eubacteria and bacteria use this
  method.

\end{itemize}
  \begin{ques*}

    Where does DNA replication start?

  \end{ques*}

  There are two possibilities:


\begin{enumerate}
\item\label{item:1} Always starts from the same location on DNA.

  What
  are some of the characteristics of the sequences at replication
  origins?
\begin{itemize}
\item Easy to open, A-T rich
\item Recognized by and bound by indicator proteins
\end{itemize}
\item Random
\end{enumerate}

There is a single origin of replication in bacteria, while multiple in
eukaryotes.

\begin{ques*}

How does DNA replication proceed in bacteria?

\end{ques*}

The style of replication only apples to circular genomes

\begin{ques*}

  What happens at the DNA replication forks?

\end{ques*}

With time there are more DNA separation, and also Okazaki fragments
occuring which stitch to the leading strand.  The replication fork is
assymetrical, and the leading strand is replicated continously, while
the lagging strand is replicated discontinuosly.

\subsection{Overview of DNA replication}

\begin{enumerate}
\item\label{item:2} Separate the DNA strands
\item Synthesise the DNA
\item Proofread newly synthesized DNA
\end{enumerate}

Ingredients for synthesis: origin, primers, dNTPs, ATP (as an energy
source), DNA polymerase, accessory proteins

\subsection{DNA Synthesis}

This reaction is catalyzed by DNΑ polymerase.

This reaction relies on accurate base pairing to make the DNA.

First, take a primer. A primer is the first part of the growing polypeptide chain.

The appropriate base pairing will induce the dNTP to catalyze the
reaction which will cut off the pyrophosphate group that, in turn, is
gradually degraded.

Steps in the bacterial DNA replication are as follows:

\begin{enumerate}
\item\label{item:3} origin of replication

  Initiator proteins for replication in E.coli bind to origin and help
  the helicase to bind, for which ATP is required.

\item binding of initiator proteins

  First, primase binds
  to helicase, forms a primosome, to which the rest of the replication
  machinery binds.

\item unwinding by helicase, brought by a helicase-loading protein

  Helicases unwind and separate strands. The predominant helicase goes
  5'-3' along the lagging strand template.

  There are 5 subunits in each helicase.

\item binding of single-strand binding proteins

  Following the action of helicase, single strand binding proteins
  keep DNA strands separated.

  A helicase protein separates the strands by binding ssDNA and
  prevents strands from H-bonding.

\item RNA primers made by primase

  In order to beign, DNA polymerase requires a bound primer.

  The purpose of the primase in replication, which proceeds in the
  direction 5'-3', is to synthesise an RNA primer.

  Note that primase, DNA primase and RNA primase are all the same.

\item DNA polymerase

  Close to fingers and in the palm of the DNA polymerase is the
  synthesis centre.

  If there is a mismatch of pairs, 3'-5' exonuclease site in the wrist
  area will chop it off.

\item sliding clamp (also known as a $\beta$ clamp) holds polymerase
  onto DNA

  Polymerase does not hold to the strand, so sliding clamps keep the
  polymerase catalyzing the reactions.

\item nick sealing by DNA ligase


  A special DNA repair system is responsible for removal of the RNA
  primer and replacing it with a correctly matched DNA sequence.



\end{enumerate}








\end{document}