% -*- coding: utf-8; -*-
%%% Local Variables:
%%% mode: latex
%%% TeX-engine: xetex
%%% TeX-master: t
%%% End:
\documentclass[11pt]{scrartcl}
\usepackage[fancy, beaue, pset, anon]{masty}
\pSet{\nt{Losev}{1}{Introduction to Representation Theory}}
  \usepackage{lineno}
  % ----------------------------------------------------------------------
  % Page setup
  % ----------------------------------------------------------------------

  \pagenumbering{gobble}

  % ----------------------------------------------------------------------
  % Custom commands
  % ----------------------------------------------------------------------

  % alignment

  \newcommand*{\LongestHence}{$\Rightarrow$}% function name
  \newcommand*{\LongestName}{$f_o(-x)+f_e(-x)$}% function name
  \newcommand*{\LongestValue}{$(-a)x +(-a)(-y)$}% function value
  \newcommand*{\LongestText}{\defi}%

  \newlength{\LargestHenceSize}%
  \newlength{\LargestNameSize}%
  \newlength{\LargestValueSize}%
  \newlength{\LargestTextSize}%

  \settowidth{\LargestHenceSize}{\LongestHence}%
  \settowidth{\LargestNameSize}{\LongestName}%
  \settowidth{\LargestValueSize}{\LongestValue}%
  \settowidth{\LargestTextSize}{\LongestText}%

  % Choose alignment of the various elements here: [r], [l] or [c]

  \newcommand*{\mbh}[1]{{\makebox[\LargestHenceSize][r]{\ensuremath{#1}}}}%
  \newcommand*{\mbn}[1]{{\makebox[\LargestNameSize][r]{\ensuremath{#1}}}}%
  \newcommand*{\mbv}[1]{\ensuremath{\makebox[\LargestValueSize][r]{\ensuremath{#1}}}}%
  \newcommand*{\mbt}[1]{\makebox[\LargestTextSize][l]{#1}}%

  \newcommand{\R}[1]{\label{#1}\linelabel{#1}}
  \newcommand{\lr}[1]{line~\lineref{#1}}

  % ----------------------------------------------------------------------
  % Launch!
  % ----------------------------------------------------------------------

  \begin{document}

  \section{Introduction to Representation Theory}

  \subsection{Definitions}
  Let $G$ be a group.

  \begin{definition}
    A \textit{representation} of the group $G$ is fully defined by a
    homomorphism $\rho: G \to GL(V)$, where $GL(V)$ is the set of
    automorphisms.

    Let $V$ be a representation of $G$. If $g\in G$, $v\in V$, then we
    write $gv = \rho(g)v$ and $g_{V} = \rho(g)$.

    A \textit{subrepresentation} $U \su V$is a subspace $gu \in U$ for $u \in U$.

    If $V_{1}$ and $V_{2}$ are representations of $G$, then
    $V_{1}\oplus V_{2}$ is a representation
    $g(v_{1}, v_{2}) = (gv_{1}, gv_{2})$.

    A homomorphism of representations $V_{1 \to V_{2}}$ is a linear
    transformation $\phi: V_{1} \to V_{2}$ such that
    $\phi(gv_{1}) = g\phi(v_{1})$.

    Denote the space of linear transformations $V_{1} \to V_{2}$ as
    $\Hom(V_{1}, V_{2})$, and the space of homomorphisms
    as $\Hom_{G}(V_{1}, V_{2})$.
    
  \end{definition}
  \begin{example}

    If $G = S_{n}$, then $V = \CC^{n}$ is a representation of $G$.
    
    $V_{1} = \set{x, \cdots, x; x\in\CC} $ is a subrepresentation.

  \end{example}

  \subsection{Group Algebra}

  \begin{definition}
    Let $G$ be a finite group. A group ring
    $G \su \ZZ G = \set{\sum_{g\in G}x_{g}g; x_{g} \in \ZZ}$ is a set
    of additions and multiplications of the basis elements -- as in G.
  \end{definition}

  Suppose a group algebra $\CC G$ is given, so that $\ZZ G \su \CC G$.
  Then the representation of the algebra $\CC G$ is a homomorphism
  $\rho: \CC G \to \Hom(V, V)$.

  \subsection{Complete Representability (?)}

  Let $G$ be a finite group, and suppose that $V$ denotes a finite
  representation.

  A representation $V$ is irreducible, if any subrepresentation is
  $\set{0}, V$.

  A representation $V$ is almost reducible, if for all
  subrepresentations $U\su V$ there exists a subrepresentation
  $U'\su V$ such that $V = U\oplus U'$.

  Note that if $V$ is almost reducible, then $V$ is isomorphic to the
  direct sum of irreducible representations.

  \begin{theorem}[Maschke's Theorem]
    All representations mod $\CC$ are almost reducible.
  \end{theorem}

  \begin{proof}
    \hfill

    Suppose that the inner product is Hermitian, and let $U\ su V$
    be a subspace. Then $U \oplus U^{\perp} = V$.

    We say that $\ipr{\cdot}{\cdot}$ is $G$-invariant, if for all
    $g\in G$ and $u, v \in V$ we have $(gu, gv) = (u, v)$.

    If $\ipr{\cdot}{\cdot}$ is $G$-invariant and $U$ is a
    subrepresntation, then $U^{\perp}$ is a subrepresentation.

    It is enough to show that there exists a $G$-invariant Hermitian
    dot product.

    Let $\ipr{\cdot}{\cdot}$ be an arbitrary dot product.

    Then
    $\ipr{\cdot}{\cdot}_{inv}: (u, v)_{inv} = \sum_{h\in G}(hu, hv)$.

    We check the invariance: $(gu, gv)_{inv} = \sum_{h\in G}(hgu, hgv) = \sum_{h\in G}(hu, hv) = (u, v)_{inv}$.
    
  \end{proof}

  \begin{lemma}[Schur's Lemma]
    If $G$ is a finite group, and $U, V$ are irreducible
    representations. If $U$ is not isomorphic to $V$, then
    $\Hom_{G}(U, V) = 0$. Moreover,
    $\Hom_{G}(V,V) = \set{x \cdot \id V; x\in \CC}$.
  \end{lemma}

  \begin{proof}
    \hfill

    Let $\phi \in \Hom_{G}(U, V)$. Then $\ker \phi \su U$,
    $\img \phi \su V$ are subrepresentations.

    Note that $u\in\ker \phi$ if and only if $\phi(u) = 0$, which is
    equivalent to $\phi(gu) = 0$ and $gu \in \ker \phi$.

    Thus, if $\phi\neq 0$, then
    $\ker \phi \neq U = \set{0}, \img \phi = V $. Hence, $\phi$ is an
    isomorphism.

    Let $x$ be an eigenvalue of $\phi$, so that $\phi - x \id_{V}$ is
    irrevertible, and thus equal to $0$.
  \end{proof}

  Therefore, if $U, V$ are representations and $U$ is irreducible,
  then the multiplicity of $U$ in $V$ is $\dim \Hom_{G}(V, U)$.

  Note that the proof of Schur's lemma holds for all irreducible $V$,
  and thus
  $\Hom_{G}(V_{1}\oplus V_{2}, U) = \Hom_{G}(V_{1}, U) \oplus
  \Hom_{G}(V_{2}, U)$, which means that it holds for all $V$.

  \begin{example}

    Let $V = \CC G$. Then the multiplicity of $U$ in $\CC G$ is equal
    to $\dim \Hom_{G}(\CC G, U)$.
  \end{example}

  Let $F \in \CC G$ and $F = \sum_{g} x_{g} g$. Suppose
  $x_{g} = x_{hgh^{-1}}$ for all $h, g \in G$. Then for all
  irreducible $V$ $F_{V}$ is a constant operator.

  \begin{proof}
    \hfill

    For all $h\in G$, $Fh = hF$. Thus,
    $F = hFh^{-1} = \sum_{g\in G}x_{g}(hgh^{-1}) = \sum_{g}
    x_{hgh^{-1}}(hgh^{-1})$.

    Note that $F_{v}$ is a homomorphism of representations. Thus,
    $Fh = hF \ra V\to V$. $F_{V}h_{v} = h_{V}F_{V}$, which happens if
    and only if $F_{V} \in \Hom_{G}(V, V)$.
  \end{proof}

  \subsection{Characters}

  Suppose that $G$ is a finite group and $V$ is a finite-dimensional
  representation. A \textit{character} $\ch_{V}: G \to \CC$ is such that
  $\ch_{V}(g) = \tr(g_{V})$.

  \begin{example}

    Let $\CC$ be a trivial representation.

    Then $\ch_{\CC}(g) =1$,
    $\ch_{\CC G}(g) =
    \begin{cases}
      \abs{G}, g = e\\
      0, \text{ otherwise }
    \end{cases}$, where the last case is justified by the fact that
    there are no diagonal matrix elements of $h\mapsto gh$.

  \end{example}
  
  Note that $\ch_{V_{1}\oplus V_{2}} = \ch_{V_{1}} + \ch_{V_{2}}$.

  \begin{lemma}
    $\ch_{V}(hgh^{-1}) = \ch_{V} (g)$.
  \end{lemma}

  \subsection{Orthogonality of Characters}

  Let $Cl(G) = \set{f: G \to \CC | f(hgh^{-1}) = f(g)}$. Note that
  $\ch_{V} \in Cl(G)$.

  Recall that the Hermitian dot product
  $(F_{1}, F_{2}) = \frac{1}{\abs{G}}\sum_{g\in
    G}\ol{F_{1}(g)}F_{2}(g)$.

  \begin{theorem}
    Characters of irreducible representations is an orthonormal basis of $Cl(G)$.    
  \end{theorem}

  For representations $U, V$, we have $(ch_{U}, ch_{V}) = \dim \Hom_{G}(U, V)$.

  If $F \in Cl(G)$, then $F = \sum_{g\in G} F(g) g \in \CC G$.
  Moreover, if $V$ is irreducible, then we know that $F_{V} = x\cdot \id_{V}$.

  Thus, $x = \frac{\abs{G}}{\dim V}(\ol{F}, \ch_{V})$.

  To prove this, note that 
  \begin{align}
    x &= \frac{\tr(F_{V})}{\dim V} \\
      &= \frac{1}{\dim V} \sum_{g\in G} F(g)\tr(g_{v})\\
      &= \frac{1}{\dim V} \sum F(g) \ch_{V}(g)\\
      &= \frac{\abs{G}}{\dim V}(\ol{F}, \ch_{V}).
  \end{align}

  Proving the theorem, we can use Schur's lemma and previous comments
  to show that a basis is orthonormal.

\end{document}
