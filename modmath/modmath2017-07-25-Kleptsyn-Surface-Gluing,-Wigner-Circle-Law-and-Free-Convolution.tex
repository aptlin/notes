% -*- coding: utf-8; -*-
%%% Local Variables:
%%% mode: latex
%%% TeX-engine: xetex
%%% TeX-master: t
%%% End:
\documentclass[11pt]{scrartcl}
\usepackage[fancy, beaue, pset, anon]{masty}
\pSet{\nt{Kleptsyn}{2}{Surface Gluing, Wigner's Circle Law and Free Convolutions }}
  \usepackage{lineno}
  % ----------------------------------------------------------------------
  % Page setup
  % ----------------------------------------------------------------------

  \pagenumbering{gobble}

  % ----------------------------------------------------------------------
  % Custom commands
  % ----------------------------------------------------------------------

  % alignment

  \newcommand*{\LongestHence}{$\Rightarrow$}% function name
  \newcommand*{\LongestName}{$f_o(-x)+f_e(-x)$}% function name
  \newcommand*{\LongestValue}{$(-a)x +(-a)(-y)$}% function value
  \newcommand*{\LongestText}{\defi}%

  \newlength{\LargestHenceSize}%
  \newlength{\LargestNameSize}%
  \newlength{\LargestValueSize}%
  \newlength{\LargestTextSize}%

  \settowidth{\LargestHenceSize}{\LongestHence}%
  \settowidth{\LargestNameSize}{\LongestName}%
  \settowidth{\LargestValueSize}{\LongestValue}%
  \settowidth{\LargestTextSize}{\LongestText}%

  % Choose alignment of the various elements here: [r], [l] or [c]

  \newcommand*{\mbh}[1]{{\makebox[\LargestHenceSize][r]{\ensuremath{#1}}}}%
  \newcommand*{\mbn}[1]{{\makebox[\LargestNameSize][r]{\ensuremath{#1}}}}%
  \newcommand*{\mbv}[1]{\ensuremath{\makebox[\LargestValueSize][r]{\ensuremath{#1}}}}%
  \newcommand*{\mbt}[1]{\makebox[\LargestTextSize][l]{#1}}%

  \newcommand{\R}[1]{\label{#1}\linelabel{#1}}
  \newcommand{\lr}[1]{line~\lineref{#1}}

  % ----------------------------------------------------------------------
  % Launch!
  % ----------------------------------------------------------------------

  \begin{document}

  \section{Surface Gluing, Wigner's Circle Law and Free Convolutions }
  
  \subsection{Revision}

  \begin{definition}
    $k$th moment of a random variable $X$ is $m_{k} = \EE(X^{k})$.
  \end{definition}

  \begin{definition}
    The \textit{Laplace transform of a function} $\rho$ is a function
    $\Phi(t) = \int_{\RR}e^{tx}\rho(x)\dif x = \EE(e^{tX})$.
  \end{definition}

  \begin{note*}
    $\Phi(0) = 1$.
  \end{note*}

  What happens if we differentiate $\Phi$?
  \begin{align}
    \frac{\dif }{\dif t}\Phi &= \frac{\dif }{\dif t}\int_{-\infty}^{\infty}e^{tx}\rho(x) \dif x\\
                             &= \int_{-\infty}^{\infty} \frac{\dif }{\dif t}e^{tx}\rho(x)\dif x\\
                             &= \int_{-\infty}^{\infty}x e^{tx}\rho(x) \dif x\\
                             &= \EE(Xe^{tX}).
  \end{align}  


  Therefore, $(\frac{\dif }{\dif t})^{n}\Phi(t) = \EE(X^{n}e^{tX})$,
  which means that
  $(\frac{\dif }{\dif t})^{k}|_{t=0}\Phi(t) = \EE(X^{k}) = m_{k}$.

  Now,
  \begin{align}
    \int_{-\infty}^{\infty}e^{tx}\rho(x) \dif x &= \frac{1}{\sqrt{2\pi }}\int_{-\infty}^{\infty}e^{tx}e^{- \frac{x^{2}}{2}}\dif x\\
                                                &= \frac{1}{\sqrt{2\pi}}\int_{-\infty}^{\infty} e^{-\frac{1}{2}(x-t)^{2}}\dif (x-t) \* e^{\frac{t^{2}}{2}}\\
                                                &= e^{\frac{t^{2}}{2}}.
  \end{align}

  Therefore,
  $\Phi(t) = e^{\frac{t^{2}}{2}} = 1 + \frac{t^{2}}{2} +
  \frac{(\frac{t^{2}}{2})^{2}}{2!} + \dots$, and hence $m_{2} = 1$.
  Similarly, by looking at the Taylor series for $\Phi(t)$, we obtain
  that $m_{4} = 3$. In general,
  $m_{2k} = (\frac{\dif }{\dif
    t})^{2k}[\frac{(\frac{t^{2}}{2})^{k}}{k!}]$.

  It is worthwhile to note that
  $(\frac{\dif }{\dif t})^{2k}|_{t = 0}(e^{\frac{t^{2}}{2}})$ is the
  number of unique ways to pair up $2k$ numbers into $k$ pairs, which
  can be deduced to be $ \frac{(2k)!}{k!2^{k}}$, and thus
  $m_{2k} = \frac{(2k)!}{k!2^{k}}$.

  \begin{theorem}
    Let $X_{1}, \dots, X_{2k}$ be random variables forming a Gaussian
    vector with the mean 0. Then 
    \begin{equation*}
      \EE(X_{1}\dots X_{2k}) = \sum_{\text{the number of ways to pair up $2k$ elements into $k$ pairs $(j_{i}, j'_{i})$}} \prod_{i=1}^{u}\EE(X_{j_{i}}X_{j'_{i}}).
    \end{equation*}

  \end{theorem}

  \begin{theorem}[Wigner Semicircle Law]

    Eigenvalues of random matrices fall into a compact semicircle
    region, provided a matrix is big enough.

  \end{theorem}

  Now, let $f:S_{1}^{N} \to \RR$, such that $Lip(f) \leq C$, if
  $\abs{f(x)- f(y)} < C\abs{x-y}$ for all $x$ and $y$ in $S_{1}^{N}$.
  \begin{theorem}[Concentration of Measure]
    If $f$ is dependent on many independent variables in a Lipschitz
    way, then $f$ is essentially constant.
  \end{theorem}

  
\end{document}
