% -*- coding: utf-8; -*-
%%% Local Variables:
%%% mode: latex
%%% TeX-engine: xetex
%%% TeX-master: t
%%% End:
\documentclass[11pt]{scrartcl}
\usepackage[fancy, beaue, pset, anon]{masty}
\pSet{\nt{Volk}{3}{Brain Maths}}
  \usepackage{lineno}
  % ----------------------------------------------------------------------
  % Page setup
  % ----------------------------------------------------------------------

  \pagenumbering{gobble}

  % ----------------------------------------------------------------------
  % Custom commands
  % ----------------------------------------------------------------------

  % alignment

  \newcommand*{\LongestHence}{$\Rightarrow$}% function name
  \newcommand*{\LongestName}{$f_o(-x)+f_e(-x)$}% function name
  \newcommand*{\LongestValue}{$(-a)x +(-a)(-y)$}% function value
  \newcommand*{\LongestText}{\defi}%

  \newlength{\LargestHenceSize}%
  \newlength{\LargestNameSize}%
  \newlength{\LargestValueSize}%
  \newlength{\LargestTextSize}%

  \settowidth{\LargestHenceSize}{\LongestHence}%
  \settowidth{\LargestNameSize}{\LongestName}%
  \settowidth{\LargestValueSize}{\LongestValue}%
  \settowidth{\LargestTextSize}{\LongestText}%

  % Choose alignment of the various elements here: [r], [l] or [c]

  \newcommand*{\mbh}[1]{{\makebox[\LargestHenceSize][r]{\ensuremath{#1}}}}%
  \newcommand*{\mbn}[1]{{\makebox[\LargestNameSize][r]{\ensuremath{#1}}}}%
  \newcommand*{\mbv}[1]{\ensuremath{\makebox[\LargestValueSize][r]{\ensuremath{#1}}}}%
  \newcommand*{\mbt}[1]{\makebox[\LargestTextSize][l]{#1}}%

  \newcommand{\R}[1]{\label{#1}\linelabel{#1}}
  \newcommand{\lr}[1]{line~\lineref{#1}}

  % ----------------------------------------------------------------------
  % Launch!
  % ----------------------------------------------------------------------

  \begin{document}

  \section{Brain Maths III}

  \subsection{Revision}

  Some neurons persist in a sleeping state until they are acted upon
  by electric signals or chemicals. This can be represented by
  equilibrium points of vector fields in $\RR^{n}$, which can also be
  perturbed until they fall back into the stable state.

  Some neurons can also turn into an excited state, when they give off
  spikes periodically or quasiperiodically. This behaviour, in turn,
  can be modelled by closed trajectories, called \textit{attracting
    cycles} in the parameter space, which are resistant to
  perturbations.

  We will study how stable states can turn into attracting cycles.

  \subsection{}

  Informally, a vector field $\dot{x} = f(x)$, where $f(x)$ is a
  multidimensional function, is called structurally stable if there
  exists $\epsilon > 0$ such that for all $g$ with
  $d_{C'}(f, g) < \epsilon$ the behaviour of $f$ and $g$ is similar.

  \subsection{Smooth Classification}

  We say that $f\sim g$ if there exists a diffeomorphism
  $h: \RR^{n} \to \RR^{n}$ such that $gh(x) = (Dh)_{x}(f(x))$.

  \begin{example}

    Suppose that $\dim = 1$, and let $f(0) = 0$ and $h(0) = 0$, which
    means that $g(0) = 0$. If $h$ is a diffeomorphism, we thus have
    $Df(0) = Dg(0)$. Now, if $f = h(x)$, then
    $g(y) = \frac{\partial h(x)}{\partial x} \* f(x) = \frac{\partial
    }{\partial}h(h^{-1}y)\*f(h^{-1}(y))$. Thus,
    $\frac{\partial g(0)}{\partial y} = \frac{\partial h(0)}{\partial
      x} \* \frac{\partial f(0)}{\partial x}\* \frac{\partial
      h^{-1}(0)}{\partial y}$, and hence
    $\frac{\partial g(0)}{\partial y} = \frac{\partial f(0)}{\partial
      x}$.

  \end{example}

  \subsection{Topological Classification}

  Let $h$ be a homeomorphism. For all $x$ and time $t$, we want the
  flux to be in the form $\Phi_{t}^{g}(h(x)) = h(\Phi^{f}_{t}(x))$.
  Note that $\Phi_{t}$ is smooth.

  Now, there exists $x\in \RR^{n}$ and $T > 0$ such that
  $\Phi_{T}^{f}(x) = x$. From the condition above, we also have
  $\Phi_{T}^{g}(h(x)) = h(x)$.

  \subsection{Orbital Topological Equivalence}

  If there exists a homeomorphism transforming oriented trajectories
  of $f$ into the oriented trajectories of $g$, then we can classify
  our vector fields up to orbital topological equivalence.

  \begin{example}

    Let $\dim = 1$. For $\epsilon > 0$, if $\norm{f- g} < \epsilon$
    and $\norm{f' - g '} < \epsilon$, then $d_{C'}(f, g) < \epsilon$.

  \end{example}

  \begin{theorem}
    Suppose $\dot{x} = f(x)$ is given. It is structurally stable, if
    \begin{itemize}
    \item all equilibrium points are isolated
    \item for all equilibrium points we have $f'(p) \neq 0$.
    \end{itemize}
  \end{theorem}

  \begin{note*}
    Note that structurally stable vector fields form an open set in
    the space of all real vector spaces.
  \end{note*}

  \subsection{Bifurcations}

  Let $f_{\alpha}$ be a family of vector fields dependent on the
  parameter $\alpha$, where $\alpha \in (- \alpha_{0}, \alpha_{0})$.
  We say that $\alpha$ is \textit{regular}, if $f_{\alpha}$ is structurally
  stable. Moreover, we say that $\alpha$ is a \textit{bifurcation point}, if
  $f_{\alpha}$ is not structurally stable.

  \begin{theorem}
    For $\dim = 1$, a typical family of vector fields $f_{\alpha}$ has
    only a saddle-knot bifurcation point.
  \end{theorem}

  The two-dimensional case is more complex.

  % If our vector space $f(p) \neq 0$, then it can be shown that it is
  % smoothly equivalent to a 
  
\end{document}
