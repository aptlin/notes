% -*- coding: utf-8; -*-
%%% Local Variables:
%%% mode: latex
%%% TeX-engine: xetex
%%% TeX-master: t
%%% End:
\documentclass[11pt]{scrartcl}
\usepackage[fancy, beaue, pset, anon]{masty}
\pSet{\nt{Ilyashenko}{1}{Structural Stability}}
  \usepackage{lineno}
  % ----------------------------------------------------------------------
  % Page setup
  % ----------------------------------------------------------------------

  \pagenumbering{gobble}

  % ----------------------------------------------------------------------
  % Custom commands
  % ----------------------------------------------------------------------

  % alignment

  \newcommand*{\LongestHence}{$\Rightarrow$}% function name
  \newcommand*{\LongestName}{$f_o(-x)+f_e(-x)$}% function name
  \newcommand*{\LongestValue}{$(-a)x +(-a)(-y)$}% function value
  \newcommand*{\LongestText}{\defi}%

  \newlength{\LargestHenceSize}%
  \newlength{\LargestNameSize}%
  \newlength{\LargestValueSize}%
  \newlength{\LargestTextSize}%

  \settowidth{\LargestHenceSize}{\LongestHence}%
  \settowidth{\LargestNameSize}{\LongestName}%
  \settowidth{\LargestValueSize}{\LongestValue}%
  \settowidth{\LargestTextSize}{\LongestText}%

  % Choose alignment of the various elements here: [r], [l] or [c]

  \newcommand*{\mbh}[1]{{\makebox[\LargestHenceSize][r]{\ensuremath{#1}}}}%
  \newcommand*{\mbn}[1]{{\makebox[\LargestNameSize][r]{\ensuremath{#1}}}}%
  \newcommand*{\mbv}[1]{\ensuremath{\makebox[\LargestValueSize][r]{\ensuremath{#1}}}}%
  \newcommand*{\mbt}[1]{\makebox[\LargestTextSize][l]{#1}}%

  \newcommand{\R}[1]{\label{#1}\linelabel{#1}}
  \newcommand{\lr}[1]{line~\lineref{#1}}

  % ----------------------------------------------------------------------
  % Launch!
  % ----------------------------------------------------------------------

  \begin{document}

  \section{Structural Stability}

  \subsection{Introduction}

  Suppose that a differential equation $\dot{x} = v(x)$ is given,
  where $x\in \RR^{n}$. A solution of this equation is a vector
  function $\phi: I \to \RR^{n}$ such that
  $\dot{\phi(t)} = v(\phi(t))$. This is an analytic description.

  Now, consider $\Omega \su \RR^{n}$, a vector field on a real
  manifold. Imagine putting a toy boat on this manifold. The boat
  would move according to the analytic solution we have discussed
  above. The path according to which the boat moves is called an
  \textit{orbit}, which is a phase curve $\img \phi$.

  \subsection{Stationary Points and Limit Cycles}

  Orbits of any differential equation can be classified into three
  types: topologically equivalent to a point, circle or a line.

  \begin{definition}
    If $v(a) = 0$, we say that $a$ is a stationary point.
  \end{definition}

  In turn, closed orbits may contain other closed orbits.

  \begin{definition}
    An isolated closed orbit is called a \textit{limit cycle}.
  \end{definition}

  \begin{example}

    Consider a plane with polar coordinates, and suppose that
    $\dot{\phi} = 1$ and $\dot{r} = r(1-r)$. In this way, all the
    boats move from $0$ to $1$ and from the point at infinity to one.

    There is also a limit cycle corresponding to a circle with the
    radius $1$.

  \end{example}

  A vector field is called \textit{typical}, if a small perturbation
  of vector field does not destroy the key properties of a vector
  field. Now, let's consider stationary points and limit cycles of a
  \textit{typical} vector field.

  Stationary points in $\RR^{2}$ come in a wide variety. For example,
  a saddle point is, informally, a repelling stationary point, unless
  the boat moves straight to its center. There are also knots, for
  which a phase diagram looks like a plot for a point charge, and
  foci, which is spiral.

  \begin{theorem}
    Open, everywhere dense set in $Vect^{1}(S^{2})$ consists of vector
    fields with a finite number of foci.
  \end{theorem}

  What is the asymptotic behaviour of other trajectories?

  Trajectories of a vector field with a finite number of stationary
  points on a two-dimensional sphere asymptotically approach either a
  stationary point, a cycle or a polycycle. A polycycle is a
  polyhedron made of separatrices, which can make up a complex
  picture.

  \begin{theorem}
    Trajectories of a vector field from an open, everywhere dense set
    in $Vect^{1}(S^{2})$ can revolve around either stationary points
    or limit cycles, if there is only a finite number of special
    points or cycles.
  \end{theorem}

  Smale have significantly advanced the study of differential
  equations on manifolds, proving that the previous theorem holds for
  all vector fields in $Vect^{k}(M^{n})$, where $M$ is compact and $n$
  arbitrary.

  \subsection{Smale Horseshoe}

  Suppose a unit square is given, which we divide into 5 equal parts,
  both in the horizontal and vertical direction. Colour the second and
  fourth rectangles thus obtained. The Smale horseshoe map $f$ then
  transforms the horizontal rectangles into the vertical, and the
  vertical rectangles into the horizontal. At first sight, this map
  does not immediately seem to be useful, but Smale's idea have
  revolutionised the study of dynamic systems.

  \subsection{Symbolic Dynamics}

  Let $\Lambda = \set{p; f^{n}p \text{ is defined for all $n \in \ZZ$}}$.

  \begin{definition}
    A destiny for a point $p\in \Lambda$ is
    $\omega = \dots \omega_{-n}\dots \omega_{0}\dots \omega_{n}\dots$
    such that $\omega_{n} = j $ if and only if $f^{n}(p)\in D_{j}$.
  \end{definition}

  \begin{theorem}
    \label{t3}
    Each sequence of 0 and 1 is realisable as a unique destiny of some
    point.
  \end{theorem}

  Let $p\in \Lambda$ and $\omega = \omega(p)$. What is $\omega(f(p))$?
  If $\sigma_{w}$ is a shift of $\omega$ to the left once, then the
  answer is $\sigma_{w} \omega$. As a result, $f$ has a countable number
  of periodic orbits.

  \begin{note*}
    Suppose that a point $p$ is periodic. Then her destiny is a
    periodic sequence.

    Now, let $\omega$ be a periodic sequence. Then by Theorem \ref{t3}
    we know that there exists $p\in \Lambda$ such that
    $\omega = \omega(p)$, which means that $p$ is also periodic.

    Now, assume that the period of $\omega$ is $n$. Then
    $\sigma^{n}\omega = \omega$. But we know that
    $\omega(f^{n}p) = \sigma^{n}\omega = \omega = \omega^{p}$, and thus
    $p = f^{n}(p)$.
  \end{note*}
  
\end{document}
