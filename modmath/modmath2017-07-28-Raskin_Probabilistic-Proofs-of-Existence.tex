% -*- coding: utf-8; -*-
%%% Local Variables:
%%% mode: latex
%%% TeX-engine: xetex
%%% TeX-master: t
%%% End:
\documentclass[11pt]{scrartcl}
\usepackage[fancy, beaue, pset, anon]{masty}
\pSet{\nt{Raskin}{1}{Probabilistic Proofs of Existence}}
  \usepackage{lineno}
  % ----------------------------------------------------------------------
  % Page setup
  % ----------------------------------------------------------------------

  \pagenumbering{gobble}

  % ----------------------------------------------------------------------
  % Custom commands
  % ----------------------------------------------------------------------

  % alignment

  \newcommand*{\LongestHence}{$\Rightarrow$}% function name
  \newcommand*{\LongestName}{$f_o(-x)+f_e(-x)$}% function name
  \newcommand*{\LongestValue}{$(-a)x +(-a)(-y)$}% function value
  \newcommand*{\LongestText}{\defi}%

  \newlength{\LargestHenceSize}%
  \newlength{\LargestNameSize}%
  \newlength{\LargestValueSize}%
  \newlength{\LargestTextSize}%

  \settowidth{\LargestHenceSize}{\LongestHence}%
  \settowidth{\LargestNameSize}{\LongestName}%
  \settowidth{\LargestValueSize}{\LongestValue}%
  \settowidth{\LargestTextSize}{\LongestText}%

  % Choose alignment of the various elements here: [r], [l] or [c]

  \newcommand*{\mbh}[1]{{\makebox[\LargestHenceSize][r]{\ensuremath{#1}}}}%
  \newcommand*{\mbn}[1]{{\makebox[\LargestNameSize][r]{\ensuremath{#1}}}}%
  \newcommand*{\mbv}[1]{\ensuremath{\makebox[\LargestValueSize][r]{\ensuremath{#1}}}}%
  \newcommand*{\mbt}[1]{\makebox[\LargestTextSize][l]{#1}}%

  \newcommand{\R}[1]{\label{#1}\linelabel{#1}}
  \newcommand{\lr}[1]{line~\lineref{#1}}

  % ----------------------------------------------------------------------
  % Launch!
  % ----------------------------------------------------------------------

  \begin{document}

  \section{Probabilistic Proofs of Existence}

  \subsection{Introduction}

  Constructions can have different nature.

  Sometimes explicit construction of mathematical objects is very
  challenging. In these cases, we can construct an approximating
  object, which may not have all the \textit{good} properties we want, but
  which at least lacks all the \textit{bad} ones we want to avoid.

  \subsection{Introduction to the Ramsey Theory}

  Suppose that a complete graph is given with $n$ edges. Let's choose
  a subgraph with $k$ nodes and colour its nodes red or blue. If we
  choose a subgraph randomly, then the probability that it is coloured
  is $2^{\frac{k(k-1)}{2}}2$. If we look at the total probability,
  without any regard for which subgraph we choose, we get
  $\cv{n;k}2^{1- \frac{k(k-1)}{2}}$. The upper bound is thus
  $n^{k}2^{1- \frac{k(k-1)}{2}}$. Can we do better?

  \subsection{Markov's Inequality}

  Suppose that a random variable $X$ is given. We define the expected
  value of $X$ as $\EE(X) = \sum x_{i}\Pr(x_{i})$.

  \begin{theorem}[Markov's Inequality]
    Suppose $X \geq 0$. Then $\Pr(X \geq a) \leq \frac{\EE(X)}{a}$.
  \end{theorem}

    Why does it hold? It is easier to see if we substitute $X$ for $Y$,
  where $Y \leq X$, and $Y = \begin{cases}
      0, X < a\\
      a, X \geq a
    \end{cases}$.

    We define variance as $\Var(X) = \EE(X-\EE(X))^{2} = \sigma^{2}$.
      
  \begin{theorem}[Chebyshev's Inequality]
    
    \begin{equation*}
      \Pr\left(\frac{\abs{X-\EE(X)}}{\sigma} \geq a\right) \leq \frac{1}{a^{2}},
    \end{equation*}

  \end{theorem}

  \begin{theorem}[Chernov's Inequality]
    Suppose that $X_{1}, \dots, X_{n}$ are independent random
    variables such that $X_{1}, \dots, X_{n} \in \set{0, 1}$. Let
    $X = \sum_{j} X_{j}$, and take $\delta > 0$. Then
    \begin{equation*}
      \Pr(X > (1+\delta) \EE(X)) \leq e^{\frac{-\delta \log(1+\delta)\EE(X)}{2}}
    \end{equation*}

  \end{theorem}

  \subsection{Error-Correcting Codes}

  Suppose that we code an $n$-bit word with $m$ bits which is
  transmitted via a channel introducing $d$ errors.

  Note that in this case
  $\sum_{j=0}^{d}\cv{m;j} \geq \cv{m;d} \geq \frac{(m-d)^{d}}{d!}$,
  and thus we can write $2^{n}\frac{(m-d)^{d}}{d!} \leq 2^{m}$.

\end{document}
