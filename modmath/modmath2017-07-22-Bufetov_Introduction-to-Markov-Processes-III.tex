% -*- coding: utf-8; -*-
%%% Local Variables:
%%% mode: latex
%%% TeX-engine: xetex
%%% TeX-master: t
%%% End:
\documentclass[11pt]{scrartcl}
\usepackage[fancy, beaue, pset, anon]{masty}
\pSet{\nt{Bufetov}{3}{Introduction to Markov Processes III}}
  \usepackage{lineno}
  % ----------------------------------------------------------------------
  % Page setup
  % ----------------------------------------------------------------------

  \pagenumbering{gobble}

  % ----------------------------------------------------------------------
  % Custom commands
  % ----------------------------------------------------------------------

  % alignment

  \newcommand*{\LongestHence}{$\Rightarrow$}% function name
  \newcommand*{\LongestName}{$f_o(-x)+f_e(-x)$}% function name
  \newcommand*{\LongestValue}{$(-a)x +(-a)(-y)$}% function value
  \newcommand*{\LongestText}{\defi}%

  \newlength{\LargestHenceSize}%
  \newlength{\LargestNameSize}%
  \newlength{\LargestValueSize}%
  \newlength{\LargestTextSize}%

  \settowidth{\LargestHenceSize}{\LongestHence}%
  \settowidth{\LargestNameSize}{\LongestName}%
  \settowidth{\LargestValueSize}{\LongestValue}%
  \settowidth{\LargestTextSize}{\LongestText}%

  % Choose alignment of the various elements here: [r], [l] or [c]

  \newcommand*{\mbh}[1]{{\makebox[\LargestHenceSize][r]{\ensuremath{#1}}}}%
  \newcommand*{\mbn}[1]{{\makebox[\LargestNameSize][r]{\ensuremath{#1}}}}%
  \newcommand*{\mbv}[1]{\ensuremath{\makebox[\LargestValueSize][r]{\ensuremath{#1}}}}%
  \newcommand*{\mbt}[1]{\makebox[\LargestTextSize][l]{#1}}%

  \newcommand{\R}[1]{\label{#1}\linelabel{#1}}
  \newcommand{\lr}[1]{line~\lineref{#1}}

  % ----------------------------------------------------------------------
  % Launch!
  % ----------------------------------------------------------------------

  \begin{document}

  \section{Introduction to Markov Processes III}

  We have already seen that
  $\frac{\dif p_{ij}}{\dif t} = \sum_{k\in S} q_{ik} p_{kj}(t)$.

  Moreover,
  $p_{ij}(t) = (\exp(tQ))_{ij} =
  \sum_{k=0}^{\infty}\frac{t^{k}Q^{k}}{k!}$.

  Thus, $\frac{\dif }{\dif t} P = QP$.

  We can rewrite the first equation above as follows:
  \begin{equation*}
    \frac{\dif  p_{ij}}{\dif t} = -c(i)p_{ij}(t) + \sum_{k\neq i}q_{ik}p_{kj(tQ)}.
  \end{equation*}

  Let's solve the equation in the form
  $\frac{\dif }{\dif t}p = - cp(t)+g $. We obtain $p = Re^{-ct}$,
  where $\frac{\dif R}{\dif S} = ge^{cS}$. Thus,
  $R = \int_{0}^{t}ge^{cs}\dif s$, where
  $p = \int_{0}^{t}ge^{c(s-t)}\dif s$. Hence,
  
  \begin{equation*}
    p_{ij}(t) = \delta_{ij}e^{-ct} + \int_{0}^{t}e^{c(s- t)}(\sum_{k\neq i}q_{ik}p_{kj}(s))\dif s.
  \end{equation*}

  Thus, we can write
  
  \begin{equation*}
    p_{ij}(t) = \delta_{ij}e^{- c(i)t} + \int_{0}^{t}e^{c(i)(s-t)}(\sum_{k\neq i} q_{ik}p_{kj}(s))\dif s
  \end{equation*}

  Let's construct explicitly a solution of this equation with a
  specific probabilistic meaning. Then we will check that the solution
  satisfies the Kolmogorov-Chapman equation, and discuss the
  uniqueness of solutions.

  The method of finding a solution is that of sequential approximations:

  \begin{align}
    p_{ij}^{(0)}(t) &= \delta_{ij}e^{-c(i)t}\\
    p_{ij}^{(n+1)}(t) &= \delta_{ij}e^{-c(i)t} + \int_{0}^{t}e^{c(i)(s-t)}\sum_{k\neq i}q_{ik}p_{kj}^{(n)}(s)\dif s
  \end{align}

  Note that $p_{ij}^{(n+1)}(t) \geq p_{ij}^{(n)}(t)$, which can be shown by induction.

  Thus,
  $p_{ij}^{(n+1)}(t) - p_{ij}^{(n)}(t) = \ol{p_{ij}}(t) = \lim_{n\to
    \infty}p_{ij}^{(n)}(t)$, which means that
  \begin{equation*}
    p_{ij}^{(n+1)}(t) - p_{ij}^{(n)}(t) = \int_{0}^{t}e^{c(i)(s-t)}\sum_{k\neq i}q_{ik}(p_{kj}^{(n)}(s) - p_{kj}^{(n-1)}(s)) \dif s.
  \end{equation*}

  For all $i$, we have $\sum_{j}p_{ij}^{(n)}(t) \leq 1$ and
  $\sum_{j}\ol{p_{kj}}(t) \leq 1$.

  Let $\wh{p_{ij}}$ be a solution. We know that
  $\wh{p_{ij}} \geq \ol{p_{ij}}$, given that
  $\wh{p_{ij}} \geq p_{ij}^{(n)}$.

  The constructed solution is therefore minimal. Therefore, in the
  non-explosive case, when $\sum_{j}\ol{p_{ij}} = 1$, the solution is
  unique.

  Remember that $p_{ij}^{(0)} = \delta_{ij}e^{-c(i)t}$, and thus
  $p_{ij}^{(1)}(t) = \delta_{ij}e^{-c(i)t} + \int_{0}^{t}
  e^{c(i)(s-t)}q_{ij}e^{-c(j)s}\dif s$.

  Hence, $p_{ij}^{(2)}(t) = \delta_{ij}e^{-c(i)t} + \int_{0}^{t} e^{c(i)(s-t)-c(j)s}q_{ij}$.

  In general, 
  \begin{equation*}
    p_{ij}^{n}(t) = \delta_{ij}e^{-c(i)t} + \sum_{i\neq k_{1}, k_{1}\neq k_{2}, \dots, k_{r-2} \neq k_{r-1} }q_{ik_{1}}q_{ik_{2}}\dots q_{r-1}\times I,
  \end{equation*}

  where
  $I = \int \dots \int \exp(- c(i)s_{1} - c(k_{1}) s_{2} -
  \set{s_{1}+\dots+s_{r} <t} - \dots - c(k_{r-1})s_{r} - c(j)(t- s_{1}
  - \dots - s_{r})) ds_{1}\dots ds_{r}$.

  Let $\Lambda = \delta_{ij}c(i)$, $\Pi = \set{p_{ij}}, \pi_{ij} = \frac{q_{ij}}{c(i)}$.

  Then
  $P(t) = e^{-t\Lambda} + \sum_{r = 1+ s_{1}+\dots +s_{r} <
    t}^{\infty}\int e^{-s_{1}\Lambda}\Lambda\Pi \dots e^{-(t - s_{1} -
    \dots - s_{r})\Lambda}ds_{1}\dots ds_{r}$.

  

  
\end{document}
