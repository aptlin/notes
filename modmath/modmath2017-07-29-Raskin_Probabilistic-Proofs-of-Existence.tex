% -*- coding: utf-8; -*-
%%% Local Variables:
%%% mode: latex
%%% TeX-engine: xetex
%%% TeX-master: t
%%% End:
\documentclass[11pt]{scrartcl}
\usepackage[fancy, beaue, pset, anon]{masty}
\pSet{\nt{Raskin}{2}{Probabilistic Proofs of Existence}}

% ----------------------------------------------------------------------
% Page Setup
% ----------------------------------------------------------------------

\pagenumbering{gobble}

% ----------------------------------------------------------------------
% Custom Commands
% ----------------------------------------------------------------------

% ----------------------------------------------------------------------
% Launch!
% ----------------------------------------------------------------------

\begin{document}

\section{Probabilistic Proofs of Existence II}

\subsection{Lov\'asz Local Lemma}

Suppose there is a list of unfortunate events happening with some
probability, which we will say less than some $p$. Represent them in a
graph, drawing edges between the dependent events. Suppose that each
node in this dependency graph has a degree less than or equal to $d$.
We can show that, if either $4pd \leq 1$ or $ep(d+1) \leq 1$ is
satisfied, then there is a positive probability that no unfortunate
events has happened. In the context of our previous discussion of
randomly constructed objects, in this case we would obtain that our
object does not have properties we wanted to avoid.
\begin{example}

  Suppose we have $n$ colours and a circle with $100n$ marked points,
  and we want to colour the points in such a way that there are 100
  points painted in each colour, and the whole palette is spread over
  the circle, so that no two points of the same colour are
  neighbouring.

  Choose a pair of points at random. There is a $\frac{1}{10000}$
  probability that there is a local problem at these points. Now, by
  looking at the pairs of nodes which share at least 1 colour, we have
  $d = 400$ (connect two dots of the same colour, of which there are
  100, and count the pairs to which these points belong, of which
  there are 4), which, by direct plugging into the formula, satisfies
  the condition of the Lov\'asz lemma.

\end{example}

\begin{exercise}

  Suppose that we have a circular table and $n$ guests, some of whom
  are quarrelling. How likely are we to position our guests in such a
  way that the distance between the quarrelling guests is maximised,
  provided we position them randomly?

\end{exercise}

\begin{example}

  Suppose that we write lines of sequences, of length $n$ on $n$th
  line, such that for each line there is no programme which can
  compress a sequence up to $\frac{n}{2}$ bits.

  There are $2^{n}$ sequences in total, and $2^{\frac{n}{2}}$
  compressed sequences, if their length is $\frac{n}{2}$ or less. Then
  the probability of getting a line described above is
  $2^{-\frac{n}{2}}$, which is very small.

  
\end{example}
\begin{example}[Expanders]

Suppose that a graph $V$ and its subgraph $V'$ are given such that
$\abs{V'} < \frac{\abs{V}}{2}$. There are edges inside a subgraph and
borderline edges of a subgraph connected to the nodes in a graph which
are in the complement of the subgraph. Denote these borderline edges
as $\partial V$.

Suppose that the maximal degree of an edge in $V$ is less than or
equal to $1000$, and $\abs{\partial V} \geq 10 \abs{V'}$.

By the Chernov inequality, we thus obtain an approximation

\begin{equation*}
  \sum_{j= 901}C_{\abs{V}-1}\left(\frac{100}{\abs{V}}\right)\left(1-\frac{100}{\abs{M}}\right),
\end{equation*}

which is quite small. 

\end{example}



\end{document}
