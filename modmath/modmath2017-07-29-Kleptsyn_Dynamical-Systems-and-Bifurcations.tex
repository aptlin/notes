% -*- coding: utf-8; -*-
%%% Local Variables:
%%% mode: latex
%%% TeX-engine: xetex
%%% TeX-master: t
%%% End:
\documentclass[11pt]{scrartcl}
\usepackage[fancy, beaue, pset, anon]{masty}
\pSet{\nt{Kleptsyn}{1}{Dynamic Systems and Bifurcations}}

% ----------------------------------------------------------------------
% Page Setup
% ----------------------------------------------------------------------

\pagenumbering{gobble}

% ----------------------------------------------------------------------
% Custom Commands
% ----------------------------------------------------------------------

% ----------------------------------------------------------------------
% Launch!
% ----------------------------------------------------------------------

\begin{document}

\section{Dynamical Systems and Bifurcations}

\subsection{Introduction}

Suppose that $(X, f)$ is given, where $X$ is a manifold and $f$ is a
mapping from $X$ to $X$. Usually we also say that $f$ is continuous,
and $X$ is compact. $(X, f)$ defines a \textit{dynamical system}.

If $f$ is $C^{k}$-smooth, then a dynamical system is $C^{k}$-\textit{smooth}.

If $f$ is invertible and $f^{-1}$ is continuous, then $(X, f)$ is an
\textit{invertible} dynamical system.

\subsection{Examples}
\begin{example}

  When a link between differential equations and dynamical systems was
  realised, a problem of Laplacian determinism came to the attention,
  which led to the study of order and chaos.

  We can imagine a circle as a glued interval from $0$ to $1$, or,
  equivalently, as $\RR / \ZZ$.

  Take $f(x) = 2x \mod 1$. If we know the initial point with an error
  of $10^{-30}$, after every step the error is doubled. Therefore,
  after 100 iterations, we cannot say anything certain about the
  position of our dot. This is an example of \textit{dynamical chaos},
  or \textit{trajectory divergence}. The same behaviour is seen, for
  example, in weather forecasts, since we cannot measure temperature
  and pressure everywhere all the time.

  There are, however, examples with more regular behaviour. Take, for
  instance, $R_{\alpha}: X \mapsto X + \alpha \mod 1$, with
  $\alpha \not\in \QQ$.

\end{example}

\begin{definition}
  A sequence
  $x_{0}, x_{1} = f(x_{0}), \dots, x_{n} = f^{n}(x_{0}), \dots$ is
  called a \textit{trajectory} of a point $x_{0}$.
\end{definition}

\begin{definition}
  A point $x$ is called \textit{periodic}, if there exists $n$ such
  that $f^{n}(x) = x$.
\end{definition}

\begin{theorem}
  If $\alpha \not \in \QQ$, then for all $x_{0} \in S^{1}$ the
  trajectory of $x_{0}$ under the action of $R_{\alpha}$ is dense
  everywhere.
\end{theorem}

\begin{definition}
  A periodic point $x$ of period $1$, $f(x) = x$, is called a fixed
  point.
\end{definition}

\begin{definition}
  For $\dim = 1$, a multiplicative fixed point is a fixed point such that $f'(x_{0}) = \lambda$.
\end{definition}

This means that $f(x) - x_{0} \approx (x-x_{0})$, so that
$\abs{\lambda} < 1$ is an \textit{attracting point}, and if
$\abs{\lambda} > 1$, then $x_{0}$ is \textit{repelling}.

\subsection{Revertible Chaos}

Take $A = \begin{pmatrix}
  2 & 1\\
  1 & 1
\end{pmatrix}$. Note that $\sqrt{A} = \begin{pmatrix}
  1 & 1\\
  1 & 0
\end{pmatrix}$ maps $\cv{x;y}$ to $\cv{x+ y;x}$.

Since a Fibonacci sequence is asymptotically geometric with the golden
ratio as the multiplicative factor, then in the long term a standard
plane with the Descartes coordinate system, then the slope of the
horizontal axes coincides with $\phi$. Notice that the inverse maps
$\cv{u;v}$ to $\cv{v;u-v}$.

Take a two-dimensional torus on $\Pi^{2} = \RR^{2} / \ZZ^{2}$, and map
the trajectories defined by the aforementioned $f$. Then due to the
topology of a torus chaos would emerge, and its development is easy to
track.

\subsection{Complex Dynamicals}

Suppose that we are given $f: \CC P^{1} \to \CC P^{1}$.

Assume further that $f$ is a polynomial.

Let $K_{f} = \set{z; f^{n}(z)\not \to \infty}$.

Notice that in a projective space the point at infinity is attracting
with respect to $f$. We can simplify the study of dynamicals in a
projective space by making a substitution $ z = \frac{1}{w}$:

\begin{align}
  \frac{1}{f(\frac{1}{w})} &= \frac{1}{(\frac{1}{w})^{n} + a_{n-1}(\frac{1}{w})^{n-1} + \dots}\\
                           &= \frac{w^{n}}{1+ a_{n-1}w + a_{n-2}w^{2}+\dots}.
\end{align}

For example, $f(z) = z^{2}$ is mapped to a disc.

\begin{definition}
  A Julia set of a polynomial $f$ is $J_{f} = \partial K_{f}$.
\end{definition}

\begin{note*}
  Julia sets are often fractals.
\end{note*}

\subsection{}

\begin{definition}
  Let $f$ be a rational function, $f: \CC P^{1} \to \CC P^{1}$. 


\begin{note*}
  Rational functions are \textit{good} mappings of a Riemann sphere to itself.
\end{note*}

Consider a torus $E$ in $\CC / \Lambda$,where
$\Lambda = \ZZ \oplus i \ZZ$, and take a mapping $f: E \to E$ such
that $z \mapsto 2z \mod \Lambda$.

Let's make a sphere out of our torus:
$E /_{(z\sim -z)} \simeq \CC P^{1}$. Define a map
$p: E \to \CC P^{1}$, which is also known as Weirstra\ss\ $p$-function,
$p(z) = \frac{1}{z^{2}} + \sum_{\gamma \in \Lambda \ \set{0}} \left(
  \frac{1}{(z-\gamma)^{2}} - \frac{1}{\gamma^{2}}\right)$.

\end{definition}

This map defines an exciting example of a dynamical system, which
plays a key role in the Latt\`es construction.

\subsection{Bifurcations}

\begin{definition}
  A \textit{bifurcation} is a rapid change in the qualitative
  behaviour of a dynamical system.
\end{definition}

Define $f_{\epsilon} = x + x^{2} + \epsilon$. Fixed points of
$f_{\epsilon}$ are thus at $x = \pm \sqrt{-\epsilon}$. Note that, for
example, if $\epsilon$ is less than 0, then $-\sqrt{\epsilon}$ is
attracting, and $\sqrt{\epsilon}$ is repelling. This is an example of
a saddle-knot bifurcation.

If the mapping is perturbed, we can study what happens with fixed
points. For example, a fixed point can degenerate into an orbit with a
period $2$. On the other hand, the stability interval can eventually
narrow down to a point at which a small perturbation pushes the
trajectory to drastic instability. This is, for instance, what happens
in the \textit{Andronov-Hopf} regime.


\end{document}
