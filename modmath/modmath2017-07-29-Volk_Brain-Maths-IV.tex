% -*- coding: utf-8; -*-
%%% Local Variables:
%%% mode: latex
%%% TeX-engine: xetex
%%% TeX-master: t
%%% End:
\documentclass[11pt]{scrartcl}
\usepackage[fancy, beaue, pset, anon]{masty}
\pSet{\nt{Volk}{4}{Brain Maths}}

% ----------------------------------------------------------------------
% Page Setup
% ----------------------------------------------------------------------

\pagenumbering{gobble}

% ----------------------------------------------------------------------
% Custom Commands
% ----------------------------------------------------------------------

% ----------------------------------------------------------------------
% Launch!
% ----------------------------------------------------------------------

\begin{document}

\section{Brain Maths IV}

\subsection{Revision}

Let $\dot{x} = f(x)$ for $x\in \RR^{2}$ be such that $f(0)$,
$Df(0) = \begin{pmatrix}
  \* & \*\\
  \* & \*
\end{pmatrix}$, and $\Spec Df = \set{\lambda_{1}, \lambda_{2}}$.

If $\lambda_{1}, \lambda_{2} < 0$, then the equilibrium points of $f$
are stable knots.

If $\lambda_{1} < 0 < \lambda_{2}$, then the equilibrium points are
saddles.

If the eigenvalues are complex conjugates of each other with the
nonzero imaginary parts, then the equilibrium points of $f$ are foci.

\begin{theorem}[Grobman-Hartman]

  If an equilibrium point is hyperbolic (i.e., $\Re (\lambda_{i}) \neq 0$),
  then, locally, $f$ is topologically dual to $Df$.

\end{theorem}

As a consequence, a vector field in the neighbourhood of a hyperbolic
equilibrium point is stable.

Thus, if we also have that $d_{C}(f, g) < \epsilon$, then
$f \sim Df \sim Dg \sim g$.

\subsection{Hadamard-Perron Theorem}

\begin{theorem}
  Suppose that $Df(0)$ is a saddle point.

  Then there exist unique $w^{s}$ and $w^{u}$ such that for all
  $x\in w^{s}$ we have $f^{n}(x) \to 0$ as $n\to +\infty$ and
  $w^{u}\to 0$ as $n\to -\infty$.

\end{theorem}

\begin{theorem}
  Stable equilibrium of a typical 1-parametric family $f_{\alpha}$ can
  degenerate in either of two cases:

  \begin{itemize}
  \item $\lambda_{1} = 0$, $\lambda_{2} \neq 0$
    In this case, we obtain a saddle-knot bifurcation.
  \item $\lambda_{1, 2} = \pm iw$, $w \neq 0$
    This is an Andronov-Hopf bifurcation.
  \end{itemize}
\end{theorem}

Now, if $\alpha = 0$ and one of the eigenvalues is also equal to 0,
then another theorem applies:

\begin{theorem}[Central Manifold Theorem]
  There exists $W^{c} \in C^{0}$ and unique $W^{s}, W^{u}$ such that
  $T_{0}W^{i} = E^{i}$, where $i\in \set{s, u, c}$.
\end{theorem}

\subsection{Reduction Principle}

It is worthwhile to note that we can make up a system which is
topologically equivalent to our saddle-knot:


\begin{equation*}
  \begin{cases}
    \dot{x} = f_{\alpha}(x)\\
    \dot{\alpha} = 0
  \end{cases} \sim
  \begin{cases}
    \dot{x_{1}} = v_{\alpha}(x)\\
    \dot{x_{2}} = x_{2}\\
    \dot{x_{3}} = - x_{3}
  \end{cases},
\end{equation*}

where $x_{2}$ and $x_{3}$ are called \textit{saddle extensions}. This
allows us to classify our equilibrium points and predict how stable
they are.

\subsection{Attractive Cycles}

Cycles can degenerate due to bifurcations (eg homoclinic orbits of
saddle-knots can form). The way by which the cycle is tranformed
depends on the nature of the eigenvalues of $Df$.

If the parameter of bifurcation changes, for example, for neurons, if
the current increases, then a cycle can form at a particular
frequency. For example, a phenomenon of \textit{ghost attractors} can
arise, which usually means that a cycle is very slow..

The classification into integrators and resonators originates exactly
from the differences in the way attractor cycles behave under
bifurcations.

\end{document}
