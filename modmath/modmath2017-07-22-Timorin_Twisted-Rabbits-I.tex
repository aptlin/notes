% -*- coding: utf-8; -*-
%%% Local Variables:
%%% mode: latex
%%% TeX-engine: xetex
%%% TeX-master: t
%%% End:
\documentclass[11pt]{scrartcl}
\usepackage[fancy, beaue, pset, anon]{masty}
\pSet{\nt{Timorin}{1}{Twisted Rabbits}}
  \usepackage{lineno}
  % ----------------------------------------------------------------------
  % Page setup
  % ----------------------------------------------------------------------

  \pagenumbering{gobble}

  % ----------------------------------------------------------------------
  % Custom commands
  % ----------------------------------------------------------------------

  % alignment

  \newcommand*{\LongestHence}{$\Rightarrow$}% function name
  \newcommand*{\LongestName}{$f_o(-x)+f_e(-x)$}% function name
  \newcommand*{\LongestValue}{$(-a)x +(-a)(-y)$}% function value
  \newcommand*{\LongestText}{\defi}%

  \newlength{\LargestHenceSize}%
  \newlength{\LargestNameSize}%
  \newlength{\LargestValueSize}%
  \newlength{\LargestTextSize}%

  \settowidth{\LargestHenceSize}{\LongestHence}%
  \settowidth{\LargestNameSize}{\LongestName}%
  \settowidth{\LargestValueSize}{\LongestValue}%
  \settowidth{\LargestTextSize}{\LongestText}%

  % Choose alignment of the various elements here: [r], [l] or [c]

  \newcommand*{\mbh}[1]{{\makebox[\LargestHenceSize][r]{\ensuremath{#1}}}}%
  \newcommand*{\mbn}[1]{{\makebox[\LargestNameSize][r]{\ensuremath{#1}}}}%
  \newcommand*{\mbv}[1]{\ensuremath{\makebox[\LargestValueSize][r]{\ensuremath{#1}}}}%
  \newcommand*{\mbt}[1]{\makebox[\LargestTextSize][l]{#1}}%

  \newcommand{\R}[1]{\label{#1}\linelabel{#1}}
  \newcommand{\lr}[1]{line~\lineref{#1}}

  % ----------------------------------------------------------------------
  % Launch!
  % ----------------------------------------------------------------------

  \begin{document}

  \section{Twisted Rabbits}

  Twisted rabbits have arised in the study of discrete dynamic systems.

  \subsection{Introduction}
  \begin{definition}
    The \textit{Douady rabbit} is a fractal defined by
    $P_{c}(z) = z^{2} + c$ such that 0 is periodic with the period 3,
    which means that $P_{c}P_{c}P_{c}(0) = 0$, and $P_{c} \neq 0$.

    Thus, $(c^{2} + c)^{2} + c = 0$, which means that
    $c^{3} + 2 c^{2} + c + 1 = 0$. We require that $\img c > 0$.
  \end{definition}

  The mapping $P_{c}$ of a point defines its orbit, which is a set
  $\set{z, P_{c}(z), P_{c}(P_{c}(z)) = P^{\circ 2}_{c}(z), \dots}$.

  Only a finite area of the complex plane does not approach infinity
  exponentially fast. This set is called a \textbf{Julia set}.

  \begin{definition}
    A filled Julia set is a set
    $K(P_{c}) = \set{z; P_{c}^{\circ n}(z)\not \to \infty}$.
  \end{definition}

  To proceed, we will need a concept of a Jordan curve, which is
  obtained when a circle is mapped continuously to a plane in such a
  way that the mapped points do not coincide.

  The \textit{body} of a rabbit looks like a circle. It also has a
  pair of \textit{ears}. Then a scaled copy of itself looks like a
  pair of paws. A fractal structure is obtained. Our rabbit is an
  example of a filled Julia set. Note that no ears of any degree
  intersect each other.

  The zero point is located at the centre.

  Let's $v = P_{c}(0) = c$ and $w = P^{\circ 2}_{c}(0) = P_{c}(v)$. We
  can see that $0\mapsto v \mapsto w \mapsto 0$.

  Since the derivative at zero is zero, all the neighbouring points
  on mapping get even closer.

  It is worthwhile to note that the picture shows the dynamic
  behaviour of the system. Thus, all the points get eventually into
  the periodic cycle of $0, v, w$, which act as attractors. There is
  also a stable point $\alpha$, to which the ears are attached, and
  around which all the points jump around. If $\alpha$ is omitted, our
  rabbit would fall apart into three parts.

  Another important point is $\beta$, which is a limit of the images
  of $0$.

  Drawing a rabbit is an easy and relatively fast programming task,
  which is partially due to the fast exponential runaway of the mapped
  points.

  Now, we introduce a concept of a \textit{Thurston lamination}.

  Draw a unit circle. Now we take points and assign the angle measure,
  which, for our purposes, would range from $(0, 1]$. First, take $0$,
  and assign the angle of $1$. Then mark the points with the angles
  from $\frac{1}{7}$ to $\frac{6}{7}$.

  Denote a map doubling the angle as $\sigma_{2}$.

  Connect the points with angles $\frac{1}{7}$, $\frac{2}{7}$ and
  $\frac{4}{7}$ with lines, from $\frac{1}{7}$ to $\frac{2}{7}$, from
  $\frac{2}{7}$ to $\frac{4}{7}$, and from $\frac{4}{7}$ to
  $\frac{1}{7}$.

  To obtain the representation of a rabbit on a circle, we need to
  find an equivalence relation which is invariant with respect to our
  mapping.

  We map our triangle to a new triangle connecting the points with
  angles of $\frac{1}{14}, \frac{9}{14}$ and $\frac{11}{14}$.

  Our goal is to construct the images of our triangles so that they do
  not intersect. We can prove that it can always be achieved.

  There is also a \textit{stripe} defined by the images, which lacks
  triangles covering the entire stripe.

  \begin{note*}
    A filled Julia set is connected.
  \end{note*}

  In this way, we obtain a countable number of triangles in the unit
  circle.

  \subsection{Relatives}

  What happens if we take other values of $c$?

  If, for example, $\img(c) < 0$, we obtain an antirabbit, which can
  be identified by the order of the ears.

  If, however, $\img(c_{*}) = 0$, then we obtain an \textit{aeroplane}.

  To study the aeroplane, we construct a Cantor set, which is a
  segment with a countable number of disjoint open intervals omitted
  in such a way that there is no interval left at the end. Cantor sets
  are also identified as \textit{Cantor dust}. In particular, we build
  a Cantor necklace, where circles are inserted into the omitted
  intervals, and then another necklaces grow on the circles
  themselves.

  Again, let's draw a unit circle and divide it into seven parts.

  In the case of the aeroplane, however, we would not get triangles in
  the lamination. First, we connect the points with angles
  $\frac{1}{7}$ and $\frac{6}{7}$, then with angles $\frac{2}{7}$ and
  $\frac{5}{7}$, and finally with $\frac{3}{7}$ and $\frac{4}{7}$. The
  critical stripe is obtained in the region defined by the chords
  connecting $\frac{2}{7}$ and $\frac{5}{7}$, and $\frac{3}{14}$ and
  $\frac{11}{14}$. The difference of the lamination of the aeroplane
  from the lamination corresponding to the rabbit is that the number
  of chords is uncountable.

  \subsection{Generalisations}

  The mapping that we used for the rabbit plays an important role.
  However, to prove results in complex analysis, only the intrinsic
  properties of the mapping are used, not the explicit equation. This
  is due to the fact that the form of the equation does not tell much
  about the topological properties of the system obtained.

  This is why we use such tools like Thurston laminations.

  The mapping $z^{2} + c$ can be viewed as the mapping of the sphere
  $S^{2}$ into itself.

  The preimage of a small disc in the topological sense (an area bound
  by a Jordan curve) consists of several discs, on which the mapping
  is $z \mapsto z^{k}$, where $k \in \ZZ^{+} $.

  Suppose that we take a map $f$ between two discs, and we define two
  homeomorphisms conserving the orientation, $\phi_{1}$ and
  $\phi_{2}$, on each of the discs.

  Note that $z^{2} + c$ is a branched covering.

  All branch coverings have critical points, of which there is a
  finite amount.

  To proceed, we need a concept of a postcritically finite branched
  covering, which is a set of orbits of the critical points:
  
  \begin{equation*}
    P(f) = \cup \text{ orbits of all critical points}.
  \end{equation*}

  The theory of postcritically finite branched coverings was actively
  studied by Thurston, and led to important results in Thurston theory.

\end{document}
