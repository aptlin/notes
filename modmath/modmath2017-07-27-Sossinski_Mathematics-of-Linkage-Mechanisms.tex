% -*- coding: utf-8; -*-
%%% Local Variables:
%%% mode: latex
%%% TeX-engine: xetex
%%% TeX-master: t
%%% End:
\documentclass[11pt]{scrartcl}
\usepackage[fancy, beaue, pset, anon]{masty}
\pSet{\nt{Sossinski}{1}{Mathematics of Linkage Mechanisms}}
  \usepackage{lineno}
  % ----------------------------------------------------------------------
  % Page setup
  % ----------------------------------------------------------------------

  \pagenumbering{gobble}

  % ----------------------------------------------------------------------
  % Custom commands
  % ----------------------------------------------------------------------

  % alignment

  \newcommand*{\LongestHence}{$\Rightarrow$}% function name
  \newcommand*{\LongestName}{$f_o(-x)+f_e(-x)$}% function name
  \newcommand*{\LongestValue}{$(-a)x +(-a)(-y)$}% function value
  \newcommand*{\LongestText}{\defi}%

  \newlength{\LargestHenceSize}%
  \newlength{\LargestNameSize}%
  \newlength{\LargestValueSize}%
  \newlength{\LargestTextSize}%

  \settowidth{\LargestHenceSize}{\LongestHence}%
  \settowidth{\LargestNameSize}{\LongestName}%
  \settowidth{\LargestValueSize}{\LongestValue}%
  \settowidth{\LargestTextSize}{\LongestText}%

  % Choose alignment of the various elements here: [r], [l] or [c]

  \newcommand*{\mbh}[1]{{\makebox[\LargestHenceSize][r]{\ensuremath{#1}}}}%
  \newcommand*{\mbn}[1]{{\makebox[\LargestNameSize][r]{\ensuremath{#1}}}}%
  \newcommand*{\mbv}[1]{\ensuremath{\makebox[\LargestValueSize][r]{\ensuremath{#1}}}}%
  \newcommand*{\mbt}[1]{\makebox[\LargestTextSize][l]{#1}}%

  \newcommand{\R}[1]{\label{#1}\linelabel{#1}}
  \newcommand{\lr}[1]{line~\lineref{#1}}

  % ----------------------------------------------------------------------
  % Launch!
  % ----------------------------------------------------------------------

  \begin{document}

  \section{Mathematics of Flat Linkage Mechanisms}

  \subsection{Introduction}

  Suppose that a linkage mechanism $ABCDEF$ is given, with each
  linkage having a known length. Some hinges are fixed and some are
  allowed to move. Suppose that $A$ and $E$ are fixed, and have
  coordinates $(0, 0)$ and $(1, 0)$ respectively, while moving points
  $B, C, D$ and $F$ have coordinates $(x_{1},
  y_{1})$,$(x_{2}, y_{2})$, $(x_{3}, y_{3})$ and $(x_{4}, y_{4})$.
  
  The configuration space of a linkage mechanism can be defined as a
  topological space, with a standard euclidean metric. Since the links
  between each hinge are straight lines, we can easily write the
  equations describing the relations between $\set{x_{i}}$ and
  $\set{y_{i}}$.

  \begin{problem*}[Direct Problem]
    \hfill

    What can we say about classes of configuration spaces from classes
    of linkage mechanisms up to homeomorphisms?
  \end{problem*}

  \begin{problem*}[Inverse Problem]
    \hfill

    Given a class of configuration spaces, what can we say about the
    corresponding class of linkage mechanisms?

  \end{problem*}

  \begin{theorem}[Thurston]
    There exists a flat linkage mechanism such that its configuration
    space is your signature.
  \end{theorem}
  \begin{definition}
    $L$ is a linkage mechanism in a general position if
    $\epsilon_{1}l_{1}+\epsilon_{2}l_{2}+\dots + \epsilon_{n}l_{n}
    \neq 0$ for all $\epsilon_{i} \in \set{-1, 1}$.
  \end{definition}

  \begin{definition}
    For polyhedral linkage mechanisms in a general position, defined
    by a set $\set{l_{1}, l_{2}, \dots, l_{n}}$ of lengths
  \end{definition}
  \begin{exercise}

    The configuration space $Conf L$ is unique up to the order of
    linkages.

  \end{exercise}

  \begin{theorem}
    For all $g$ there exists $L_{g}$ such that $Conf L = M_{g}^{2}$,
    where $M_{g}$ is a manifold of genus $g$.
  \end{theorem}

  \begin{theorem}
    For all $n$-dimensional smooth oriented manifolds $M^{n}$ there
    exists $L_{M}$ such that $Conf L_{M} = \sqcup^{k} M^{n}$, where
    $k$ is some finite positive integer.
  \end{theorem}

  \begin{theorem}
    For all algebraic varieties $X$ in $\RR^{n}$ there exists $L_{X}$
    there exists $Conf L_{X} = \sqcup ^{k} X$, where $k$ is some
    finite positive integer.
  \end{theorem}




  \end{document}
