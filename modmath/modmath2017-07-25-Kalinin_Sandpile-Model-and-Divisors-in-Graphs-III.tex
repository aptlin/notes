% -*- coding: utf-8; -*-
%%% Local Variables:
%%% mode: latex
%%% TeX-engine: xetex
%%% TeX-master: t
%%% End:
\documentclass[11pt]{scrartcl}
\usepackage[fancy, beaue, pset, anon]{masty}
\pSet{\nt{Kalinin}{3}{Sandpile Model and Divisors in Graphs}}
  \usepackage{lineno}
  % ----------------------------------------------------------------------
  % Page setup
  % ----------------------------------------------------------------------

  \pagenumbering{gobble}

  % ----------------------------------------------------------------------
  % Custom commands
  % ----------------------------------------------------------------------

  % alignment

  \newcommand*{\LongestHence}{$\Rightarrow$}% function name
  \newcommand*{\LongestName}{$f_o(-x)+f_e(-x)$}% function name
  \newcommand*{\LongestValue}{$(-a)x +(-a)(-y)$}% function value
  \newcommand*{\LongestText}{\defi}%

  \newlength{\LargestHenceSize}%
  \newlength{\LargestNameSize}%
  \newlength{\LargestValueSize}%
  \newlength{\LargestTextSize}%

  \settowidth{\LargestHenceSize}{\LongestHence}%
  \settowidth{\LargestNameSize}{\LongestName}%
  \settowidth{\LargestValueSize}{\LongestValue}%
  \settowidth{\LargestTextSize}{\LongestText}%

  % Choose alignment of the various elements here: [r], [l] or [c]

  \newcommand*{\mbh}[1]{{\makebox[\LargestHenceSize][r]{\ensuremath{#1}}}}%
  \newcommand*{\mbn}[1]{{\makebox[\LargestNameSize][r]{\ensuremath{#1}}}}%
  \newcommand*{\mbv}[1]{\ensuremath{\makebox[\LargestValueSize][r]{\ensuremath{#1}}}}%
  \newcommand*{\mbt}[1]{\makebox[\LargestTextSize][l]{#1}}%

  \newcommand{\R}[1]{\label{#1}\linelabel{#1}}
  \newcommand{\lr}[1]{line~\lineref{#1}}

  % ----------------------------------------------------------------------
  % Launch!
  % ----------------------------------------------------------------------

  \begin{document}

  \section{Sandpile Model and Divisors in Graphs}

  \subsection{Revision}

  We have seen the concepts of topplings, relaxations, revertible
  states and forbidden configurations. We have also stidued the
  Riemann-Roch theorem, waves, $\Delta F$, the sandpile group and its
  unit.

  \subsection{}
  % No note needed in the notes for the  previous lecture.

  \begin{exercise}

    Suppose that $\phi = (3 + \psi)^{0}$. Then for all $\lambda$ there
    exists $\psi \geq 0$ such that $(\lambda + \psi)^{0} = \phi$.

  \end{exercise}

  \begin{soln}
    \hfill

    Note that
    $\lambda \to \lambda^{0} \to 3 \to 3+ \psi \to (3+\psi)^{0}$.
    Thus, $\phi = (\lambda + (3-\lambda^{0})+\psi)^{0}$.
  \end{soln}

  \begin{exercise}

    Find $\Delta F$ for $F(i, j) = i(i+1) + j(j+1)$.

  \end{exercise}

  \begin{soln}
    \hfill

    Note that $\Delta(i(i+1)) = -4(i+1)i+2(i+1)i +(i-1)i +(i+1)(i+2)$,
    which means that $\Delta(i(i+1)) = -4i +2i -i +3i + 2 = 2$.

    Therefore, $\Delta F = 4$.
    
  \end{soln}

  Now we prove that the sandpile group \textit{is} a group.

  There are three ways to look at this problem.

  Consider an integer lattice, and choose two linearly independent vectors.

  We say that $(i, j) \sim (i', j')$ if
  $(i-i', j-j') = k_{1}v_{1} + k_{2}v_{2}$ for $k_{1}, k_{2} \in \ZZ$.

  Let $\phi$ and $\phi'$ be two states in the form
  $\Gamma \setminus \delta\Gamma \to \ZZ$. We say that
  $\phi \sim \phi'$ if $\phi - \phi' = \Delta F$, where $F$ is in the
  form $F:\Gamma \to \ZZ$. In other words, $\phi$ and $\phi'$ are
  equivalent if there is a sequence of topplings and antitopplings
  transforming $\phi$ into $\phi'$.

  Let $N$ be the number of vertices in
  $\Gamma \setminus \delta\Gamma$, and consider
  $\ZZ^{n} / \ZZ\< v_{1}v_{2}\dots v_{N}\>$, where $\ZZ^{n}$ is a set
  all states. Note that $\ZZ^{n} / \ZZ\< v_{1},v_{2},\dots, v_{N}\>$ is a
  group.

  \begin{lemma}
    Each equivalence class has exactly one revertible state.
  \end{lemma}
  \begin{note*}
    In this way, the set of equivalence classes is in one-to-one
    correspondence with the set of revertible states of the group.
  \end{note*}

  \begin{proof}
    \hfill

    Consider $\Delta F$, where $F \equiv -1$. Suppose that $\Gamma$ is
    rectangluar. Then $\Delta F$ is such that there are $2$'s at the
    vertices of a rectangular, and $1$'s at the other border
    positions. The rest is filled with $0$'s. This state is called the
    \textit{Kreitz unity} and denoted as $\beta$. Each equivalence
    class has more than or equal to 1 revertible state. Let $\phi$ be arbitrary,
    then $(\phi + k\beta)^{0}$ is revertible, where $k$ is some big
    natural number.

    \begin{exercise}

      Approximate $k$.

    \end{exercise}

    Now we prove that each equivalence class has less than or equal to
    1 state.

    Let $\phi_{1}$ and $\phi_{2}$ be revertible.

    Let $D = \set{v \in \Gamma; F(v) = \min F}$.

    Then $D$ is a forbidden configuration for $\phi_{2}$.

    Suppose that $\min F = 0$. Choose a corner in $D$. Then
    $3 \leq \phi_{1}(v) \leq \phi_{2}(v) + 2$. Then
    $\phi_{2}(v) \leq 1$.

    We, however, need to account for the case when the region defined
    by $D$ is not compleely inside $\Gamma\setminus\delta\Gamma$.
  \end{proof}
  \begin{note*}
    $\Delta F \sim \< 0 \>$. Thus, $\< 0 \> = \phi + \Delta 1$.
  \end{note*}

  \begin{exercise}

    If two states are equivalent, there exists another state from
    which the first two states receive sand.

  \end{exercise}

  \subsection{Group Unity Element}

  We canq prove that the sandpile group unity element $E$ is
  $(\<8\> - \<8\>^{0})^{0}$.

  \begin{theorem}
    Consider a rectangle $n$ by $m$ such that $m > 10 n$. Then there
    is a stripe of 2s in the middle, with the distance from the left
    and range margin less than or equal to $n$.
  \end{theorem}

  \begin{proof}
    \hfill

    Consider $((n^{2} + n)\beta = \phi)$.

    \begin{lemma}
      $\phi^{0} = E$.
    \end{lemma}

    \begin{proof}
      \hfill

      $\phi^{0} \sim E$, and thus we only need to show that
      $\phi^{0} \geq \< 2 \>$, which means that $\phi^{0}$ is
      revertible.
      \begin{exercise}

        Show that $\< 2 \>$ is revertible.

      \end{exercise}
      \begin{exercise}

        If $\phi \geq \psi$ and $\psi$ is revertible, then $\psi^{0}$
        is revertible.

      \end{exercise}

      \begin{exercise}

        $\phi$ is revertible if and only if
        $(\phi+ \beta)^{0} = \phi$, where $\beta$ is the Kreitz unity.
        (Hint: if $(\phi + \beta)^{0} \neq \phi$, then there exists a
        forbidden state.)

      \end{exercise}

      Let $F(i, k) = (n-k)^{2}k + (n-k)$. It is easy to check that
      $\phi + \Delta F$ is in the form with a stripe of twos in the
      middle, which means that $\Delta F = 2$ almost everywhere.

      Define a map $G$ as follows for $k > [n \sqrt{2}]$:.
      
      \begin{equation*}
        G(k, i) = \frac{([n \sqrt{2} -k])([n \sqrt{2}] - k -9)}{2} = 0,
      \end{equation*}

      \begin{exercise}

        Continue the proof by using $G$ on a rectangle with a stripe of
        twos in the middle.

      \end{exercise}      
    \end{proof}

  \end{proof}

  
  
\end{document}

