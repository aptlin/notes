% -*- coding: utf-8; -*-
%%% Local Variables:
%%% mode: latex
%%% TeX-engine: xetex
%%% TeX-master: t
%%% End:
\documentclass[11pt]{scrartcl}
\usepackage[fancy, beaue, pset, anon]{masty}
\pSet{\nt{Kalinin}{2}{Model of Sand Moving and Divisors in Graphs}}
  \usepackage{lineno}
  % ----------------------------------------------------------------------
  % Page setup
  % ----------------------------------------------------------------------

  \pagenumbering{gobble}

  % ----------------------------------------------------------------------
  % Custom commands
  % ----------------------------------------------------------------------

  % alignment

  \newcommand*{\LongestHence}{$\Rightarrow$}% function name
  \newcommand*{\LongestName}{$f_o(-x)+f_e(-x)$}% function name
  \newcommand*{\LongestValue}{$(-a)x +(-a)(-y)$}% function value
  \newcommand*{\LongestText}{\defi}%

  \newlength{\LargestHenceSize}%
  \newlength{\LargestNameSize}%
  \newlength{\LargestValueSize}%
  \newlength{\LargestTextSize}%

  \settowidth{\LargestHenceSize}{\LongestHence}%
  \settowidth{\LargestNameSize}{\LongestName}%
  \settowidth{\LargestValueSize}{\LongestValue}%
  \settowidth{\LargestTextSize}{\LongestText}%

  % Choose alignment of the various elements here: [r], [l] or [c]

  \newcommand*{\mbh}[1]{{\makebox[\LargestHenceSize][r]{\ensuremath{#1}}}}%
  \newcommand*{\mbn}[1]{{\makebox[\LargestNameSize][r]{\ensuremath{#1}}}}%
  \newcommand*{\mbv}[1]{\ensuremath{\makebox[\LargestValueSize][r]{\ensuremath{#1}}}}%
  \newcommand*{\mbt}[1]{\makebox[\LargestTextSize][l]{#1}}%

  \newcommand{\R}[1]{\label{#1}\linelabel{#1}}
  \newcommand{\lr}[1]{line~\lineref{#1}}

  % ----------------------------------------------------------------------
  % Launch!
  % ----------------------------------------------------------------------

  \begin{document}

  \section{Sandpile Model and Divisors in Graphs}
  \subsection{Revision}
  \begin{soln}[1]
    \hfill

    Let $N = \sum_{v \in \Gamma} \phi(v)$. Note that at the point
    closest to the boundare less than or equal to $N+1$ topplings. For
    the point second closest to the boundary the number of topplings
    is bounded from above by $4(N+1)$. As we move closer and closer to
    the centre, we can define a bound, which means that the total
    number of topplings is finite.

  \end{soln}

  % \begin{soln}[3]
  %   \hfill
  %   Let $F = \min \set{G; \phi + \Delta G \leq 3}$.

  %   We prove that $F \geq \min$ and $F\leq min$.

  %   Note that $0 = F_{0} \leq F_{1} < \dots < F_{n} = F$ and
  %   $F_{0} \equiv G$, while $\phi(v) \geq 4$.
  % \end{soln}
  \begin{exercise}

    Let $\phi$ be a revertible state. Then there exists $\psi$ such
    that $\phi = (\< 100 \> + \psi)^{0}$.

  \end{exercise}
  \begin{exercise}

    If for $\phi$ there exists $\psi$ such that
    $(3+ \psi)^{0} = \phi$, then for all $\beta$ there exists $\psi$
    such that $\phi = (\beta + \psi)^{0}$.

  \end{exercise}
  \begin{soln}[4]
    \hfill

    We are proving that a revertible state does not have forbidden
    configurations.

    Let $v$ be a vertex in $D$ of final toppling in the relaxation
    leading to $\phi$, where $\phi$ is revertible.

    Then $D\setminus v$ from the previous step is also in a forbidden
    state, which is a contradiction.

  \end{soln}

  \subsection{Decomposition of Relaxations into Waves}
  Think about the following statement, and try to make it work:
  \begin{theorem}[ill-defined]
  Let $\phi$ be stable. Then
  $(\phi + \delta_{v})^{0} = (W_{v})^{\infty}\phi + \delta_{v}$, where
  $\delta_{v} = \begin{cases}
    1 \text{ in } v\\
    0 \text{ otherwise }
  \end{cases}$ $w_{v}$ is a \textit{wave operator} at $v$, defined by
  the equation $w_{v}\phi = \begin{cases}
    \phi\to \phi + \delta\\
    \text{ toppling in } v \text{ if possible } \to \phi\\
    \phi' - \delta_{v} \to \text{ relaxation }
  \end{cases}$, and $v$ is a vertex in
  $\Gamma \setminus \delta\Gamma$.
  \end{theorem}

  \begin{note*}
    If $v$ has a neighbouring vertex with $c\geq 3$ grains, then the
    definition of $w_{v}$ is as noted above. Otherwise,
    $w_{v}\phi = \phi$.
  \end{note*}

  \begin{exercise}

    When $\phi$ is mapped to $w_{v}\phi$, there is either 1 or no
    topplings at each vertex.

  \end{exercise}

  \subsection{Discrete Harmonic Functions}

  Recall that $\Delta F(v) = \sum_{w\sim v} F(w) - \deg(v)F(v)$.

  \begin{exercise}

    If $F:\ZZ^{2} \to \ZZ_{\geq 0}$ is harmonic (i.e.
    $\Delta F = 0$),then $F$ is a constant.

  \end{exercise}

  \begin{exercise}

    If $F: \ZZ^{2} \to \RR$ is harmonic, then $F$ is a constant.

  \end{exercise}

  \begin{exercise}

    Compute $\Delta F(i, j)$, where $F$ is a linear function such that
    $F(i, j) = A_{i} + B_{j} + C$, $F(i, j) = ij$,
    $F(i, j) = \frac{1}{2}(i(i+1)+j(j+1))$, $[\frac{1}{3}i^{2}]$, and
    $[\frac{1}{3}(i^{2}+j^{2}+7ij + i)]$.

  \end{exercise}

  \begin{exercise}[Mega]

    Find all
    $F(i, j) = [\alpha i^{2} + \beta j^{2} + \gamma ij + \dots +
    \delta i + \lambda j + \mu]$ such that $0 \leq \Delta F \leq 3$.

  \end{exercise}

  \subsection{Riemann-Roch's Formula}
  \begin{definition}
    Let $D$ and $D'$ be divisors.

    We say that $D \sim D'$ if there exists a function $F$ from the
    set of nodes to $\ZZ$ such that $D-D' = \Delta F$.
  \end{definition}

  \begin{definition}
    A divisor $D = \sum a_{i}v_{i}$ over the edges $\set{v_{i}}$ is
    called \textit{effective}, if $a_{i} \geq 0$.
  \end{definition}

  \begin{definition}
    Let $D$ be a divisor.

    A rank $r(D)$ is defined as $\max_{\text{effective divisors $D'$
        of degree $s$}} s$, if $D-D' \sim \text{effective divisor}$,
    and $- 1$ otherwise.
  \end{definition}

  \begin{theorem}
    Suppose that a graph $G$ is given. Let $g$ be the genus of $G$,
    which is the number of edges minus the number of nodes plus 1.

    Then
    
    \begin{equation*}
      r(D) - r(k-D) = d - g + 1,
    \end{equation*}

    where $D$ is a divisor, which is a formal sum in the form
    $D = \sum a_{i}v_{i}$, with $a_{i} \in \ZZ$ and $\set{v_{i}}$ are
    the nodes of $G$, and $d = \sum a_{i}$ is the degree of a divisor,
    and $K$ is also a divisor in the form
    $k = \sum_{r\in G}(\deg v - i) v$.

  \end{theorem}

  \subsection{Key Words}
  \textit{discrete Laplacian, discrete harmonic functions}
\end{document}
