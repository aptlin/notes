% -*- coding: utf-8; -*-
%%% Local Variables:
%%% mode: latex
%%% TeX-engine: xetex
%%% TeX-master: t
%%% End:
\documentclass[11pt]{scrartcl}
\usepackage[fancy, beaue, pset, anon]{masty}
\pSet{\nt{Smirnov}{1}{Domino Tilings}}
  \usepackage{lineno}
  % ----------------------------------------------------------------------
  % Page setup
  % ----------------------------------------------------------------------

  \pagenumbering{gobble}

  % ----------------------------------------------------------------------
  % Custom commands
  % ----------------------------------------------------------------------

  % alignment

  \newcommand*{\LongestHence}{$\Rightarrow$}% function name
  \newcommand*{\LongestName}{$f_o(-x)+f_e(-x)$}% function name
  \newcommand*{\LongestValue}{$(-a)x +(-a)(-y)$}% function value
  \newcommand*{\LongestText}{\defi}%

  \newlength{\LargestHenceSize}%
  \newlength{\LargestNameSize}%
  \newlength{\LargestValueSize}%
  \newlength{\LargestTextSize}%

  \settowidth{\LargestHenceSize}{\LongestHence}%
  \settowidth{\LargestNameSize}{\LongestName}%
  \settowidth{\LargestValueSize}{\LongestValue}%
  \settowidth{\LargestTextSize}{\LongestText}%

  % Choose alignment of the various elements here: [r], [l] or [c]

  \newcommand*{\mbh}[1]{{\makebox[\LargestHenceSize][r]{\ensuremath{#1}}}}%
  \newcommand*{\mbn}[1]{{\makebox[\LargestNameSize][r]{\ensuremath{#1}}}}%
  \newcommand*{\mbv}[1]{\ensuremath{\makebox[\LargestValueSize][r]{\ensuremath{#1}}}}%
  \newcommand*{\mbt}[1]{\makebox[\LargestTextSize][l]{#1}}%

  \newcommand{\R}[1]{\label{#1}\linelabel{#1}}
  \newcommand{\lr}[1]{line~\lineref{#1}}

  % ----------------------------------------------------------------------
  % Launch!
  % ----------------------------------------------------------------------

  \begin{document}

  \section{Domino Tilings}

  \subsection{Introduction}

  There are several key questions we may ask about domino tilings:
  \begin{itemize}
  \item Does a tiling exist?
  \item How many tilings are there?
  \item How does a random tiling look like?
  \end{itemize}

  There are 4 types of domino parts, depending on the colour
  configuration a part would have if the shape was coloured like a
  chessboard. Shape of the base element significantly affects the
  patterns emerging with the increasing number of copies. For example,
  some tilings possess regions of \textit{freezing}, where only one
  colour is predominant.

  A domino tiling can be represented as a covering of a corresponding
  dual graph with dimers.

  \begin{theorem}[Kusteleyn, Temperdey-Fisher (1961)]
    The number of tiling of a rectangle $n$ by $m$ is
    
    \begin{equation*}
      \sqrt{\prod_{i=1}^{m}\prod_{k=1}^{n}2(\cos \frac{\pi i}{m+1} + i
        \cos \frac{\pi k}{n+1})}.
    \end{equation*}
  \end{theorem}
  \begin{proof}
    \hfill The number of tilings is $\sqrt{\perm A}$, where $A$ is the
    adjacency matrix for a dual graph and $\perm$ is a permanent,
    which can be calculated similarly to a determinant, but without
    any considerations for the sign of the terms.

    The formula for the number of tesselations can be obtained as
    follows.

    Take, for instance, two tiled squares, and put one over the other.
    Traverse the combined squares by following their patterns
    depending on the corresponding chessboard colouring until you
    return to the initial position, which yields a cycle.

    \begin{lemma}
      Any tesselation can be obtained from the other by applying one
      transformation, which depends on the shape of a base unit.
    \end{lemma}


    We can add signs to the terms in $A$ to obtain a new matrix $K$
    such that $\perm A = \abs{\det K}$.

    The following method of building $K$ is due to R.Kenyon(2000).
    Assign $i$ to all the vertical edges and $1$ to the horizontal.
    The other way would include a chessboard-like assignment of signs
    to the edges.

    Note that $\det K = \prod^{nm}\lambda_{i}$.

    % Assume $\nu = 2\cos \frac{\pi x}{m+1}$ and $f_{x}$

    \begin{exercise}

      Continue the proof.

    \end{exercise}
    
  \end{proof}
  \begin{theorem}[Kuperberg et al]
    The number of tilings of an aztec diamond with a side of size $n$
    is $2^{\frac{n(n+1)}{2}}$.
  \end{theorem}

  \begin{definition}

    Take a rectangle, a choose two internal squares denoted as $x$ and
    $y$. Then a \textit{coupling function} $C(x, y) =K^{-1}_{x, y}$.
  \end{definition}
  \begin{note*}
    $C(x, y)$ corresponds to the number of tesselations with a sign.
    For instance, $C(x, x+1)$ is the number of domino tilings with $x, x+1$ as entries.
  \end{note*}

  Thus we obtain
  $\sum_{\alpha = \pm 1, \pm i} \alpha C(x, y+\alpha) = \begin{cases}
    0, x\neq y\\
    1, x=y
  \end{cases}$.

  There exists an operator $D$ such that $D C(x, x) = D C(x, y) = 0$.

  \subsection{Measuring Order}

  One of the methods to predict the pattern is to compute the average
  height function. See, for example, Cohn, Kenyon and Propp for an in-depth
  study.
  
\end{document}
