% -*- coding: utf-8; -*-
%%% Local Variables:
%%% mode: latex
%%% TeX-engine: xetex
%%% TeX-master: t
%%% End:
\documentclass[11pt]{scrartcl}
\usepackage[fancy, beaue, pset, anon]{masty}
\pSet{\nt{Bragilevsky}{3}{Curry-Howard Correspondence}}
  \usepackage{lineno}
  % ----------------------------------------------------------------------
  % Page setup
  % ----------------------------------------------------------------------

  \pagenumbering{gobble}

  % ----------------------------------------------------------------------
  % Custom commands
  % ----------------------------------------------------------------------

  % alignment

  \newcommand*{\LongestHence}{$\Rightarrow$}% function name
  \newcommand*{\LongestName}{$f_o(-x)+f_e(-x)$}% function name
  \newcommand*{\LongestValue}{$(-a)x +(-a)(-y)$}% function value
  \newcommand*{\LongestText}{\defi}%

  \newlength{\LargestHenceSize}%
  \newlength{\LargestNameSize}%
  \newlength{\LargestValueSize}%
  \newlength{\LargestTextSize}%

  \settowidth{\LargestHenceSize}{\LongestHence}%
  \settowidth{\LargestNameSize}{\LongestName}%
  \settowidth{\LargestValueSize}{\LongestValue}%
  \settowidth{\LargestTextSize}{\LongestText}%

  % Choose alignment of the various elements here: [r], [l] or [c]

  \newcommand*{\mbh}[1]{{\makebox[\LargestHenceSize][r]{\ensuremath{#1}}}}%
  \newcommand*{\mbn}[1]{{\makebox[\LargestNameSize][r]{\ensuremath{#1}}}}%
  \newcommand*{\mbv}[1]{\ensuremath{\makebox[\LargestValueSize][r]{\ensuremath{#1}}}}%
  \newcommand*{\mbt}[1]{\makebox[\LargestTextSize][l]{#1}}%

  \newcommand{\R}[1]{\label{#1}\linelabel{#1}}
  \newcommand{\lr}[1]{line~\lineref{#1}}

  % ----------------------------------------------------------------------
  % Launch!
  % ----------------------------------------------------------------------

  \begin{document}

  \section{Curry-Howard Correspondence III}

  \subsection{Introduction}

  How can we formalise the notion of a proof? For example, we can say
  that $\Gamma \vdash M: \phi$, where $\Gamma$ is the context, $M$ is
  a proof and $\phi$ is some statement. Now we can denote an inference path as follows:

  \begin{itemize}
  \item $\Gamma \vdash M: \phi \to \psi$, $\Gamma \vdash N:\phi$
  \item $\Gamma \vdash (MN): \psi$
  \end{itemize}

  We can also write our presuppositions before the proof and the
  statement, thus identifying explicitly elements of the context.

  We can also parametrise our proofs with the language of abstractions:

  \begin{itemize}
  \item $\Gamma, x: \phi \vdash M:\psi$
  \item $\Gamma \vdash (\lambda x: \phi. M):\phi \to \psi$
  \end{itemize}

  \subsection{Types}

  Let $\Phi(\to)$ denote the implicative fragment of intuitionistic
  $\lambda$-calculus. Call them \textit{simple types}.

  Let $\Gamma = \set{x_{1}:\tau_{1}, \dots, x_{n}:\tau_{n}}$,
  where $x_{i}$ are variables and $\tau_{i}$ are simple types, be our
  context. Let $\range(\Gamma)$ be the set of all types.

  Now, let $M$ be a $\lambda$-term. If $M$ is of type $\tau$, we write
  $\Gamma\vdash M: \tau$.

  If we know from the context that $x$ has a type $\tau$, we
  write $\Gamma, x: \tau \vdash x:\tau$. Denote this rule as \textit{Var}.

  We also introduce the \textit{Abs} rule:

  \begin{itemize}
  \item $\Gamma, x:\tau \vdash M:\sigma$
  \item $\Gamma\vdash (\lambda x.M):\tau \to \sigma$
  \end{itemize}

  Similarly, we give a rule for application:

  \begin{itemize}
  \item $\Gamma \vdash M:\tau \to \sigma$, $\Gamma \vdash N:\tau$
  \item $\Gamma \vdash (MN):\sigma$
  \end{itemize}

  \begin{example}

    Given the context $t:\tau_{1}, s:\tau_{2}$, we write
    $t:\tau_{1}, s:\tau_{2} \vdash (\lambda x.t)s:\tau_{1}$.

  \end{example}

  \begin{example}

    We can also write the following deduction:

    \begin{itemize}
    \item $t:\tau_{1}, s:\tau_{2} \vdash t:\tau_{1}$
    \item
      $t:\tau_{1}, s:\tau_{2} \vdash (\lambda x.t):\tau_{2}\to
      \tau_{1}$ , $t:\tau_{1}, s:\tau_{2} \vdash s:\tau_{2}$ 
    \item $t:\tau_{1}, s:\tau_{2} \vdash (\lambda x. t)s:\tau_{1}$
    \end{itemize}

  \end{example}

  Simple typed $\lambda$-calculus is denoted as $\lambda_{\to}$. In
  our case, we have constructed $\lambda_{\to}$ in the interpretation
  of Curry.

  A similar construction $\lambda_{\to}$ was developed by Church.

  Now, consider a spreading combinator
  $S = \lambda x. \lambda y. \lambda z. xz(yz)$. Having made it typed,
  we can write it as
  
  \begin{equation*}
    \lambda x: p \to q \to r. \lambda y: p\to q. \lambda z: p.
    xz(yz):(p\to q \to r) \to (p\to q) \to p \to r.
  \end{equation*}

  If some $\lambda$-term $M$ is typed and $\beta$-convertible, then
  the type is conserved.

  \subsection{Curry-Howard Correspondence}

  \begin{theorem}
    \hfill
    \begin{itemize}
    \item If $\Gamma\vdash M:\phi:\phi$ in $\lambda_{\to}$, then
      $\Gamma\vdash \phi$ in the implicative fragment of the
      intuitionistic $\lambda$-calculus.
    \item If $\Delta \vdash \phi$, then there exists a term and a
      context $\Gamma$ such that
      $\range(\Gamma) = \Delta:F\vdash M:\phi$.
    
    \end{itemize}
  \end{theorem}

  \subsection{Extensions}

  For a pair of $\lambda$-terms, we can define another $\lambda$-term
  with a combined type:

  \begin{equation*}
    \< M, N \> : \sigma \times \tau
  \end{equation*}

  Thus, $\pi_{1}(\< M, N \>) \to_{\beta} M$ and
  $\pi_{2}(\<M, N\>) \to_{\beta} N$. We can write:

  \begin{itemize}
  \item $\Gamma \vdash M:\sigma$, $\Gamma \vdash N:\tau$
  \item $\Gamma \vdash \<M, N \>: \sigma \times \tau$
  \end{itemize}

  Moreover,

  \begin{itemize}
  \item $\Gamma \vdash M: \sigma \times \tau$
  \item $\Gamma \vdash \pi_{1}(M):\sigma$.
  \end{itemize}

  How can we assign a type to a disjunction?

  We can define constructors:

  \begin{align}
    \iota_{1}^{\sigma \Or \tau}(M): \sigma \Or \tau\\
    \iota_{2}^{\sigma \Or \tau}(N): \sigma \Or \tau
  \end{align}

  To decide what to do with a $\lambda$-term, we can define a decision
  rule in the form
  
  \begin{equation*}
    \text{case}\ M\ \text{of}\ [x]P\ \text{or}\ [y]Q
  \end{equation*}

  Thus, for example, for the disjunctive $\lambda$-term we can write

  \begin{equation*}
    \text{case}\ \iota_{1}^{\sigma\Or \tau}(M)\ \text{of} \ [x]P\
    \text{or}\ [y]Q \to_{\beta} [M/x]P 
  \end{equation*}

  For further information, see \textit{Lectures on Curry-Howard
    Isomorphism}.
  

\end{document}
