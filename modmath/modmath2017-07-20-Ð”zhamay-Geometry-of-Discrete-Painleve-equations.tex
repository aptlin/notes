% -*- coding: utf-8; -*-
%%% Local Variables:
%%% mode: latex
%%% TeX-engine: xetex
%%% TeX-master: t
%%% End:
\documentclass[11pt]{scrartcl}
\usepackage[fancy, beaue, pset, anon]{masty}
\pSet{\nt{Dzhamay}{1}{Geometry of Discrete Painleve Equations}}
  \usepackage{lineno}
  % ----------------------------------------------------------------------
  % Page setup
  % ----------------------------------------------------------------------

  \pagenumbering{gobble}

  % ----------------------------------------------------------------------
  % Custom commands
  % ----------------------------------------------------------------------

  % alignment

  \newcommand*{\LongestHence}{$\Rightarrow$}% function name
  \newcommand*{\LongestName}{$f_o(-x)+f_e(-x)$}% function name
  \newcommand*{\LongestValue}{$(-a)x +(-a)(-y)$}% function value
  \newcommand*{\LongestText}{\defi}%

  \newlength{\LargestHenceSize}%
  \newlength{\LargestNameSize}%
  \newlength{\LargestValueSize}%
  \newlength{\LargestTextSize}%

  \settowidth{\LargestHenceSize}{\LongestHence}%
  \settowidth{\LargestNameSize}{\LongestName}%
  \settowidth{\LargestValueSize}{\LongestValue}%
  \settowidth{\LargestTextSize}{\LongestText}%

  % Choose alignment of the various elements here: [r], [l] or [c]

  \newcommand*{\mbh}[1]{{\makebox[\LargestHenceSize][r]{\ensuremath{#1}}}}%
  \newcommand*{\mbn}[1]{{\makebox[\LargestNameSize][r]{\ensuremath{#1}}}}%
  \newcommand*{\mbv}[1]{\ensuremath{\makebox[\LargestValueSize][r]{\ensuremath{#1}}}}%
  \newcommand*{\mbt}[1]{\makebox[\LargestTextSize][l]{#1}}%

  \newcommand{\R}[1]{\label{#1}\linelabel{#1}}
  \newcommand{\lr}[1]{line~\lineref{#1}}

  % ----------------------------------------------------------------------
  % Launch!
  % ----------------------------------------------------------------------

  \begin{document}

  \section{Geometry of Discrete Painleve Equations}

  \subsection{Discrete Integrable Systems}

  Suppose there is a parameter $t:\ZZ \to X$, where $X$ is the
  configuration space. We will denote the $n$th step $x_{n}$ as $x$,
  and let $x_{n-1}$ be denoted as $\ul{x}$ and $x_{n+1}$ as $\ol{x}$.

  We will look at the mapping from $\mathbf{x}$ to $\mathbf{\ol{x}}$, from $(x, y)$ to $(\ol{x}, \ol{y})$.

  Moreover, we will require that $\ol{x} = \frac{p(x, y)}{q(x,y)}$. We
  will require that a similar condition holds for $\ol{y}$. This kind
  of relation is described as a \textit{birational mapping}.

  \subsection{QRT}
  
  Now, let's consider a biquadratic curve $\Gamma$, where $\Gamma$ is
  a set of zeros of some polynomial $p(x, y) \in \CC[x, y]$. Since
  this polynomial is bivariate, there are two characteristics of its
  degree, which can be described as $(\deg_{x} p, \deg_{x} p)$.

  Suppose that
  $p(x, y) = a_{00}x^{2}y^{2} + a_{10}xy^{2} + a_{20} y^{2} +
  a_{01}x^{2}y + \dots + a_{22}$.

  Note that this polynomial can be written in the matrix form%
  
  \begin{equation}
    p(x, y) = X^{T}AY = \vv{x^{2}; x; 1} \begin{pmatrix}
      a_{00} & a_{01} & \dots\\
      \vdots & & \vdots\\
      \dots & & a_{22}
    \end{pmatrix} \cv{y^{2}; y; 1}.
  \end{equation}

  Consider two involutions $r_{x}$ and $r_{y}$, so that
  $r_{x}^{2} = r_{y}^{2} = \id$.

  The transformation QRT is then $r_{x} \circ r_{y}$.

  It is worthwhile to note that $\Gamma$ is isomorphic (?) to a torus,
  which in turn is isomorphic to $\CC / \Lambda$ (cf. Hans Dustermaat
  2010).

  Moreover, we can explicitly state the form of $r_{x}$ and $r_{y}$
  by applying standard methods of solving a quadratic to $p(x, y)$.

  Now, let $A$ and $B$ be two complex $3\times 3$ matrices.

  We can study \textit{bundles}, families of curves defined by
  $\Gamma_{A}$ and $\Gamma_{B}$. Thus,
  $\lambda_{0}XAY + \lambda_{1}X^{T}BY = 0$.

  Suppose we choose a point in plane. Then there exists a curve which
  contains this point and cuts $\Gamma_{A}$ and $\Gamma_{B}$ at the
  points of intersections, called base points, so that we obtain
  $[\lambda_{0}: \lambda_{1}] = -[X_{*}^{T}BY_{*}:X_{*}^{T}AY_{*}]$.
  This mapping is QRT.

  Now, suppose that $r_{y}$ is such that it maps $(x_{*}, y_{*})$ to
  $(x_{*}, y_{*}')$. Then
  $y_{*}' = \frac{f_{1}(x_{*}) y_{*} -
    f_{0}(x_{*})}{f_{2}(x_{*})y_{*}-f_{1}(x_{*})}$, where
  $f_{0}, f_{1}$ and $f_{2}$ are such that
  $(X^{T}A\times X_{*}^{T}B) = \langle f_{0}(x_{*}), f_{1}(x_{*}),
    f_{2}(x_{*})\rangle$. Then $QRT = r_{x} \circ r_{y} = \phi \circ \phi$,
  where $\phi = \sigma \circ r_{y}$ and $\sigma$ maps $(x, y; A, B)$
  to $(y,x; A^{T}, B^{T})$.

  Considering the previous comments, we can now the form for $\phi$:

  \begin{equation}
    \ol{x}  = \frac{f_{1}(x)y-f_{0}(x)}{f_{2}(x)y - f_{1}(x)}\\
    \ol{y}= x
  \end{equation}

  \subsection{Technical Details}

  Consider a mapping $\CC\times \CC \to \PP_{\CC}'\times \PP_{\CC}'$,
  where $\PP_{\CC}'$ is the Riemann sphere. In this context we talk
  about the equivalence classes $\CC^{2} - \set{(0, 0)} / \sim$,
  where $(x_{0}, x_{1}) \tilde (\mu x_{0}, \mu x_{1})$, and
  $\mu \neq 0$.

  To treat the subject matter properly, we also need a concept of a
  \textit{divisor}.

  Let $X$ be an algebraic variety. Then
  $\Div(X) = \spn_{\ZZ}\set{\text{continuous algebraic varieties of
      codimension 1}}$.

  In our case, $X = \PP_{\CC}'$. Then we can evaluate
  $D = \sum_{i=1}^{n} a_{i} \set{p_{i}}$, where $a_{i} \in \ZZ$.

  Let's find a divisor of a rational function
  $f = 3 \frac{(x-1)^{2}(x+2)}{(x-i)(x+4)(x-5)}$. A divisor of $f$,
  $(f)$, is then $(f) = 2\set{1}+{-2} - \set{i} - \set(-4) - \set{5}$.

  Suppose now that $f(x) = x(x-1)$. Let $X = \frac{1}{x}$. Then
  $f(x) = \frac{1}{X}(\frac{1}{X} - 1)$. Notice that the degree of a
  divisor is 0. Indeed, it is zero always on the Riemann sphere.

  Let $X = \PP' \times \PP'$. If we look at its classes, we can define a Picard lattice as follows:
  
  \begin{equation}
    Pic(X) = Cl(X) = \spn{\CH_{x}, \CH_{y}}, 
  \end{equation}
  where $[H_{x}] = \CH_{x}$ and $[H_{y}] = \CH_{y}$.

  For example, the class $[p(x, y) = X^{T} A Y]$ is equivalent to $ 2\CH_{x} + 2\CH_{y}$.

  We can also look at the intersection forms. For example,
  $\CH_{x} \cdot \CH_{y} = 1$ and
  $\CH_{x} \cdot \CH_{x} = 0 = \CH_{y} \cdot \CH_{y}$.

  It is also worthwhile to note that $(2\CH_{x} + 2\CH_{y})^{2} = 8$.

  \subsection{Concrete Example}

  Suppose $A = \begin{pmatrix}
    0 & 0 & 0\\
    0 & 1 & 0\\
    0 & 0 & 0
  \end{pmatrix}$ and $B = \begin{pmatrix}
    1 & \alpha & 1 \\
    \alpha & 0 & - \alpha\\
    1 & -\alpha & 1
  \end{pmatrix}$.

  Then $\phi = \begin{cases}
    \ol{x} = \frac{(x-a)(x-a^{-1})}{y(x+a)(x+a^{-1})}\\
    \ol{y} = x
  \end{cases}$.

  Then for $A$ we can write that $x_{0}x_{1}y_{0}y_{1} = 0$, and for
  $B$ we have
  $x^{2}y^{2}+\alpha(x^{2}y + xy^{2})+ (x^{2}+y^{2}) - \alpha(x+y) + 1
  = 0$ , which can be rewritten as
  $(xy+a^{-1}(x+y) - 1)(xy+a(x+y) - 1) = 0$.

  We can also write $\phi^{-1}: \begin{cases}
    \ul{x} = y\\
    \ul{y} = \frac{(y-a)(y-a^{-1})}{x(y+a)(y+a^{-1})}.
  \end{cases}$

  

\end{document}
