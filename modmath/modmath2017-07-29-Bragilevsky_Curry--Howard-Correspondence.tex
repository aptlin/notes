% -*- coding: utf-8; -*-
%%% Local Variables:
%%% mode: latex
%%% TeX-engine: xetex
%%% TeX-master: t
%%% End:
\documentclass[11pt]{scrartcl}
\usepackage[fancy, beaue, pset, anon]{masty}
\pSet{\nt{Bragilevsky}{4}{Curry-Howard Correspondence}}

% ----------------------------------------------------------------------
% Page Setup
% ----------------------------------------------------------------------

\pagenumbering{gobble}

% ----------------------------------------------------------------------
% Custom Commands
% ----------------------------------------------------------------------

% ----------------------------------------------------------------------
% Launch!
% ----------------------------------------------------------------------

\begin{document}

\section{Curry-Howard Correspondence IV}

Curry-Howard has a much deeper meaning than just the correspondence
between two logical systems.

Brouwer was mistaken in his belief that the model of classical logic
cannot be embedded into the constructive setting.

\subsection{$\lambda_{\mu}$-Calculus}

Let $V$ be a set of variables, let $A$ denote a set of all addresses,
and suppose tat $\Phi$ is a set of types (formulae). We also introduce
two operations: $\to$, $\bot$.

Let $x\in V$ and $a\in A$, and suppose a new context is defined as
$M \defeq x | (MM) | (\lambda x:\sigma.M) | (\mu a: !!\not \sigma. M) |
([a]M)$.

We can understand an address as a channel device through which
information can be transmitted. This operation is akin to exceptional
procedures in programming languages.

Now, we introduce two rules:

\begin{itemize}
\item $\Gamma, a: \not \sigma \vdash M: \bot$
\item $\Gamma \vdash (\mu a : \not \sigma. M):\sigma$
\end{itemize}

and

\begin{itemize}
\item $\Gamma, a : \not \sigma \vdash M:\sigma$
\item $\Gamma, a : \not \sigma \vdash [a]M: \not \sigma$
\end{itemize}

For example, the Pierce tautology (?) can be written as a $\lambda$-term as follows:

\begin{equation*}
\lambda x: (p\to q)\to p.\mu a : \not p.[a](x (\lambda z:p.\mu b: \not q. [a]z))
\end{equation*}

This is one of the examples how classical and intuitionistic logic can
be put into correspondence.

\subsection{$\lambda$-Cube}

We can draw a range of related logical systems, with the logical model
and model of computation represented as a pair, in a figure to obtain a cube.


\end{document}
