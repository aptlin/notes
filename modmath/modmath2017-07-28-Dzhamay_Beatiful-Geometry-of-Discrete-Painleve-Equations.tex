% -*- coding: utf-8; -*-
%%% Local Variables:
%%% mode: latex
%%% TeX-engine: xetex
%%% TeX-master: t
%%% End:
\documentclass[11pt]{scrartcl}
\usepackage[fancy, beaue, pset, anon]{masty}
\pSet{\nt{Dzhamay}{5}{Geometry of Discrete Painleve Equations}}
  \usepackage{lineno}
  % ----------------------------------------------------------------------
  % Page setup
  % ----------------------------------------------------------------------

  \pagenumbering{gobble}

  % ----------------------------------------------------------------------
  % Custom commands
  % ----------------------------------------------------------------------

  % alignment

  \newcommand*{\LongestHence}{$\Rightarrow$}% function name
  \newcommand*{\LongestName}{$f_o(-x)+f_e(-x)$}% function name
  \newcommand*{\LongestValue}{$(-a)x +(-a)(-y)$}% function value
  \newcommand*{\LongestText}{\defi}%

  \newlength{\LargestHenceSize}%
  \newlength{\LargestNameSize}%
  \newlength{\LargestValueSize}%
  \newlength{\LargestTextSize}%

  \settowidth{\LargestHenceSize}{\LongestHence}%
  \settowidth{\LargestNameSize}{\LongestName}%
  \settowidth{\LargestValueSize}{\LongestValue}%
  \settowidth{\LargestTextSize}{\LongestText}%

  % Choose alignment of the various elements here: [r], [l] or [c]

  \newcommand*{\mbh}[1]{{\makebox[\LargestHenceSize][r]{\ensuremath{#1}}}}%
  \newcommand*{\mbn}[1]{{\makebox[\LargestNameSize][r]{\ensuremath{#1}}}}%
  \newcommand*{\mbv}[1]{\ensuremath{\makebox[\LargestValueSize][r]{\ensuremath{#1}}}}%
  \newcommand*{\mbt}[1]{\makebox[\LargestTextSize][l]{#1}}%

  \newcommand{\R}[1]{\label{#1}\linelabel{#1}}
  \newcommand{\lr}[1]{line~\lineref{#1}}

  % ----------------------------------------------------------------------
  % Launch!
  % ----------------------------------------------------------------------

  \begin{document}

  \section{Geometry of Discrete Painleve Equations}

  \subsection{Introduction}

  \subsubsection{What is a Painleve equation?}

  The classic eqample is $\frac{\dif^{2}y }{\dif t^{2}} = 6y^{2} + 1$.

  The key property of Painleve equations is nonlinearity. Moreover,
  their general solution is free of movable critical points, which
  means that it is independent of any constants of integration.

  % The surprising fact is that Painleve .

  \subsubsection{Why should we study them?}

  The original motivation of Painleve et al was to find new nonlinear
  special functions. Examples of special functions, like the
  exponential or trigonometric functions arising from solutions of the
  first- and second-order differential equations, are well-known.

  There are six Painleve equations, which have parameters, and
  equations from different classes can be transformed into each other
  by degeneration of parameters.

  In turn, \textit{discrete} Painleve equations are second-order non-autonomous
  nonlinear recurrence relations.

  It is difficult to determine whether a given recurrence relation is
  a discrete Painleve equation. One of the best ways to approach the
  problem uses tools of algebraic geometry developed by Sakai.

  \begin{note*}
    There is a notion of entropy in discrete Painleve equations.
  \end{note*}

  Discrete Painleve equations have a much richer classification
  scheme. For example, each equation has a corresponding pair of orthogonal
  sublattices.

  \begin{note*}

    Discrete Painleve equations have applications in the study of the
    longest increasing subsequences in permutations. 

  \end{note*}

  Solutions of Painleve equations are called \textit{Painleve
    transcendents}, which are purely nonlinear special functions.

  We can also approach the study of Painleve equations through geometry,
  which was actively developed by Okamoto in 1970-1980s.

    
\end{document}
