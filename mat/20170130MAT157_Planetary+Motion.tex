% -*- coding: utf-8; -*-
%%% Local Variables:
%%% mode: latex
%%% TeX-engine: xetex
%%% TeX-master: t
%%% End:
\documentclass[11pt]{scrartcl}
\usepackage[fancy, beaue, pset, anon]{masty}
\pSet{\nt{MAT157}{X}{Planetary Motion}}
\usepackage{lineno}
% ----------------------------------------------------------------------
% Page setup
% ----------------------------------------------------------------------

\pagenumbering{gobble}

% ----------------------------------------------------------------------
% Custom commands
% ----------------------------------------------------------------------

% alignment

\newcommand*{\LongestHence}{$\Rightarrow$}% function name
\newcommand*{\LongestName}{$f_o(-x)+f_e(-x)$}% function name
\newcommand*{\LongestValue}{$(-a)x +(-a)(-y)$}% function value
\newcommand*{\LongestText}{\defi}%

\newlength{\LargestHenceSize}%
\newlength{\LargestNameSize}%
\newlength{\LargestValueSize}%
\newlength{\LargestTextSize}%

\settowidth{\LargestHenceSize}{\LongestHence}%
\settowidth{\LargestNameSize}{\LongestName}%
\settowidth{\LargestValueSize}{\LongestValue}%
\settowidth{\LargestTextSize}{\LongestText}%

% Choose alignment of the various elements here: [r], [l] or [c]

\newcommand*{\mbh}[1]{{\makebox[\LargestHenceSize][r]{\ensuremath{#1}}}}%
\newcommand*{\mbn}[1]{{\makebox[\LargestNameSize][r]{\ensuremath{#1}}}}%
\newcommand*{\mbv}[1]{\ensuremath{\makebox[\LargestValueSize][r]{\ensuremath{#1}}}}%
\newcommand*{\mbt}[1]{\makebox[\LargestTextSize][l]{#1}}%

\newcommand{\R}[1]{\label{#1}\linelabel{#1}}
\newcommand{\lr}[1]{line~\lineref{#1}}

% ----------------------------------------------------------------------
% Launch!
% ----------------------------------------------------------------------

\begin{document}

\section{Kepler's Laws and Planetary Motion}

Achievements in observing the planetary motion was brought by:
\begin{itemize}
\item Ptolemy
\item Tycho Brahe
\item Kepler
\end{itemize}

\subsection{Kepler's Laws}

\begin{enumerate}
\item\label{item:1} Planets move in ellipses (not quite true,
  cf. \textit{conic sections}), with the sun at a focus.
\item Equal areas are swept in equal times.
\item If $a$ is the length of the major axis of an ellipse and $T$ is
  the period of one revolution of a planet around the sun, then $\frac{a^3}{T^2} = \text{const}$.
\end{enumerate}

Kepler inferred these laws from empirical data. Newton developed a
\textit{theory of fluxes} and proved Kepler's laws formally.

Consider a two-body system with one of them acting as the sun (a fixed
point) and another acting as a planet and having a velocity which is a
sum of two linearly independent vectors, with one of the components
pointed to the sun, so that all the motion lies in a plane.

Introduce polar coordinates with the origin at the sun and the vector
$r(t)$ corresponding to the position of the orbiting planet at time $t$.

Let's write

\begin{equation*}
c(t) = r(t)(\cos(\theta(t)), \sin(\theta(t))).
\end{equation*}

Write $e(t) = (\cos(\theta(t)), \sin(\theta(t)))$ so that
$c(t) = r(t)e(t)$.

Consider a derivative of $c(t)$:

\begin{align}
c'(t) &= (r'(t)\cos(\theta(t))-r(t)\sin(\theta(t))\theta'(t), r'(t)\sin(\theta(t)) + r(t)\cos(\theta(t))\theta'(t)).
\end{align}

If $u = (a, b)$ and $v = (c, d)$, write $\det(u, v) = \det 
\begin{pmatrix}
  a & b \\
  c & d
\end{pmatrix}$.

Therefore,
\begin{align}
  \det(c, c') &= r^2(t)\theta'(t)\det 
  \begin{pmatrix}
    \cos \theta & \sin \theta\\
    -\sin \theta & \cos \theta
  \end{pmatrix}\\
  &= r^2(t) \theta'(t)
\end{align}

Note that the lower sum for the section area of any curve in polar
coordinates is

\[\sum_ir_i^2(\theta_i-\theta_{i-1}),\]

and the upper sum is $\sum_iR_i^2(\theta_i-\theta_{i-1})$, and thus
$\half \int_{\theta_1}^{\theta_2} r^2(\theta(t))\theta'(t)\dif t$.

By the Fundamental Theorem of Calculus,

$A'(t) = \frac{1}{2}r^2(t)\theta'(t) = \half \det(c, c')$.

Note that Kepler's Second Law amounts to $A'' = 0$, i.e. $A'$ is a constant.

\begin{align}
  A'' &= \half  \frac{\dif \det(c, c')}{\dif t}\\
      &= \half(\det(c',c') + \det(c, c''))\\
      &= \half \det(c, c'')
\end{align}

Observe that $A'' = 0$ if and only if $c''$ as a multiple of $c$, i.e. the acceleration is always $c$ and the corresponding force is along $c$, which makes it a central force.

\begin{theorem}[Kepler's Second Law]
Equal areas are swept out in equal times if and only if the force is central.
\end{theorem}

Newton hypothesized that gravity is an ``inverse-square'' force, with the strength proportional to $\frac{1}{r^2}$.

This means that 
\begin{equation*}
c''(t) = -\frac{K}{r^2(t)}e(t).
\end{equation*}

Since we know that $r^2\theta' = H$, where $H$ is some constant, so $\frac{c''(t)}{\theta'(t)} = - \frac{K}{H}e(t)$.

Note that $\frac{c''(t)}{\theta'(t)}$ is the derivative of $c'(\theta^{-1})(\theta(t))$.

Write $D(\theta) = c'(\theta^{-1}(\theta))$.

So $D' = - \frac{K}{H}e(\theta) = -\frac{K}{H}(\cos \theta, \sin \theta)$.

We can solve the following:

\begin{equation*}
D = - \frac{K}{H}(\sin \theta +A, -\cos \theta +B)
\end{equation*}

Putting this into $\det(c, c') = r^2\theta'(t)$ and dealing with the associated algebra and geometry, it is possible to show that

\begin{equation*}
r = L(1+\epsilon \cos\theta),
\end{equation*}

which is the equation of a conic with one focus at the origin and eccentricity of $\epsilon$:
\begin{itemize}
\item if $0<\epsilon< 1$, then the trajectory is an ellipse
\item if $\epsilon = 1$, then the trajectory is a parabola
\item if $\epsilon > 1$, then the trajectory is a hyperbola
\end{itemize}

A similar argument also shows that to get orbits that are conic
sections, a central force has to be an inverse-square force.











\end{document}