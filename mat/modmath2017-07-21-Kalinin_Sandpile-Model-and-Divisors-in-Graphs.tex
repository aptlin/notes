% -*- coding: utf-8; -*-
%%% Local Variables:
%%% mode: latex
%%% TeX-engine: xetex
%%% TeX-master: t
%%% End:
\documentclass[11pt]{scrartcl}
\usepackage[fancy, beaue, pset, anon]{masty}
\pSet{\nt{Kalinin}{1}{Sandpile Model and Divisors in Graphs}}
  \usepackage{lineno}
  % ----------------------------------------------------------------------
  % Page setup
  % ----------------------------------------------------------------------

  \pagenumbering{gobble}

  % ----------------------------------------------------------------------
  % Custom commands
  % ----------------------------------------------------------------------

  % alignment

  \newcommand*{\LongestHence}{$\Rightarrow$}% function name
  \newcommand*{\LongestName}{$f_o(-x)+f_e(-x)$}% function name
  \newcommand*{\LongestValue}{$(-a)x +(-a)(-y)$}% function value
  \newcommand*{\LongestText}{\defi}%

  \newlength{\LargestHenceSize}%
  \newlength{\LargestNameSize}%
  \newlength{\LargestValueSize}%
  \newlength{\LargestTextSize}%

  \settowidth{\LargestHenceSize}{\LongestHence}%
  \settowidth{\LargestNameSize}{\LongestName}%
  \settowidth{\LargestValueSize}{\LongestValue}%
  \settowidth{\LargestTextSize}{\LongestText}%

  % Choose alignment of the various elements here: [r], [l] or [c]

  \newcommand*{\mbh}[1]{{\makebox[\LargestHenceSize][r]{\ensuremath{#1}}}}%
  \newcommand*{\mbn}[1]{{\makebox[\LargestNameSize][r]{\ensuremath{#1}}}}%
  \newcommand*{\mbv}[1]{\ensuremath{\makebox[\LargestValueSize][r]{\ensuremath{#1}}}}%
  \newcommand*{\mbt}[1]{\makebox[\LargestTextSize][l]{#1}}%

  \newcommand{\R}[1]{\label{#1}\linelabel{#1}}
  \newcommand{\lr}[1]{line~\lineref{#1}}

  % ----------------------------------------------------------------------
  % Launch!
  % ----------------------------------------------------------------------

  \begin{document}

  \section{Sandpile Model and Divisors in Graphs}

  \subsection{Introduction}

  \begin{definition}
    Let $\Gamma$ be a finite connected subgraph.

    Define a \textbf{boundary} $\delta \Gamma$ as a set of vertices which
    neighbours do not belong to $\Gamma$.

    Define a \textbf{state} as $\phi: \Gamma \to \ZZ_{\geq 0}$ over a set of
    vertices.

    If $v\in \Gamma \setminus \delta \Gamma, \phi(v) \geq 4$, then we can
    make a \textbf{toppling} $\phi \to \phi'$ such that:
    \begin{align}
      &\phi'(v) = \phi(v) - 4\\
      &\phi'(w) = \phi(w) + 1, \text{ if } W \sim V, \text{ where $\sim$ means that $W$ is a neighbouring subgraph}\\
      & \phi'(w) = \phi(w) \text{ otherwise}
    \end{align}

    \textbf{Relaxation} of $\phi$ is a sequence of topplings while they are possible.
  \end{definition}

  \begin{exercise}

    Relaxation always ends.

  \end{exercise}  


  \begin{exercise}

    Order of topplings does not matter, so that the relaxation is
    unique.

  \end{exercise}

  Denote the result of the relaxation as $\phi^{0}$.

  \begin{definition}
    The function of topplings is a function
    $F: \Gamma  \to \ZZ_{\geq 0}$ mapping $\phi$
    to $\phi^{0}$.

    Let $F(v)$ denote the number of topplings in $v$.
  \end{definition}

  \begin{exercise}

    Prove that $F$ is well-defined.

  \end{exercise}

  \begin{exercise}

    $\phi^{0} = \phi + \Delta F$.

  \end{exercise}

  \begin{definition}
    $(\Delta F)(i, j) = F(i+1, j) + F(i-1, j) + F(i, j-1) + F(i,
    j+1) - 4 F(i, j)$.
  \end{definition}

  \begin{exercise}

    The toppling function $F$ is a dot-wise minimal function among
    functions $G: \Gamma \to \ZZ_{\geq 0}$ such that
    $\phi + \Delta G \leq 3$.

  \end{exercise}

  \begin{note*}
    A dot-wise minimal function $F$ is such that
    $F(v) = \min(G(v) \text{ for all $G$})$ and
    $\phi + \Delta G \leq 3$ is satisfied.
  \end{note*}

  \begin{definition}
    Let $\<k\>$ denote the state in which each node has $k$ sand grains.
    Denote the maximally stable state as $\< 3\>$.
  \end{definition}

  \begin{definition}
    The state $\phi$ is \textbf{revertible} if there exists $\psi \geq 0$ such
    that $\phi = (\< 3 \> + \psi)^{0}$, where $+$ is defined dot-wise.
  \end{definition}

  \begin{definition}
    $\phi \oplus \psi = (\phi + \psi)^{0}$.
  \end{definition}
  \begin{theorem}
    Revertible states with the operation $\oplus$ form a group, so
    that $(\phi \oplus \psi) + \beta = \phi \oplus (\psi + \beta)$,
    $\psi + \phi = \phi + \psi$, there exists $\phi \oplus e = \phi$,
    and there exist inverse elements.
  \end{theorem}

  \begin{note*}
    $\<0\>$ is not the unit element in the sand group.
  \end{note*}

  \subsection{Model on a Random Graph}

  Suppose that $\Gamma$ is a random graph, where $I$ is a special vertex of $\Gamma$ called a \textit{dump}.

  The dump is $\delta \Gamma$ pulled into one node.

  Suppose that $\Gamma$ is in relaxation, and there are no topplings in a dump.

  Let's measure the frequency of change over the area of the sand
  avalanche. The graph would be linear and declining with the growing
  area. This phenomenon is called \textit{self-organised criticality}.

  \subsection{Forbidden Configurations}

  \begin{example}

    In a revertible state there cannot be a graph with two states
    without any topplings, which can be seen as follows.

    Look at the node of the last toppling, which can be either in the
    vertex 1 or vertex 2 to obtain a contradiction.

  \end{example}

  \begin{definition}
    Suppose that $D \su \Gamma$. The state $\phi$ on $D$ i called
    forbidden, if $\phi(v)$ is less than the number of neighbours of $v \in D$.
  \end{definition}

  \begin{exercise}

    Prove that a revertible state does not contain forbidden
    configurations.

  \end{exercise}

  \begin{exercise}

    The unit in the sand group is equal to
    $(\< 8 \> - \<8\>^{0})^{0}$. 

  \end{exercise}

  To prove the theorem, take a state $\phi$, add itself and relax.
  Repeat the procedure for all the nodes. This procedure will
  eventually cycle over some nodes. Now, take another state $\psi$,
  and add it to one of the cycle nodes. How can you proceed?

\end{document}
