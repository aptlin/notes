% -*- coding: utf-8; -*-
%%% Local Variables:
%%% mode: latex
%%% TeX-engine: xetex
%%% TeX-master: t
%%% End:
\documentclass[11pt]{scrartcl}
\usepackage[fancy, beaue, pset, anon]{masty}
\pSet{\nt{MAT157}{}{Convergence of Functions}}
\usepackage{lineno}
% ----------------------------------------------------------------------
% Page setup
% ----------------------------------------------------------------------

\pagenumbering{gobble}

% ----------------------------------------------------------------------
% Custom commands
% ----------------------------------------------------------------------

% alignment

\newcommand*{\LongestHence}{$\Rightarrow$}% function name
\newcommand*{\LongestName}{$f_o(-x)+f_e(-x)$}% function name
\newcommand*{\LongestValue}{$(-a)x +(-a)(-y)$}% function value
\newcommand*{\LongestText}{\defi}%

\newlength{\LargestHenceSize}%
\newlength{\LargestNameSize}%
\newlength{\LargestValueSize}%
\newlength{\LargestTextSize}%

\settowidth{\LargestHenceSize}{\LongestHence}%
\settowidth{\LargestNameSize}{\LongestName}%
\settowidth{\LargestValueSize}{\LongestValue}%
\settowidth{\LargestTextSize}{\LongestText}%

% Choose alignment of the various elements here: [r], [l] or [c]

\newcommand*{\mbh}[1]{{\makebox[\LargestHenceSize][r]{\ensuremath{#1}}}}%
\newcommand*{\mbn}[1]{{\makebox[\LargestNameSize][r]{\ensuremath{#1}}}}%
\newcommand*{\mbv}[1]{\ensuremath{\makebox[\LargestValueSize][r]{\ensuremath{#1}}}}%
\newcommand*{\mbt}[1]{\makebox[\LargestTextSize][l]{#1}}%

\newcommand{\R}[1]{\label{#1}\linelabel{#1}}
\newcommand{\lr}[1]{line~\lineref{#1}}

% ----------------------------------------------------------------------
% Launch!
% ----------------------------------------------------------------------

\begin{document}

\section{Convergence of Functions}


Consider $\sum_{n=1}^{\infty}a_n(x-a)^n$.

Suppose the series converges at $x=x_0$.

The terms $a_n(x_0-a)^n$ must go to 0.

So there exists $M\in\RR$ such that $\abs{a_n(x_0-a)^n}< M$ for all $n\in\NN$.

Consider a pont $r$ such that $\abs{r-a}< \abs{x_0-a}$:

\begin{align}
\sum_{i=1}^{\infty}a_n(r-a)^n = \sum_{i=1}^{\infty}a_n(x_0-a)^n\* (\frac{r-a}{x_0-a})^n.
\end{align}

Then $\abs{a_n(r-a)^n} = \abs{a_n(x_0-a)^n(\frac{r-a}{x_0-a})^n} \leq M \abs{\frac{r-a}{x_0-a}}^n$.

Thus, $\sum_{i=1}^{\infty}a_n(r-a)^n$ converges absolutely.

Now consider the series as a function:
\begin{equation*}
f_{n}(x) = \sum_{n=0}^na_n(x-a)^n.
\end{equation*}

As $n$ goes to infinity, what is its derivative?

Note that $f_n'(x) = \sum_{n=0}^{\infty}na_n(x-a)^{n-1}$.

Then the ratio test says that $\frac{f_{n+1}'(x)}{f_n'(x)} = \frac{n+1}{n}\frac{a_{n+1}}{a_n}(x-a)$.

Suppose $f_n(x) \to f(x)$ for all $x\in [a, b]$. Suppose that $f_n(x)$
is continuous for all $n\in\NN$. 

On $[0, 1]$, as $n\to \infty$, $x^n$ tends to zero for $x<1$ and to one for $x=1$.

\begin{example}

  Let $f_n(x) $ be such that $f_n(x) = \begin{cases}
    n\leq x\leq n+1\\
    \text{ otherwise}
  \end{cases}$.

Suppose now that $\lim_{n\to\infty}g_n(x) = 0$ for all $x\in\RR$ such that

\begin{equation*}
  \frac{1}{2}=\lim_{n\to\infty}\int_0^1 g_n(x) \dif x +\int_{0}^1\lim_{n=1}^{\infty}g_{n} = 0
\end{equation*}
\end{example}
\begin{definition}
  A sequence of functions $f_n$ \textit{converges uniformly} to $f(x)$ if, given $\epsilon > 0$, there exists $N\in\RR$ such that $\abs{f_n(x) - f(x)} < \epsilon$ for any $n > N$ and for any $x\in D(f)$.
\end{definition}
\begin{remark}
We can also define postive convergence, if $\lim_{n\to\infty}f_n(x) = f(x)$ for all $x\in D(f)$.
\end{remark}

\begin{theorem}

Suppose that $f_n$ and $f$ are integrable.

If $f_n \to f$ uniformly on $[a, b]$, then $\lim_{n\to\infty}\int_a^bf_n(x) \dif x = \int_a^bf(x) \dif x = \int _a^b\lim_{n\to\infty}f_n(x) \dif x$. 
\end{theorem}

\begin{proof}
  \hfill

  Given $\epsilon > 0$, we can find $N$ such that $\abs{f_n(x) - f(x) < \frac{\epsilon}{b-a}}$ for all $x\in [a, b]$ and for all $n > N$.

  Then $\abs{\int_a^bf_n(x) \dif x - \int_a^bf(x)\dif x} \leq \int_a^b \abs{f_n(x) - f(x)} \dif x < \frac{\epsilon}{b-a}\int_a^b \dif x = \epsilon$.
\end{proof}


\end{document}
