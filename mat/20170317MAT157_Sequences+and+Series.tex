% -*- coding: utf-8; -*-
%%% Local Variables:
%%% mode: latex
%%% TeX-engine: xetex
%%% TeX-master: t
%%% End:
\documentclass[11pt]{scrartcl}
\usepackage[fancy, beaue, pset, anon]{masty}
\pSet{\nt{MAT157}{}{Sequences and Series}}
\usepackage{lineno}
% ----------------------------------------------------------------------
% Page setup
% ----------------------------------------------------------------------

\pagenumbering{gobble}

% ----------------------------------------------------------------------
% Custom commands
% ----------------------------------------------------------------------

% alignment

\newcommand*{\LongestHence}{$\Rightarrow$}% function name
\newcommand*{\LongestName}{$f_o(-x)+f_e(-x)$}% function name
\newcommand*{\LongestValue}{$(-a)x +(-a)(-y)$}% function value
\newcommand*{\LongestText}{\defi}%

\newlength{\LargestHenceSize}%
\newlength{\LargestNameSize}%
\newlength{\LargestValueSize}%
\newlength{\LargestTextSize}%

\settowidth{\LargestHenceSize}{\LongestHence}%
\settowidth{\LargestNameSize}{\LongestName}%
\settowidth{\LargestValueSize}{\LongestValue}%
\settowidth{\LargestTextSize}{\LongestText}%

% Choose alignment of the various elements here: [r], [l] or [c]

\newcommand*{\mbh}[1]{{\makebox[\LargestHenceSize][r]{\ensuremath{#1}}}}%
\newcommand*{\mbn}[1]{{\makebox[\LargestNameSize][r]{\ensuremath{#1}}}}%
\newcommand*{\mbv}[1]{\ensuremath{\makebox[\LargestValueSize][r]{\ensuremath{#1}}}}%
\newcommand*{\mbt}[1]{\makebox[\LargestTextSize][l]{#1}}%

\newcommand{\R}[1]{\label{#1}\linelabel{#1}}
\newcommand{\lr}[1]{line~\lineref{#1}}

% ----------------------------------------------------------------------
% Launch!
% ----------------------------------------------------------------------

\begin{document}
\section{Sequences}

Consider a function $f(x) = e^{-x}\cos(2x+1)$.

We can use the Squeeze Theorem to show that $\lim_{x\to \infty}f(x) = 0$.

Therefore, $\lim_{n\to \infty}a_n = 0$.

\begin{theorem}
  If $a_n = f(n)$, for some function $f(x)$ and
  $lim_{x\to \infty} f(x) = L$, then $\lim_{n\to \infty} a_n = L$.
\end{theorem}

\begin{example}

  Consider $a_{n} = a^n$ for some real constant $a$.

  If $a > 0$, define $f(x) = a^x = e^{x\log a}$.

  Since $\lim_{x\to \infty} e ^{x\log a} = \exp(\lim_{x\to \infty} x \log a) =
  \begin{cases}
    0, \text{ if } \log a < 0\\
    \infty, \text{ if } \log a > 0
  \end{cases}$, then \[\lim _{n\to \infty}a_{n} =
  \begin{cases}
    0, \text{ if } 0 < a \leq 1\\
    \infty, \text{ if } a  > 1
  \end{cases}\].

If $a <0$,then $a_{n} = (-1)^n \abs{a}^n$, and hence $0$ if
$\abs{a}< 1$. If $\abs{a}> 1$ or $a = -1$, then the sequence diverges.

\end{example}

\begin{example}

  Suppose that $f(x) = \sin(\pi x)$. Then $a_n = f(n) = 0$, and thus
  $\lim_{x\to \infty} a_n = 0$, but $\lim_{x\to \infty}\sin x $ does
  not exists.

\end{example}

A sequence is said to be bounded if $\abs{a_n} < M$ and bounded above if $a_n< M$.

A sequence is said to be strictly increasing if $a_n < a_{n+1}$ and nondecreasing if $a_n\leq a_{n+1}$.

\begin{theorem}
  If $\set{a_{n}}$ is bounded and nondecreasing, then $\set{a_n}$ converges.
\end{theorem}

\begin{proof}
  \hfill

  Let $L  = \sup a_n$.

  Suppose some $\epsilon > 0$ is given.

  Therefore, there exists $M$ such that $L-a_M< \epsilon$ .

  But for any $n \geq M$, we have that $a_n\geq M$. So
  $L - a_n\leq L-a_m <\epsilon$, which means that
  $\lim_{n\to \infty}a_n = L$.
\end{proof}

Suppose now that $\lim_{x\to\alpha}f(x) = L$.

If $\set{a_n}$ is a sequence, then the domain of $f(x)$ so that
$a_n\neq \alpha$ for any $n\in\RR$ and
$\lim_{x\to \infty}a_n = \alpha$, then $\lim_{n\to\alpha} f(a_n) = L$.

More is true. If $\lim_{n\to\infty}f(a_n) = L$ for all sequences of the stated type, then $\lim_{x\to \alpha}f(x) = L$.

Note that we can always find a nondecreasing or a nonincreasing subsequence.

\begin{theorem}
A bounded sequence always has a convergent subsequence.
\end{theorem}

\begin{proof}
  \hfill

Pick a nondecrueasing or nonincreasing subsequence. Since it is bounded, we know that it has a limit.
\end{proof}

\section{Series}

In our current framework, $\sum_{i=1}^{\infty}$ does not make sense.

We define the $n$th partial sumas $S_n = \sum_{i=1}^na_n$.

If the sequence $\set{s_n}$ converges, say to $S$, then we say that
$\sum_{n=1}^{\infty} a_n $ converges and
$\sum_{n=1}^{\infty} a_n = S$.

\subsection{Geometric Series}

Remember that $\sum_{n=0}^{\infty}r^n= \frac{1-r^{n+1}}{1-r}$.

Do these partial sums tend to a limit?

If $\abs{r}< 1$, then $r^{n+1} \to 0$.

So $\sum_{n=0}^{\infty}r^n = \frac{1}{1-r}$ converges.

When $r = 1$ and $r=-1$, it diverges.


\end{document}
