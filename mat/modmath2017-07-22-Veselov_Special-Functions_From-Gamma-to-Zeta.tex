% -*- coding: utf-8; -*-
%%% Local Variables:
%%% mode: latex
%%% TeX-engine: xetex
%%% TeX-master: t
%%% End:
\documentclass[11pt]{scrartcl}
\usepackage[fancy, beaue, pset, anon]{masty}
\pSet{\nt{Veselov}{2}{Special Functions: From Gamma to Zeta II}}
  \usepackage{lineno}
  % ----------------------------------------------------------------------
  % Page setup
  % ----------------------------------------------------------------------

  \pagenumbering{gobble}

  % ----------------------------------------------------------------------
  % Custom commands
  % ----------------------------------------------------------------------

  % alignment

  \newcommand*{\LongestHence}{$\Rightarrow$}% function name
  \newcommand*{\LongestName}{$f_o(-x)+f_e(-x)$}% function name
  \newcommand*{\LongestValue}{$(-a)x +(-a)(-y)$}% function value
  \newcommand*{\LongestText}{\defi}%

  \newlength{\LargestHenceSize}%
  \newlength{\LargestNameSize}%
  \newlength{\LargestValueSize}%
  \newlength{\LargestTextSize}%

  \settowidth{\LargestHenceSize}{\LongestHence}%
  \settowidth{\LargestNameSize}{\LongestName}%
  \settowidth{\LargestValueSize}{\LongestValue}%
  \settowidth{\LargestTextSize}{\LongestText}%

  % Choose alignment of the various elements here: [r], [l] or [c]

  \newcommand*{\mbh}[1]{{\makebox[\LargestHenceSize][r]{\ensuremath{#1}}}}%
  \newcommand*{\mbn}[1]{{\makebox[\LargestNameSize][r]{\ensuremath{#1}}}}%
  \newcommand*{\mbv}[1]{\ensuremath{\makebox[\LargestValueSize][r]{\ensuremath{#1}}}}%
  \newcommand*{\mbt}[1]{\makebox[\LargestTextSize][l]{#1}}%

  \newcommand{\R}[1]{\label{#1}\linelabel{#1}}
  \newcommand{\lr}[1]{line~\lineref{#1}}

  % ----------------------------------------------------------------------
  % Launch!
  % ----------------------------------------------------------------------

  \begin{document}

  \section*{Special Functions: From Gamma to Zeta II}

  Consider an equation $x^{2} + y^{2} + z^{2} = 3xyz$, where $x, y, z \in \NN$.

  Solutions to this equation are called \textit{Markov triples}.

  $(1, 1, 1), (1, 1, 2)$ and $(1, 2, 5)$ are one of the solutions.

  Define an involution $\tau: (x, y, z) \to (x, y, 3y - z)$ mapping a
  Markov triple to a Markov triple. Why?

  Note the following:

  \begin{align}
    z^{2} - 3xyz + x^{2}+y^{2} &= 0\\
    z_{1}+z_{2} &= 3xy\\
    z_{2} &= 3xy - z_{1}
  \end{align}

  \begin{theorem}[Markov]
    All Markov triples are obtained from $(1, 1, 1)$ by application of
    $\tau$ and permutations.
  \end{theorem}

  For example,
  $(1, 1, 1) \overset{\tau}{\to} (1, 1, 2) \to (1, 2, 1)
  \overset{\tau}{\to} (1, 2, 5)$.

  Note that this procedure can be represented as a tree. This tree is
  spacial, with the numbers written in the complement of the tree,
  rather than the nodes themselves.

  The building unit of a tree consists of $z$ between two branches,
  connected to another two branches. Over the edge connecting two
  branches there is $x$, and $y$ is under the branch, while the other
  branch encloses $3xy - z$.

  It is easy to spot Fibonacci numbers in the resulting tree. In fact,
  every second number is Fibonacci, which is left as an exercise.

  \subsection{Tropicalisation}

  Tropicalisation is a transformation replacing each operator in the
  expression according to a predefined rules: $+$ is replaced with
  $\max$, $\times$ with $+$ and $/$ with $-$.

  Tropicalised equation for Markov triples is thus
  $\max(2a, 2b, 2c) = a+b+c$. Without loss of generality, suppose that
  $c$ is maximal, and thus we obtain that $a+b = c$.

  What kind of a tree do we obtain from a tropicalised Markov equation?

  The surprising fact is that in this case the step-by-step results of
  the Euclidean algorithm are represented by the tree, because the
  building unit consists of three branches connected to one note, with
  $a$, $b$ and $a+b$ in its complement.

  \subsection{Quantisation}

  The modern approach applies the same principles to matrices.

  For example, take the rule $AB = C$ for 2-by-2 matrices $A$ and $B$,
  with the inital matrices $\begin{pmatrix}
    3 & 4\\
    2 & 3
  \end{pmatrix}$ and $\begin{pmatrix}
    1 & 1\\
    1 & 2
  \end{pmatrix}$, which generate the commutator of $\SL_{2}(\ZZ)$.

  In this case, the Cohn tree (1953) is obtained.

  The Markov numbers less than 1000, which constitute Markov triples,
  are 1, 2, 5, 13, 29, 34, 89, 169, 194, 233, 433, 610, 985. Zognev
  has shown that $m_{n} \sim \frac{1}{3}e^{c \sqrt{n}}$.

  \begin{description}

  \item[Unicity Conjecture (Frobenius, 1913)]
    Each Markov triple is uniquely defined by its maximal element.
  \end{description}

  \subsection{Wonders of Markov Triples}
  Markov has shown that the set of Markov triples is equivalent to the
  set of the most irrational numbers up to an equivalence class. In
  other words,
  
  \begin{equation*}
    (x, y, z) \lra d = \frac{b}{x} + \frac{y}{xz} - \frac{3}{2} + \frac{\sqrt{9z^{2}-4}}{2z}, 
  \end{equation*}
  where $by - ax = z$.

  For example, if we consider $(1, 1, 1)$, we obtain $b - a = 1$,
  $b = 1$, $a=0$, and thus $d = \frac{1+\sqrt{5}}{2}$.

  For $(1, 1, 2)$, we get $d = 1 + \sqrt{2}$, which is called the
  silver ratio.

  For $(1, 2, 5)$, we obtain $\frac{9+\sqrt{221}}{10}$

  How can we measure how irrational a number is?

  One of the ways requires the notion of the Markov constant.

  Define
  $\mu(\alpha) = \inf \set{c: \abs{\alpha - \frac{p}{q}} <
    \frac{c}{q^{2}} \text{ for infinitely many $\frac{p}{q}$}}$.

  \subsubsection{Markov Spectrum}

  Note that $\mu(\alpha) > \frac{1}{3}$ is equivalent to
  $\mu(\alpha) = \frac{m}{\sqrt{9m^{2}-4}}$, where $m\in M$.

  Markov managed to study the spectrum successfully for some
  $\mu > \frac{1}{3}$. The spectrum was also classified for $\mu$ from
  0 to Freiman's constant. What happens between Freiman's constant and
  $\frac{1}{3}$ is not yet known.

  \subsubsection{Second Wonder}
  Gorshkov (1953) and Cohn (1955) studied a punctured
  $\ZZ_{3}$-symmetric torus $T^{2}_{*}$ with a hyperbolic metric, with
  a complete puncture. The result was that the lengths of simple
  closed geodesics is in the form $\frac{2}{3} \cosh m$, where
  $m\in M$.

  It was also established that real Markov's surface $x^{2} + y^{2} + z^{2} = xyz$, where
  $x, y, z >0$, is a Teichm\"uller space of punctured tori.
  

  \subsubsection{Third Wonder}

  Rudakov (1988) has established that the set of Markov triples can be
  put into the correspondence with the unique vector bundles on
  $\PP^{2}$.
  
  \subsubsection{Fourth Wonder}

  Kontsevich, Manin (1994) and Dubrovin(1996) have established that
  the generating function for $N_{d}$, which is the number of
  solutions of rational curves of degree $d$, containing $3d-1$ points
  on $\PP^{2}$, satisfies the Painleve (?) equation PVI with the Markov initial conditions.

  \subsubsection{Fifth Wonder}

  Prokhorov, Hacking (2010) have show that the set of Markov triples
  is equivalent to the toric degeneration of $\PP^{2}$, weighted
  projective plane $\PP^{2}(x, y, z)$, where $(x, y, z)$ is a Markov
  triple.
\end{document}
