% -*- coding: utf-8; -*-
%%% Local Variables:
%%% mode: latex
%%% TeX-engine: xetex
%%% TeX-master: t
%%% End:
\documentclass[11pt]{scrartcl}
\usepackage[fancy, beaue, pset, anon]{masty}
\pSet{\nt{MAT157}{}{Taylor Approximation}}
\usepackage{lineno}
% ----------------------------------------------------------------------
% Page setup
% ----------------------------------------------------------------------

\pagenumbering{gobble}

% ----------------------------------------------------------------------
% Custom commands
% ----------------------------------------------------------------------

% alignment

\newcommand*{\LongestHence}{$\Rightarrow$}% function name
\newcommand*{\LongestName}{$f_o(-x)+f_e(-x)$}% function name
\newcommand*{\LongestValue}{$(-a)x +(-a)(-y)$}% function value
\newcommand*{\LongestText}{\defi}%

\newlength{\LargestHenceSize}%
\newlength{\LargestNameSize}%
\newlength{\LargestValueSize}%
\newlength{\LargestTextSize}%

\settowidth{\LargestHenceSize}{\LongestHence}%
\settowidth{\LargestNameSize}{\LongestName}%
\settowidth{\LargestValueSize}{\LongestValue}%
\settowidth{\LargestTextSize}{\LongestText}%

% Choose alignment of the various elements here: [r], [l] or [c]

\newcommand*{\mbh}[1]{{\makebox[\LargestHenceSize][r]{\ensuremath{#1}}}}%
\newcommand*{\mbn}[1]{{\makebox[\LargestNameSize][r]{\ensuremath{#1}}}}%
\newcommand*{\mbv}[1]{\ensuremath{\makebox[\LargestValueSize][r]{\ensuremath{#1}}}}%
\newcommand*{\mbt}[1]{\makebox[\LargestTextSize][l]{#1}}%

\newcommand{\R}[1]{\label{#1}\linelabel{#1}}
\newcommand{\lr}[1]{line~\lineref{#1}}

% ----------------------------------------------------------------------
% Launch!
% ----------------------------------------------------------------------

\begin{document}

Suppose $f(x)$ is given, and suppose 
\begin{equation*}
  f(x) = f(a) +R.
\end{equation*}

Thus, $R = f(x) - f(a) = \int_a^xf'(t)\dif t$. 

Note that
\begin{equation*}
\int_a^x f'(t)\dif t = [f'(t)(t-x)]^x_a-\int_a^xf''(t)(t-x)\dif t =
0-f'(a)(a-x)-\int_a^xf''(t)(t-x)\dif t.
\end{equation*}

Therefore, for $f(x) = fax) + R_1$ we obtain that 
\[f(x)=f(a)+f'(x)(x-a) + \int_a^xf''(t)(x-t)\dif t.\]

Similarly, for $R_2= \int_a^xf''(t)(x-t)\dif t$, we get that
\begin{align}
  f(x) &= f(a) +f'(a)(x-a) +\frac{f''(a)}{2}(x-a)^2+\int_a^xf'''(t)\frac{(x-t)^2}{2}\dif t.
\end{align}

We therefore can prove by induction that
\begin{align}
  f(x) &= f(a) +f'(a)(x-a) +\cdots+f^{(n)}(a)\frac{(x-a)^{n}}{n!}+\int_a^xf^{n+1}(t)\frac{(x-t)^n}{n!}\dif t, 
\end{align}

and thus $f(x) = P_n(x) +R_n(x)$ for $R_n(x) = \int_a^xf^{(n+1)}(t) \frac{(x-t)^n}{n!} \dif t$.


If, for example, $f(x) = \sin x$, then for all $k\in\ZZ^+$ and for all
$x\in D(f)$, then $\abs{f^{(k)}(x)}\leq 1$, and hence
$\abs{R_n(x)}\leq \int_a^x \frac{(x-t)^n}{n!}\dif t =
\frac{(x-a)^{n+1}}{(n-1)!}$. In this way, if we are to compute the
approximation of $\sin(1)$ to within 0.001, then any $n\in\ZZ^+$ such
that $(n+1)! > 1000$ will suffice.

\subsection{Lagrange Form}

Assume that a bound for $f^{(n+1)}$ is known on the interval from
$a$ to $x$. Therefore, $m \leq f^{(n+1)}(t) \leq W$ for any $t$ between
$a$ and $x$, which means that 
\begin{align}
  \abs{R_n(x)}&\leq \int_a^xM \frac{(x-t)^n}{n!}\dif t\\
              &\leq M \frac{\abs{x-a}^{n+1}}{(n+1)!}.
\end{align}

If $f^{(n+1)}$ is continuous, then the IVT shows that there exists
$z\in\RR$ between $a$and $x$ such that
$R_n(x) = f^{(n+1)}(z) \frac{\abs{x-a}^{n+1}}{(n+1)!}$.

Another way to see it is by looking at $P_{n+1}(x)$.

This form of a remainder is called a \textbf{Lagrange form}.

Suppose now that that $f(x) = e^x$.

Then $R_n(1) = \frac{e^z}{(n+1)!}$ for some $z\in [0, 1]$.

Since $\log(4) >1$, we know that $e< 4$, and thus $R_n(1) < \frac{4}{(n+1)!}$. 

Therefore, $f(1) = 2 \nicefrac{2}{3}+R_3(1)$.

\end{document}