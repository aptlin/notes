% -*- coding: utf-8; -*-
%%% Local Variables:
%%% mode: latex
%%% TeX-engine: xetex
%%% TeX-master: t
%%% End:
\documentclass[11pt]{scrartcl}
\usepackage[fancy, beaue, pset, anon]{sdll}
\pSet{\nt{MAT157}{II.3}{Integrability Condition}}
\usepackage{lineno}
% ----------------------------------------------------------------------
% Page setup
% ----------------------------------------------------------------------

\pagenumbering{gobble}

% ----------------------------------------------------------------------
% Custom commands
% ----------------------------------------------------------------------

% alignment

\newcommand*{\LongestHence}{$\Rightarrow$}% function name
\newcommand*{\LongestName}{$f_o(-x)+f_e(-x)$}% function name
\newcommand*{\LongestValue}{$(-a)x +(-a)(-y)$}% function value
\newcommand*{\LongestText}{\defi}%

\newlength{\LargestHenceSize}%
\newlength{\LargestNameSize}%
\newlength{\LargestValueSize}%
\newlength{\LargestTextSize}%

\settowidth{\LargestHenceSize}{\LongestHence}%
\settowidth{\LargestNameSize}{\LongestName}%
\settowidth{\LargestValueSize}{\LongestValue}%
\settowidth{\LargestTextSize}{\LongestText}%

% Choose alignment of the various elements here: [r], [l] or [c]

\newcommand*{\mbh}[1]{{\makebox[\LargestHenceSize][r]{\ensuremath{#1}}}}%
\newcommand*{\mbn}[1]{{\makebox[\LargestNameSize][r]{\ensuremath{#1}}}}%
\newcommand*{\mbv}[1]{\ensuremath{\makebox[\LargestValueSize][r]{\ensuremath{#1}}}}%
\newcommand*{\mbt}[1]{\makebox[\LargestTextSize][l]{#1}}%

\newcommand{\R}[1]{\label{#1}\linelabel{#1}}
\newcommand{\lr}[1]{line~\lineref{#1}}

% ----------------------------------------------------------------------
% Launch!
% ----------------------------------------------------------------------

\begin{document}

\section{Integrability Condition}
\label{sec:alg}

\subsection{Review}
\label{subsec:rev}


\begin{theorem}
  If \(f\) is bounded on \([a, b]\), then \(f\) is also integrable on
  \([a,b]\) if and only if for all \( \epsilon > 0\) there exists a
  partition \(P\) of \([a, b]\) such that

  \begin{equation*}
    U(f, P) - L(f, P) < \epsilon
  \end{equation*}

\end{theorem}


\begin{proof}


  Assume that \(f\) is given such that \(f\) is bounded.

  Suppose the condition \(  U(f, P) - L(f, P) < \epsilon\) is true for any \(P\).

  Since
  \(L(f, P) \leq \sup \set{L(f, P')} \leq \inf\set{U(f, P')}\leq U(f,
  P)\), it follows that
  \(\inf\set{U(f, P')} - \sup\set{L(f, P')} < \epsilon\).

  Since this is true for all \(\epsilon>0\),
  \(\inf\set{U(f, P')} = \sup\set{L(f, P')}\). Thus, \(f\) is
  integrable.

  Conversely, suppose that \(f\) is integrable. Thus,
  \(\inf\set{U(f, P)} = \sup\set{L(f, P)}\) for any \(P\).

  Therefore, there exist partitions \(P', P''\) for any \(\epsilon>0\)
  such that \(\inf\set{U(f, P'')} - \sup\set{L(f, P')} < \epsilon\).

  Let \(P\) be the partition which contains both \(P',\ P''\). According to the lemma,

  \(L(f, P') \leq L(f, P)\) and \(U(f, P) \leq U(f, P'')\). Therefore,
  \(U(f, P) - L(f, P) < \epsilon\), as required.

  For any \(P\), \(\)
\end{proof}

\subsection{Continuity and Integrability}
\label{subsec:cont}

\begin{theorem}
  If \(f\) is continuous on \([a, b]\), then \(f\) is integrable on
  \([a, b]\).
\end{theorem}
\begin{proof}

Since \(f\) is continuous on \([a, b]\), it is also bounded on
\([a,b]\).

It has been shown that \(f\) is uniformly continuous on \([a,
b]\). Thus, there is some \(\delta > 0\) for all \(x\) and \(y\) in
\([a,b]\) such that if \(\abs{x-y}<\delta\), then
\(\abs{f(x) - f(y)} < \frac{\epsilon}{2(b-a)}\).

Choose a partition \(P=\set{t_{0}, t_{1}, \dots, t_{n}}\) such that
\(\abs{t_{i} - t_{i-1}} < \delta\). Then for each \(i\) we obtain

\begin{equation*}
  \abs{f(x) - f(y)} < \epsilon\text{ for all \(x, y\) in \([t_{i-1}, t_{i}]\)}.
\end{equation*}

Therefore,
\(M_{i} - m_{i} \leq \frac{\epsilon}{2(b-a)} < \frac{\epsilon}{b-a}\).

This holds for any \(i\), and thus

\begin{equation*}
  U(f, P) - L(f, P) = \sum_{i=1}^{n} (M_{i} - m_{i})(t_{i} - t_{i-1}) < \frac{\epsilon}{b-a}\sum_{i=1}^{n}(t_{i} - t_{i-1}) = \epsilon
\end{equation*}

\end{proof}


\end{document}