% -*- coding: utf-8; -*-
%%% Local Variables:
%%% mode: latex
%%% TeX-engine: xetex
%%% TeX-master: t
%%% End:
\documentclass[11pt]{scrartcl}
\usepackage[fancy, beaue, pset, anon]{masty}
\pSet{\nt{MAT157}{}{Math Plum}}
\usepackage{lineno}
% ----------------------------------------------------------------------
% Page setup
% ----------------------------------------------------------------------

\pagenumbering{gobble}

% ----------------------------------------------------------------------
% Custom commands
% ----------------------------------------------------------------------

% alignment

\newcommand*{\LongestHence}{$\Rightarrow$}% function name
\newcommand*{\LongestName}{$f_o(-x)+f_e(-x)$}% function name
\newcommand*{\LongestValue}{$(-a)x +(-a)(-y)$}% function value
\newcommand*{\LongestText}{\defi}%

\newlength{\LargestHenceSize}%
\newlength{\LargestNameSize}%
\newlength{\LargestValueSize}%
\newlength{\LargestTextSize}%

\settowidth{\LargestHenceSize}{\LongestHence}%
\settowidth{\LargestNameSize}{\LongestName}%
\settowidth{\LargestValueSize}{\LongestValue}%
\settowidth{\LargestTextSize}{\LongestText}%

% Choose alignment of the various elements here: [r], [l] or [c]

\newcommand*{\mbh}[1]{{\makebox[\LargestHenceSize][r]{\ensuremath{#1}}}}%
\newcommand*{\mbn}[1]{{\makebox[\LargestNameSize][r]{\ensuremath{#1}}}}%
\newcommand*{\mbv}[1]{\ensuremath{\makebox[\LargestValueSize][r]{\ensuremath{#1}}}}%
\newcommand*{\mbt}[1]{\makebox[\LargestTextSize][l]{#1}}%

\newcommand{\R}[1]{\label{#1}\linelabel{#1}}
\newcommand{\lr}[1]{line~\lineref{#1}}

% ----------------------------------------------------------------------
% Launch!
% ----------------------------------------------------------------------

\begin{document}

\section{Math Plum}

Let $\set{x}$ be the distance from $x$ to the nearest integer.

Consider $\frac{1}{10} \set{10 x}$.

Note that this function scales the graph of $\set{x}$ down by the
factor of $10$ in both directions.

Define the following function:
\begin{equation}
  f(x) = \sum_{n=0}^{\infty}\frac{1}{10^n} \set{10^nx}.
\end{equation}

Note that $f(x)$ is well-defined by the Weierstrass $M$-test, since
$\frac{1}{10^n} \set{10^nx} \leq \frac{1}{10^n}\frac{1}{2}
$. Moreover, it converges uniformly and $f$ is continuous.

However, $f$ is \textbf{not} differentiable \textbf{anywhere}.

Consider $\frac{f(a+h)-f(a)}{h}$.

We will find a sequence $\set{h_m}$ so that the sequence
$\frac{f(a+h_m)-f(a)}{h_{m}}$ diverges, which would imly the
difference quotient has no limit.

Fix $a$ such that $0 < a < 1$, and let $a= 0.a_1a_2\dots$.

Take $h_m = 
\begin{cases}
  10^{-m}, \text{ if $a_m \neq 4, 9$ }\\
  -10^{-m}, \text{ if $a_m = 4, 9$}
\end{cases}
$ (think about why the nonuniqueness of decimal representation would not affect the overall argument).


It is easy to see that $h_m\to 0$.

Consider $\frac{f(a+h_m) - f(a)}{h_m}$:

\begin{align}
&= \pm 10 (f(a+h_m) - f(a))\\ 
&= \pm 10^m (\sum_{n=0}^{\infty}10^{-n}(\set{10^n(a+h_m)} - \set{10^na})).
\end{align}

If $n \geq m$, then $10^nh_m=10^{n-m} \in \ZZ$, $\set{10^k(a+h_m)} = \set{10^na}$.

So only finitely many terms are nonzero, and thus it equals $\sum_{n=0}^{m}\pm 10^{-n}(\set{10^n(a+b_{m}) - \set{10^na}})$

If $0.a_{n+1}a_{n+2}\dots a_m  < \frac{1}{2}$, then 

Thus,$\frac{f(a+h_m) - f(a)}{h_m}$, which is equal to
$\pm 10^m\sum_{n=0}^m10^{n}(\pm 10^{n-m}) = \pm \sum_{n=0}^m\pm 1$.

It is even if $m$ is odd, and $odd$ if $m$ is even. 

Therefore, it does not converge.
\end{document}
