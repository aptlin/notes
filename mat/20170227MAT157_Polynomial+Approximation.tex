% -*- coding: utf-8; -*-
%%% Local Variables:
%%% mode: latex
%%% TeX-engine: xetex
%%% TeX-master: t
%%% End:
\documentclass[11pt]{scrartcl}
\usepackage[fancy, beaue, pset, anon]{masty}
\pSet{\nt{MAT157}{}{Polynomial Approximation}}
\usepackage{lineno}
% ----------------------------------------------------------------------
% Page setup
% ----------------------------------------------------------------------

\pagenumbering{gobble}

% ----------------------------------------------------------------------
% Custom commands
% ----------------------------------------------------------------------

% alignment

\newcommand*{\LongestHence}{$\Rightarrow$}% function name
\newcommand*{\LongestName}{$f_o(-x)+f_e(-x)$}% function name
\newcommand*{\LongestValue}{$(-a)x +(-a)(-y)$}% function value
\newcommand*{\LongestText}{\defi}%

\newlength{\LargestHenceSize}%
\newlength{\LargestNameSize}%
\newlength{\LargestValueSize}%
\newlength{\LargestTextSize}%

\settowidth{\LargestHenceSize}{\LongestHence}%
\settowidth{\LargestNameSize}{\LongestName}%
\settowidth{\LargestValueSize}{\LongestValue}%
\settowidth{\LargestTextSize}{\LongestText}%

% Choose alignment of the various elements here: [r], [l] or [c]

\newcommand*{\mbh}[1]{{\makebox[\LargestHenceSize][r]{\ensuremath{#1}}}}%
\newcommand*{\mbn}[1]{{\makebox[\LargestNameSize][r]{\ensuremath{#1}}}}%
\newcommand*{\mbv}[1]{\ensuremath{\makebox[\LargestValueSize][r]{\ensuremath{#1}}}}%
\newcommand*{\mbt}[1]{\makebox[\LargestTextSize][l]{#1}}%

\newcommand{\R}[1]{\label{#1}\linelabel{#1}}
\newcommand{\lr}[1]{line~\lineref{#1}}

% ----------------------------------------------------------------------
% Launch!
% ----------------------------------------------------------------------

\begin{document}

\section{Polynomial Approximation}

The tangent line at $x=a$ can be thought as the best linear
approximation to $f(x)$ when $x=a$. Can we do better approximations
with polynomials?

We choose polynomials because of their properties which make them
amenable to differentiation and integration.

Suppose $a=0$.

Consider a polynomial of degree $2$:

\begin{align}
  P_2(x) &= a_0 + a_1 x + a_2 x^2\\
  P_2'(x) &= a_1 + 2 a_2x\\
  P_2''(x) &= 2a_2
\end{align}

If $a_0=f(0)$, $a_1=f'(0)$ and $a_2 = \frac{1}{2}f''(0)$, then $P_2$ would satisfy
\begin{align}
  P_2(0) &= f(0)\\
  P_2'(x) &= f'(0)\\
  P_2''(x) &= f''(0)
\end{align}

We shall prove that $P_2(x)$ is the best approximation of $f$ in terms
of the polynomial of the second degree.


Now, consider $P_n(x)=\sum_{k=0}^n a_kx^k$. Then it is easy to show
that $P_n^{(m)}(0) = m! a_m$.

Take $a_m = \frac{1}{m!}f^{(m)}(0)$.

This approximation has a name.

\begin{definition}
  The \textit{Taylor polynomial} of degree $n$ for $f(x)$ near $x=0$ is

  \begin{equation*}
    P_n(x) = \sum_{k=0}^n \frac{1}{m!} f^{(m)}(0)x^k 
 \end{equation*}
\end{definition}

Note that a Taylor polynomial can also be defined at any point $a$ by translation:
  \begin{equation*}
    P_n(x) = \sum_{k=0}^n \frac{1}{m!} f^{(m)}(a)(x-a)^k
  \end{equation*}

  \begin{example}

Let $f(x) = \sin x$.  Then $f(x)$ for $x< 1$ is approximated well by $g_n(x) = \sum_{k=0}^n x-\frac{x^3}{3!}+\frac{x^5}{5!} -\cdots$.
\end{example}

Suppose now that $f(x) = \begin{cases}
  e^{-\frac{1}{x^2}-}, x\neq 0\\
  0, x = 0
\end{cases}$
\end{document}