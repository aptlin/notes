% -*- coding: utf-8; -*-
%%% Local Variables:
%%% mode: latex
%%% TeX-engine: xetex
%%% TeX-master: t
%%% End:
\documentclass[11pt]{scrartcl}
\usepackage[fancy, beaue, pset, anon]{masty}
\pSet{\nt{MAT247}{}{More on JCF}}
\usepackage{lineno}
% ----------------------------------------------------------------------
% Page setup
% ----------------------------------------------------------------------

\pagenumbering{gobble}

% ----------------------------------------------------------------------
% Custom commands
% ----------------------------------------------------------------------

% alignment

\newcommand*{\LongestHence}{$\Rightarrow$}% function name
\newcommand*{\LongestName}{$f_o(-x)+f_e(-x)$}% function name
\newcommand*{\LongestValue}{$(-a)x +(-a)(-y)$}% function value
\newcommand*{\LongestText}{\defi}%

\newlength{\LargestHenceSize}%
\newlength{\LargestNameSize}%
\newlength{\LargestValueSize}%
\newlength{\LargestTextSize}%

\settowidth{\LargestHenceSize}{\LongestHence}%
\settowidth{\LargestNameSize}{\LongestName}%
\settowidth{\LargestValueSize}{\LongestValue}%
\settowidth{\LargestTextSize}{\LongestText}%

% Choose alignment of the various elements here: [r], [l] or [c]

\newcommand*{\mbh}[1]{{\makebox[\LargestHenceSize][r]{\ensuremath{#1}}}}%
\newcommand*{\mbn}[1]{{\makebox[\LargestNameSize][r]{\ensuremath{#1}}}}%
\newcommand*{\mbv}[1]{\ensuremath{\makebox[\LargestValueSize][r]{\ensuremath{#1}}}}%
\newcommand*{\mbt}[1]{\makebox[\LargestTextSize][l]{#1}}%

\newcommand{\R}[1]{\label{#1}\linelabel{#1}}
\newcommand{\lr}[1]{line~\lineref{#1}}

% ----------------------------------------------------------------------
% Launch!
% ----------------------------------------------------------------------

\begin{document}

\section{Existence and Uniqueness of Jordan Canonical Form}

\begin{theorem}
  If a characteristic polynomial $f(t)$ of $T$ splits, then
  \begin{enumerate}[label=\alph*)]
  \item $V = \bigoplus_{i=1}^r K_{\lambda_i}$, where $\lambda_i$ for $i\in [1, r]\cap\NN$ are distinct eigenvalues.
  \item $\dim K_{\lambda} = m_{\lambda}$ for any eigenvalue $\lambda$.
  \end{enumerate}
\end{theorem}
\begin{proof}
  \hfill

  First we show that $V = \sum_{i=1}^rK_{\lambda_i}$ by induction on $r$.

  Let $W = \img (T-\lambda I)^{m_{\lambda}}$.

  Since $\dim V = \dim K_{\lambda_1} +\dim W$, while
  $K_{\lambda_1}\cap W = \set{0}$,

  then $\dim(K_{\lambda_1}+W) = \dim(K_{\lambda_1})+\dim(W) = \dim V$,
  and thus $V = K_{\lambda_1}\oplus W$.

  Note hat $W$ is $T$-invariant and $T_{W}$ has eigenvalues
  $\lambda_2,\dots, \lambda_r$.

  If $x\in W$, $x = (T-\lambda I)^{m_{\lambda}}(y)$ for some $y\in
  V$.

  Therefore,
  $Tx = T(T-\lambda I)^{m_{\lambda}}(y)=(T-\lambda
  I)^{m_{\lambda}}T(y) \in W$.

  Thus, $W$ is $T$-invariant.

  If $Tv = T|_W(v) = \mu v$ and $v\neq 0$, then $\mu$ is an eigenvalue of $T$, then $\mu$ is an eigenvalue of $T$, and hence $\mu \in \set{\lambda_1, \dots, \lambda_r}$. If $\mu = \lambda_1$, then $v\in W\cap K_{\lambda_1} = \set{0}$  by the previous remark.

  The generalised eigenspaces of $T|_W$ are
  $K_{\lambda_2}, \dots, K_{\lambda_r}$. Notice that
  $K_{\lambda}^W = K_{\lambda}\cap W$, where $K_{\lambda}^{W}$ is 

  By Theorem 7.1(b), since
  $(T-\lambda_1I)^{m_{\lambda_{1}}}:K_{\lambda_i}\to K_{\lambda_i}$
  for all $i\neq 1$, then
  $K_{\lambda_i} \suq \img((T-\lambda_1I)) = W$ for all $i\neq 1$, and
  therefore $K_{\lambda_i}^W = K_{\lambda_i}\cap W = K_{\lambda_i}$ for all $i\neq 1$.

  Thus, $K_{\lambda_1}^W = \set{0}$.

  Now we apply the inductive hypothesis to $T_W: W\to W$ (the
  characteristic polynomial splits and it has $r-1$ eigenvalues).

  Then $W = \sum_{i=2}^rK^W_{\lambda_i} = \sum_{i=2}^rK_{\lambda_i}$, and hence
  $V = K_{\lambda_1}+W  = \sum_{i=1}^rK_{\lambda_i}$.

  Since $V = \sum _{i=1}^rK_{\lambda_i}$ and
  $\dim V \leq \sum_{i=1}^r\dim K_{\lambda_i} \leq \sum_{i=1}^rm_i =
  \dim V$,

  and thus $\dim V = \sum_{i=1}^r \dim K_{\lambda_i}$ and
  $\dim K_{\lambda_i} = m_i$, which means that
  $V = \bigoplus_{i=1}^rK_{\lambda_i}$.
\end{proof}

Now we can fin a nice basis for each $K_{\lambda}$ separately.

If $x \in K_{\lambda}$ and $x\neq 0$, there is a smallest $l \geq 1$
such that $(T-\lambda I)^lx = 0$.

We call a set
$\set{(T-\lambda I)^{l-1}x, (T-\lambda I)^{l-2}x, \dots, (T-\lambda
  I)x, x}$ a \textbf{cycle of generalised eigenvectors} corresponding
to $\lambda$ of length $l$. Let's call $(T-\lambda I)^{l-1}$ an
\textit{initial vector} and $x$ an \textit{end vector}.

The initial vector is in $N(T-\lambda I) = E_{\lambda}$, and hence it
is an eigenvector for $\lambda$.

\begin{theorem}
  \label{sec:exist-uniq-jord}
  If $\gamma$ is a basis of $V$ which is a disjoint union of cycles
  $\gamma_i$ for $1\leq i\leq r$ of generalised eigenvectors, let
  $W_i=\spn(\gamma_i)$.

  \begin{enumerate}[label=\alph*)]
  \item $W_i$ is $T$-invariant and $[T_{W_i}]_{\gamma_i}$ is a Jordan block.
  \item $[T]_\gamma$ is in JCF.
  \end{enumerate}
\end{theorem}

\begin{proof}
  \hfill

  \begin{enumerate}[label=\alph*)]
  \item Fix $i$.

    Suppose
    \[\gamma_i = \set{(T-\lambda I)^{l-1}(x), \dots, (T-\lambda I)x,
        x}.\]

  Note that $W_i= \spn(\gamma_i)$, but $\gamma_i\suq \gamma$, so
  $\gamma_i$ is linearly independent and thus a basis of $W_i$.

  Let $v_j = (T-\lambda I)^{l-j}x$ for $1 \leq j \leq l$.

  We know that

  \begin{align}
    (T-\lambda J)v_j &= (T-\lambda J)^{l-j+1}(x)\\
                     &= (T-\lambda J)^{l-(j-1)}x\\
                     &=
                       \begin{cases}
                         v_{j-1} \text{ if } j> 1\\
                         0 \text{ if } j=1
                       \end{cases}
  \end{align}

  Therefore,
  \begin{equation*}
    Tv_j =
    \begin{cases}
      \lambda v_j + v_{j-1}, \text{ if } j>1\\
      \lambda v_j, \text{ if } j=1
    \end{cases}.
  \end{equation*}

  So $Tv_j\in W_i$ for all $j$, and thus $W_i$ is $T$-invariant and
  $[T_{W_i}]_{\gamma_i}$ is a Jordan block.
\item Note that, by definition, $\gamma = \bigcup_{i=1}^r
  \gamma_{i}$. The matrix representation $[T]_{\gamma}$ has Jordan
  blocks on a diagonal, and thus $[T]_{\gamma}$ is in a Jordan
  Canonical Form.
\end{enumerate}
\end{proof}

\begin{theorem}
  Suppose $\gamma_1,\dots, \gamma_r$ are cycles of generalised
  eigenvalues corresponding to the \textbf{same} eigenvalue $\lambda$.

  If the initial vectors are linearly independent, then the sets $\gamma_i$ are disjoint and $\gamma = \bigcup_{i=1}^r\gamma_i$ is linearly independent.
\end{theorem}
\begin{proof}
  \hfill

  Let $W=\spn \gamma$. From Theorem \ref{sec:exist-uniq-jord}, $W$ is $T$-invariant.

  Let $U = T-\lambda I:W\to W$.

  Note that $\gamma_i = \set{U^{l_i-1}x_i, \dots, Ux_i, x_i}$.

  We proceed by induction on the number of vectors
  $l_1+\dots+l_r$.

  If $\sum_{i=1}^rl_i = 1$, there is a one-dimensional cycle which is linearly independent trivially.

  Suppose $U^{l_1-1}(x_1), \dots, U^{l_r-1}x_r$, which are all in
  $E_{\lambda}= \ker (U)$, are linearly independent, and then
  $\dim \ker(U) \geq r$.

  On the other hand,
  $\gamma_i' = \set{U^{l_i-1}x_i, U^2x_i, Ux_i}$ is a cycle of length $\lambda_i-1$ contained in $\img U$.

  The total number of vectors is $r$ fewer than before, so we can
  apply induction to $\gamma_1', \dots, \gamma_r'$.  Therefore,

  $\bigcup_{i=1}^r\gamma_r'$ is a linearly independent disjoint union.

  Therefore,
  $\dim \img(U)\geq \sum_{i=1}^r(l_i-1) = -r + \sum_{i=1}^rl_{i}$.

  Hence, by the dimension theorem,
  \begin{align}
    d &= \dim \img U +\dim \ker U \geq (\sum_{i=1}^r\lambda_i -r)+r\\
      &= \sum_{i=1}^rl_i\geq \abs{\gamma} \geq \dim W = d.
  \end{align}

  Thus, the equality holds, so $\abs{\gamma} = \sum l_i$, and thus
  $\gamma$ is a disjoint union.

  Therefore, $\abs{\gamma} = \dim W$ and thus $\gamma$ is a basis of
  $W$, and thus it is linearly independent.
\end{proof}




\end{document}