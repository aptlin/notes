% -*- coding: utf-8; -*-
%%% Local Variables:
%%% mode: latex
%%% TeX-engine: xetex
%%% TeX-master: t
%%% End:
\documentclass[11pt]{scrartcl}
\usepackage[fancy, beaue, pset, anon]{masty}
\pSet{\nt{MAT157}{XVIII}{Solids of Revolution}}
\usepackage{lineno}
% ----------------------------------------------------------------------
% Page setup
% ----------------------------------------------------------------------

\pagenumbering{gobble}

% ----------------------------------------------------------------------
% Custom commands
% ----------------------------------------------------------------------

% alignment

\newcommand*{\LongestHence}{$\Rightarrow$}% function name
\newcommand*{\LongestName}{$f_o(-x)+f_e(-x)$}% function name
\newcommand*{\LongestValue}{$(-a)x +(-a)(-y)$}% function value
\newcommand*{\LongestText}{\defi}%

\newlength{\LargestHenceSize}%
\newlength{\LargestNameSize}%
\newlength{\LargestValueSize}%
\newlength{\LargestTextSize}%

\settowidth{\LargestHenceSize}{\LongestHence}%
\settowidth{\LargestNameSize}{\LongestName}%
\settowidth{\LargestValueSize}{\LongestValue}%
\settowidth{\LargestTextSize}{\LongestText}%

% Choose alignment of the various elements here: [r], [l] or [c]

\newcommand*{\mbh}[1]{{\makebox[\LargestHenceSize][r]{\ensuremath{#1}}}}%
\newcommand*{\mbn}[1]{{\makebox[\LargestNameSize][r]{\ensuremath{#1}}}}%
\newcommand*{\mbv}[1]{\ensuremath{\makebox[\LargestValueSize][r]{\ensuremath{#1}}}}%
\newcommand*{\mbt}[1]{\makebox[\LargestTextSize][l]{#1}}%

\newcommand{\R}[1]{\label{#1}\linelabel{#1}}
\newcommand{\lr}[1]{line~\lineref{#1}}

% ----------------------------------------------------------------------
% Launch!
% ----------------------------------------------------------------------

\begin{document}

\section{Ellipsoids}

Note that the general equation for an ellipse is $\frac{y^2}{b^2} = 1- \frac{x^2}{a^2}$.

Thus, the equation of an upper half of an ellipse is $y = b \sqrt{1-\frac{x^2}{a^2}}$.

Thus, the volume of an ellipse is given by the following integral:
\begin{equation*}
  \int_{-a}^a \pi b^2(1-\frac{x^2}{a^2}) \dif x = \frac{4}{3}\pi ab^2
\end{equation*}

\section{Method of Shells}
Another method of evaluating a volume is the \textit{method of slices}.

Suppose a circle of radius $r$ is drawn on the Cartesian plane such that its centre is on the $y$-axis at the distance of $r$ from the origin. Revolve this circle around the $x$-axis to obtain a torus, also called an \textit{annulus}.

Although the standard method of computing its volume can be applied,
another way is available.

Note that the equation of the given circleis $x^2+(y-R)^2= r^2$.

Thus, the upper and lower semicircle is given by $x = \pm\sqrt{r^2-(y-R)^2}$.

Slice the circle in horizontal stripes and revolve this stripes around the $x$-axis. In such a way, a set of cylindrical shells is obtained. We may easily apply the standard method to evaluate the volume of each shell of thickness $\dif y$, radius $r$ and width of $2\sqrt{r^2-(y-R)^2}$.

The volume of a shell is thus given by $2\pi y \* 2 \sqrt{r^2-(y-R)^2} \dif y$.

Therefore, the total volume of a torus is $\int_{R-r}^{R+r}2\pi y \* 2 \sqrt{r^2-(y-R)^2} \dif y$, and hence

\begin{align}
  V & = \int_{R-r}^{R+r}2\pi y \* 2 \sqrt{r^2-(y-R)^2} \dif y                        \\
    & = 4\pi\int_{R-r}^{R+r} y \sqrt{r^2-(y-R)^2} \dif y                             \\
    & = 4\pi (\int_{-r}^ru \sqrt{r^2-u^2}\dif u + R \int_{-r}^r \sqrt{r^2-u^2}\dif u \\
    & = 2\pi^2Rr^2
\end{align}

\section{Surface Area}

The surface area of a solid of revolution can also be computed by considering thin slices of its surface.

Suppose a graph of a function $f(x)$ is drawn in the first quart of the Cartesian plane, and the corresponding solid of revolution is drawn.

Thenthe radius of a slice is $f(x)$ and the circumference is $2\pi f(x)$.

Note that the width of a strip is given by

\begin{equation}
  \label{eq:1}
\sqrt{(\dif x) ^2 + (\dif y)^2}= \sqrt{1+f'(x)^2} \dif x.
\end{equation}

Therefore, the surface area of a solid is given by
$ 2\pi\int_a^b f(x) \sqrt{1+f(x)^2}\dif x$.  For a sphere,
$f(x) = \sqrt{R^2-x^2}$ and hence $f'(x) = \frac{-x}{\sqrt{R^2-x^2}}$,
which, subbing into the obtained equation (\ref{eq:1}), gives us the surface area of $4\pi R^2$.

Consider now a horn obtained by rotating the graph of
$f(x) = \frac{1}{x}$ from 1 to $\infty$ around the $x$-axis.

Thus,

\[V = \int_1^{\infty}\pi \frac{1}{x^2}\dif x = \lim_{R\to 0} \int_1^R \pi \frac{1}{x^2}\dif x = \pi\] 
\end{document}