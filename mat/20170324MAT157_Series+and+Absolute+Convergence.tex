% -*- coding: utf-8; -*-
%%% Local Variables:
%%% mode: latex
%%% TeX-engine: xetex
%%% TeX-master: t
%%% End:
\documentclass[11pt]{scrartcl}
\usepackage[fancy, beaue, pset, anon]{masty}
\pSet{\nt{MAT157}{}{Series and Absolute Convergence}}
\usepackage{lineno}
% ----------------------------------------------------------------------
% Page setup
% ----------------------------------------------------------------------

\pagenumbering{gobble}

% ----------------------------------------------------------------------
% Custom commands
% ----------------------------------------------------------------------

% alignment

\newcommand*{\LongestHence}{$\Rightarrow$}% function name
\newcommand*{\LongestName}{$f_o(-x)+f_e(-x)$}% function name
\newcommand*{\LongestValue}{$(-a)x +(-a)(-y)$}% function value
\newcommand*{\LongestText}{\defi}%

\newlength{\LargestHenceSize}%
\newlength{\LargestNameSize}%
\newlength{\LargestValueSize}%
\newlength{\LargestTextSize}%

\settowidth{\LargestHenceSize}{\LongestHence}%
\settowidth{\LargestNameSize}{\LongestName}%
\settowidth{\LargestValueSize}{\LongestValue}%
\settowidth{\LargestTextSize}{\LongestText}%

% Choose alignment of the various elements here: [r], [l] or [c]

\newcommand*{\mbh}[1]{{\makebox[\LargestHenceSize][r]{\ensuremath{#1}}}}%
\newcommand*{\mbn}[1]{{\makebox[\LargestNameSize][r]{\ensuremath{#1}}}}%
\newcommand*{\mbv}[1]{\ensuremath{\makebox[\LargestValueSize][r]{\ensuremath{#1}}}}%
\newcommand*{\mbt}[1]{\makebox[\LargestTextSize][l]{#1}}%

\newcommand{\R}[1]{\label{#1}\linelabel{#1}}
\newcommand{\lr}[1]{line~\lineref{#1}}

% ----------------------------------------------------------------------
% Launch!
% ----------------------------------------------------------------------

\begin{document}

\section{Series and Absolute Convergence}

\begin{definition}
  A series $\sum_{n=1}^{\infty}a_n$ \textbf{converges absolutely} if
  $\sum_{n=1}^{\infty}\abs{a_n}$ converges.
\end{definition}

A series which converges but does not converge absolutely is said to
\textbf{converge conditionally}.

\begin{theorem}
If $s=\sum_{n=1}^{\infty}a_n$ converges absolutely,  then $s$ converges.
\end{theorem}

\begin{proof}
  \hfill

  Since $s$ converges absolutely, then $\sum_{n=1}^{\infty} \abs{a_n}$ converges.

  Therefore, $\sum_{i=0}^{m-n} \abs{n+i} \to 0$ a $m,n \to \infty$.

  But
  $\abs{\sum_{i=1}^{m-n}a_{n+i}} \leq \sum_{i=1}^{m-n}\abs{a_{n+i}}$,
  which means that $\sum_{i=1}^{m-n}a_{n+i}$ goes to 0.

  Therefore, $\sum a_n$ converges.
\end{proof}

Write $a_n^+ = \begin{cases}
  a_n \text{ if $a_n\geq 0$}\\
  0 \text{ otherwise}
\end{cases}$ and $a_n^- = \begin{cases}
  a_n \text{ if $a_n\leq 0$}\\
  0 \text{ otherwise}
\end{cases}$.

Then $\sum a_n = \sum a_n^+ +\sum a_n^-$.

\begin{theorem}
$\sum a_n$ converges absolutely if $\sum a_n^+$ and $\sum a_n^-$ converge absolutely.

In this case, $\sum a_n = \sum a_n^+ +\sum a_n^-$ and $\sum \abs{a_n} = \sum a_n^+ -\sum a_n^-$.

\end{theorem}

\begin{example}

Consider $\log 2 = 1-\frac{1}{2}+\frac{1}{3}-\frac{1}{4}+\dots$.

We can \textit{rearrange} the series so that it converges to 17.

Note that

\begin{equation*}
1+ \frac{1}{3}+\frac{1}{5}+\dots+\frac{1}{2n+1}-\frac{1}{2}-\frac{1}{4}-\dots+\frac{1}{2n+5}+\dots
\end{equation*}

Every term of the original series appears eventually in the rearranged series, and it converges to 17.

Note that there is nothing special about the choice of 17 in this
cas. Given any $A$, we can rearrange any conditionally convergent series to converge to $A$.

\end{example}

\begin{theorem}
  If $\sum a_n converges$ absolutely, then any rearrangement converges
  absolutely and the sums are equal.
\end{theorem}

\begin{remark}
  The addition symbol in series is a misnomer. A series is a limit of
  a sequence, not the sum in the usual sense.
\end{remark}

\begin{theorem}[Leibniz Theorem]
If $a_1 > a_2 > a_3 > \dots > a_n > 0$ and $\lim_{n\to\infty}a_n = 0$, then $\sum_{i=1}^{\infty} (-1)^na_n$ converges.
\end{theorem}

\begin{example}

  Take $\sum (-1)^{n+1}\frac{1}{n}$  and $\sum (-1)^{n+1}\frac{1}{n^2}$.

  It can be shown that the first series converges unconditionally,
  while the second converges conditionally.

\end{example}

\subsection{Power Series}

Series in the form $\sum_{n=0}^{\infty}a_n(x-a)^n$ are called \textbf{power series}.

For example, all Taylor series are power series.

Take $s=\sum_{i=1}^{\infty} \frac{\abs{x}^2}{n!}$ and apply the Ratio Test:
\begin{equation*}
  \frac{\frac{\abs{x}^{n+1}}{(n+1)!}}{\frac{\abs{x}^n}{n!}} = \frac{\abs{x}}{n+1}\to 0.
\end{equation*}

Therefore, $s=\sum_{i=1}^{\infty} \frac{\abs{x}^2}{n!}$ converges
absolutely and therefore converges.

Consider now $\log (x+1) = \sum_{i=1}^n(-1)^{n+1}\frac{x^n}{n}$.

Using the ratio test we obtain that 

\begin{equation*}
  \frac{(-1)^{n+2}\frac{x^{n+1}}{n+1}}{(-1)^n \frac{x^{n}}{n}}.
\end{equation*}

The sums convergewhen $\abs{x> 1}$, and diverges if $\abs{x< 1}$.

What happens when $\abs{x}=1$?

Take $x=1$.

Then $\sum (-1)^{n+1}\frac{1}{n}$ converges to $\log 2$, at least
conditionally, and $\sum_{i=1}^{\infty}(-1)^{n+1}\frac{(-1)^n}{n}$ diverges (harmonic series are obtained).

\begin{theorem}
  A power series $\sum a_n(x-a)^n$ converges absolutely on some
  interval $(a-r, a+r)$ and diverges if $\abs{x-a}> r$ 

  It may diverge, converge conditionall or converge absolutely at each endpoint.
\end{theorem}

\begin{remark}
Note that $r$ might denote $\infty$, which would mean that it converges absolutely on $\RR$.
\end{remark}

\begin{example}

Consider $\frac{1}{1+x^2} = 1-x^2+x^4-x^6+\dots$.

Note that it converges on $(-1, 1)$ but not at $\pm 1$.

\end{example}
\end{document}
