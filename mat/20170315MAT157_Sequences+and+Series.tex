% -*- coding: utf-8; -*-
%%% Local Variables:
%%% mode: latex
%%% TeX-engine: xetex
%%% TeX-master: t
%%% End:
\documentclass[11pt]{scrartcl}
\usepackage[fancy, beaue, pset, anon]{masty}
\pSet{\nt{MAT157}{}{Sequences and Series}}
\usepackage{lineno}
% ----------------------------------------------------------------------
% Page setup
% ----------------------------------------------------------------------

\pagenumbering{gobble}

% ----------------------------------------------------------------------
% Custom commands
% ----------------------------------------------------------------------

% alignment

\newcommand*{\LongestHence}{$\Rightarrow$}% function name
\newcommand*{\LongestName}{$f_o(-x)+f_e(-x)$}% function name
\newcommand*{\LongestValue}{$(-a)x +(-a)(-y)$}% function value
\newcommand*{\LongestText}{\defi}%

\newlength{\LargestHenceSize}%
\newlength{\LargestNameSize}%
\newlength{\LargestValueSize}%
\newlength{\LargestTextSize}%

\settowidth{\LargestHenceSize}{\LongestHence}%
\settowidth{\LargestNameSize}{\LongestName}%
\settowidth{\LargestValueSize}{\LongestValue}%
\settowidth{\LargestTextSize}{\LongestText}%

% Choose alignment of the various elements here: [r], [l] or [c]

\newcommand*{\mbh}[1]{{\makebox[\LargestHenceSize][r]{\ensuremath{#1}}}}%
\newcommand*{\mbn}[1]{{\makebox[\LargestNameSize][r]{\ensuremath{#1}}}}%
\newcommand*{\mbv}[1]{\ensuremath{\makebox[\LargestValueSize][r]{\ensuremath{#1}}}}%
\newcommand*{\mbt}[1]{\makebox[\LargestTextSize][l]{#1}}%

\newcommand{\R}[1]{\label{#1}\linelabel{#1}}
\newcommand{\lr}[1]{line~\lineref{#1}}

% ----------------------------------------------------------------------
% Launch!
% ----------------------------------------------------------------------

\begin{document}

\section{Sequences and Series}

\begin{definition}
A sequence is a list of numbers $a_1, a_2, \dots ,a_n, \dots$.
\end{definition}

Sequences can be thought as a function from $\NN\to \RR$, with $na_n\in \RR$.

\begin{definition}
  Let $L$ be a number such that for any $\epsilon_{n} > 0$ there
  exists $N$ such that $\abs{a_n-L} < \epsilon$ for all values of $n>N$.

  If $\set{a_n}$ has a limit $L$, we say that it \textbf{converges} to $L$. If there is no limit, then a sequence is said to diverge.

  Take $c_n=$For any $M>0$, there exists $N$ so that $c_n >M$ and for all $n> N$. In this case we say that $f(n)$diverges.

  Consider now the sequence such that $h_n=1+ \frac{1}{n}$ if and only if $n$ is odd and $h_n=0$ if $n$ is even.

  We define $\lim_{n\to \infty}h_m = L$, which means that for a given $\epsilon > 0$ there exists $N >0 $ such that $h_{n < L+\epsilon}$ forr $n> N$ and for any $\epsilon >0$ and $M$ there exists $n >M$ such that $\abs{h_n - L}< \epsilon$.
\end{definition}

\end{document}
