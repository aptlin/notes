% -*- coding: utf-8; -*-
%%% Local Variables:
%%% mode: latex
%%% TeX-engine: xetex
%%% TeX-master: t
%%% End:
\documentclass[11pt]{scrartcl}
\usepackage[fancy, beaue, pset, anon]{sdll}
\pSet{\nt{MAT157}{V}{Applications of the Fundamental Theorem of Calculus}}
\usepackage{lineno}
% ----------------------------------------------------------------------
% Page setup
% ----------------------------------------------------------------------

\pagenumbering{gobble}

% ----------------------------------------------------------------------
% Custom commands
% ----------------------------------------------------------------------

% alignment

\newcommand*{\LongestHence}{$\Rightarrow$}% function name
\newcommand*{\LongestName}{$f_o(-x)+f_e(-x)$}% function name
\newcommand*{\LongestValue}{$(-a)x +(-a)(-y)$}% function value
\newcommand*{\LongestText}{\defi}%

\newlength{\LargestHenceSize}%
\newlength{\LargestNameSize}%
\newlength{\LargestValueSize}%
\newlength{\LargestTextSize}%

\settowidth{\LargestHenceSize}{\LongestHence}%
\settowidth{\LargestNameSize}{\LongestName}%
\settowidth{\LargestValueSize}{\LongestValue}%
\settowidth{\LargestTextSize}{\LongestText}%

% Choose alignment of the various elements here: [r], [l] or [c]

\newcommand*{\mbh}[1]{{\makebox[\LargestHenceSize][r]{\ensuremath{#1}}}}%
\newcommand*{\mbn}[1]{{\makebox[\LargestNameSize][r]{\ensuremath{#1}}}}%
\newcommand*{\mbv}[1]{\ensuremath{\makebox[\LargestValueSize][r]{\ensuremath{#1}}}}%
\newcommand*{\mbt}[1]{\makebox[\LargestTextSize][l]{#1}}%

\newcommand{\R}[1]{\label{#1}\linelabel{#1}}
\newcommand{\lr}[1]{line~\lineref{#1}}

% ----------------------------------------------------------------------
% Launch!
% ----------------------------------------------------------------------

\begin{document}

\section{Applications of the Fundamental Theorem of Calculus}

\begin{corollary}
  If $f$ is continuous on $[a, b]$ and $f = g'$ for some function $g$, then
  \begin{equation*}
    \int_{a}^bf = g(b) - g(a).
  \end{equation*}
\end{corollary}

\begin{proof}
  Let $F(x) = \int_a^xf$. Then $F' = f = g'$ on $[a, b]$. Thus, there
  is a number $c\in \RR$ such that
  \begin{equation*}
    F = g + c
  \end{equation*}

  Note that $F(a) = g(a) + c = 0$ by definition of $F$, and thus
  $c = -g(a)$, which means that $F(x) = g(x) - g(a)$. In particular,
  this holds for $x=b$, which gives $F(b) = \int_a^bf = g(b) - g(a)$.
\end{proof}

\begin{theorem}[The Second Fundamental Theorem of Calculus]
  If $f$ is integrable on $[a,b]$ and $f = g'$ for some function $g$, then

  \begin{equation*}
\int_a^bf = g(b) - g(a).
  \end{equation*}

\end{theorem}
\begin{proof}
  Let $P = \set{t_0, t_1, \dots, t_n}$ be any partition of $[a,b]$.

  By the Mean Value Theorem, there is a point $x_i$ in
  $[t_{i-1}, t_i]$ such that
  \begin{equation*}
    g(t_i) - g(t_{i-1}) = g'(x_i)(t_i-t_{i-1}) = f(x_i)(t_i-t_{i-1})
  \end{equation*}

  If $m_i= \inf\set{f(x);t_{i-1}\leq x_i\leq t_i}$ and
  $M_i= \sup\set{f(x);t_{i-1}\leq x_i\leq t_i}$, then

  \begin{equation*}
    m_i(t_i-t_{i-1})\leq f(x_i)(t_{i} - t_{i-1}) \leq M_{i}(t_{i} - t_{i-1}),
  \end{equation*}

  and thus
  \begin{equation*}
    m_i(t_i-t_{i-1})\leq g(t_{i-1})-g(t_i) \leq M_{i}(t_{i} - t_{i-1}),
  \end{equation*}
  giving

  \begin{equation*}
    \sum_{i=1}^nm_{i}(t_{i} - t_{i-1}) \leq g(b) - g(a) \leq    \sum_{i=1}^nM_{i}(t_{i} - t_{i-1}).
  \end{equation*}

  Therefore, $L(f, P) \leq g(b) - g(a) \leq U(f, P)$ for every partition $P$, which means that


  \begin{equation*}
    \int_a^bf = g(b) - g(a)
  \end{equation*}


\end{proof}

\end{document}