% -*- coding: utf-8; -*-
%%% Local Variables:
%%% mode: latex
%%% TeX-engine: xetex
%%% TeX-master: t
%%% End:
\documentclass[11pt]{scrartcl}
\usepackage[fancy, beaue, pset, anon]{masty}
\pSet{\nt{MAT157}{VII}{Trigonometric Functions}}
\usepackage{lineno}
% ----------------------------------------------------------------------
% Page setup
% ----------------------------------------------------------------------

\pagenumbering{gobble}

% ----------------------------------------------------------------------
% Custom commands
% ----------------------------------------------------------------------

% alignment

\newcommand*{\LongestHence}{$\Rightarrow$}% function name
\newcommand*{\LongestName}{$f_o(-x)+f_e(-x)$}% function name
\newcommand*{\LongestValue}{$(-a)x +(-a)(-y)$}% function value
\newcommand*{\LongestText}{\defi}%

\newlength{\LargestHenceSize}%
\newlength{\LargestNameSize}%
\newlength{\LargestValueSize}%
\newlength{\LargestTextSize}%

\settowidth{\LargestHenceSize}{\LongestHence}%
\settowidth{\LargestNameSize}{\LongestName}%
\settowidth{\LargestValueSize}{\LongestValue}%
\settowidth{\LargestTextSize}{\LongestText}%

% Choose alignment of the various elements here: [r], [l] or [c]

\newcommand*{\mbh}[1]{{\makebox[\LargestHenceSize][r]{\ensuremath{#1}}}}%
\newcommand*{\mbn}[1]{{\makebox[\LargestNameSize][r]{\ensuremath{#1}}}}%
\newcommand*{\mbv}[1]{\ensuremath{\makebox[\LargestValueSize][r]{\ensuremath{#1}}}}%
\newcommand*{\mbt}[1]{\makebox[\LargestTextSize][l]{#1}}%

\newcommand{\R}[1]{\label{#1}\linelabel{#1}}
\newcommand{\lr}[1]{line~\lineref{#1}}

% ----------------------------------------------------------------------
% Launch!
% ----------------------------------------------------------------------

\begin{document}

The intuitive understanding of angles can be made more precise by
considering a unit circle.

In this way, an angle $0\leq \theta \leq 2\pi$ corresponds to a unique
point on the unit circle.

Since the equation of a semicircle on the axes $xOy$ is
$y = \sqrt{1-x^{2}}$ and the area of the unit circle is $\pi$, we thus
define $\pi = 2\int_{-1}^1\sqrt{1-x^{2}}\dif x$. The area of the
sector corresponding to an angle $\theta$ with $\cos \theta = x$, if
$-1 \leq x \leq 1$, is


\begin{equation*}
A(x) = \frac{x\sqrt{1-x^2}}{2} + \int_x^1\sqrt{1-t^2} \dif t.
\end{equation*}

Consider now $A'(x)$:


\begin{equation*}
  A'(x) = \half(\sqrt{1-x^2} - \frac{x^{2}}{ \sqrt{1-x^2}}) - \sqrt{1-x^{2}} = \frac{-1}{2 \sqrt{1-x^{2}} }
\end{equation*}

Note that $A'$ is well-defined for $x \in (-1, 1)$.

We also note that $A(-1) = \frac{\pi}{2}$ and $A(0) = 0$. The graph of $A$ is such that it is:

\begin{itemize}
\item decreasing (since $A' < 0$)
\item injective
\end{itemize}
Therefore, it has an inverse.

\begin{ques*}

How to express $\cos \theta$ as a function of $2A$?

\end{ques*}

\begin{answer*}

  Note that $\cos \theta$ is the inverse function of $2A$. Given
  $2A$, which determines an angle $\theta$, define $\cos \theta$ to be
  the unique $x$ such that $2A = 2A(x)$, thus
  $\theta = 2A(cos \theta)$.

\end{answer*}

We define $\sin \theta = \sqrt{1-\cos^2\theta}$ and $B = 2A$. $B$ is
the inverse of $\cos \theta$, and thus, from above,


\begin{equation*}
  \frac{\dif \cos \theta}{\dif \theta } = \frac{1}{B'(\cos\theta)} = - \sin \theta
\end{equation*}

Similarly,

\begin{equation*}
  \frac{\dif \sin \theta}{\dif \theta } = \frac{\dif  \sqrt{1 - \cos^{2}\theta}}{\dif \theta} = \cos \theta
\end{equation*}
\end{document}