% -*- coding: utf-8; -*-
%%% Local Variables:
%%% mode: latex
%%% TeX-engine: xetex
%%% TeX-master: t
%%% End:
\documentclass[11pt]{scrartcl}
\usepackage[fancy, beaue, pset, anon]{sdll}
\pSet{\nt{MAT157}{II.2}{Intro to Integral Calculus II}}
\usepackage{lineno}
% ----------------------------------------------------------------------
% Page setup
% ----------------------------------------------------------------------

\pagenumbering{gobble}

% ----------------------------------------------------------------------
% Custom commands
% ----------------------------------------------------------------------

% alignment

\newcommand*{\LongestHence}{$\Rightarrow$}% function name
\newcommand*{\LongestName}{$f_o(-x)+f_e(-x)$}% function name
\newcommand*{\LongestValue}{$(-a)x +(-a)(-y)$}% function value
\newcommand*{\LongestText}{\defi}%

\newlength{\LargestHenceSize}%
\newlength{\LargestNameSize}%
\newlength{\LargestValueSize}%
\newlength{\LargestTextSize}%

\settowidth{\LargestHenceSize}{\LongestHence}%
\settowidth{\LargestNameSize}{\LongestName}%
\settowidth{\LargestValueSize}{\LongestValue}%
\settowidth{\LargestTextSize}{\LongestText}%

% Choose alignment of the various elements here: [r], [l] or [c]

\newcommand*{\mbh}[1]{{\makebox[\LargestHenceSize][r]{\ensuremath{#1}}}}%
\newcommand*{\mbn}[1]{{\makebox[\LargestNameSize][r]{\ensuremath{#1}}}}%
\newcommand*{\mbv}[1]{\ensuremath{\makebox[\LargestValueSize][r]{\ensuremath{#1}}}}%
\newcommand*{\mbt}[1]{\makebox[\LargestTextSize][l]{#1}}%

\newcommand{\R}[1]{\label{#1}\linelabel{#1}}
\newcommand{\lr}[1]{line~\lineref{#1}}

% ----------------------------------------------------------------------
% Launch!
% ----------------------------------------------------------------------

\begin{document}

\begin{lemma*}
  For partitions \(Q, P\) such that \(P \su Q\)

  \begin{align*}
    L(f, P) &\leq L(f, Q)\\
    U(f, P) &\geq U(f, Q)
  \end{align*}

\end{lemma*}

\begin{proof}

  Consider a special case when \(Q\) have just one extra point \(u\)
  in addition to all the points in \(P\)
  (\(t_{0} < t_{1} < \dots < t_{k-1} < u < t_{k} < \dots < t_{n} \)).

  \begin{note*}
    The \(\inf\) on \([t_{k}, u]\) and \([u, t_{k+1}]\) may be larger than the \(\inf\) on \([t_{k},t_{k+1}]\).\\
    The \(\sup\) on \([t_{k}, u]\) and \([u, t_{k+1}]\) may be less than the \(\sup\) on \([t_{k},t_{k+1}]\).
  \end{note*}
\begin{align}
  L(f, P) &= \sum_{i=1}^{n}m_{i}(t_{i} - t_{i-1})\\
  L(f, Q)  &= \sum_{i=1}^{k-1}m_{i}(t_{i} - t_{i-1}) + m'(u-t_{k-1}) + m''(t_{k} - u) +  \sum_{i=k+1}^{n}m_{i}(t_{i} - t_{i-1}),
\end{align}
where \(m' = \inf\set{f(x); t_{k-1} \leq x \leq u}\) and \(m'' = \inf\set{f(x); u \leq x \leq t_{k}}\).

By the remark given above, \(m_{k} \leq m'\) and \(m_{k} \leq m''\), and thus


\begin{equation*}
m_{k}(t_{k} - t_{k-1}) = m_{k}(t_{k} - u) + m_{k}(u - t_{k-1}) \leq m'(t_{k} - u) + m''(u-t_{k-1}),
\end{equation*}
which shows that \(L(f, P) \leq L(f, Q)\).

The argument for \(U(f, P) \geq U(f, Q)\) is similar.

The general case is deduced by considering a sequence of partitions
differing only by one point:


\begin{equation*}
P=P_{1}, P_{2}, \dots, P_{\alpha} = Q
\end{equation*}

Then by the lemma
\(L(f, P_{1}) \leq L(f, P_{2}) \leq \dots \leq L(f, P_{\alpha}) = L(f,
Q)\) and
\(U(f, P_{1}) \geq U(f, P_{2}) \geq \dots \geq U(f, P_{\alpha}) = U(f, Q)\).
\end{proof}
\begin{theorem}
  \label{sec:1}
  Let \(P_{1}\) and \(P_{2}\) be partitions of \([a, b]\), and let the
  function \(f\) be bounded on \([a, b]\). Then


  \begin{equation*}
    L(f, P_{1}) \leq U(f, P_{2})
  \end{equation*}

\end{theorem}

\begin{proof}
  Consider a partition \(P\) which contains points both in \(P_{1}\)
  and \(P_{2}\). By the lemma above it follows that


  \begin{equation*}
    L(f, P_{1}) \leq L(f, P) \leq U(f, P) \leq U(f, P_{2}).
  \end{equation*}
\end{proof}

From Theorem \ref{sec:1} it follows that
\(\sup \set{L(f, P)} \leq \inf\set{U(f, P)}\).

\begin{example}
  Let's find an example such that \(\sup \set{L(f, P)} < \inf\set{U(f, P)}\).
  Consider the following function:

\[f(x) = \begin{cases}
    0,\text{ x irrational}\\
    1,\text{ x rational}
  \end{cases}\]

For any partition \(P = [t_{0}, t_{1}, \dots, t_{n}]\), \(m_{i} = 0\),
since the irrational numbers are dense, and \(M_{i} = 1\), since the
rational numbers are dense.

Therefore,

\begin{align}
  L(f, P) &= \sum_{i=1}^{n}0(t_{i} - t_{i-1}) = 0\\
  U(f, P) &= \sum_{i=1}^{n}1(t_{i} - t_{i-1}) = b - a
\end{align}
\end{example}

We now can define the area \(R(f, a, b)\) formally.

\begin{definition}
  A function \(f\) which is bounded on \([a, b]\) is called
  \textbf{integrable} on \([a, b]\) if
  \(\sup \set{L(f, P): \text{ P is a partition of \([a,b]\)}} =
  \inf\set{U(f, P): \text{ P is a partition of \([a,b]\)}}\).

  In this case, the unique number
  \(\sup \set{L(f, P)} = \inf\set{U(f, P)}\) is denoted as
  \(\int_{a}^{b}f\).

\end{definition}

\begin{theorem}
  If \(f\) is bounded on \([a, b]\), then \(f\) is also integrable on
  \([a,b]\) if and only if for all \( \epsilon > 0\) there exists a
  partition \(P\) of \([a, b]\) such that

  \begin{equation*}
    U(f, P) - L(f, P) < \epsilon
  \end{equation*}

\end{theorem}

This restatement is convenient, as shown by the following example.
\begin{example}

  Consider \(f\) defined on \([0, 2]\) by
  \begin{equation*}
    f = \begin{cases}
      0, x \neq 1\\
      1, x = 1
    \end{cases}
  \end{equation*}

  Consider a partition \(P= [t_{0}, t_{1}, \dots, t_{n}]\) such that
  \(t_{k-1} < 1 < t_{k}\).

  Note that \(m_{i}\) and \(M_{i}\) are both zero for all intervals in
  \(P\) but for \([t_{k-1}, t_{k}]\), for which \(m_{k} = 0\) and
  \(M_{k} = 1\).

  Therefore, \(U(f, P) - L(f, P) = t_{k} - t_{k-1}\). Since a partition
  can be chosen such that \(t_{k} - t_{k-1}\) is arbitrarily small and
  positive, it follows that \(f\) is integrable.

  Since \(L(f, P) \leq 0 \leq U(f, P)\) and the integral is unique, it
  is equal to zero.


\end{example}
\end{document}