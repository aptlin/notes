% -*- coding: utf-8; -*-

\documentclass[11pt]{scrartcl}
\usepackage[fancy, beaue, pset, anon]{sdll}
\pSet{\nt{MAT157}{Lecture II.1}{Intro to Integral Calculus}}
\usepackage{lineno}
% ----------------------------------------------------------------------
% Page setup
% ----------------------------------------------------------------------

\pagenumbering{gobble}

% ----------------------------------------------------------------------
% Custom commands
% ----------------------------------------------------------------------

% alignment

\newcommand*{\LongestHence}{$\Rightarrow$}% function name
\newcommand*{\LongestName}{$f_o(-x)+f_e(-x)$}% function name
\newcommand*{\LongestValue}{$(-a)x +(-a)(-y)$}% function value
\newcommand*{\LongestText}{\defi}%

\newlength{\LargestHenceSize}%
\newlength{\LargestNameSize}%
\newlength{\LargestValueSize}%
\newlength{\LargestTextSize}%

\settowidth{\LargestHenceSize}{\LongestHence}%
\settowidth{\LargestNameSize}{\LongestName}%
\settowidth{\LargestValueSize}{\LongestValue}%
\settowidth{\LargestTextSize}{\LongestText}%

% Choose alignment of the various elements here: [r], [l] or [c]

\newcommand*{\mbh}[1]{{\makebox[\LargestHenceSize][r]{\ensuremath{#1}}}}%
\newcommand*{\mbn}[1]{{\makebox[\LargestNameSize][r]{\ensuremath{#1}}}}%
\newcommand*{\mbv}[1]{\ensuremath{\makebox[\LargestValueSize][r]{\ensuremath{#1}}}}%
\newcommand*{\mbt}[1]{\makebox[\LargestTextSize][l]{#1}}%

\newcommand{\R}[1]{\label{#1}\linelabel{#1}}
\newcommand{\lr}[1]{line~\lineref{#1}}

% ----------------------------------------------------------------------
% Launch!
% ----------------------------------------------------------------------

\begin{document}

\section*{Integral Calculus}
\label{sec:intcalc}

Motivation: finding areas

Although triangulation provides a viable approach to compute areas of
geometric shapes, it is not obvious that the non-unique partitions add
up to the same area.

Problems also arise in the calculation of the area of a circle. What's
\(\pi\)?

We introduce the notion of the \emph{area under the graph between
  \(a\) and \(b\)}, corresponding to the signed area under the graph
of \(y = f(x)\) above the \(x\)-axis and between the lines \(x=a\) and
\(x=b\).

This area can be approximated by computing \(f\) on a \emph{partition} \(P\)
of \([a, b]\), a set of points
\(a = t_{0} < t_{1} < t_{2} < \dots < t_{n} = b\).

For each \(i = 1, \dots , n\) define
\(m_{i} = \inf \set{f(x); t_{i-1} \leq x \leq t_{i}}\),
\(M_{i} = \sup\set{f(x); t_{i-1} \leq x \leq t_{i}}\).

\begin{note*}
  Since \(f\) is not assumed to be continuous, note that \(\inf\) and
  \(\sup\) are used instead of \(\min\) and \(\max\).
\end{note*}

Assume \(f\) is bounded, so that \(m_{i}\) and \(M_{i}\) exist.

Define the \emph{lower sum} \(L(f, P)\) of the partition as follows:

\begin{equation*}
  L(f, P) = \sum_{i=1}^{n}m_{i}(t_{i} - t_{i-1})
\end{equation*}

Define the \emph{upper sum} similarly:

\begin{equation*}
  U(f, P) = \sum_{i=1}^{n}M_{i}(t_{i} - t_{i-1})
\end{equation*}

\begin{description}

\item[e.g.] for a constant function \(f(x) = c > 0\),
  \(U(f, P) = L(f, P)\) for any partition \(P\).

\end{description}

\begin{note*}

For a partition P,

  \begin{equation*}
    L(f, P) = \sum_{i=1}^{n}m_{i}(t_{i} - t_{i-1}) \leq \sum_{i=1}^{n}M_{i}(t_{i} - t_{i-1}) = U(f, P),
  \end{equation*}

  since \(m_{i} \leq M_{i}\) for all \(i\) by definition.
\end{note*}

\end{document}
%%% Local Variables:
%%% mode: latex
%%% TeX-engine: xetex
%%% TeX-master: t
%%% End: