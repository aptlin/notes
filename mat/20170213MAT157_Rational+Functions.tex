% -*- coding: utf-8; -*-
%%% Local Variables:
%%% mode: latex
%%% TeX-engine: xetex
%%% TeX-master: t
%%% End:
\documentclass[11pt]{scrartcl}
\usepackage[fancy, beaue, pset, anon]{masty}
\pSet{\nt{MAT157}{XVI}{Rational Functions}}
\usepackage{lineno}
% ----------------------------------------------------------------------
% Page setup
% ----------------------------------------------------------------------

\pagenumbering{gobble}

% ----------------------------------------------------------------------
% Custom commands
% ----------------------------------------------------------------------

% alignment

\newcommand*{\LongestHence}{$\Rightarrow$}% function name
\newcommand*{\LongestName}{$f_o(-x)+f_e(-x)$}% function name
\newcommand*{\LongestValue}{$(-a)x +(-a)(-y)$}% function value
\newcommand*{\LongestText}{\defi}%

\newlength{\LargestHenceSize}%
\newlength{\LargestNameSize}%
\newlength{\LargestValueSize}%
\newlength{\LargestTextSize}%

\settowidth{\LargestHenceSize}{\LongestHence}%
\settowidth{\LargestNameSize}{\LongestName}%
\settowidth{\LargestValueSize}{\LongestValue}%
\settowidth{\LargestTextSize}{\LongestText}%

% Choose alignment of the various elements here: [r], [l] or [c]

\newcommand*{\mbh}[1]{{\makebox[\LargestHenceSize][r]{\ensuremath{#1}}}}%
\newcommand*{\mbn}[1]{{\makebox[\LargestNameSize][r]{\ensuremath{#1}}}}%
\newcommand*{\mbv}[1]{\ensuremath{\makebox[\LargestValueSize][r]{\ensuremath{#1}}}}%
\newcommand*{\mbt}[1]{\makebox[\LargestTextSize][l]{#1}}%

\newcommand{\R}[1]{\label{#1}\linelabel{#1}}
\newcommand{\lr}[1]{line~\lineref{#1}}

% ----------------------------------------------------------------------
% Launch!
% ----------------------------------------------------------------------

\begin{document}

\section{Rational Functions}

The key in integrating a function in the form $\frac{p(x)}{q(x)}$ is factoring the denominator. The following theorem can be proven:

\begin{theorem}
Any real polynomial can be factored as a unique product up to the order of linear terms in the form $ax+b$ and irreducible quadratic terms $Ax^2+Bx+C$ with $B^2-4AC < 0$.
\end{theorem}

Given a rational function $\frac{p(x)}{q(x)}$, if $\deg p \geq \deg q$, then a polynomial division can be performed:

\[
\frac{p(x)}{q(x)} = P(x) + \frac{r(x)}{q(x)},
\]

with $\deg(r)<\deg(q)$ and $P$ a polynomial. Thus, it suffices to show how to integrate $\frac{p}{q}$ when $\deg p < \deg q$.

The following result can be shown:

\begin{theorem}
  Any rational function in the form $\frac{p}{q}$ when
  $\deg p \leq \deg q$ can be expressed as a sum of terms, each of which
  has only one term in the denominator, possibly repeated.
\end{theorem}

\begin{description}

\item[e.g.] $\frac{1}{x^2-1}= -\frac{1}{2(x+1)}+\frac{1}{2(x-1)}$

\end{description}



\end{document}