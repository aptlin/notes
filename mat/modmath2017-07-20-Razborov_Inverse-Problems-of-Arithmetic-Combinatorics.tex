% -*- coding: utf-8; -*-
%%% Local Variables:
%%% mode: latex
%%% TeX-engine: xetex
%%% TeX-master: t
%%% End:
\documentclass[11pt]{scrartcl}
\usepackage[fancy, beaue, pset, anon]{masty}
\pSet{\nt{Razborov}{1}{Inverse Problems of Arithmetic Combinatorics}}
  \usepackage{lineno}
  % ----------------------------------------------------------------------
  % Page setup
  % ----------------------------------------------------------------------

  \pagenumbering{gobble}

  % ----------------------------------------------------------------------
  % Custom commands
  % ----------------------------------------------------------------------

  % alignment

  \newcommand*{\LongestHence}{$\Rightarrow$}% function name
  \newcommand*{\LongestName}{$f_o(-x)+f_e(-x)$}% function name
  \newcommand*{\LongestValue}{$(-a)x +(-a)(-y)$}% function value
  \newcommand*{\LongestText}{\defi}%

  \newlength{\LargestHenceSize}%
  \newlength{\LargestNameSize}%
  \newlength{\LargestValueSize}%
  \newlength{\LargestTextSize}%

  \settowidth{\LargestHenceSize}{\LongestHence}%
  \settowidth{\LargestNameSize}{\LongestName}%
  \settowidth{\LargestValueSize}{\LongestValue}%
  \settowidth{\LargestTextSize}{\LongestText}%

  % Choose alignment of the various elements here: [r], [l] or [c]

  \newcommand*{\mbh}[1]{{\makebox[\LargestHenceSize][r]{\ensuremath{#1}}}}%
  \newcommand*{\mbn}[1]{{\makebox[\LargestNameSize][r]{\ensuremath{#1}}}}%
  \newcommand*{\mbv}[1]{\ensuremath{\makebox[\LargestValueSize][r]{\ensuremath{#1}}}}%
  \newcommand*{\mbt}[1]{\makebox[\LargestTextSize][l]{#1}}%

  \newcommand{\R}[1]{\label{#1}\linelabel{#1}}
  \newcommand{\lr}[1]{line~\lineref{#1}}

  % ----------------------------------------------------------------------
  % Launch!
  % ----------------------------------------------------------------------

  \begin{document}

  \section{Inverse Problems of Arithmetic Combinatorics}

  Take a set of $n$ elements $A$. It does not really matter what the
  elements are -- only the size of the set is key.

  Consider a finite abelian group $A$. Suppose that for any $a\in \ZZ$,
  $a\neq 0$, the sum of $n$ $a$'s is not equal to zero.

  Now, let's think about Minkowski sums of the set $A$ with itself.
  What can we say about its size?

  Trivially, we can bound $\abs{A+A}$ as
  $\abs{A} \leq \abs{A+A} \leq \abs{A}^{2}$.

  \begin{theorem}
    $\abs{A+A} \geq 2 \abs{A} - 1$.
  \end{theorem}

  \begin{proof}
    \hfill

    Write two columns representing elements as hinges, with one hinge
    over another if an element is bigger than the other. Imagine a
    stick lying on these hinges. Assume first that the stick lies on
    the lowest hinge. Each time another element is added on the right
    to the element on the left, the stick goes up.

  \end{proof}

  \begin{theorem}[Cauchy-Davenport's Theorem]
    Suppoe that two sets $A$ and $B$ are given such that
    $A, B \suq \ZZ_{p}$, where $p$ is prime. Then
    $\abs{A+B} \geq \min(\abs{A} + \abs{B} - 1, p)$.
  \end{theorem}

  Now, denote $\sigma[A] = \frac{\abs{A+A}}{\abs{A}}$ as $\sigma(A)$.
  Then
  $\sigma[A] \leq 3 \ra \exists P (\abs{P} = 2 \abs{A} | A \suq P))$.

  We can construct Freiman isomorphisms, such that
  $a_{1} + a_{2} \neq a_{3} + a_{4} \lra \phi(a_{1}) + \phi(a_{2})
  \neq \phi(a_{3}) + \phi(a_{4})$, while
  $a_{1} + \dots + a_{8} = a_{9} + \dots + a_{16}$.

  In this way, we construct sets with the low value of $\sigma$, so
  that $\sigma[A] \leq \frac{2^{d}}{\rho}$ and
  $\frac{\abs{A}}{Q} \geq \rho$ by taking a parallelopiped, choosing a
  dense subset and projecting onto the integer lattice.

  \begin{theorem}[Freiman's Theorem]
    Suppose that $K>0$ is a constant.

    Let $A\suq \ZZ$ and $\sigma[A] \leq K$. Then $A\suq Q$, and $Q$ is
    a $\delta(K)$-dimensional arithmetic progression, where $\delta$
    is some function, and $\abs{A} \geq \rho(K) \abs{Q}$.
  \end{theorem}

  For applications of Freiman's theorem, see Rouge inqualities,
  stating that, if $\sigma[A] \leq K$, then $\abs{A-A}$,
  $\abs{A+A - A}$, $\dots$, $\abs{mA-nA} \leq K^{m+n}\abs{A}$.

  Another important result was obtained by Bogolyubov:

  \begin{lemma}[Bogolyubov's Lemma]
    For any $A$ there exists a generalised arithmetic progression $Q$
    such that $Q\suq 2A - 2A$ and $\abs{Q} \ geq \rho \abs{2A-2A}$.
  \end{lemma}

  \subsection{Discrete Fourier Transformation}

  Consider homomorphisms in the form $\chi: \ZZ_{p} \to \CC^{*}$ such
  that $\chi(a+b) = \chi(a)\chi(b)$, $\chi(1) = \xi$, and
  $\chi(0) = 1 = \xi^{p}$.

  Then $\chi_{m}(a) = e^{\frac{2 \pi ami}{p}}$.

  Take $f:\ZZ_{p} \to \CC$. We have
  $\wh{f}(\chi) = \frac{1}{p} \sum_{a\in \ZZ_{p}} f(a)\chi(a)$.

  Define $\mathbf{1}_{A}(a = \begin{cases}
    1, a\in A\\
    0, a\not\in A
  \end{cases})$.

  Parseval's inequality states that
  $Spec_{\alpha}(A) \leq \alpha^{2}$, while
  $\wh{1}_{A}(\chi) \geq \alpha$.

  Now, let's consider convolutions of characteristic functions.
  $(f * g)(z) = \sum_{x+y = z} f(x)g(y) $.

  For example, if f and g are probability distributions, then their convolution is also a probability distribution.

\end{document}
