% -*- coding: utf-8; -*-
%%% Local Variables:
%%% mode: latex
%%% TeX-engine: xetex
%%% TeX-master: t
%%% End:
\documentclass[11pt]{scrartcl}
\usepackage[fancy, beaue, pset, anon]{masty}
\pSet{\nt{A.P.Veselov}{1}{Special Functions: From Gamma to Zeta}}
  \usepackage{lineno}
  % ----------------------------------------------------------------------
  % Page setup
  % ----------------------------------------------------------------------

  \pagenumbering{gobble}

  % ----------------------------------------------------------------------
  % Custom commands
  % ----------------------------------------------------------------------

  % alignment

  \newcommand*{\LongestHence}{$\Rightarrow$}% function name
  \newcommand*{\LongestName}{$f_o(-x)+f_e(-x)$}% function name
  \newcommand*{\LongestValue}{$(-a)x +(-a)(-y)$}% function value
  \newcommand*{\LongestText}{\defi}%

  \newlength{\LargestHenceSize}%
  \newlength{\LargestNameSize}%
  \newlength{\LargestValueSize}%
  \newlength{\LargestTextSize}%

  \settowidth{\LargestHenceSize}{\LongestHence}%
  \settowidth{\LargestNameSize}{\LongestName}%
  \settowidth{\LargestValueSize}{\LongestValue}%
  \settowidth{\LargestTextSize}{\LongestText}%

  % Choose alignment of the various elements here: [r], [l] or [c]

  \newcommand*{\mbh}[1]{{\makebox[\LargestHenceSize][r]{\ensuremath{#1}}}}%
  \newcommand*{\mbn}[1]{{\makebox[\LargestNameSize][r]{\ensuremath{#1}}}}%
  \newcommand*{\mbv}[1]{\ensuremath{\makebox[\LargestValueSize][r]{\ensuremath{#1}}}}%
  \newcommand*{\mbt}[1]{\makebox[\LargestTextSize][l]{#1}}%

  \newcommand{\R}[1]{\label{#1}\linelabel{#1}}
  \newcommand{\lr}[1]{line~\lineref{#1}}

  % ----------------------------------------------------------------------
  % Launch!
  % ----------------------------------------------------------------------

  \begin{document}

  \section{A.P.Veselov (ULoughborough): Special Functions: From Gamma to Zeta}
  \subsection{The Story of $\zeta$}

  Integrable systems and special functions share at least one property
  -- they have not been properly defined. In particular, special
  functions are those which were considered as special by (?) Watson
  and Whittaker.

  Almost all the special functions arise from the study of
  differential equations, but $\Gamma$ and $\zeta$ are important
  exceptions.

  We will need two formulaes: the sum of consecutive $n-1$ integers,
  equal to $\frac{n(n-1)}{2}$, and the sum of the squares of
  consecutive $n-1$ integers, equal to $\frac{n(n-1)(2n-1)}{6}$.
  Moreover,
  $S_{m}(n) = \sum_{j=1}^{n-1}j^{m} = \frac{1}{m+1}(B_{m+1}(n) -
  B_{m+1})$, where $B_{m+1}$ is Bernoulli's constant. (?)

  It is worthwhile to note a function with a range of wonderful
  properties:

  \begin{equation}
    \frac{te^{tx}}{e^{t}-1} = \sum_{k=0}^{\infty}\frac{B_{k}(x)}{k!}t^{k},
  \end{equation}

  where $B_{k}(x) = \sum_{j=0}^{k}C_{k}^{j}B_{k}x^{kj}$ and
  $C=(B+x)^{k}$. From this equation we can derive that
  $B_{m}(x) = (-1)^{m}B_{m}(1-x)$, which is symmetric around
  $x=\frac{1}{2}$.
  
  In 1689 Jacob Bernoulli formulated Basel's problem, which, however,
  was already known to Wallis in 1668. The problem required to find
  what the limit of $\sum_{i=0}^{\infty} \frac{1}{i^{2}}$ is. Euler
  was obsessed with the problem, calculating the limit up to 20
  significant figures, eventually finding the answer --
  $\frac{\pi^{2}}{6}$.

  Bernoulli's numbers have a special significance. Let's list the
  first few of them -- $B_{0} = 1$, $B_{1} = -\frac{1}{2}$,
  $B_{2} = \frac{1}{6}$, $B_{3} = B_{5} = \dots = 0$,
  $B_{4} = -\frac{1}{30} = B_{8}$, $B_{6} = \frac{1}{42}$,
  $B_{10} = \frac{5}{66}$, $B_{12} = - \frac{691}{2730}$.

  The exciting fact is that we, knowing Bernoulli's numbers, can
  represent $\zeta$ in a particularly nice form:
  
  \begin{equation}
    \zeta(2k) = (-1)^{k-1}\frac{(2\pi)^{2k}}{2(2k)!}B_{2k}.
  \end{equation}

  Euler knew that $\zeta(s) = \sum_{n=1}^{\infty} \frac{1}{n^{s}}$,
  where $s > 1$ to guarantee the convergence. However, he was also
  adept at seeing connections between different representations, and
  derived that

  \begin{equation}
    \zeta(s) = \prod_{p\in \PP}\frac{1}{1-\frac{1}{p^{s}}} = \prod(1+\frac{1}{p^{s}} + \frac{1}{p^{2s}} + \dots).
  \end{equation}

  What did Riemann add to the definition of the $\zeta$ function?

  In 1859, Riemann analytically continued its definition to
  $s = \sigma + it \in \CC$.

  

  \subsection{The Story of $\Gamma$}

  In January 1730 Goldbach wrote a letter to Euler stating his idea of
  generalising the factorial function for non-integer numbers. Euler,
  in turn, constructed such a function for $x > 0$:

  
  \begin{equation}
    \Gamma(x) = \int_{0}^{\infty}t^{x-1}e^{-t}\dif t,
  \end{equation}

  which has a nice property that $\Gamma(x+1) = x\Gamma(x)$.

  We can analytically continue the function to negative real numbers
  by using this property. Moreover, there exists a \textit{formula of completion}:

  
  \begin{equation}
    \Gamma(x)\Gamma(1-x) = \frac{\pi}{\sin(\pi x)},
  \end{equation}

  which follows from the fact that
  $\frac{1}{\Gamma(x)\Gamma(1-x) = x
    \prod(1-\frac{x^{2}}{k^{2}}) = \frac{\sin \pi x}{\pi}}$.

  From this, we can deduce that $(-\frac{1}{2})! = \sqrt(\pi)$, which
  helps us compute the Gaussian integral
  $I = \int_{-\infty}^{\infty}e^{-x^{2}}\dif x = \sqrt(\pi)$.

  Moreover, calculating the area of a unit ball, we obtain
  $Vol S^{n-1} = \frac{2(\sqrt{\pi})^{N}}{\Gamma(\frac{N}{2})} = N
  Vol(B^{N})$.

  \subsection{Connection between $\zeta$ and $\Gamma$}  

  How do $\zeta$ and $\Gamma$ relate?
  
  A beatiful connection can be derived --
  $\Gamma(s)\zeta(s) = \int_{x=0}^{\infty} \frac{x^{s-1}}{e^{x} - 1}
  \dif x$.

  The idea for the proof is as follows. First, note that

  \begin{equation}
    \frac{1}{e^{x}- 1} = \frac{1}{e^{x}(1-e^{-x})} = \sum_{k=1}^{\infty}e^{-kx},
  \end{equation}

  which means that
  $\sum_{k=1}^{\infty} \int_{x=0}^{\infty}x^{s-1}e^{-kx} \dif x$, and
  thus, if $t = kx$, then the equation can be rewritten as
  $(\infty_{t=0}^{\infty}t^{s-1}e^{-t}\dif
  t)(\sum_{k=1})^{\infty}\frac{1}{k^{s}}$.

  Moreover,
  $\zeta(s) = \frac{\Gamma(1-s)}{2\pi i}\int \frac{z^{s-1}}{e^{-z}-1}
  dz$. This analytically continued function is meromorphic, with the
  pole at $s = 1$.

  The symmetry of $zeta(s)$ is striking, and allows us to infer immediately its trivial zeros:

  \begin{equation}
    \zeta(s) = (2^{s}\pi^{s-1}\sin(\frac{\pi s}{2}) \Gamma(1-s) \zeta(1-s)).
  \end{equation}

  Trivial zeros occur when $\sin (\frac{\pi s}{2}) = 0$, and thus
  $s = -2k$, $k\in \NN$.

  Moreover, we can deduce that
  $\zeta(-n) = (-1)^{n} \frac{B_{n+1}}{n+1}$.

  \subsection{Around the Riemann Hypothesis}

  Empirical evidence from direct calculation have shown that all the
  non-trivial zeros of the Riemann zeta function lie on the critical
  line with $\Re = \frac{1}{2}$.

  If we look at the original work by Riemann, he noted that the first
  few dozen non-trivial zeros do not come in pairs.

  One of the applications which a resolution of the Riemann Hypothesis
  would bring is the illumination of how primes are distributed. The
  Prime Number Theorem states that $\pi(x) \sim \frac{x}{\log x}$,
  where $\pi(x)$ is the prime-counting function, yielding the number
  of primes less than or equal to $x$, and $f(x) \tilde g(x)$ means
  that $\lim_{x \to \infty} \frac{f(x)}{g(x)} = 1$.

  Chebyshev have shown that, for $\psi(x) = \sum \log p$, where
  $p^{k} \leq x$ and $p\in \PP$, the Prime Number Theorem is
  equivalent to the theorem that $\psi(x) \tilde x$.

  If the Riemann Hypothesis holds, then the approximation given by the
  PNT is the best possible.
\end{document}
