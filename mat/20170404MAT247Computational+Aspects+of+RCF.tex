% -*- coding: utf-8; -*-
%%% Local Variables:
%%% mode: latex
%%% TeX-engine: xetex
%%% TeX-master: t
%%% End:
\documentclass[11pt]{scrartcl}
\usepackage[fancy, beaue, pset, anon]{masty}
\pSet{\nt{MAT247}{}{Computational Aspects of RCF}}
\usepackage{lineno}
% ----------------------------------------------------------------------
% Page setup
% ----------------------------------------------------------------------

\pagenumbering{gobble}

% ----------------------------------------------------------------------
% Custom commands
% ----------------------------------------------------------------------

% alignment

\newcommand*{\LongestHence}{$\Rightarrow$}% function name
\newcommand*{\LongestName}{$f_o(-x)+f_e(-x)$}% function name
\newcommand*{\LongestValue}{$(-a)x +(-a)(-y)$}% function value
\newcommand*{\LongestText}{\defi}%

\newlength{\LargestHenceSize}%
\newlength{\LargestNameSize}%
\newlength{\LargestValueSize}%
\newlength{\LargestTextSize}%

\settowidth{\LargestHenceSize}{\LongestHence}%
\settowidth{\LargestNameSize}{\LongestName}%
\settowidth{\LargestValueSize}{\LongestValue}%
\settowidth{\LargestTextSize}{\LongestText}%

% Choose alignment of the various elements here: [r], [l] or [c]

\newcommand*{\mbh}[1]{{\makebox[\LargestHenceSize][r]{\ensuremath{#1}}}}%
\newcommand*{\mbn}[1]{{\makebox[\LargestNameSize][r]{\ensuremath{#1}}}}%
\newcommand*{\mbv}[1]{\ensuremath{\makebox[\LargestValueSize][r]{\ensuremath{#1}}}}%
\newcommand*{\mbt}[1]{\makebox[\LargestTextSize][l]{#1}}%

\newcommand{\R}[1]{\label{#1}\linelabel{#1}}
\newcommand{\lr}[1]{line~\lineref{#1}}

% ----------------------------------------------------------------------
% Launch!
% ----------------------------------------------------------------------

\begin{document}

\section{Computational Aspects of RCF}

\subsection{Review}

Suppose $x\in V$, $x\neq 0$, and let $T \in \End(V)$.

The $T$-annihilator of $x$ is a monic polynomial $p(t)$ of the least
degree such that $p(T)x = 0$.


Note that the characteristic polynomial of $T$, $f(t)$, can be
represented in the form $(-1)^{\dim V}\prod_{i=1}^s\phi_s(t)^{m_s}$,
so that $V = \bigoplus_{i=1}^sK_{\phi_i}$ and
$\dim K_{\phi_i} = m_i\deg \phi_i$.

Each $K_{\phi_i}$ has a basis $\beta_i$ which is a disjoint union of
$T$-cyclic bases in the form $\set{x, Tx, \dots, T^{k-1}x}$.

Let $\beta = \bigcup_{i=1}^s \beta_i$. Then $[T]_{\beta}$ is in RCF.

\subsection{How to Find RCF}

For each $K_{\phi_i}$ we can write a dot diagram consisting of
representations of cycles inside some $\beta_i$.

The $T$-annihilator of $x_j$ is $\phi_j(t)^{k_j}$ for some $k_j\in\ZZ^+$.

Order these $x_j$'s so that $k_1\geq k_2\geq \dots \geq k_s$.

Note that $\abs{\beta_{x_j}}=k_j\deg \phi_j $.

Moreover, the number of dots in the first $s$ rows is
$\frac{\nll \phi_i(T)^s }{\deg \phi_i} = \frac{\dim K_{\phi_i}}{\deg
  \phi_i} = m_i$.

\begin{example}

  Suppose that $\FF = \ZZ_5$ and let $T \in \End(V)$ be such that the
  characteristic polynomial of $T$ is $f(t)= (t^2+1)(t^2+2)(t^3+3)$.

  Since $\FF = \ZZ_5$, then $f(t) = (t-2)(t+2)(t^2+2)(t^3+3)$.

  Since $m_1=m_2=m_3=m_4=1$, we know that each dot diagram consists of only one dot.

  Each $K_{\phi_i}$ has a $T$-cyclic basis, and a $T$-annihilator is
  $\phi_i(t)$. 

  Thus, the RCF is $
  \begin{pmatrix}
  C_{t-2}& & &\\
  & C_{t+2} & &\\
  & & C_{t^2+2} &\\
  & & & C_{t^3+3}
\end{pmatrix}
$.
\end{example}

\begin{example}
\label{sec:how-find-rcf}
Suppose that $\FF = \ZZ_5$.

Let $A = 
\begin{pmatrix}
1 & 1& 1& 0 &0\\
2& 4 &2 &4 &0\\
& & 2&1 &1\\
& & 4&3 &0\\
& & & &0
\end{pmatrix}
$.

Then $f(t)= -(t^2+2)(t^2+2)t = -t(t^2+2)^2$.

Note that $\phi_1(t)=t$ and $m_1=1$, while $\phi_2(t)=t^2+2$ and $m_2=2$.

The number of dots in the first row is
$\frac{\nll(A^{2}+2I)}{\deg \phi_2} = \frac{4}{2} =2$.

So the dot diagram of $K_{\phi_2}$ consists of two dots in a row, which means that
$R(A) = 
\begin{pmatrix}
0 & -2 & & &\\
1 & 0 & & &\\
& & 0 & -2 &\\
& & 1  & 0 &\\
& & & &0
\end{pmatrix}
$.

\end{example}

\subsection{How to Find a Rational Canonical Basis}

In Example \ref{sec:how-find-rcf}, $K_{\phi_2} = \ker (A^2+2I)$ is
$4$-dimensional, so each element $x\in K_{\phi_2}\setminus \set{0}$
has $L_A$-annihilator $t^2+2$, so a $2$-dimensional $L_A$-cyclic
subspace is generated.

Pick any vector $x\in K_{\phi_2}\setminus \set{0}$ to obtain the first
cycle, and pick any other vector not in the span of $\beta_x$ to get
another cycle.
\end{document}
