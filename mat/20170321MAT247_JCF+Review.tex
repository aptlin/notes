% -*- coding: utf-8; -*-
%%% Local Variables:
%%% mode: latex
%%% TeX-engine: xetex
%%% TeX-master: t
%%% End:
\documentclass[11pt]{scrartcl}
\usepackage[fancy, beaue, pset, anon]{masty}
\pSet{\nt{MAT247}{}{Minimal Polynomial}}
\usepackage{lineno}
% ----------------------------------------------------------------------
% Page setup
% ----------------------------------------------------------------------

\pagenumbering{gobble}

% ----------------------------------------------------------------------
% Custom commands
% ----------------------------------------------------------------------

% alignment

\newcommand*{\LongestHence}{$\Rightarrow$}% function name
\newcommand*{\LongestName}{$f_o(-x)+f_e(-x)$}% function name
\newcommand*{\LongestValue}{$(-a)x +(-a)(-y)$}% function value
\newcommand*{\LongestText}{\defi}%

\newlength{\LargestHenceSize}%
\newlength{\LargestNameSize}%
\newlength{\LargestValueSize}%
\newlength{\LargestTextSize}%

\settowidth{\LargestHenceSize}{\LongestHence}%
\settowidth{\LargestNameSize}{\LongestName}%
\settowidth{\LargestValueSize}{\LongestValue}%
\settowidth{\LargestTextSize}{\LongestText}%

% Choose alignment of the various elements here: [r], [l] or [c]

\newcommand*{\mbh}[1]{{\makebox[\LargestHenceSize][r]{\ensuremath{#1}}}}%
\newcommand*{\mbn}[1]{{\makebox[\LargestNameSize][r]{\ensuremath{#1}}}}%
\newcommand*{\mbv}[1]{\ensuremath{\makebox[\LargestValueSize][r]{\ensuremath{#1}}}}%
\newcommand*{\mbt}[1]{\makebox[\LargestTextSize][l]{#1}}%

\newcommand{\R}[1]{\label{#1}\linelabel{#1}}
\newcommand{\lr}[1]{line~\lineref{#1}}

% ----------------------------------------------------------------------
% Launch!
% ----------------------------------------------------------------------

\begin{document}

\section{Review}

We have show that a Jordan Canonical Form is unique using dot diagrams
for $T_{K_{\lambda}}$ and any eigenvalue $\lambda$.

For this purpose we have found a cycle basis for $K_{\lambda}$, and,
ordering subbases by their sizes with the subbasis of greatest size
being the first, we can construct a dot diagram, which has several
nice properties.

For example, the number of dots in the diagram is equal to the
dimension of $K_{\lambda}$, and the number of dots in the first $s$
rows is equal to the nullity of $(T-\lambda I)^s$.

We now can finde a Jordan Canonical basis for the sequences such that,
for example, $\ker (A-2i) = \spn{\cv{0;1;0;0}, \cv{0;0;0;1}}$.

Taking $(A-3I) v= \cv{1;-1;-1}$, we can solve for $v$, and thus obtain a cycle basis. Having done that, the JCF is easily found.

Suppose now a dot diagram is given. Starting in the top left, we
obtain thatfor vectors $v_{1}$ and $v_2$ such that $(A-I)v_1 $ and
$(A-I)v_2$ are in $\ker(A-I)\cap \img (A-I)$. We already saw that
$\rank (A-I) = 2$, and hence these form a matrix.

We now solve for $v_{1}$ and $v_{2}$ by noting that they are
eigenvectors to obtain  $\cv{0;0;1;0;0}$ and $v_2=\cv{0;0;0;1;0}$.

Extending $(A-I)v_{1}$, $(A-I)v_{2}$ to a basis of $\ker(A-I)$, eg by
taking $v_3 =\cv{0;0;0;1;-1}$, we can see that $[L_{A}]_{\beta}$ is in
JCF.

We can therefore formulate a general strategy for finding  Jordan Canonical basis:

\begin{itemize}
\item Figure out dot diagrams
\item For a fixed eigenvalue $\lambda$, working from left to right in
  the 1st row of the dot diagram (among initial eigenvectorrs. 
\item Solve for the end vectors by considering the system of the form $(T-\lambda I)^{l-1}v =$ initial vectors.
\end{itemize}
\begin{example}

  For the first two initial vectors we have that $(T-\lambda I)^2v_1$
  and $(T-\lambda)^2v_2 \in \ker(T-\lambda)\cap(\img(T-\lambda I)^{2})$.

\end{example}
\end{document}
