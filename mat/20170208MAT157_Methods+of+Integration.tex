% -*- coding: utf-8; -*-
%%% Local Variables:
%%% mode: latex
%%% TeX-engine: xetex
%%% TeX-master: t
%%% End:
\documentclass[11pt]{scrartcl}
\usepackage[fancy, beaue, pset, anon]{masty}
\pSet{\nt{MAT157}{XIV}{Methods of Integration}}
\usepackage{lineno}
% ----------------------------------------------------------------------
% Page setup
% ----------------------------------------------------------------------

\pagenumbering{gobble}

% ----------------------------------------------------------------------
% Custom commands
% ----------------------------------------------------------------------

% alignment

\newcommand*{\LongestHence}{$\Rightarrow$}% function name
\newcommand*{\LongestName}{$f_o(-x)+f_e(-x)$}% function name
\newcommand*{\LongestValue}{$(-a)x +(-a)(-y)$}% function value
\newcommand*{\LongestText}{\defi}%

\newlength{\LargestHenceSize}%
\newlength{\LargestNameSize}%
\newlength{\LargestValueSize}%
\newlength{\LargestTextSize}%

\settowidth{\LargestHenceSize}{\LongestHence}%
\settowidth{\LargestNameSize}{\LongestName}%
\settowidth{\LargestValueSize}{\LongestValue}%
\settowidth{\LargestTextSize}{\LongestText}%

% Choose alignment of the various elements here: [r], [l] or [c]

\newcommand*{\mbh}[1]{{\makebox[\LargestHenceSize][r]{\ensuremath{#1}}}}%
\newcommand*{\mbn}[1]{{\makebox[\LargestNameSize][r]{\ensuremath{#1}}}}%
\newcommand*{\mbv}[1]{\ensuremath{\makebox[\LargestValueSize][r]{\ensuremath{#1}}}}%
\newcommand*{\mbt}[1]{\makebox[\LargestTextSize][l]{#1}}%

\newcommand{\R}[1]{\label{#1}\linelabel{#1}}
\newcommand{\lr}[1]{line~\lineref{#1}}

% ----------------------------------------------------------------------
% Launch!
% ----------------------------------------------------------------------

\begin{document}

\section{Methods of Integration}
\subsection{Integration by Parts}
\begin{example}
  \begin{align}
    \int \cos^2(x) \dif x & = \cos x \sin x - \int -\sin x \sin x \dif x \\
                          & =\cos x \sin x + x - \int \cos^2x,
  \end{align}

  and thus
  \begin{align}
    \int \cos^2(x) \dif x & = \half(\cos x \sin x + x).
  \end{align}
\end{example}
\begin{example}

  \begin{align}
    \int \log x \dif x & = x \log x - \int x \frac{1}{x} \\
                       & = x\log x - x
  \end{align}
\end{example}
\subsection{Reverse Chain Rule}

\begin{example}

  $\frac{\dif }{\dif x}e^{-x^2} = -2xe^{-x^2}$, and thus $-\int 2x e^{-x^2} = e^{-x^2}$.
\end{example}
\begin{example}

$\int x^2 \sin x^3 = \frac{1}{3}(-\cos x^3$.

\end{example}
\begin{example}

  $\int \cos x \sin ^4x \dif x = \frac{1}{5}\sin^5x$.

\end{example}

\begin{theorem}
  If $\int_a^bf\circ g(x) g'(x) \dif x = \int_{g(a)}^{g(b'}f(i) \dif u$
\end{theorem}
\begin{example}

  \begin{align}
    \int \frac{x}{1+x^2} & = \half \int \frac{1}{u} \dif u \\
                         & =\half \log(1+x^2)
  \end{align}
\end{example}

\begin{example}

  \begin{align}
    \int \frac{\log^5}{x} \dif x &= \int u^{5} \dif u\\
                                 &= \frac{1}{6}\int\log^{}
  \end{align}

\end{example}

\begin{example}
  The following example is better solved using integration by parts:
  
  \begin{align}
    \int x e^{2x} \dif x & = \frac{xe^{2x}}{2} - \frac{1}{2}\int e^{2x}\dif x \\
                         & =(\frac{x}{2}-\frac{1}{4})e^{2x}
  \end{align}

\end{example}
\end{document}