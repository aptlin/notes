%%% Local Variables:
%%% mode: latex
%%% TeX-master: t
%%% End:

\documentclass[11pt]{scrartcl}
\usepackage[beaue, pset, anon]{sdll}
\pSet{\hw{MAT247}{I}{4}}
\usepackage{lineno}
% ----------------------------------------------------------------------
% Page setup
% ----------------------------------------------------------------------

\pagenumbering{gobble}

% ----------------------------------------------------------------------
% Custom commands
% ----------------------------------------------------------------------

% alignment

\newcommand*{\LongestHence}{$\Rightarrow$}% function name
\newcommand*{\LongestName}{$f_o(-x)+f_e(-x)$}% function name
\newcommand*{\LongestValue}{$(-a)x +(-a)(-y)$}% function value
\newcommand*{\LongestText}{\defi}%

\newlength{\LargestHenceSize}%
\newlength{\LargestNameSize}%
\newlength{\LargestValueSize}%
\newlength{\LargestTextSize}%

\settowidth{\LargestHenceSize}{\LongestHence}%
\settowidth{\LargestNameSize}{\LongestName}%
\settowidth{\LargestValueSize}{\LongestValue}%
\settowidth{\LargestTextSize}{\LongestText}%

% Choose alignment of the various elements here: [r], [l] or [c]

\newcommand*{\mbh}[1]{{\makebox[\LargestHenceSize][r]{\ensuremath{#1}}}}%
\newcommand*{\mbn}[1]{{\makebox[\LargestNameSize][r]{\ensuremath{#1}}}}%
\newcommand*{\mbv}[1]{\ensuremath{\makebox[\LargestValueSize][r]{\ensuremath{#1}}}}%
\newcommand*{\mbt}[1]{\makebox[\LargestTextSize][l]{#1}}%

\newcommand{\R}[1]{\label{#1}\linelabel{#1}}
\newcommand{\lr}[1]{line~\lineref{#1}}

% ----------------------------------------------------------------------
% Launch!
% ----------------------------------------------------------------------

\begin{document}

% ----------------------------------------------------------------------
% Body
% ----------------------------------------------------------------------
\begin{linenumbers}
  \begin{claim*}
    Suppose that $M\in M_{n\times n}(\FF)$ has three distinct
    eigenvalues $\lambda, \mu, \nu$ and that
    $\dim E_{\lambda} = n - 2$. Then $M$ is diagonalizable.
  \end{claim*}

  \begin{proof}
    Let $T = \LL_{M}$.
    Note that $T$ is diagonalizable if and only if
    \[\dim V = \dim E(\lambda, T) + \dim E(\mu, T) + \dim E(\nu, T).\]

    We prove that this condition indeed holds.

    Since $\mu$ and $\nu$ are distinct eigenvalues,
    $\dim E(\mu, T) \geq 1$ and $\dim E(\nu, T) \geq 1$. Given that
    $\dim E_{\lambda} = n-2$, we obtain

    \[ \dim E(\lambda, T) + \dim E(\mu, T) + \dim E(\nu, T) \geq n = \dim V.\]

    Since $V = E(\lambda, T) \oplus E(\mu, T) \oplus E(\nu, T)$,
    \[\dim E(\lambda, T) + \dim E(\mu, T) + \dim E(\nu, T) \leq n =
      \dim V.\]

    Therefore,
    \[\dim V = \dim E(\lambda, T) + \dim E(\mu, T) + \dim E(\nu, T),\]
    and thus $T$ is diagonalizable.
  \end{proof}

  \begin{problem*}
    Give an example of a matrix with precisely three distinct
    eigenvalues that is not diagonalizable.
  \end{problem*}

  \begin{soln}
    By the Claim above, if $M$ is not diagonalizable but has three
    distinct eigenvalues, neither of them has
    $\dim E_{\lambda_{i}} = n - 2$.

    Moreover, if $M$ is not diagonalizable, then
    \[\dim V > \dim E(\lambda, T) + \dim E(\mu, T) + \dim E(\nu, T).\]

    Suppose $M \in M_{4 \times 4}(\QQ)$ is defined over $\QQ$.

    Take $M =
    \begin{pmatrix}
      0 & 0 & 0  & 1 \\
      0 & 1 & 0  & 0 \\
      0 & 0 & -1 & 0 \\
      0 & 0 & 5 & 0
    \end{pmatrix}
    $.

    Consider $\det(M-\lambda I) = 0$.

    \begin{align}
      \det(M-\lambda I) =
      \det\begin{pmatrix}
        -\lambda & 0 & 0  & 1 \\
        0 & 1-\lambda & 0  & 0 \\
        0 & 0 & -1-\lambda & 0\\
        0 & 0 & 5 & -\lambda
      \end{pmatrix} = 0
    \end{align}

    Expanding along the first column and using the fact that
    $M-\lambda I$ has a normal Jordan form, we obtain
    \begin{align}
      \det(M-\lambda I) =
      -\lambda\det\begin{pmatrix}
        1-\lambda & 0          & 0 \\
        0         & -1-\lambda & 0 \\
        0         & 5          & -\lambda
      \end{pmatrix} = -\lambda^{2}(1-\lambda)(1+\lambda),
    \end{align}


    which gives possible eigenvalues of $0, 1, -1$. Note that they are
    distinct.

    For $\lambda=0$,
    \begin{align}
      \begin{pmatrix}
        0 & 0 & 0  & 1 \\
        0 & 1 & 0  & 0 \\
        0 & 0 & -1 & 0 \\
        0 & 0 & 5 & 0
      \end{pmatrix}\cv{w;x;y;z} = \bm{0},
    \end{align}
    if $x=0$, $y=0,z=0$. Thus, $\cv{1;0;0;0}$ spans $E_{0}$.

    For $\lambda = 1$,
    \begin{align}
      \begin{pmatrix}
        -1 & 0 & 0  & 1 \\
        0  & 0 & 0  & 0 \\
        0  & 0 & -2 & 0 \\
        0  & 0 & 5  & -1
      \end{pmatrix}\cv{w;x;y;z} = \bm{0},
    \end{align}
    if $w = z$, $y=0$ and $z = 5y = 0 = w$. Thus, $\cv{0;1;0;0}$ spans $E_{1}$.

  For $\lambda = -1$,
  \begin{align}
    \begin{pmatrix}
      1 & 0 & 0 & 1 \\
      0 & 2 & 0 & 0 \\
      0 & 0 & 0 & 0 \\
      0 & 0 & 5 & 1
    \end{pmatrix}\cv{w;x;y;z} = \bm{0},
  \end{align}
  if $w = -z$, $x=0$ and $z = -5y$. Thus, $\cv{5;0;1;-5}$ spans $E_{-1}$.

  Since there are only three eigenvectors, while the domain of
  $T = \LL_{M}$ has the dimension $4$, there is no basis for the
  domain consisting of eigenvectors, and thus $M$ is not
  diagonalizable, while there are three distinct eigenvalues
  corresponding to $M$.
  \end{soln}

\end{linenumbers}
\end{document}