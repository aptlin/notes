%%% Local Variables:
%%% mode: latex
%%% TeX-master: t
%%% End:

\documentclass[12pt]{article}
\usepackage[utf8]{inputenc}
%-----------------------------------------------------------
\usepackage{fullpage}
\usepackage{hyperref}
\usepackage{graphicx}
\usepackage{color}

\definecolor{mygrey}{gray}{0.90}
\raggedbottom
\raggedright
\setlength{\tabcolsep}{0in}

\usepackage{amsthm, amsmath, amssymb}
\usepackage{chngcntr}


% Adjust margins to 0.5in on all sides
\addtolength{\oddsidemargin}{-0.5in}
\addtolength{\evensidemargin}{-0.5in}
\addtolength{\textwidth}{1.0in}
\addtolength{\topmargin}{-0.5in}
\addtolength{\textheight}{1.0in}
% ----------------------------------------------------------------------
% Custom definitions
\def\Re{\mathbb{R}}
\def\P{\mathbb{P}}
\def\F{\mathbb{F}}
\def\defi{Definition of }
\def\mclo{Multiplicative Closure of }
\def\aclo{Additive Closure of }
\def\dist{Distributive Law}
\def\ainv{Existence of an Additive Inverse }
\def\minv{Existence of a Multiplicative Inverse }
\def\uainv{Uniqueness of an Additive Inverse }
\def\uminv{Uniqueness of a Multiplicative Inverse }
\def\comm{Commutative Law }
\def\tric{Trichotomy Law }
\def\assoc{Associative Law }
\def\aid{Existence of an Additive Identity }
\def\mid{Existence of a Multiplicative Identity }
\def\canc{Cancellation Property }
\def\die{Distinctness of an Additive Identity and Multiplicative Identity}

\def\ra{\Rightarrow}
\def\equ{\Leftrightarrow}
\def\v{\vspace{0.1in}}
\def\s{\\\v}
% -----------------------------------------------------------
%Custom commands
\newcommand{\resitem}[1]{\item #1 \vspace{-2pt}}
\newcommand{\resheading}[1]{{\large \colorbox{mygrey}{\begin{minipage}{\textwidth}{\textbf{#1 \vphantom{p\^{E}}}}\end{minipage}}}}
\newcommand{\ressubheading}[4]{
\begin{tabular*}{7.0in}{l@{\extracolsep{\fill}}r}
                \textbf{#1} & #2 \\
                \textit{#3} & \textit{#4} \\
\end{tabular*}\vspace{-6pt}}


\newcommand*{\LongestName}{$\Rightarrow\ (-a)(x-y)$}% function name
\newcommand*{\LongestValue}{$(-a)x +(-a)(-y)$}% function value
\newcommand*{\LongestText}{\defi - and \dist }%

\newlength{\LargestNameSize}%
\newlength{\LargestValueSize}%
\newlength{\LargestTextSize}%

\settowidth{\LargestNameSize}{\LongestName}%
\settowidth{\LargestValueSize}{\LongestValue}%
\settowidth{\LargestTextSize}{\LongestText}%

% Choose alignment of the various elements here: [r], [l] or [c]
\newcommand*{\mbn}[1]{{\makebox[\LargestNameSize][r]{\ensuremath{#1}}}}%
\newcommand*{\mbv}[1]{\ensuremath{\makebox[\LargestValueSize][l]{\ensuremath{#1}}}}%
\newcommand*{\mbt}[1]{\makebox[\LargestTextSize][l]{#1}}%

\newtheorem{theorem}{Theorem}[section]
\newtheorem*{theorem*}{Theorem}
\newtheorem{corollary}{Corollary}[theorem]
\newtheorem{lemma}[theorem]{Lemma}
\newtheorem*{lemma*}{Lemma}
\newtheorem{subtheorem}{Lemma}[theorem]
\theoremstyle{definition}
\newtheorem{definition}{Definition}[section]
\theoremstyle{remark}
\newtheorem*{remark}{Remark}
% -----------------------------------------------------------

\pagenumbering{gobble}

\counterwithin*{equation}{theorem}
\counterwithin*{equation}{corollary}
\counterwithin*{equation}{subtheorem}
\begin{document}

Let $\F$ be any field.

\begin{lemma}
  $\forall a \in \mathbb{F}: a\cdot0=0$
  \label{eq:zero}
\end{lemma}

\begin{proof}
\begin{align}
  \mbn{0+0}\ & \mbv{=0} & \mbt{\aid}\\
  \mbn{\Rightarrow a\cdot(0+0)}\ & \mbv{=a\cdot 0 + a \cdot 0} & \mbt{\dist}\\
  \mbn{}\ & \mbv{=a \cdot 0} & \mbt{\defi =}\\
  \mbn{(a\cdot 0 + a \cdot 0)-(a\cdot 0)}\ & \mbv{=a \cdot 0 -(a \cdot 0)} & \mbt{\defi =}\\
  \mbn{\Rightarrow a\cdot0 +(a\cdot0-a\cdot0)}\ & \mbv{=0} & \mbt{\assoc}\\
  \mbn{}\ & \mbv{} & \mbt{and \ainv}\nonumber\\
  \mbn{\Rightarrow a\cdot0 +0}\ & \mbv{=0} & \mbt{\ainv}\\
  \mbn{\Rightarrow a\cdot0}\ & \mbv{=0} & \mbt{\aid}\nonumber
\end{align}
\end{proof}


  \begin{lemma}
  \label{eq:zerzer}
  $\forall a,b \in \F: ab=0 \equ a=0 \vee b=0$
\end{lemma}
\begin{proof}
  \vspace{0.1in}
  By \comm and Lemma \ref{eq:zero}, $a=0 \ra ab=ba=b\cdot 0=0$.\\
  \vspace{0.1in}
  Similarly, $b=0 \ra ab=a\cdot 0=0$.
  \vspace{0.1in}
  If $ab=0$ and $b\neq 0$, $\exists\ b^{-1}: abb^{-1}=0\cdot b^{-1}$,
  hence by \comm and \minv $a\cdot 1=b^{-1}\cdot 0$, then by \mid
  and Lemma \ref{eq:zero} $a=0$.\\
  \vspace{0.1in}
  If $ab=0$ and $a\neq 0$, $\exists\ a^{-1}: a^{-1}ab=a^{-1}\cdot 0$,
  hence by \comm and Lemma \ref{eq:zero} $aa^{-1}b=0$, then by \minv
  $1\cdot b=0$, and by \comm and \mid $b\cdot 1=b=0$.\\
  \vspace{0.1in}
  If $a=0 \wedge b=0$, then by Lemma \ref{eq:zero} $ab=0\cdot0=0$
\end{proof}


\section*{}
\begin{theorem}
  $1+1+1+1=0 \in \F \ra 1+1=0$
\end{theorem}
\begin{proof}
  Consider $(1+1)\cdot(1+1)$.
  \begin{align}
    \mbn{(1+1)(1+1)}\ & \mbv{=(1+1) \cdot 1 + (1+1) \cdot 1} & \mbt{\dist} \\
    \mbn{1\cdot(1+1)+1\cdot(1+1)}\ & \mbv{=1+1+1+1} & \mbt{\dist\ and \comm}
  \end{align}

  Therefore, $1+1+1+1=(1+1)(1+1)=0$ by assumption.\s



Now, since $(1+1)(1+1)=0$ and $1+1=1+1$, by Lemma \ref{eq:zerzer}
$1+1=0$ as required.
\end{proof}


\end{document}