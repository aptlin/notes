%%% Local Variables:
%%% mode: latex
%%% TeX-master: t
%%% End:

\documentclass[12pt]{article}
\usepackage[utf8]{inputenc}
%-----------------------------------------------------------
\usepackage{fullpage}
\usepackage{hyperref}
\usepackage{graphicx}
\usepackage{color}

\definecolor{mygrey}{gray}{0.90}
\setlength{\tabcolsep}{0in}

\usepackage{amsthm, amsmath, amssymb}
\usepackage{chngcntr}


% Adjust margins to 0.5in on all sides
\addtolength{\oddsidemargin}{-0.5in}
\addtolength{\evensidemargin}{-0.5in}
\addtolength{\textwidth}{1.0in}
\addtolength{\topmargin}{-0.5in}
\addtolength{\textheight}{1.0in}
% ----------------------------------------------------------------------
% Custom definitions
\def\Re{\mathbb{R}}
\def\P{\mathbb{P}}
\def\defi{Definition of }
\def\mclo{Multiplicative Closure of }
\def\aclo{Additive Closure of }
\def\dist{Distributive Law}
\def\ainv{Existence of an Additive Inverse }
\def\minv{Existence of a Multiplicative Inverse }
\def\comm{Commutative Law }
\def\tric{Trichotomy Law }
\def\assoc{Associative Law }
\def\aid{Existence of an Additive Identity }
\def\mid{Existence of a Multiplicative Identity }
\def\canc{Cancellation Property }
\def\ra{\Rightarrow}
\def\equ{\Leftrightarrow}
% -----------------------------------------------------------
%Custom commands
\newcommand{\resitem}[1]{\item #1 \vspace{-2pt}}
\newcommand{\resheading}[1]{{\large \colorbox{mygrey}{\begin{minipage}{\textwidth}{\textbf{#1 \vphantom{p\^{E}}}}\end{minipage}}}}
\newcommand{\ressubheading}[4]{
\begin{tabular*}{7.0in}{l@{\extracolsep{\fill}}r}
                \textbf{#1} & #2 \\
                \textit{#3} & \textit{#4} \\
\end{tabular*}\vspace{-6pt}}


\newcommand*{\LongestName}{$\Rightarrow\ (-a)(x-y)$}% function name
\newcommand*{\LongestValue}{$(-a)x +(-a)(-y)$}% function value
\newcommand*{\LongestText}{\defi - and \dist }%

\newlength{\LargestNameSize}%
\newlength{\LargestValueSize}%
\newlength{\LargestTextSize}%

\settowidth{\LargestNameSize}{\LongestName}%
\settowidth{\LargestValueSize}{\LongestValue}%
\settowidth{\LargestTextSize}{\LongestText}%

% Choose alignment of the various elements here: [r], [l] or [c]
\newcommand*{\mbn}[1]{{\makebox[\LargestNameSize][r]{\ensuremath{#1}}}}%
\newcommand*{\mbv}[1]{\ensuremath{\makebox[\LargestValueSize][l]{\ensuremath{#1}}}}%
\newcommand*{\mbt}[1]{\makebox[\LargestTextSize][l]{#1}}%

\newtheorem{theorem}{Theorem}[section]
\newtheorem{corollary}{Corollary}[theorem]
\newtheorem{lemma}[theorem]{Lemma}
\newtheorem{subtheorem}{Lemma}[theorem]
\theoremstyle{definition}
\newtheorem{definition}{Definition}[section]
\theoremstyle{remark}
\newtheorem*{remark}{Remark}
% -----------------------------------------------------------

\pagenumbering{gobble}
\counterwithin*{equation}{theorem}
\counterwithin*{equation}{corollary}
\counterwithin*{equation}{subtheorem}
\begin{document}



\begin{tabular*}{7.5in}{l@{\extracolsep{\fill}}r}
\textbf{\large Alexander Illarionov}\\
\end{tabular*}
\textcolor{mygrey}{\noindent\makebox[\linewidth]{\rule{7.5in}{1pt}}}\\
\vspace{0.1in}

\begin{tabular*}{7.5in}{l@{\extracolsep{\fill}}r}
  \textbf{MAT 157 Problem Set I} & 02016/09/19
\end{tabular*}
\textcolor{mygrey}{\noindent\makebox[\linewidth]{\rule{7.5in}{1pt}}}\\
\vspace{0.1in}

Let $a,b,x,y \in \Re$.\\
Let $\mathbb{F}$ be a field.
\begin{lemma}{Cancellation Property}
  \begin{align}
    \mbn{\forall a,b,c \in \mathbb{F}: a+c=b+c}\ & \mbv{\Leftrightarrow a=b} & \mbt{}\\
    \mbn{\forall a,b,c \in \mathbb{F}, c\neq0: ac=bc}\ & \mbv{\Leftrightarrow a=b} & \mbt{}
  \end{align}
\end{lemma}
\begin{proof}
  Suppose $a+c=b+c$.
  \begin{align}
    \mbn{\exists\ (-c): c + (-c)}\ & \mbv{=0} & \mbt{\ainv}\\
    \mbn{\Rightarrow\ (a+c)+(-c)}\ & \mbv{=(b+c)+(-c)} & \mbt{\defi =}\\
    \mbn{\Rightarrow\ a+(c+(-c))}\ & \mbv{=b+(c+(-c))} & \mbt{\assoc}\\
    \mbn{\Rightarrow\ a+0}\ & \mbv{=b+0} & \mbt{\ainv}\\
    \mbn{\Rightarrow\ a}\ & \mbv{=b} & \mbt{\aid}\\
  \end{align}
  Suppose now $ac=bc$.
  \begin{align}
    \mbn{\exists\ c^{-1}: c  c^{-1}}\ & \mbv{=1} & \mbt{\ainv}\\
    \mbn{\Rightarrow\ (ac)c^{-1}}\ & \mbv{=(bc)c^{-1}} & \mbt{\defi =}\\
    \mbn{\Rightarrow\ a(cc^{-1})}\ & \mbv{=b(cc^{-1})} & \mbt{\assoc}\\
    \mbn{\Rightarrow\ a\times1}\ & \mbv{=b\times1} & \mbt{\minv}\\
    \mbn{\Rightarrow\ a}\ & \mbv{=b} & \mbt{\aid}
  \end{align}
\end{proof}


\begin{lemma}
  $\forall a \in \mathbb{F}: a\times0=0$
  \label{eq:zero}
\end{lemma}

\begin{proof}
\begin{align}
  \mbn{0+0}\                                       & \mbv{=0}                        & \mbt{\aid}               \\
  \mbn{\Rightarrow a\times(0+0)}\                  & \mbv{=a\times 0 + a \times 0}   & \mbt{\dist}              \\
  \mbn{}\                                          & \mbv{=a \times 0}               & \mbt{\defi =}            \\
  \mbn{(a\times 0 + a \times 0)-(a\times 0)}\      & \mbv{=a \times 0 -(a \times 0)} & \mbt{\defi =}            \\
  \mbn{\Rightarrow a\times0 +(a\times0-a\times0)}\ & \mbv{=0}                        & \mbt{\assoc}             \\
  \mbn{}\                                          & \mbv{}                          & \mbt{and \ainv}\nonumber \\
  \mbn{\Rightarrow a\times0 +0}\                   & \mbv{=0}                        & \mbt{\ainv}              \\
  \mbn{\Rightarrow a\times0}\                      & \mbv{=0}                        & \mbt{\aid}\nonumber
\end{align}
\end{proof}

\begin{lemma}
  \label{eq:negp}
  $\forall a,b \in \Re : (-a)b=-ab$
\end{lemma}
\begin{proof}

  \begin{align}
    \mbn{a+(-a)}\                & \mbv{=0}                         & \mbt{\ainv}                            \\
    \mbn{\Rightarrow ab+(-a)b}\  & \mbv{=ba+b(-a)}                  & \mbt{\comm}                            \\
    \mbn{\Rightarrow b(a+(-a))}\ & \mbv{=b\times0}                  & \mbt{\dist}                            \\
    \mbn{}\                      & \mbv{}                           & \mbt{and \ainv}                        \\
    \mbn{}\                      & \mbv{=0}                         & \mbt{Lemma \ref{eq:zero}}              \\
    \mbn{\Rightarrow ab+(-a)b}\  & \mbv{=0}                         & \mbt{\defi =}                          \\
    \mbn{\Rightarrow(-a)b+ab}\   & \mbv{=0}                         & \mbt{\comm}                            \\
    \mbn{(-a)b+ab-ab}\           & \mbv{=0-ab}                      & \mbt{\defi =}                          \\
    \mbn{\Rightarrow(-a)b+0}\    & \mbv{=-ab}                       & \mbt{\ainv}                            \\
    \mbn{}\                      & \mbv{}                           & \mbt{and \aid}\nonumber                \\
    \mbn{}\                      & \mbv{=(-a)b}                     & \mbt{\aid}
  \end{align}
\end{proof}
\begin{corollary}
\label{eq:negm}
  $\forall a \in \Re: -b=(-1)b$
\end{corollary}
\begin{proof}
  From Lemma \ref{eq:negp}, if $a=1$, then $(-1)b=-1 \times b$
  \begin{align}
    \mbn{-1\times b}\             & \mbv{=-b\times1}                 & \mbt{\comm}                            \\
    \mbn{\ra (-1)b}\              & \mbv{=-b}                        & \mbt{\defi =}                          \\
    \mbn{}\                       & \mbv{}                           & \mbt{and \mid}\nonumber
  \end{align}
\end{proof}
\begin{lemma}
  \label{eq:negneg}
  $-(-a)=a$
\end{lemma}
\begin{proof}
  \begin{align}
    \mbn{a+(-a)}\                 & \mbv{=0}                         & \mbt{\ainv}                            \\
    \mbn{(-1)(a+(-a))}\           & \mbv{=(-1)0}                     & \mbt{\defi =}                          \\
    \mbn{(-1)a+(-1)(-a)}\         & \mbv{=0}                         & \mbt{\dist}                            \\
    \mbn{}\                       & \mbv{}                           & \mbt{and Lemma \ref{eq:zero}}\nonumber \\
    \mbn{\equ -a-(-a)}\           & \mbv{=0}                         & \mbt{Corollary \ref{eq:negm}}          \\
    \mbn{a+(-a-(-a))}\            & \mbv{=a+0}                       & \mbt{\defi =}                          \\
    \mbn{(a-a)-(-a))}\            & \mbv{=a}                         & \mbt{\assoc}                           \\
    \mbn{}\                       & \mbv{}                           & \mbt{and \aid}                         \\
    \mbn{0-(-a)}\                 & \mbv{=a}                         & \mbt{\ainv}                            \\
    \mbn{-(-a)}\                  & \mbv{=a}                         & \mbt{\aid}
  \end{align}
\end{proof}
\section{}
\begin{theorem}
  $ a > 0 \wedge x > y \Rightarrow  ax > ay$
\end{theorem}
\begin{proof}
  \begin{align}
    \mbn{x > y}\                  & \mbv{\Leftrightarrow x-y \in \P} & \mbt{\defi $\P$ and $>$}               \\
    \mbn{a \in \P}\               & \mbv{\Rightarrow a(x-y)\in\P}    & \mbt{\mclo $\P$ }                      \\
    \mbn{a(x-y)=ax-ay \in\P}\     & \mbv{\Rightarrow ax-ay > 0}      & \mbt{\dist}                            \\
    \mbn{}\                       & \mbv{\Leftrightarrow ax>ay}      & \mbt{\defi $>$}
  \end{align}
\end{proof}
\begin{theorem}
  $a < 0 \wedge x > y \Rightarrow ax < ay$
\end{theorem}
\begin{proof}
  \begin{align}
    \mbn{a < 0}\                  & \mbv{\Rightarrow -a \in \P}      & \mbt{\tric}                            \\
    \mbn{x > y}\                  & \mbv{\Leftrightarrow x-y \in \P} & \mbt{\defi $\P$ and $>$}               \\
    \mbn{\Rightarrow\ (-a)(x-y)}\ & \mbv{=(-a)x +(-a)(-y) \in \P}    & \mbt{\dist, \defi $-$}                 \\
    \mbn{}\                       & \mbv{}                           & \mbt{and \mclo $\P$}\nonumber          \\
    \mbn{}\                       & \mbv{=-ax-a(-y)}                 & \mbt{Lemma \ref{eq:negp}}              \\
    \mbn{-ax-a(-y)}\              & \mbv{=-ax-(-y)a}                 & \mbt{\comm}                            \\
    \mbn{-ax-(-y)a}\              & \mbv{=-ax+ya}                    & \mbt{Lemma \ref{eq:negneg}}            \\
    \mbn{\ra -ax+ay}\             & \mbv{=ay-ax}                     & \mbt{\comm}                            \\
    \mbn{}\                       & \mbv{\ra ay-ax \in \P}           & \mbt{\mclo $\P$}                       \\
    \mbn{}\                       & \mbv{\ra ay>ax}                  & \mbt{\defi $>$}                        \\
    \mbn{}\                       & \mbv{\equ ax<ay}                 & \mbt{\defi $<$}
  \end{align}
\end{proof}

\begin{theorem}
$-0=(-1)0$
\end{theorem}
\begin{proof}
From Corollary \ref{eq:negm}, if $a=0$, by \ainv $\exists -0: -0=(-1)0$.
\end{proof}

\begin{theorem}
$\forall x,y \in \Re: x,y > 0 \wedge \frac{1}{x} < \frac{1}{y} \ra x >y$
\end{theorem}
\begin{proof}
  \begin{align}
    \mbn{x,y > 0}\ & \mbv{\ra x,y \in \P} & \mbt{\defi $>$}\\
    \mbn{}\ & \mbv{\ra xy \in \P} & \mbt{\mclo $\P$}\\
    \mbn{\frac{1}{x}<\frac{1}{y}}\ & \mbv{\ra \frac{1}{y} - \frac{1}{x} \in \P} & \mbt{\defi $<$}\\
    \mbn{}\ & \mbv{\ra xy(\frac{1}{y} - \frac{1}{x}) \in \P} & \mbt{\mclo $\P$}\\
    \mbn{}\ & \mbv{\ra(xy)\frac{1}{y} - (xy)\frac{1}{x} \in \P} & \mbt{\dist}\\
    \mbn{}\ & \mbv{\ra(xy)\frac{1}{y}-(yx)\frac{1}{x} \in \P} & \mbt{\comm}\\
    \mbn{}\ & \mbv{\ra x(y\times\frac{1}{y})-y(x\times\frac{1}{x}) \in \P} & \mbt{\assoc}\\
    \mbn{}\ & \mbv{\ra x \times 1 - y \times 1 \in \P} & \mbt{\ainv}\\
    \mbn{}\ & \mbv{\ra x - y \in \P} & \mbt{\mid}\\
    \mbn{}\ & \mbv{\ra x>y} & \mbt{\defi $>$}
\end{align}
\end{proof}

\begin{theorem}
\label{eq:sqsq}
  $x^2=y^2 \equ x=y \vee x=-y$
\end{theorem}
\begin{subtheorem}
    $x=y \vee x=-y \ra x^2=y^2$
\end{subtheorem}
\begin{proof}
  Suppose $x=y$.
  \begin{align}
    \mbn{x \times x}\ & \mbv{=x \times y} & \mbt{\defi =} \\
    \mbn{y \times y}\ & \mbv{=x \times y} & \mbt{\defi =} \\
    \mbn{\ra x^2}\ & \mbv{=y^2} & \mbt{Transitive Law}
  \end{align}
  Suppose $x=-y$.
  \begin{align}
    \mbn{x \times x}\ & \mbv{=x\times(-y)} & \mbt{\defi =} \\
    \mbn{(-y)(-y)}\ & \mbv{=x \times (-y)} & \mbt{\defi =} \\
    \mbn{(-y)(-1)y}\ & \mbv{=(-1)(-y)y} & \mbt{Corollary \ref{eq:negm} and \comm} \\
    \mbn{}\ & \mbv{=-(-y)y} & \mbt{Corollary \ref{eq:negm}} \\
    \mbn{}\ & \mbv{=y^2} & \mbt{Lemma \ref{eq:negneg}} \\
    \mbn{\ra x^2}\ & \mbv{=y^2} & \mbt{Transitive Law}
  \end{align}
\begin{subtheorem}
  \label{eq:zerzer}
  $\forall a,b \in \Re: ab=0 \equ a=0 \vee b=0$
\end{subtheorem}
\begin{proof}
  \vspace{0.1in}
  By \comm and Lemma \ref{eq:zero}, $a=0 \ra ab=ba=b\times 0=0$.\\
  \vspace{0.1in}
  Similarly, $b=0 \ra ab=a\times 0=0$.
  \vspace{0.1in}
  If $ab=0$ and $b\neq 0$, $\exists\ b^{-1}: abb^{-1}=0\times b^{-1}$,
  hence by \comm and \minv $a\times 1=b^{-1}\times 0$, then by \mid
  and Lemma \ref{eq:zero} $a=0$.\\
  \vspace{0.1in}
  If $ab=0$ and $a\neq 0$, $\exists\ a^{-1}: a^{-1}ab=a^{-1}\times 0$,
  hence by \comm and Lemma \ref{eq:zero} $aa^{-1}b=0$, then by \minv
  $1\times b=0$, and by \comm and \mid $b\times 1=b=0$.\\
  \vspace{0.1in}
  If $a=0 \wedge b=0$, then by Lemma \ref{eq:zero} $ab=0\times0=0$
\end{proof}
\begin{subtheorem}
  \label{eq:diffsq}
  $\forall a,b \in \mathbb{F}: (a+b)(a-b)=a^2-b^2 $
\end{subtheorem}
\begin{proof}
\begin{align}
  \mbn{(a+b)(a-b)}\ & \mbv{=(a+b)a+(a+b)(-b)} & \mbt{\dist} \\
  \mbn{}\ & \mbv{=(a(a+b))-(b(a+b))} & \mbt{\comm} \\
  \mbn{}\ & \mbv{=(a^2+ab)-(ba-b^2)} & \mbt{\dist} \\
  \mbn{}\ & \mbv{=a^2+(ab-(ba-b^2))} & \mbt{\assoc} \\
  \mbn{}\ & \mbv{=a^2+((ab-ba)-b^2)} & \mbt{\assoc} \\
  \mbn{}\ & \mbv{=a^2+((ab-ab)-b^2)} & \mbt{\comm} \\
  \mbn{}\ & \mbv{=a^2+(0-b^2)} & \mbt{\ainv} \\
  \mbn{}\ & \mbv{=a^2-b^2} & \mbt{\aid}
\end{align}
\end{proof}
\vspace{0.3in}
If $x^2=y^2$, by \canc and Lemma \ref{eq:diffsq} $(x-y)(x+y)=0$.\\
\vspace{0.1in}
Therefore, by Lemma \ref{eq:zerzer} and \canc $x=y \vee x=-y$
\end{proof}

\begin{theorem}
  $x^3=y^3 \equ x=y$
\end{theorem}
\begin{proof}
  By \ref{eq:sqsq}, if $x=y$, then $x^2=y^2$.
  \begin{align}
    \mbn{x^2=y^2}\ & \mbv{\ra x^3=xy^2} & \mbt{\defi =} \\
    \mbn{y^2=x^2}\ & \mbv{\ra y^3=xy^2} & \mbt{\defi =} \\
    \mbn{}\ & \mbv{\ra x^3=y^3} & \mbt{Transitive Law} \\
  \end{align}
  \begin{subtheorem}
    \label{eq:cub}
    $x^3-y^3=(x-y)(x^2+xy+y^3)$
  \end{subtheorem}
  \begin{proof}
    \begin{align}
      \mbn{(x-y)(x^2+xy+y^2)}\ & \mbv{=x^3+x^2y+} & \mbt{\dist}\\
      \mbn{}\ & \mbv{+xy^2-yx^2-yxy-y^3} & \mbt{}\nonumber \\
      \mbn{}\ & \mbv{=x^3+x^2y+xy^2-} & \mbt{\comm} \\
      \mbn{}\ & \mbv{-x^2y-xy^2-y^3} & \mbt{}\nonumber \\
      \mbn{}\ & \mbv{=x^3+(x^2y-x^2y)+} & \mbt{\comm} \\
      \mbn{}\ & \mbv{+(xy^2-xy^2)-y^3} & \mbt{}\nonumber \\
      \mbn{}\ & \mbv{=x^3+0+0-y^3} & \mbt{\ainv} \\
      \mbn{}\ & \mbv{=x^3-y^3} & \mbt{\aid}
    \end{align}
  \end{proof}

\begin{definition}
  A number $x \in \Re$ is called a \textit{postive square root} of a number
  $a\in\Re$ if $x \in \P$ and $x^2=a$. $x$ is denoted as $\sqrt{a}$.
\end{definition}
Suppose now that there is an equation $ax^2+bx+c$ with
$a,b,c,x\in\Re \wedge a\neq0$.
\begin{definition}
  A \textit{discriminant $\Delta$ of a quadratic} is defined as
  $\Delta=\sqrt{b^2-4ac}.$
\end{definition}
\begin{definition}
A number $x\in\Re$ such that $ax^2+bx+c=0$ is called the real root of the equation.
\end{definition}

\begin{remark}
For a real root of the equation to exist, $\Delta$ must be an element in $\P$.
\end{remark}

\vspace{0.3in}

If $x^3=y^3$, by Lemma \ref{eq:cub}, \canc and Lemma \ref{eq:zerzer}
$x=y \vee x^2+xy+y=0$.\\
Consider the case when $x^2+xy+y=0$. Note that it is an equation in
the form $ax^2+bx+c$, hence, by the remark above,
$\Delta=\sqrt{-3}\not\in\Re$. Therefore, the real root of this
equation does not exist. Hence, $x=y$ is the only case satisfying
the conditions.

\end{proof}

\begin{lemma}
  \label{eq:psq}
  $\forall a\in\Re:a^2\geq 0$
\end{lemma}
\begin{proof}
  Suppose that $a>0$. Therefore, $a\in\P$ by \defi $\P$. Hence,
  $a^2\in\P$ by \mclo $\P$.\\
  Suppose now that $a<0$. Therefore, $-a\in\P$ by \tric. Hence,
  $(-a)(-a)\in\P$ by \mclo $\P$. But by \comm, Lemma \ref{eq:negp}
  and Lemma \ref{eq:negneg}, $(-a)(-a)=a^2$. Thus, $\forall a\in\Re,a<0:a^2>0$.\\
  Suppose now that $a=0$. Thus, by Lemma \ref{eq:zero} $aa=0\times0=0$.

\end{proof}
\begin{lemma}
  \label{eq:pinv}
  $\forall a\in\P:a^{-1}\in\P$
\end{lemma}
\begin{proof}
  Suppose it is not the case, i.e. $\exists\ a^{-1} < 0$. Therefore,
  $-a^{-1}\in\P$ by \tric. But by Lemma \ref{eq:negp}, \comm and \ainv
  $a(-a^{-1})=-1 \not\in\P$, which contradicts \mclo $\P$. Thus, $a^{-1}\in\P$
\end{proof}

\begin{theorem}{AM-GM Inequality\\\vspace{0.2in}}
\begin{equation*}\forall a,b \in R, a\geq0,b\geq0: \sqrt{ab}\leq\frac{a+b}{2}\end{equation*}
\end{theorem}
\begin{proof}
  \begin{subtheorem}
    \label{eq:dsq}
    $(a-b)^2=a^2-2ab+b^2$
  \end{subtheorem}
  \begin{proof}
    \begin{align}
      \mbn{}\ & \mbv{=(a-b)a-(a-b)b} & \mbt{\dist} \\
      \mbn{}\ & \mbv{} & \mbt{\comm , Lemma \ref{eq:negm}} \\
      \mbn{}\ & \mbv{=a(a-b)-b(a-b)} & \mbt{\comm} \\
      \mbn{}\ & \mbv{=a^2-ab-ba+b^2} & \mbt{\dist} \\
      \mbn{}\ & \mbv{} & \mbt{Lemma \ref{eq:negp}, Lemma \ref{eq:negneg}} \\
      \mbn{}\ & \mbv{=a^2-ab-ab+b^2} & \mbt{\comm} \\
      \mbn{}\ & \mbv{=a^2-2ab+b^2} & \mbt{\defi =}
    \end{align}
  \end{proof}
  \begin{subtheorem}
    \label{eq:diff}
    $4ab\leq(a+b)^2$
  \end{subtheorem}
  \begin{proof}
  \begin{align}
    \mbn{(a+b)^2}\ & \mbv{\in\P} & \mbt{Lemma \ref{eq:psq}} \\
    \mbn{}\ & \mbv{=(a+b)a+(a+b)b} & \mbt{\dist} \\
    \mbn{}\ & \mbv{=a(a+b)+b(a+b)} & \mbt{\comm} \\
    \mbn{}\ & \mbv{=a^2+ab+ba+b^2} & \mbt{\dist} \\
    \mbn{}\ & \mbv{=a^2+ab+ab+b^2} & \mbt{\comm} \\
    \mbn{}\ & \mbv{=a^2+2ab+b^2} & \mbt{\defi = }\\
    \mbn{\ra (a+b)^2-4ab}\ & \mbv{=a^2-2ab+b^2} & \mbt{\comm} \\
    \mbn{}\ & \mbv{=(a-b)^2\geq0} & \mbt{Lemma \ref{eq:dsq}, Lemma \ref{eq:psq}}\\
    \mbn{}\ & \mbv{\equ (a+b)^2-4ab\geq0} & \mbt{\defi =} \\
    \mbn{}\ & \mbv{\equ (a+b)^2\geq 4ab} & \mbt{\ainv} \\
    \mbn{}\ & \mbv{} & \mbt{\aid}\nonumber \\
    \mbn{}\ & \mbv{} & \mbt{\aclo $\P$}\nonumber
  \end{align}
  \end{proof}
  \begin{subtheorem}
    \label{eq:equiv}
    $\forall a,b \in R, a\geq0,b\geq0: 4ab\leq(a+b)^2 \ra \sqrt{ab}\leq\frac{a+b}{2}$
  \end{subtheorem}
  \begin{proof}
    If $a=0 \vee b=0$, LHS $\equ 0\leq b^2 \vee 0\leq a^2$, RHS
    $\equ 0 \leq \frac{b}{2} \vee 0 \leq \frac{a}{2}$, which are both
    trivially true from the condition that $a,b\geq0$.
    \\\vspace{0.1in} Suppose now that
    $a>0 \wedge b>0$.\\\vspace{0.1in}
    $(a+b)^2-4ab=(a+b-2\sqrt{ab})(a+b+2\sqrt{ab}) \in P$.
    \vspace{0.1in} Since $a,b\in\P$, by Lemma \ref{eq:pinv}
    $(a+b+2\sqrt{ab})^{-1}\in\P$ and hence by \mclo $\P$
    $a+b-2\sqrt{ab} \in \P$. Thus, $\frac{a+b}{2}\geq\sqrt{ab}$ is
    implied by $4ab\leq(a+b)^2$

  \end{proof}
  \vspace{0.1in}
  Therefore, from Lemma \ref{eq:diff} and Lemma \ref{eq:equiv},
  \begin{equation*}\forall a,b \in R, a\geq0,b\geq0: \sqrt{ab}\leq\frac{a+b}{2}\end{equation*}

\end{proof}
\section{}
\subsection{}
If $x=-2\wedge y=1$, $x^2>y^2$ but $x<y$.
\subsection{}
$\sqrt{12}=2\sqrt{3}$.  \\Suppose that $\sqrt{3}$ is rational,
i.e. $\exists\ a,b \in \mathbb{Z}, (a,b)=1: \sqrt{3}=
\frac{a}{b}$.\\
Therefore, $a^2=3b^2$. Hence, $3|a^2$. But then $3|a$, and thus
$9|a^2$. Assume $\exists k\in\mathbb{Z}: a^2=9k$. So $b^2=3k$, and,
similarly, $3|b$. Therefore $(a,b)$ is at least $3$, which is a
contradiction. Since there are no $a, b$ satisfying the conditions,
there are no $2a, b$ such that $2\sqrt{3}\in\mathbb{Q}$. Ergo,
$\sqrt{12}$ is irrational.
\subsection{}
If $a=0 \wedge  b=\sqrt{3}$, $a+b=\sqrt{3}$, which is irrational, but
$0 \in \mathbb{Q}$.
\subsection{}
If $a=1 \wedge b=-1$, min$(|a|,|b|)=1$, while $|1-1|=0$, which is a
contradiction.
\section{}
Suppose $\exists\ a,b \in \mathbb{Z}, (a,b)=1: \sqrt{3}+\sqrt{5}=\frac{a}{b}$.\\
Therefore, $5=\frac{a^2}{b^2}-2\times\sqrt{3}\times\frac{a}{b}+3$, or
$2=\frac{a^2}{b^2}-2\times\sqrt{3}\times\frac{a}{b}$. Hence,
$\frac{1}{2\times\frac{a}{b}}(\frac{a^2}{b^2}-2)=\sqrt{3}$. LHS
implies that $\sqrt{3}$ is rational, which is a contradiction. Thus,
$\sqrt{3}+\sqrt{5}$ is irrational.
\section{}
\subsection{}
$=2-\sqrt{3}$, since $4>3$
\subsection{}
$(\sqrt{5}+\sqrt{3})^2-25=8+2\sqrt{15}-25=2\sqrt{15}-7=\sqrt{60}-\sqrt{49}>0$\\
Hence, $=\sqrt{5}+\sqrt{3}-5$
\subsection{}
Since $17>12$, $\sqrt{17}-\sqrt{12}>0$, and the dominator becomes
$\sqrt{12}+1-\sqrt{17}$.\\
$1+12+2\sqrt{12}-17=4\sqrt{3}-4=4(\sqrt{3}-\sqrt{1})>0$.\\
$10-(\sqrt{5}+1)^2=10-(6+2\sqrt5)=4-2\sqrt{5}=2(\sqrt{4}-\sqrt{5})<0$.\\
Therefore, $=\frac{\sqrt{5}+1-\sqrt{10}}{1+\sqrt{12}-\sqrt{17}}$.
\subsection{}
$3-\frac{6^5}{5^5}=3-\frac{36\times36\times6}{625\times5}=
3(1-\frac{12\times36\times6}{625\times5})=3(\frac{3125-432\times6}{3125})=
3(\frac{3125-2592}{3125})>0$\\
Therefore, $=3^{\frac{1}{5}}-\frac{6}{5}$
\section{}
\begin{theorem}
$\forall n \in \mathbb{N}: \sum_{i=1}^{n} i^3=\frac{n^2(n+1)^2}{4}$
\end{theorem}
\begin{proof}
\begin{align}
  \mbn{1^3}\ & \mbv{=\frac{1^2(1+1)^2}{4}} & \mbt{Base Case} \\
  \mbt{\text{Assume that: }}\ & \mbv{} & \mbt{}\nonumber \\
  \mbn{\exists k \in \mathbb{N}: \sum_{i=1}^{k} i^3}\ & \mbv{=\frac{k^2(k+1)^2}{4} } & \mbt{Inductive Step} \\
  \mbt{\text{Consider the case n=k+1: }}\ & \mbv{} & \mbt{}\nonumber \\
  \mbn{\sum_{i=1}^{k} i^3 + (k+1)^3}\ & \mbv{=\frac{k^2(k+1)^2}{4}+(k+1)^3} & \mbt{Inductive Hypothesis} \\
  \mbn{\equ (k+1)^2\frac{k^2+4k+4}{4}}\ & \mbv{=\frac{(k+1)^2(k+2)^2}{4}} & \mbt{Rearrangement} \\
  \mbt{which is exactly the hypothesis in case $n=k+1$.}\ & \mbv{} & \mbt{}\nonumber
\end{align}
Therefore, if the hypothesis is true in case $n=k$, it is true for
$n=k+1$.\\
But the hypothesis holds in case $n=1$, hence
\begin{equation*}\forall n\in\mathbb{N}:\sum_{i=1}^{n} i^3=\frac{n^2(n+1)^2}{4}\end{equation*}.
\end{proof}
\section{}
\subsection{}
Consider the trivial case of two professors, Prof Y and Prof Z. After
the statement of Professor X, Prof Y thinks that Prof Z must resign,
while at the same time Prof Z knows that Prof Y should leave his
place. Since both of them are not aware of their own mistakes, they
would not resign in the first meeting. In the second meeting, however,
they can infer the existence of their own mistakes, and hence both
must resign.\\
\vspace{0.1in} In general case, the reasoning is similar. We are given
that any Professor knows about 16 mistakes which are not their
own. Hence, during the first 16 meetings no one would resign, but when
the meeting finishes, everyone must go.

\subsection{}
The statement of Professor X has given more information to each
individual about how much the others know about his own possible
mistakes, of which he was not aware before the statement.
\end{document}
