
%%% Local Variables:
%%% mode: latex
%%% TeX-master: t
%%% End:

\documentclass[11pt]{scrartcl}
\usepackage[beaue, pset, anon]{masty}
\pSet{\hw{MAT247}{X}{2}}
\usepackage{lineno}
% ----------------------------------------------------------------------
% Page setup
% ----------------------------------------------------------------------

\pagenumbering{gobble}

% ----------------------------------------------------------------------
% Custom commands
% ----------------------------------------------------------------------

% alignment

\newcommand*{\LongestHence}{$\Rightarrow$}% function name
\newcommand*{\LongestName}{$f_o(-x)+f_e(-x)$}% function name
\newcommand*{\LongestValue}{$(-a)x +(-a)(-y)$}% function value
\newcommand*{\LongestText}{\defi}%

\newlength{\LargestHenceSize}%
\newlength{\LargestNameSize}%
\newlength{\LargestValueSize}%
\newlength{\LargestTextSize}%

\settowidth{\LargestHenceSize}{\LongestHence}%
\settowidth{\LargestNameSize}{\LongestName}%
\settowidth{\LargestValueSize}{\LongestValue}%
\settowidth{\LargestTextSize}{\LongestText}%

% Choose alignment of the various elements here: [r], [l] or [c]

\newcommand*{\mbh}[1]{{\makebox[\LargestHenceSize][r]{\ensuremath{#1}}}}%
\newcommand*{\mbn}[1]{{\makebox[\LargestNameSize][r]{\ensuremath{#1}}}}%
\newcommand*{\mbv}[1]{\ensuremath{\makebox[\LargestValueSize][r]{\ensuremath{#1}}}}%
\newcommand*{\mbt}[1]{\makebox[\LargestTextSize][l]{#1}}%

\newcommand{\R}[1]{\label{#1}\linelabel{#1}}
\newcommand{\lr}[1]{line~\lineref{#1}}

% ----------------------------------------------------------------------
% Launch!
% ----------------------------------------------------------------------

\begin{document}

\section{Problem}

Suppose that $T \in \End(V)$.

\begin{lemma}
\label{sec:problem}
The $T$-annihilator of $x$ is unique.
\end{lemma}

\begin{proof}
  \hfill

  Suppose that $q(t)$ and $r(t)$ are $T$-annihilators of $x$.

  Since they are both monic polynomials of the least degree, then
  $\deg q(t) = \deg r(t)$.

  By the division algorithm, there exist unique $u(t)$ and $v(t)$ such
  that $\deg v(t) < \deg r(t)$ and $q(t) = u(t)r(t)+v(t)$.

  Therefore, $q(T) = u(T)r(T)+v(T)$, and since $q(T) = r(T) = 0$, then
  $v(T) = 0$.

  Since $v(t)$ is such that $\deg v(t) < \deg r(t)$ and $r(t)$ is a
  $T$-annihilator, while $v(T) = 0$, then $v(t)$ is a zero polynomial.

  Hence, $q(t) = u(t)r(t)$. 

  Since $\deg q(t) = \deg u(t)+\deg r(t)$, then $\deg u(t) =
  0$. Therefore, $u(t) = c\in \FF$. Since $q(t)$ and $r(t)$ are both
  monic, $u(t) = c = 1$, and thus $q(t) = r(t)$.
\end{proof}

\begin{theorem}
\label{sec:problem-1}
The $T$-annihilator of $x$ divides any polynomial $g(t)$ such that $g(T)(x) = 0$.
\end{theorem}

\begin{proof}
  \hfill

  Suppose $x\in V$.

  Let $p(t)$ be a minimal polynomial of $T$, and let $q(t)$ be the
  $T$-annihilator of $x$.

  By Theorem 7.12, $p(t)$ divides any polynomial $g(t)$ such that
  $g(T)(x) = 0$. Thus, we only need to show that $q(t)$ divides
  $p(t)$, and the claim follows.

  By the division algorithm, there exist $u(t)$ and $v(t)$ such that
  $\deg v(t) < \deg q(t)$ and $p(t) = u(t)q(t)+v(t)$. 

  Since $p(T)x=0$ by definition and we are given that $q(T)x = 0$,
  then by additivity and homogeneity of $T$ we know that
  $p(T)x = 0 = u(T)q(T)x +v(T)x = v(T)x$. Therefore, $v(T)x = 0$, and
  since $\deg v(t) < \deg q(t)$, where $q(t)$ is the monic polynomial
  of the \textit{minimal} degree such that $q(T)x = 0$, then $v(t)$ is
  the zero polynomial, which means that $p(t) = u(t)q(t)$. Therefore,
  $q(t)$ divides $p(t)$, as required.


\end{proof}

\begin{theorem}
  If $W$ is the $T$-cyclic subspace generated by $x$, then the
  $T$-annihilator of $x$ equals the minimal polynomial of $T|_W$ and
  can be represented in the form $(-1)^{\dim}$ times the
  characteristic polynomial of $T|_W$.
\end{theorem}

\begin{proof}
  \hfill

  Let $T$-annihilator of $x$ be $q(t)$ and let $p(t)$ be the minimal
  polynomial of $T|_W$.

  Let $n = \dim W$.

  By definition of a minimal polynomial, $p(T|_W) = 0$.

  By definition of a $T$-annihilator, $q(T)x = 0$.

  Note that $x$ generates a cyclic basis
  $\beta_x = \set{x, Tx, \dots, T^{n-1}x}$.

  Since the product of $q(T)$ and any power of $T$ is commutative, we
  know that, for any $j\in[1, n - 1]\cap \NN$, we have
  $T^{j}(q(T)x) = T(0) = 0 = q(T)T^jx$, and hence $q(T)$ restricted to
  $W$ is a zero transformation, which means that $q(T|_W) = 0$.

  From Theorem \ref{sec:problem-1}, we know that $q(t) | p(t)$ and
  thus there exists a polynomial $u(t)$ such that $p(t) = u(t)q(t)$.

  On the other hand, by Theorem 7.12, since $q(T|_W) = 0$, we obtain
  that $p(t)|q(t)$. Therefore, there exists a polynomial $u'(t)$ such
  that $q(t)=u'(t)p(t)$.

  Since $p(t) = u(t)u'(t)p(t)$, we see that
  $\deg u(t) +\deg u'(t) = 0$, and thus $\deg u(t) = 0 = \deg
  u'(t)$. Since $q(t)$ and $p(t)$ are also monic, then $p(t) = q(t)$.

  By Theorem 7.15, since $W$ is an $n$-dimensional cyclic vector
  space, then the characteristic polynomial $f(t)$ of $T|_W$ is
  $(-1)^np(t)$, which, from the previous discussion, means that
  $f(t)= (-1)^nq(t)$ and thus $q(t) = (-1)^{-n}f(t) = (-1)^nf(t)$, as
  required.
  
\end{proof}
\end{document}
