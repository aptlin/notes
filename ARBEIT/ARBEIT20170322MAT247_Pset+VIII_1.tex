
%%% Local Variables:
%%% mode: latex
%%% TeX-master: t
%%% End:

\documentclass[11pt]{scrartcl}
\usepackage[beaue, pset, anon]{masty}
\pSet{\hw{MAT247}{VIII}{1}}
\usepackage{lineno}
% ----------------------------------------------------------------------
% Page setup
% ----------------------------------------------------------------------

\pagenumbering{gobble}

% ----------------------------------------------------------------------
% Custom commands
% ----------------------------------------------------------------------

% alignment

\newcommand*{\LongestHence}{$\Rightarrow$}% function name
\newcommand*{\LongestName}{$f_o(-x)+f_e(-x)$}% function name
\newcommand*{\LongestValue}{$(-a)x +(-a)(-y)$}% function value
\newcommand*{\LongestText}{\defi}%

\newlength{\LargestHenceSize}%
\newlength{\LargestNameSize}%
\newlength{\LargestValueSize}%
\newlength{\LargestTextSize}%

\settowidth{\LargestHenceSize}{\LongestHence}%
\settowidth{\LargestNameSize}{\LongestName}%
\settowidth{\LargestValueSize}{\LongestValue}%
\settowidth{\LargestTextSize}{\LongestText}%

% Choose alignment of the various elements here: [r], [l] or [c]

\newcommand*{\mbh}[1]{{\makebox[\LargestHenceSize][r]{\ensuremath{#1}}}}%
\newcommand*{\mbn}[1]{{\makebox[\LargestNameSize][r]{\ensuremath{#1}}}}%
\newcommand*{\mbv}[1]{\ensuremath{\makebox[\LargestValueSize][r]{\ensuremath{#1}}}}%
\newcommand*{\mbt}[1]{\makebox[\LargestTextSize][l]{#1}}%

\newcommand{\R}[1]{\label{#1}\linelabel{#1}}
\newcommand{\lr}[1]{line~\lineref{#1}}

% ----------------------------------------------------------------------
% Launch!
% ----------------------------------------------------------------------

\begin{document}
\section{Problem}

\begin{problem*}
  Suppose that the characteristic polynomial is $-t^3(t-1)^2$. Find
  all possible Jordan canonical forms up to similarity.
\end{problem*}
\begin{soln}
  \hfill

  Note that there are two possible eigenvalues, 0 and 1, with the
  corresponding multiplicities of 3 and 2. The size of the Jordan
  blocks is dependent on the size of the cycle corresponding to a
  particular eigenvector. We count the partitions of             
  
  Note that possible partitions of 3 which are unique up to reordering
  are $1+1+1$, $1+2$, and $3$. For 2, it is 1+1 and 2. Therefore,
  there are 6 possible Jordan canonical forms unique up to similarity:

  \begin{align}
    &\begin{pmatrix}
      1 & 0 & 0 & 0 & 0\\
      0 & 1 & 0 & 0 & 0\\
      0 & 0 & 0 & 0 & 0\\
      0 & 0 & 0 & 0 & 0\\
      0 & 0 & 0 & 0 & 0
    \end{pmatrix},
    \begin{pmatrix}
      1 & 0 & 0 & 0 & 0\\
      0 & 1 & 0 & 0 & 0\\
      0 & 0 & 0 & 1 & 0\\
      0 & 0 & 0 & 0 & 1\\
      0 & 0 & 0 & 0 & 0
    \end{pmatrix},
    \begin{pmatrix}
      1 & 0 & 0 & 0 & 0\\
      0 & 1 & 0 & 0 & 0\\
      0 & 0 & 0 & 0 & 0\\
      0 & 0 & 0 & 0 & 1\\
      0 & 0 & 0 & 0 & 0
    \end{pmatrix},\\
    &\begin{pmatrix}
      1 & 1 & 0 & 0 & 0\\
      0 & 1 & 0 & 0 & 0\\
      0 & 0 & 0 & 0 & 0\\
      0 & 0 & 0 & 0 & 0\\
      0 & 0 & 0 & 0 & 0
    \end{pmatrix},
    \begin{pmatrix}
      1 & 1 & 0 & 0 & 0\\
      0 & 1 & 0 & 0 & 0\\
      0 & 0 & 0 & 1 & 0\\
      0 & 0 & 0 & 0 & 1\\
      0 & 0 & 0 & 0 & 0
    \end{pmatrix},
    \begin{pmatrix}
      1 & 1 & 0 & 0 & 0\\
      0 & 1 & 0 & 0 & 0\\
      0 & 0 & 0 & 0 & 0\\
      0 & 0 & 0 & 0 & 1\\
      0 & 0 & 0 & 0 & 0
    \end{pmatrix}
  \end{align}


\end{soln}

\end{document}
