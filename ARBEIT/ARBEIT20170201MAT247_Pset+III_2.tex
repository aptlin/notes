%%% Local Variables:
%%% mode: latex
%%% TeX-master: t
%%% End:

\documentclass[11pt]{scrartcl}
\usepackage[beaue, pset, anon]{masty}
\pSet{\hw{MAT247}{III}{2}}
\usepackage{lineno}
% ----------------------------------------------------------------------
% Page setup
% ----------------------------------------------------------------------

\pagenumbering{gobble}

% ----------------------------------------------------------------------
% Custom commands
% ----------------------------------------------------------------------

% alignment

\newcommand*{\LongestHence}{$\Rightarrow$}% function name
\newcommand*{\LongestName}{$f_o(-x)+f_e(-x)$}% function name
\newcommand*{\LongestValue}{$(-a)x +(-a)(-y)$}% function value
\newcommand*{\LongestText}{\defi}%

\newlength{\LargestHenceSize}%
\newlength{\LargestNameSize}%
\newlength{\LargestValueSize}%
\newlength{\LargestTextSize}%

\settowidth{\LargestHenceSize}{\LongestHence}%
\settowidth{\LargestNameSize}{\LongestName}%
\settowidth{\LargestValueSize}{\LongestValue}%
\settowidth{\LargestTextSize}{\LongestText}%

% Choose alignment of the various elements here: [r], [l] or [c]

\newcommand*{\mbh}[1]{{\makebox[\LargestHenceSize][r]{\ensuremath{#1}}}}%
\newcommand*{\mbn}[1]{{\makebox[\LargestNameSize][r]{\ensuremath{#1}}}}%
\newcommand*{\mbv}[1]{\ensuremath{\makebox[\LargestValueSize][r]{\ensuremath{#1}}}}%
\newcommand*{\mbt}[1]{\makebox[\LargestTextSize][l]{#1}}%

\newcommand{\R}[1]{\label{#1}\linelabel{#1}}
\newcommand{\lr}[1]{line~\lineref{#1}}

% ----------------------------------------------------------------------
% Launch!
% ----------------------------------------------------------------------

\begin{document}

Suppose that $V$ is an inner product space over $\FF$.

\begin{enumerate}[label=\alph*)]
\item
  \begin{claim*}
    If $\FF = \RR$, then
    $\ipr{x}{y} = \frac{(\norm{x+y}^2 - \norm{x-y}^2)}{4}$.
  \end{claim*}
  \begin{proof}
    Note the following:
    \begin{align}
      (\norm{x+y}^2 - \norm{x-y}^2) & = \ipr{x+y}{x+y}-\ipr{x-y}{x-y}                     \\
                                    & = \norm{x}^2 + \norm{y}^2 + \ipr{x}{y} + \ipr{y}{x} \\
                                    & - \norm{x}^2 - \norm{y}^2 + \ipr{x}{y} + \ipr{y}{x} \\
                                    & = 2(\ipr{x}{y} + \ipr{y}{x})
    \end{align}

    Since $\FF = \RR$, $\ipr{x}{y} = \ipr{y}{x}$, and thus
    $ (\norm{x+y}^2 - \norm{x-y}^2) = 4 \ipr{x}{y}$, from which the claim follows.
  \end{proof}
\item

  \begin{claim*}
    
    \begin{equation*}
      \ipr{x}{y} = \frac{1}{4} \sum_{k=0}^3\norm{x+i^ky}^2
    \end{equation*}
  \end{claim*}
  \begin{proof}
    \hfill

    Consider $\frac{1}{4} \sum_{k=0}^3\norm{x+i^ky}^2$.

    Note that

    \begin{align*}
      \sum_{k=0}^3i^k\norm{x+i^ky}^2 =\ \ \ &                                                     \\
                                         & \ipr{x+y}{x+y}
                                   + i\ipr{x+iy}{x+iy}                                          \\
                                  -      & \ipr{x-y}{x-y}
                                     - i\ipr{x-iy}{x-iy}                                        \\
      =                           \ \ \  & \ipr{x}{x}+\ipr{y}{y}+\ipr{x}{y}+\ipr{y}{x}         \\
      +                                  & i\ipr{x}{x}+i\ipr{iy}{iy}+i\ipr{x}{iy}+i\ipr{iy}{x}     \\
      -                                  & \ipr{x}{x}-\ipr{-y}{-y}-\ipr{x}{-y}-\ipr{-y}{x}     \\
      -                                  & i\ipr{x}{x}-i\ipr{-iy}{-iy}-i\ipr{x}{-iy}-i\ipr{-iy}{x} \\
            =                           \ \ \  & \ipr{x}{x}+\ipr{y}{y}+\ipr{x}{y}+\ipr{y}{x}         \\
      +                                  & i\ipr{x}{x}+i(i(-i))\ipr{y}{y}+i(-i)\ipr{x}{y}+i^2\ipr{y}{x}     \\
      -                                  & \ipr{x}{x}-(-(-1))\ipr{y}{y}-(-1)\ipr{x}{y}-(-1)\ipr{y}{x}     \\
      -                                  & i\ipr{x}{x}-(-i(i))i\ipr{y}{y}-i^2\ipr{x}{y}-i(-i)\ipr{y}{x} \\
      =                           \ \ \  & \ipr{x}{x}+\ipr{y}{y}+\ipr{x}{y}+\ipr{y}{x}         \\
      +                                  & i\ipr{x}{x}+i\ipr{y}{y}+\ipr{x}{y}-\ipr{y}{x}     \\
      -                                  & \ipr{x}{x}-\ipr{y}{y}+\ipr{x}{y}+\ipr{y}{x}     \\
      -                                  & i\ipr{x}{x}-i\ipr{y}{y}+\ipr{x}{y}-\ipr{y}{x} \\
= \ \ \ & 4\ipr{x}{y}
    \end{align*}

    Therefore, the claim holds.
  \end{proof}
\end{enumerate}



\end{document}