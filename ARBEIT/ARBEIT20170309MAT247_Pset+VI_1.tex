
%%% Local Variables:
%%% mode: latex
%%% TeX-master: t
%%% End:

\documentclass[11pt]{scrartcl}
\usepackage[beaue, pset, anon]{masty}
\pSet{\hw{MAT247}{VI}{1}}
\usepackage{lineno}
% ----------------------------------------------------------------------
% Page setup
% ----------------------------------------------------------------------

\pagenumbering{gobble}

% ----------------------------------------------------------------------
% Custom commands
% ----------------------------------------------------------------------

% alignment

\newcommand*{\LongestHence}{$\Rightarrow$}% function name
\newcommand*{\LongestName}{$f_o(-x)+f_e(-x)$}% function name
\newcommand*{\LongestValue}{$(-a)x +(-a)(-y)$}% function value
\newcommand*{\LongestText}{\defi}%

\newlength{\LargestHenceSize}%
\newlength{\LargestNameSize}%
\newlength{\LargestValueSize}%
\newlength{\LargestTextSize}%

\settowidth{\LargestHenceSize}{\LongestHence}%
\settowidth{\LargestNameSize}{\LongestName}%
\settowidth{\LargestValueSize}{\LongestValue}%
\settowidth{\LargestTextSize}{\LongestText}%

% Choose alignment of the various elements here: [r], [l] or [c]

\newcommand*{\mbh}[1]{{\makebox[\LargestHenceSize][r]{\ensuremath{#1}}}}%
\newcommand*{\mbn}[1]{{\makebox[\LargestNameSize][r]{\ensuremath{#1}}}}%
\newcommand*{\mbv}[1]{\ensuremath{\makebox[\LargestValueSize][r]{\ensuremath{#1}}}}%
\newcommand*{\mbt}[1]{\makebox[\LargestTextSize][l]{#1}}%

\newcommand{\R}[1]{\label{#1}\linelabel{#1}}
\newcommand{\lr}[1]{line~\lineref{#1}}

% ----------------------------------------------------------------------
% Launch!
% ----------------------------------------------------------------------

\begin{document}
\section{Problem I}
Suppose that $V$ is a finite-dimensional inner product space over
$\FF$. Suppose that $u \in V$ satisfies $\Vert u \Vert = 1$. Define the linear
transformation $T \in \End(V)$ by $T(x) = x-2\langle x,u\rangle u$.

\begin{lemma}
  \label{sec:problem-i}
  $Tx = x$ if and only if $x$ is orthogonal to $u$.
\end{lemma}

\begin{proof}
  \hfill

  Suppose first $Tx = x$. Therefore,
  $ Tx = x-2\langle x,u\rangle u = x$, and thus
  $2\langle x,u\rangle u = 0$.

  Since $\norm{u}=1$, $u\neq 0$. Therefore, $\ipr{x}{u} = 0$, and thus
  $x$ is orthogonal to $u$.

  Suppose now that $x$ is orthogonal to $u$.

  Therefore, $\ipr{x}{u}=0$, and thus
  $Tx = x- 2\langle x,u\rangle u = x$.
\end{proof}

\begin{lemma}
  \label{sec:problem-i-1}
  $Tx = -x$ if and only if $x\in \spn u$.
\end{lemma}
\begin{proof}
  \hfill

  Suppose first $Tx = -x$.

  Therefore, $ Tx = x-2\langle x,u\rangle u = -x$, and thus
  $2\langle x,u\rangle u = 2x$.

  Since $\ipr{x}{u}\in \FF$, $x = \ipr{x}{u}u \in \spn u$.

  Suppose now $x\in\spn u$, so that there exists $k\in\FF$ such that $x=ku$.

  Hence, $\ipr{x}{u} = \ipr{ku}{u} = k\ipr{u}{u} = k$, since
  $\norm{u}=1$.

  Therefore, $Tx = x-2\langle x,u\rangle u = x -2ku = x-2x= -x$.
\end{proof}

\begin{lemma}
$T^2=I, T^{*}=T$ and $T$ is unitary/orthogonal.
\end{lemma}

\begin{proof}
  \hfill

  For any $x\in V$, note that

  \begin{align}
    T^2x & = T(Tx) = T(x-2\langle x,u\rangle u)          \\
         & =Tx - 2 \ipr{x}{u}Tu                          \\
         & =x-2\langle x,u\rangle u - 2 \ipr{x}{u}Tu     \\
         & =x - 2 \ipr{x}{u}(u+Tu)                       \\
         & =x - 2 \ipr{x}{u}(u+ u-2\langle u,u\rangle u) \\
         & =x - 2 \ipr{x}{u}(2u - 2 \norm{u}^2u)         \\
         & =x - 2 \ipr{x}{u}(2u-2u)                      \\
         & =x.
  \end{align}

  Thus, $T^2 = I$.

  For any $v, w \in W$, by definition of $T^{*}$, $\ipr{Tv}{w}=\ipr{v}{T^{*}w}$.

  Note the following:

  \begin{align}
    \ipr{Tv}{w} & = \ipr{v-2\langle v,u\rangle u}{w}        \\
                & =\ipr{v}{w} - 2 \ipr{v}{u}\ipr{u}{w}      \\
                & =\ipr{v}{w} - \ipr{v}{\ol{2 \ipr{u}{w}}u} \\
                & =\ipr{v}{w-2\ol{ \ipr{u}{w} }u}           \\
                & =\ipr{v}{w- 2\ipr{w}{u}u }                \\
                & =\ipr{v}{Tw}
  \end{align}

  Now, $\ipr{v}{T^{*}w} = \ipr{v}{Tw}$ for any $v, w\in V$. Taking
  $v = T^{*}w - Tw$, we obtain that $T^{*} = T$.

  Moreover,

  \begin{align}
    \ipr{Tx}{Tx} & = \ipr{x}{T^{*}Tx} \\
                 & =\ipr{x}{T^{2}x}   \\
                 & =\ipr{x}{Ix}       \\
                 & =\ipr{x}{x}
  \end{align}

  Since $\ipr{x}{x} \geq 0$ for any $x\in V$, then taking square roots
  of both sides we obtain that $\norm{Tx} = \norm{x}$, and thus $T$ is unitary/orthogonal.


\end{proof}
\begin{problem*}
  Find the characteristic polynomial of $T$.
\end{problem*}
\begin{soln}
  \hfill

  Since $T$ is both self-adjoint and orthogonal, all eigenvalues of
  $T$ have an absolute value of 1.

  By Lemma \ref{sec:problem-i}, if $\dim V \geq 2$, $1$ is an
  eigenvalue of $T$ (since $T$ is self-adjoint, there is an
  orthonormal basis of eigenvectors).

  By Lemma \ref{sec:problem-i-1}, if $\dim V \geq 1$, $-1$ is an
  eigenvalue of $T$.

  Since the coefficient corresponding to a monomial of the highest
  degree in the characteristic polynomial is $(-1)^n$, where
  $n=\dim V$, the characteristic polynomial of $T$ is as follows:
  \[p(\lambda) = (-1)^n(\lambda-1)(\lambda+1).\]
\end{soln}


\end{document}