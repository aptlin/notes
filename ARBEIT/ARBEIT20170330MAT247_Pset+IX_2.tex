
%%% Local Variables:
%%% mode: latex
%%% TeX-master: t
%%% End:

\documentclass[11pt]{scrartcl}
\usepackage[beaue, pset, anon]{masty}
\pSet{\hw{MAT247}{IX}{2}}
\usepackage{lineno}
% ----------------------------------------------------------------------
% Page setup
% ----------------------------------------------------------------------

\pagenumbering{gobble}

% ----------------------------------------------------------------------
% Custom commands
% ----------------------------------------------------------------------

% alignment

\newcommand*{\LongestHence}{$\Rightarrow$}% function name
\newcommand*{\LongestName}{$f_o(-x)+f_e(-x)$}% function name
\newcommand*{\LongestValue}{$(-a)x +(-a)(-y)$}% function value
\newcommand*{\LongestText}{\defi}%

\newlength{\LargestHenceSize}%
\newlength{\LargestNameSize}%
\newlength{\LargestValueSize}%
\newlength{\LargestTextSize}%

\settowidth{\LargestHenceSize}{\LongestHence}%
\settowidth{\LargestNameSize}{\LongestName}%
\settowidth{\LargestValueSize}{\LongestValue}%
\settowidth{\LargestTextSize}{\LongestText}%

% Choose alignment of the various elements here: [r], [l] or [c]

\newcommand*{\mbh}[1]{{\makebox[\LargestHenceSize][r]{\ensuremath{#1}}}}%
\newcommand*{\mbn}[1]{{\makebox[\LargestNameSize][r]{\ensuremath{#1}}}}%
\newcommand*{\mbv}[1]{\ensuremath{\makebox[\LargestValueSize][r]{\ensuremath{#1}}}}%
\newcommand*{\mbt}[1]{\makebox[\LargestTextSize][l]{#1}}%

\newcommand{\R}[1]{\label{#1}\linelabel{#1}}
\newcommand{\lr}[1]{line~\lineref{#1}}

% ----------------------------------------------------------------------
% Launch!
% ----------------------------------------------------------------------

\begin{document}
\begin{problem*}
\hfill

Suppose that the characteristic polynomial $f(t)$ of
$A \in M_{7 \times 7}(\mathbb R)$ splits, with the only zeroes being 1 and 2. 

Assume that the eigenspaces $E_1$ and $E_2$ of $L_A$ have dimension 2 and 3, respectively.

\begin{enumerate}[label=\alph*)]
\item How many such matrices $A$ are there, up to similarity?
\item If we also know that the algebraic multiplicity $m_1$
equals 3, determine the Jordan canonical form of $A$.
\end{enumerate}
\end{problem*}

\begin{soln}
  \hfill

  Let $V$ be a vector space such that $L_A \in \End(V)$. Note that
  $\dim V = 7$.

  Let $d_1$ be the multiplicity of the eigenvalue $1$, and let $d_{2}$
  be the multiplicity of the eigenvalue $2$.

  Since $1$ and $2$ are the only eigenvalues, $d_1+d_2 = \dim V=7$.

  Since $E_1$ has the dimension of $2$, there are two columns in the
  dot diagram of $K_1$. Therefore, $d_1 \geq 2$

  Similarly, since $E_2$ has the dimension of $3$, there are three
  columns in the dot diagram of $K_2$. Thus, $d_2\geq 3$ and thus
  $7-d_2 = d_1\leq 4$.

  We now look at each case.

  If $d_1 = 2$, then $d_2 = 5$, and there is only one possibility for
  the dot diagram corresponding to $K_1$ (because $2 = 1+1$). The
  number of possible dot diagrams corresponding to $K_2$ in this case
  is equal to the number of partitions of $5$ with 3 terms, which is
  equal to 2 ($1+2+2$, $1+1+3$).

  If $d_1 = 3$, then $d_2 = 4$, and there is only one possibility for
  the dot diagram corresponding to $K_1$ (the only partition of $3$ up
  to the order of elements, consisting of two terms, is
  $1+2$). Similarly, there is only one possibility for the dot diagram
  of $K_2$ (the partition of $4$ with 3 elements is $1+1+2$).

  If $d_1 = 4$, then $d_2 = 3$. The number of partitions of $4$ with 2
  terms is $2$ ($1+3$, $2+2$), and thus there are two possibilities
  for the dot diagram of $K_1$. There is only one possible dot diagram
  of $K_2$, because the only partition of $3$ with three terms is
  $1+1+1$.

  Therefore, there are $1 \* 2 + 1\* 1 + 2\* 1 = 5$ possible matrices
  $A$, equivalent up to similarity.

  If $d_1 = 3$, from the discussion above we already know that JCF can
  be inferred:

  \begin{equation*}
[A]_{\beta} = 
\begin{pmatrix}
  1 & 1 &   &  &  &  & \\
    & 1 &   &  &  &  & \\
    &   & 1 &  &  &  & \\
    &   &   & 2 & 1 &  & \\
    &   &   &   & 2 &  & \\
    &   &   &   &   & 2  & \\
    &   &   &   &   &   & 2 \\
  \end{pmatrix}
  \end{equation*}
\end{soln}





\end{document}
