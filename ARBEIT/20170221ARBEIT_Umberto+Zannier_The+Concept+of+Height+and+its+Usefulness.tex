% -*- coding: utf-8; -*-
%%% Local Variables:
%%% mode: latex
%%% TeX-engine: xetex
%%% TeX-master: t
%%% End:
\documentclass[11pt]{scrartcl}
\usepackage[fancy, beaue, pset, anon]{masty}
\pSet{\nt{Umberto Zannier}{}{Height}}
\usepackage{lineno}
% ----------------------------------------------------------------------
% Page setup
% ----------------------------------------------------------------------

\pagenumbering{gobble}

% ----------------------------------------------------------------------
% Custom commands
% ----------------------------------------------------------------------

% alignment

\newcommand*{\LongestHence}{$\Rightarrow$}% function name
\newcommand*{\LongestName}{$f_o(-x)+f_e(-x)$}% function name
\newcommand*{\LongestValue}{$(-a)x +(-a)(-y)$}% function value
\newcommand*{\LongestText}{\defi}%

\newlength{\LargestHenceSize}%
\newlength{\LargestNameSize}%
\newlength{\LargestValueSize}%
\newlength{\LargestTextSize}%

\settowidth{\LargestHenceSize}{\LongestHence}%
\settowidth{\LargestNameSize}{\LongestName}%
\settowidth{\LargestValueSize}{\LongestValue}%
\settowidth{\LargestTextSize}{\LongestText}%

% Choose alignment of the various elements here: [r], [l] or [c]

\newcommand*{\mbh}[1]{{\makebox[\LargestHenceSize][r]{\ensuremath{#1}}}}%
\newcommand*{\mbn}[1]{{\makebox[\LargestNameSize][r]{\ensuremath{#1}}}}%
\newcommand*{\mbv}[1]{\ensuremath{\makebox[\LargestValueSize][r]{\ensuremath{#1}}}}%
\newcommand*{\mbt}[1]{\makebox[\LargestTextSize][l]{#1}}%

\newcommand{\R}[1]{\label{#1}\linelabel{#1}}
\newcommand{\lr}[1]{line~\lineref{#1}}

% ----------------------------------------------------------------------
% Launch!
% ----------------------------------------------------------------------

\begin{document}

\section{Heights}

\begin{definition}
  A basic height is a mapping $h: \ol{\RR} \to \RR_{\geq 0}$, which
  has a natural correspondence with a degree of rational or algebraic
  functions.
\end{definition}

The concept of a height was first introduced by Weil, and then
developed further by Arakelov.

\subsection{Origins}

\begin{itemize}
\item Transcendental numbers
\item Diophantine equations $f_{\alpha}(x_1, \dots, x_n) = 0$, where $x_i\in\ZZ, \QQ, \CC,\RR,\QQ_p, \dots$
\end{itemize}

The concept of a height is similar to the concept of an absolute
value, but with greater generality.

For example, the statement \quote{If $\in \ZZ\setminus\set{0}$, then
  $ \abs{m}\geq 1$} describes the discreteness of $\ZZ$.

\subsection{Applications}

Suppose $\alpha \in \ol{\QQ}$ and $f(x) = \sum_{i=0}^d a_ix^{d-i}$ are
such that $f(\alpha) = 0$, and $d\geq 1$, $a_i\in\ZZ$, while
$a_0 > 0$.

Then the \textbf{Weil height} can be defined as
$R(t) = \frac{A(t)}{B(t)}$, where $A, B \in \CC[t]$ are coprime.

Then we can define
\begin{align}
  \deg R & = \sum_{x_0\in\CC} \max(0, -\ord_{x_0}(R) +
           max(0, -\ord_{\infty}(R) \\
         & = \max(\deg A, \deg B).
\end{align}

Then we can define a dictionary, which identifies the correspondence
between $\CC[t]$ and $\ZZ$, $\CC(t)$ and $\QQ$, $t-x_0$ for
$x_0\in\CC$ and prime numbers $p$, $\ord_{x_0}$ and $\ord_p$, $\infty$
and $\abs{r}_{\infty} = \abs{r}$, where $r = \frac{a}{b}$, where
$a, b$ are corresponding integers.

Armed with this dictionary, we then define a height as follows:
\begin{align}
  h(r) & = \sum_{p\in\PP} \max(0, -\ord_p(r))\log p + \log \max(1,\abs{r}) \\
       & = \sum \log \max(1, \abs{r}_p) = \log \max(\abs{a}, \abs{b})
\end{align}

We can also define the \textit{exponential height}, given by
$H(r) = \exp(h(r))$.

If $\alpha \in \ol{\QQ}$, and $\alpha \in K$ is a number field, with
$\nu$ an arbitrary value of $K$, then

\begin{equation*}
h(\alpha) = \sum_{\nu\in M_K}\log \max(1, \abs{\alpha}_{\nu, K} \geq 0
\end{equation*}

The following formula can also be proven.

\begin{theorem}
  If $\alpha_1, \dots, \alpha_d$ are roots of $f(x)$ then
  $H^d(\alpha) = a_0\prod_{i=1}^d\max(1, \abs{\alpha_i})$.
\end{theorem}

For example, $H(1+\sqrt{2} = \sqrt{1+\sqrt{2}}$.

\subsection{Properties}

\begin{itemize}
\item $h(\sum_{i=1}^r \alpha_i) \leq \sum_{i=1}^rh(\alpha_i) + \log r$
\item $h(\prod_{i=1}^r \alpha_i \leq \sum_{i=1}^rh(\alpha_i)$
\item $h(\alpha^n) = \abs{n} h(\alpha) \ra h(\xi) = 0$ for a root of unity $\xi$, and $h(\xi \alpha) = h(\alpha)$.

  Moreover, $h(p(\alpha)) = \deg p \* h(\alpha) + O(1)$, whenever $p$
  is a fixed polynomial.
\item $h(\alpha^r) = h(\alpha)$
\end{itemize}

\subsection{Applications}
Applications of heights include the proofs of Northcott's and Kronecker's Theorems.

Moreover, heights are useful in the study of multiplicative algebraic curves $\mathbb{G}(\ol{QQ}) =  \ol{\QQ}^K$ and elliptic curves $y^2 = x^3 + ax +b$.

For example, let $p$ be a point on an elliptic curve with rational
coordinates, $p\in E(\QQ)$. Then $h(mp) = m^2h(p) + O(1)$, and
$h(p) = h(x(p))$. If some point $p_0$ is some fixed point, then
$h(p+p_0) = h(p)+O(1)$. Tate-Neron have show that height in this case
behaves as a quadratic form.

In turn, Mordell-Weil have shown that the group $E(\QQ)$ is finitely
generated. The proof of the theorem is in two steps. The first,
arithmetical, step, is to observe that there is a finite set of points
on $E(\QQ)$ such that if $\QQ$ is any point on $E(\QQ)$, then
$\QQ = 2\QQ' + R$, for $Q' \in E(\QQ)$ and $R\in S$.

The second step is to show that $h(\QQ) = h(z\QQ') + O(1) = 14h(\QQ') + O(1)$.


\end{document}