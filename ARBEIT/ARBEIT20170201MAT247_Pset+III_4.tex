
%%% Local Variables:
%%% mode: latex
%%% TeX-master: t
%%% End:

\documentclass[11pt]{scrartcl}
\usepackage[beaue, pset, anon]{masty}
\pSet{\hw{MAT247}{III}{4}}
\usepackage{lineno}
% ----------------------------------------------------------------------
% Page setup
% ----------------------------------------------------------------------

\pagenumbering{gobble}

% ----------------------------------------------------------------------
% Custom commands
% ----------------------------------------------------------------------

% alignment

\newcommand*{\LongestHence}{$\Rightarrow$}% function name
\newcommand*{\LongestName}{$f_o(-x)+f_e(-x)$}% function name
\newcommand*{\LongestValue}{$(-a)x +(-a)(-y)$}% function value
\newcommand*{\LongestText}{\defi}%

\newlength{\LargestHenceSize}%
\newlength{\LargestNameSize}%
\newlength{\LargestValueSize}%
\newlength{\LargestTextSize}%

\settowidth{\LargestHenceSize}{\LongestHence}%
\settowidth{\LargestNameSize}{\LongestName}%
\settowidth{\LargestValueSize}{\LongestValue}%
\settowidth{\LargestTextSize}{\LongestText}%

% Choose alignment of the various elements here: [r], [l] or [c]

\newcommand*{\mbh}[1]{{\makebox[\LargestHenceSize][r]{\ensuremath{#1}}}}%
\newcommand*{\mbn}[1]{{\makebox[\LargestNameSize][r]{\ensuremath{#1}}}}%
\newcommand*{\mbv}[1]{\ensuremath{\makebox[\LargestValueSize][r]{\ensuremath{#1}}}}%
\newcommand*{\mbt}[1]{\makebox[\LargestTextSize][l]{#1}}%

\newcommand{\R}[1]{\label{#1}\linelabel{#1}}
\newcommand{\lr}[1]{line~\lineref{#1}}

% ----------------------------------------------------------------------
% Launch!
% ----------------------------------------------------------------------

\begin{document}

\section*{Problem 4}

Fix any matrix $A \in M_{n \times n}(F)$.

Let $A^*$ denote a conjugate transpose of $A$.

Consider the standard inner product $\langle \*,\*\rangle$ on $\FF^n$.

\begin{lemma}
  \label{tr}
  For any $A, B \in M_{n\times n}(\FF)$, $(BA)^T=A^TB^T$.
\end{lemma}
\begin{proof}
  Note that $(ba)_{ki}^T = (ba)_{ik} = \sum_{j=1}^nb_{ij}a_{jk}$, while

  $(a^Tb^T)_{ki} = \sum_{j=1}^n(a)^T_{kj}(b)^T_{ji} = \sum_{j=1}^na_{jk}b_{ij} = \sum_{j=1}^nb_{ij}a_{jk}$.
\end{proof}


\begin{enumerate}[label=\alph*)]
\item
  \label{a}
  \begin{claim*}\hfill
    
    For any $x, y \in F^n$ (thought of as column vectors) we have
    $\langle x, Ay \rangle = \langle A^* x,y\rangle$.
  \end{claim*}

  \begin{proof}\hfill
    
    Note that, by definition, for all $x, y\in \FF^n$, $\ipr{x}{Ay}=\sum_{i=1}^n(x)_i\ol{(Ay)_{i}}$, where $(x)_i$ and $(y)_i$ are $i$th entries in $x$ and $Ay$ respectively.

    Therefore, by definition of matrix multiplication,

    \begin{align}
      \ipr{x}{Ay}=x^T\times (\ol{A}\times\ol{y}) & = (x^T\times \ol{A})\times \ol{y}    \\
      \label{trans}
                                                 & = ((\ol{A})^T\times x)^T\times\ol{y} \\
                                                 & = \ipr{(\ol{A})^T\times x}{y}        \\
                                                 & = \ipr{A^{*}x}{y},
    \end{align}
    where (\ref{trans}) comes from Lemma \ref{tr}.
  \end{proof}
\item
  \begin{claim*}\hfill
    
    Suppose that $B\in M_{n\times n}(\FF)$ and that
    $ \langle x, Ay \rangle = \langle B x,y\rangle$
    for all $x, y \in \FF^n$.

    Then $B = A^*$.
  \end{claim*}

  \begin{proof}
    From \ref{a}, for any $x,y\in\FF^{n}$ we have $\ipr{x}{Ay} = \ipr{A^{*}x}{y}$.

    We are also given that $\ipr{x}{Ay} = \ipr{Bx}{y}$. Therefore,

    \begin{align}
      \ipr{A^{*}x}{y} - \ipr{Bx}{y} & = \ipr{x}{Ay} - \ipr{x}{Ay} \\
      \ipr{A^{*}x - Bx}{y}          & = 0                         \\
      \ipr{(A^{*}-B)x}{y}           & = 0
    \end{align}
  \end{proof}

  Since this holds for any $x, y\in M_{n\times n}(\FF)$, take $y=(A^{*}-B)x$.

  Suppose, on the contrary, that $A^{*}-B \neq \bm{0}$. Therefore,
  neither input is zero, and thus $ \ipr{(A^{*}-B)x}{y}$ is greater than zero, which is a contradiction. Therefore, $A^{*} = B$.

  % Let $T = A^{*}-B$

  % If $\FF = \CC$, then

  % \begin{align}
  %   \frac{1}{4}(\ipr{T(x+y)}{x+y}-\ipr{T(x-y)}{x-y}) &=\\
  %   \frac{1}{4}(\ipr{Tx}{x}+\ipr{Tx}{y}&+\ipr{Ty}{x}+\ipr{Ty}{y} -\\
  %                                                     -\ipr{Tx}{x}+\ipr{Ty}{x}&+\ipr{Tx}{y}+T{y}{x}-\ipr{Ty}{y})
  % \end{align}

  
\end{enumerate}



\end{document}