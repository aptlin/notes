
%%% Local Variables:
%%% mode: latex
%%% TeX-master: t
%%% End:

\documentclass[11pt]{scrartcl}
\usepackage[beaue, pset, anon]{masty}
\pSet{\hw{MAT247}{VIII}{4}}
\usepackage{lineno}
% ----------------------------------------------------------------------
% Page setup
% ----------------------------------------------------------------------

\pagenumbering{gobble}

% ----------------------------------------------------------------------
% Custom commands
% ----------------------------------------------------------------------

% alignment

\newcommand*{\LongestHence}{$\Rightarrow$}% function name
\newcommand*{\LongestName}{$f_o(-x)+f_e(-x)$}% function name
\newcommand*{\LongestValue}{$(-a)x +(-a)(-y)$}% function value
\newcommand*{\LongestText}{\defi}%

\newlength{\LargestHenceSize}%
\newlength{\LargestNameSize}%
\newlength{\LargestValueSize}%
\newlength{\LargestTextSize}%

\settowidth{\LargestHenceSize}{\LongestHence}%
\settowidth{\LargestNameSize}{\LongestName}%
\settowidth{\LargestValueSize}{\LongestValue}%
\settowidth{\LargestTextSize}{\LongestText}%

% Choose alignment of the various elements here: [r], [l] or [c]

\newcommand*{\mbh}[1]{{\makebox[\LargestHenceSize][r]{\ensuremath{#1}}}}%
\newcommand*{\mbn}[1]{{\makebox[\LargestNameSize][r]{\ensuremath{#1}}}}%
\newcommand*{\mbv}[1]{\ensuremath{\makebox[\LargestValueSize][r]{\ensuremath{#1}}}}%
\newcommand*{\mbt}[1]{\makebox[\LargestTextSize][l]{#1}}%

\newcommand{\R}[1]{\label{#1}\linelabel{#1}}
\newcommand{\lr}[1]{line~\lineref{#1}}

% ----------------------------------------------------------------------
% Launch!
% ----------------------------------------------------------------------

\begin{document}
\section{Problem}

\begin{problem*}
Suppose $n \ge 1$ and $V = F^n$. Find a linear transformation $T : V \to V$ such that $V$ has a basis consisting of just one cycle of generalised eigenvectors. Write down such a basis (for your T).
\end{problem*}
\begin{soln}
  \hfill

Suppose $V = F$.

Let $T = I$. Then $(1)$ is an eigenvector, and thus $(T-I)(1) = 0$. Moreover, $\set{1}$ is a basis, and it consists of one cycle.
\end{soln}

\begin{problem*}
Suppose that $S : V \to V$ satisfies $S^r = 0$ for some integer $r \ge 1$. Show that the characteristic polynomial $f(t)$ of $S$ is $(-1)^n t^n$, where $n = \dim V$. 
\end{problem*}

\begin{lemma}
Suppose that $S : V \to V$ satisfies $S^r = 0$ for some integer $r \ge 1$. Then $S^{\dim V} = 0$.
\end{lemma}
\begin{proof}
  \hfill

  Since $S$ is nilpotent, we know that $K_0(S) = V$, by definition of
  a generalised eigenspace.  Therefore, by Theorem proved in the last
  assignment (because there exists $r$ such that
  $\rank S^{r+1} = \rank S$), we know that $S^{\dim V} = 0$.
\end{proof}

\begin{soln}
  \hfill

  Choose a basis of $\ker S$. Extend it to a basis of $\ker
  S^{2}$. Repeating the procedure, finally obtain a basis $\beta$ of
  $V$. Note that such a basis exists, because $S^{\dim V} = 0$ and
  thus the process terminates.

  Consider $[S]_{\beta}$. Since the first elements in $\beta$ by
  construction are in $\ker S$, we know that at least the first column
  of $[S]_{\beta}$ consists of all zeroes.

  The next subset of elements belongs to $\ker S^2$. Suppose
  $v\in\ker S^2$. Then $S v \in \ker S$, because $S(Sv) = 0$, and thus
  $\ker S^2 \suq \ker S$ and $Sv$ can be represented as a linear
  combination of the basis elements of $\ker S$, and thus all the
  corresponding nonzero entries lie above the diagonal. Repeating the
  procedure, we see that all the elements in $[S]_{\beta}$ lie above
  the diagonal. Since $S^{\dim V} = 0$, we know that the process
  terminates.

  Hence, there is a matrix representation such that $[S]_{\beta}$ is
  upper-triangular with the diagonal entries all equal to zero, which
  by Laplacian expansion means that the characteristic polynomial is
  equal to $(-1)^{\dim V}t^n$, as required.
\end{soln}

\begin{problem*}
Suppose that $S : V \to V$ satisfies $S^r = 0$ for some integer $r \ge 1$ and suppose that $\text{nullity}(S) = 1$. Find a Jordan canonical form of S.
\end{problem*}

\begin{soln}
  \hfill

  By the previous problem, we know that there is a basis $\beta$ of
  $V$ such that $[S]_{\beta}$ is upper triangular with all the values
  on the diagonal equal to 0. Therefore, all the eigenvalues of $S$
  are equal to 0.

  Since $\nll S = 1$, we know that there is only one element in
  $\ker S$. Since if $v\in \ker S^2$, thus $Sv\ in\ker S$ , and we
  know that there is only one possible nonzero entry, right above the
  diagonal. Similarly, for $v\in \ker S^3$, $Sv\in\ker S^2$, and the
  same reasoning applies.

  Therefore, JCF of $S$ is a single Jordan block with the zeroes on
  the diagonal.
\end{soln}

\end{document}
