
%%% Local Variables:
%%% mode: latex
%%% TeX-master: t
%%% End:

\documentclass[11pt]{scrartcl}
\usepackage[beaue, pset, anon]{masty}
\pSet{\hw{MAT247}{VII}{2}}
\usepackage{lineno}
% ----------------------------------------------------------------------
% Page setup
% ----------------------------------------------------------------------

\pagenumbering{gobble}

% ----------------------------------------------------------------------
% Custom commands
% ----------------------------------------------------------------------

% alignment

\newcommand*{\LongestHence}{$\Rightarrow$}% function name
\newcommand*{\LongestName}{$f_o(-x)+f_e(-x)$}% function name
\newcommand*{\LongestValue}{$(-a)x +(-a)(-y)$}% function value
\newcommand*{\LongestText}{\defi}%

\newlength{\LargestHenceSize}%
\newlength{\LargestNameSize}%
\newlength{\LargestValueSize}%
\newlength{\LargestTextSize}%

\settowidth{\LargestHenceSize}{\LongestHence}%
\settowidth{\LargestNameSize}{\LongestName}%
\settowidth{\LargestValueSize}{\LongestValue}%
\settowidth{\LargestTextSize}{\LongestText}%

% Choose alignment of the various elements here: [r], [l] or [c]

\newcommand*{\mbh}[1]{{\makebox[\LargestHenceSize][r]{\ensuremath{#1}}}}%
\newcommand*{\mbn}[1]{{\makebox[\LargestNameSize][r]{\ensuremath{#1}}}}%
\newcommand*{\mbv}[1]{\ensuremath{\makebox[\LargestValueSize][r]{\ensuremath{#1}}}}%
\newcommand*{\mbt}[1]{\makebox[\LargestTextSize][l]{#1}}%

\newcommand{\R}[1]{\label{#1}\linelabel{#1}}
\newcommand{\lr}[1]{line~\lineref{#1}}

% ----------------------------------------------------------------------
% Launch!
% ----------------------------------------------------------------------

\begin{document}

\section{Problem II}

\begin{lemma}
  If $T$ is diagonalisable and $W\suq V$ is a $T$-invariant subspace,
  then the restriction $T|_W$ is also diagonalisable.
\end{lemma}

\begin{proof}
  \hfill

  Since $T$ is diagonalisable, the characteristic polynomial of $T$
  splits.

  Let $f(t)$ be a characteristic polynomial of $T$. By Cayley-Hamilton
  Theorem, $f(T) = 0$. Therefore, we obtain that $f(T)|_W=0$, which by
  homogeneity of $T$ and from the fact that $W$ is $T$-invariant means
  that $f(T|_W) = 0 = g(T)$. Since $f(T)$ splits, the characteristic
  polynomial of $T|_W$ also splits.

  % Therefore, we know that $E_{\lambda}=K_{\lambda}$ for any eigenvalue
  % $\lambda$ of $T$, which means that $\dim E_{\lambda} = K_{\lambda}$,
  % and thus
  % Since $T$ is diagonalisable, there exists a basis of eigenvectors
  % $\gamma = \set{v_1, \dots, v_n}$, where $n = \dim V$.
  % For $k=\dim W$, relabelling the basis elements if necessary, let
  % $\beta = \set{v_1, \dots, v_k}\suq \gamma$ be a basis of $W$. Note
  % that $v_1, \dots, v_k$ are eigenvectors of $T|_W$ by definition of
  % $T|_W$.

  We now show that for every eigenvalue $\mu$ of $T|_W$, the condition
  $E_{\mu}|_W=K_{\mu}|_W$ must hold, where $E_{\mu}|_W$ is an
  eigenspace and $K_{\mu}|_W$ is a generalised eigenspace
  corresponding to the eigenvalue $\mu$ of $T|_W$.

  Consider an eigenvalue $\mu$ of $T|_W$. Since $g(T)$ splits, at
  least one $\mu$ and a corresponding eigenvector $v_{\mu}\in E_{\mu}$ exist.

  Since
  $K_{\mu}|_W=\set{v;v\in \ker (T|_W-\mu I)^m \text{ for some }
    m\in\ZZ^{+}}$, then $E_{\mu}|_W\suq K_{\mu}|_{W}$ by definition.

  Suppose now $w\in K_{\mu}|_W$. Note that
  $K_{\mu}|_W = K_{\mu}\cap W$, and thus $w\in K_{\mu}$ and $w\in
  W$. Since $f(T)$ splits, then $K_{\mu} = E_{\mu}$. Therefore,
  $w\in E_{\mu}$. Since $W$ is $T$-invariant, then
  $w\in E_{\mu}\cap W = E_{\mu}|_W$. Thus,
  $K_{\mu}|_W\suq E_{\mu}|_W$, and hence $K_{\mu}|_W =
  E_{\mu}|_W$. Because $g(T)$ splits, we conclude that $T|_W$ is
  diagonalisable.
  
\end{proof}


\end{document}