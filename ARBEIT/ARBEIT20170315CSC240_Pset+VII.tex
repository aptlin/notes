%%% Local Variables:
%%% mode: latex
%%% TeX-engine: xetex
%%% TeX-master: t
%%% End:
\documentclass[11pt]{scrartcl}
\usepackage[beaue, pset, anon]{masty}
\usepackage{lineno}
% ----------------------------------------------------------------------
% Page setup
% ----------------------------------------------------------------------

\pagenumbering{gobble}

% ----------------------------------------------------------------------
% Custom commands
% ----------------------------------------------------------------------

% alignment

\newcommand*{\LongestHence}{$\Rightarrow$}% function name
\newcommand*{\LongestName}{$f_o(-x)+f_e(-x)$}% function name
\newcommand*{\LongestValue}{$(-a)x +(-a)(-y)$}% function value
\newcommand*{\LongestText}{\defi}%

\newlength{\LargestHenceSize}%
\newlength{\LargestNameSize}%
\newlength{\LargestValueSize}%
\newlength{\LargestTextSize}%

\settowidth{\LargestHenceSize}{\LongestHence}%
\settowidth{\LargestNameSize}{\LongestName}%
\settowidth{\LargestValueSize}{\LongestValue}%
\settowidth{\LargestTextSize}{\LongestText}%

% Choose alignment of the various elements here: [r], [l] or [c]

\newcommand*{\mbh}[1]{{\makebox[\LargestHenceSize][r]{\ensuremath{#1}}}}%
\newcommand*{\mbn}[1]{{\makebox[\LargestNameSize][r]{\ensuremath{#1}}}}%
\newcommand*{\mbv}[1]{\ensuremath{\makebox[\LargestValueSize][r]{\ensuremath{#1}}}}%
\newcommand*{\mbt}[1]{\makebox[\LargestTextSize][l]{#1}}%

% ----------------------------------------------------------------------
% Launch!
% ----------------------------------------------------------------------

\pSet{Alexander Illarionov -- 1003590937 -- CSC240 Pset III}
\begin{document}
\textbf{Administrativia}: no discussions, no extra material consulted

\section{Problem}

For any $\Delta \in \ZZ^{+}$, let $R(A,
\Delta(first, last))=$\textquote{RSUM($A, first, last$) is the largest sum of
  consecutive elements in the sequence of integers stored in the array
  $A[first..last]$ for any nonnegative integers $first$ and $last$
  such that $last - first +1 = \Delta $, $first \leq last$, and
  $last \leq len(A)$}.

\begin{description}

\item[Preconditions:] \hfill

  \begin{enumerate}
    
  \item $first\in\NN$, $last \in\NN$
  \item $first \leq last$
  \item $last \leq len(A)$
  \item for any $i\in[first, last]\cap\NN$, $A[i]\in\ZZ$
  \end{enumerate}

  
\item[Postconditions:]  \hfill

  
  \begin{enumerate}
  \item $A, first, last$ are unchanged
  \item RSUM($A, first, last$) is the largest sum of consecutive
    elements in the sequence of integers stored in the array
    $A[first..last]$
  \end{enumerate}

\end{description}

Let $f\in\NN$, $l\in\NN$ be such that $f \leq l$ and $l\leq len(A)$.

We proceed by induction on $\Delta = l - f+1$, which corresponds to
the length of the sequence of integers stored in the array $A[f..l]$.

Since $l\geq 0$, $f\geq 0$ by construction, then $\Delta = l-f+1\geq 1$.

Suppose first that $\Delta = 2^0 = 1$, and thus there is only one element if the sequence corresponding to $A[f..l]$.

Therefore, $l = f$, and hence the condition on line 2 is satisfied.

Note that an array $A[f..l]$ consists of one integer $A[f] = A[l]$. Therefore, if this integer is negative, the largest sum of consecutive  elements in $A[f..l]$ is the empty sum, which is equal to zero. If the integer is nonnegative, then the largest sum of consecutive elements is equal to the value of the only element.

On line $1$, $best$ was assigned to zero, and hence the if-statement on line 3 and the conditional assignment on line 4 guarantee, corresponding to our reasoning in the prevous paragraph, that the largest sum of consecutive elemens is returned on line 24. Therefore, $R(A, \Delta(f, l))$ holds for $\Delta(f, l)=1$ by generalisation.

Suppose now that $f \neq l$.

Assume $R(A, \Delta(f, l))$ holds for all $\Delta = i$, where $i\in [1, 2^k] \cap \ZZ^+$ and $k\in\NN$.

% Therefore, the else-statement on line 5 is executed.

Consider the case when $\Delta = 2^{k}+1$.

Consider an arbitrary $l\in\NN, f\in\NN$ such that $f \leq l$ and
$l\leq len(A)$, and $l-f+1 = 2^k+1$. Therefore, $l = f+2^k$.

In the algorithm for $A[f..l]$, the else-statement is executed on line 5, because $l\neq f$. Therefore, after the execution of the line 5, we have $best \gets 0$ and, since $\floor{\frac{l+f}{2}} = \floor{\frac{f+f+2^k}{2}} = \floor{2^{k-1}+f} = 2^{k-1}+f$, then $mid\gets 2^{k-1}+f$.

After line 7, $b \gets 0$ and $i\gets 2^{k-1}+f+1$.

Note that steps on line 9, 10 and 11 are executed only for
$i\in[2^{k-1}+f+1, 2^{k} +f]\cap\NN$, because $l = f+2^{k}$ and
$i\leq l$, while $i\geq mid+1$ (as guaranteed by line 11), if and only
if $i\in[2^{k-1}+f+1, 2^{k} +f]\cap\NN$.

Therefore, there are at most $2^k+f-2^{k-1}-f - 1 + 1 = 2^{k-1}$
executions of each line inside the loop on lines 9-11, and thus the
loop initiated on line 8 is not infinite.

Lines 9 to 11 ensure that after the execution of the loop $best$ is
equal to the largest sum of consecutive elements in $A$ from $mid+1$
to $last$, so that all the sums checked include the contribution of
$A[mid+1]$, for the following reason. Before the first iteration, $b$
is equal to 0. After the first iteration is executed, in each new
iteration after the update on line 9 $b$ is equal to the sum of the
elements in the array with the index less than or equal to the current
value of $i$, starting with $i= mid+1$, as guaranteed by line 7. Line
10 then ensures that $best$, which was equal to 0 before the first
iteration, is equal to the greatest running total of the elements with
the index less than or equal to the current value of $i$. The value of
$i$ is then updated, and the procedure repeats until all the sums of
consecutive elements starting with $A[mid+1]$ are checked.

On line 12, 0 is assigned to $best'$, and on line 13 $b$ is reset to 0, while on line 14 $i$ gets the value of $mid$.

Note that steps on line 16, 17 and 18 are executed only for $i\in[f, 2^{k-1} +f]\cap\NN$, because $mid = 2^{k-1}+f$ and $i\geq f$, while $i\leq mid$ (as guaranteed by line 18), if and only if $i\in[f, 2^{k-1} +f]\cap\NN$.

Therefore, there are at most $2^{k-1} +f- f + 1 = 2^{k-1} +1$
executions of each line inside the loop on lines 16-18, and thus the
loop initiated on line 15 is not infinite.

We prove now that the loop initiated on line 15 and defined on lines from 16 to 18 returns the maximum sum of consecutive elements chosen from the elements of the array from $A[1]$ to $A[mid]$ so that all the sums checked include the contribution of $A[mid]$.

After the first iteration of the loop is executed, in each new
iteration after the update on line 16 $b$ is equal to the sum of
elements in the array with the index greater than or equal to the
current value of $i$. Line 17 then ensures that $best'$, which was
equal to 0 before the first iteration, is equal to the greatest
running total of the elements with the index greater than or equal to
the current value of $i$ up to the index equal to $mid$. In this way,
all the sums which are assigned to $b$ after the first iteration
include the contribution of $A[mid]$. The value of $i$ is then
updated, and the procedure repeats until all the possible sums of
consecutive elements having the highest possible index of $mid$ and
the least possible index of $1$, necessarily including the
contribution of $A[mid]$, are checked.

On line 19, the best value is updated so that it is equal to the
maximum sum of consecutive elements in $A[f..l]$ such that the sum
contains the contributions of $A[mid]$ and $A[mid+1]$, since before
the execution $best$ is equal to the maximum sum of consecutive
elements from $A[mid+1]$ to $A[l]$ and $best'$ is equal to the maximum
sum of consecutive elements from $A[1]$ to $A[mid]$.

By inductive hypothesis, $R(A, \Delta(f, mid))$ and $R(A, \Delta(mid + 1, l))$ hold,
since $mid-f+1 = 2^{k-1}+1 \in [1, 2^k]\cap \ZZ^+$ and
$l-mid-1+1 = f+2^k-2^{k-1}-f-1+1 = 2^{k-1} \in [1, 2^k]$. Therefore,
line 20 assigns to $b$ the maximum sum of consecutive elements in
$A[f..mid]$, and line 23 assigns to $b$ the maximum number of
consecutive elements in $A[mid+1..l]$. Thus, line 21 checks if the
maximum sum of consecutive elements in $A[f..l]$ containing the
contributions of $A[mid]$ and $A[mid+1]$ is less than
RSUM$(A, f, mid)$. After the execution of line 21, there is only one
case left to check, namely whether the maximum sum of consecutive
elements not containing elements with indices from $1$ to $mid$ is
greater than the sum containing their contributions, the check for
which is performed by line 23. In this way, all the three cases are
verified, which are such that:
\begin{itemize}
\item the maximum sum of consecutive elements in $A[f..l]$ contains the contribution of the elements $A[mid]$ and $A[mid+1]$
\item the maximum sum of consecutive elements in $A[f..l]$ contains the contribution of  $A[mid]$, but not of $A[mid+1]$ (and thus this sum is equal to RSUM($A, f, mid)$)
\item the maximum sum of consecutive elements in $A[f..l]$ contains
  the contribution of $A[mid+1]$, but not of $A[mid]$ (and thus this
  sum is equal to RSUM($A, mid+1, l)$)
\end{itemize}

Hence, line 24 returns the maximum sum of consecutive elements in $A[f..l]$.

Moreover, by inductive hypothesis the execution of lines 20 and 22 is finite.

Note that $A, first, last$ were not changed after the execution of line 24. 

Therefore, $R(A, \Delta(f, l))$ holds by generalisation. 

Thus, for any $first\in\NN$, $last\in\NN$ such that $first \leq last$
and $last\leq len(A)$ we have shown that $R(A, \Delta(first, last))$
by induction.

Since the preconditions and postconditions are satisfied, while the
algorithm terminates, we have shown that RSUM is totally complete.

\end{document}
