
%%% Local Variables:
%%% mode: latex
%%% TeX-master: t
%%% End:

\documentclass[11pt]{scrartcl}
\usepackage[beaue, pset, anon]{masty}
\pSet{\hw{MAT247}{V}{1}}
\usepackage{lineno}
% ----------------------------------------------------------------------
% Page setup
% ----------------------------------------------------------------------

\pagenumbering{gobble}

% ----------------------------------------------------------------------
% Custom commands
% ----------------------------------------------------------------------

% alignment

\newcommand*{\LongestHence}{$\Rightarrow$}% function name
\newcommand*{\LongestName}{$f_o(-x)+f_e(-x)$}% function name
\newcommand*{\LongestValue}{$(-a)x +(-a)(-y)$}% function value
\newcommand*{\LongestText}{\defi}%

\newlength{\LargestHenceSize}%
\newlength{\LargestNameSize}%
\newlength{\LargestValueSize}%
\newlength{\LargestTextSize}%

\settowidth{\LargestHenceSize}{\LongestHence}%
\settowidth{\LargestNameSize}{\LongestName}%
\settowidth{\LargestValueSize}{\LongestValue}%
\settowidth{\LargestTextSize}{\LongestText}%

% Choose alignment of the various elements here: [r], [l] or [c]

\newcommand*{\mbh}[1]{{\makebox[\LargestHenceSize][r]{\ensuremath{#1}}}}%
\newcommand*{\mbn}[1]{{\makebox[\LargestNameSize][r]{\ensuremath{#1}}}}%
\newcommand*{\mbv}[1]{\ensuremath{\makebox[\LargestValueSize][r]{\ensuremath{#1}}}}%
\newcommand*{\mbt}[1]{\makebox[\LargestTextSize][l]{#1}}%

\newcommand{\R}[1]{\label{#1}\linelabel{#1}}
\newcommand{\lr}[1]{line~\lineref{#1}}

% ----------------------------------------------------------------------
% Launch!
% ----------------------------------------------------------------------

\begin{document}

Let $f(x_1,x_2) = \abs{x_1 + x_2-3}^2 + |2x_1 + x_2+1|^2 + |3x_1 + x_2-2|^2$.

\begin{problem*}
  \hfill
  
Find $(x_1, x_2) \in \mathbb C^2$ that minimize $f$.
\end{problem*}

\begin{soln}
  \hfill

  Since $f$ is a sum of nonnegative terms, $f \geq 0$.

  Therefore, $f$ might be equal to $0$ for some $(x_1,
  x_2)\in\CC^2$. In this case, the following system of equations
  holds:
  
  \begin{equation}
    \begin{cases}
      x_1 + x_2-3 &= 0 \\
      2x_1 + x_2+1 &= 0\\
      3x_1 + x_2-2 &= 0
    \end{cases},
  \end{equation}

  and thus

    \begin{equation}
    \begin{cases}
      x_1 + x_2 &= 3 \\
      2x_1 + x_2 &= -1\\
      3x_1 + x_2 &= 2
    \end{cases},
  \end{equation}

  Let $A = 
  \begin{pmatrix}
    1 & 1\\
    2 & 1\\
    3 & 1
  \end{pmatrix}$ and let $b = \cv{3;-1;2}$. To minimise $f$, we must
  minimise $\norm{b - A\cv{x_1;x_2}}^2$.

  This can be achieved by the least squares approximation.
  
  Note that $A^{*} = 
  \begin{pmatrix}
    1 & 2 & 3 \\
    1 & 1 & 1
  \end{pmatrix}
  $, and hence

  \begin{align}
    A^{*}A &=
    \begin{pmatrix}
    1 & 2 & 3 \\
    1 & 1 & 1
  \end{pmatrix}
   \begin{pmatrix}
    1 & 1\\
    2 & 1\\
    3 & 1
  \end{pmatrix}\\
    &=\begin{pmatrix}
      14 & 6\\
      6 & 3
    \end{pmatrix}
  \end{align}

  Note that $\rank A = 2$, since

  \begin{align}
    A=    \begin{pmatrix}
    1 & 1\\
    2 & 1\\
    3 & 1
  \end{pmatrix} \underset{R_2\to R_2-R_1}{\ras}&
        \begin{pmatrix}
    1 & 1\\
    1 & 0\\
    3 & 1
  \end{pmatrix}\\
    \underset{R_3\to R_3-3R_2}{\ras}&
        \begin{pmatrix}
    1 & 1\\
    1 & 0\\
    0 & 1
  \end{pmatrix}\\
        \underset{R_1\to R_1-R_2-R_3}{\ras}&
        \begin{pmatrix}
    0 & 0\\
    1 & 0\\
    0 & 1
        \end{pmatrix}.
  \end{align}

  Therefore, $\rank A^{*}A = \rank A$, and thus
  \begin{align}
    x_0=\cv{x_1;x_2} &= (A^{*}A)^{-1}A^{*}b\\
                 &= \frac{1}{\det A^{*}A}
                   \begin{pmatrix}
                     3 & -6\\
                     -6 & 14
                   \end{pmatrix}
                          \begin{pmatrix}
                            1 & 2 & 3 \\
                            1 & 1 & 1
                          \end{pmatrix}\cv{3;-1;2}
    \\
                     &= \frac{1}{6}
                   \begin{pmatrix}
                     3 & -6\\
                     -6 & 14
                   \end{pmatrix}\cv{7;4}\\
                 &= \frac{1}{6}\cv{-3;14}\\
                 &=\cv{-0.5;\nicefrac{7}{3}}
  \end{align}

  Since $A^{*}Ax_0-b = 0$ , then $\ipr{x}{A^{*}Ax_0-A^{*}b} = 0$ for
  all $x\in \CC^2$.

  Therefore, $\ipr{Ax}{Ax_{0}-b}=0$ for all
  $x\in \CC^{2}$. Hence, $Ax_0 - b \in \Img(L_A)^{\bot}$. Since

  $Ax_0 = b + (Ax_0-b)$, where $b\in\Img(L_A)$, we find that $Ax_0$ is
  the unique vector closest to $b$. Therefore, $x_0$ is the unique solution, since $A$ is invertible and hence injective.

  % \begin{align}
  %   f(-0.5,\nicefrac{7}{3}) = \abs{-0.5 + \nicefrac{7}{3}-3}^2 + |2(-0.5) + \nicefrac{7}{3}+1|^2 + |3(-0.5) + \nicefrac{7}{3}-2|^2
  % \end{align}
  % To find the minimal solution to the system, let's find first some solution to $AA^{*}x =b$.

  % \begin{align}
  %   AA^{*}       & =
  % \begin{pmatrix}
  %   1            & 1     \\
  %   2            & 1     \\
  %   3            & 1
  % \end{pmatrix}
  %   \begin{pmatrix}
  %   1            & 2 & 3 \\
  %   1            & 1 & 1
  % \end{pmatrix}          \\
  %                & =
  %            \begin{pmatrix}
  %              2 & 3 & 4 \\
  %              3 & 5 & 7 \\
  %              4 & 7 & 10
  %            \end{pmatrix}
  % \end{align}

  % Consider a system of equations for $x, y, z\in \CC$:

  % \begin{align}
  %   2x + 3y + 4z&=3\\
  %   3x + 5y + 7z&=-1\\
  %   4x + 7y + 10z&=2
  % \end{align}

  % Represent this system as an augmented matrix:
  % \begin{align}
  % \begin{matreq}{ccc|c}
  %              2                                    & 3 & 4                & 3                 \\
  %              3                                    & 5 & 7                & -1                \\
  %              4                                    & 7 & 10               & 2
  %            \end{matreq}
  %   \underset{R_3\to R_3-2R_1}{\ras}                & 
  %                           \begin{matreq}{ccc|c}
  %                             2                     & 3 & 4                & 3                 \\
  %                             3                     & 5 & 7                & -1                \\
  %                             0                     & 1 & 2                & -4
  %                           \end{matreq}                                                       \\
  %   \underset{R_2\to R_2-\nicefrac{2}{3}R_1}{\ras}  & 
  %                           \begin{matreq}{ccc|c}
  %                             2                     & 3 & 4                & 3                 \\
  %                             0                     & 3 & \nicefrac{13}{3} & -3                \\
  %                             0                     & 1 & 2                & -4
  %                           \end{matreq}                                                       \\
  %   \underset{R_2\to R_2-3R_3}{\ras}                & 
  %                           \begin{matreq}{ccc|c}
  %                             2                     & 3 & 4                & 3                 \\
  %                             0                     & 0 & \nicefrac{-5}{3} & 9                 \\
  %                             0                     & 1 & 2                & -4
  %                           \end{matreq}                                                       \\
  %   \underset{R_2\to \nicefrac{-3}{5}R_2}{\ras}     & 
  %                           \begin{matreq}{ccc|c}
  %                             2                     & 3 & 4                & 3                 \\
  %                             0                     & 0 & 1                & -\nicefrac{27}{5} \\
  %                             0                     & 1 & 2                & -4
  %                           \end{matreq}                                                       \\
  %   \underset{R_3\to R_3 - 2R_2}{\ras} &
  %                           \begin{matreq}{ccc|c}
  %                             2                     & 3 & 4                & 3                 \\
  %                             0                     & 0 & 1                & -\nicefrac{27}{5} \\
  %                             0                     & 1 & 0                & \nicefrac{34}{5}
  %                           \end{matreq}                                                       \\
  %   \underset{R_1\to R_1 - 3R_3 - 4 R_{2}}{\ras} &
  %                           \begin{matreq}{ccc|c}
  %                             2                     & 0 & 0                & \nicefrac{21}{5}                 \\
  %                             0                     & 0 & 1                & -\nicefrac{27}{5} \\
  %                             0                     & 1 & 0                & \nicefrac{34}{5}
  %                           \end{matreq}                                                       \\
  %   \underset{R_1\to \half R_1}{\ras} &
  %                           \begin{matreq}{ccc|c}
  %                             1                     & 0 & 0                & 2.1                 \\
  %                             0                     & 0 & 1                & -\nicefrac{27}{5} \\
  %                             0                     & 1 & 0                & \nicefrac{34}{5}
  %                           \end{matreq}                                                       \\        
  % \end{align}

  % Therefore, $x = 2.1$, $y = \nicefrac{34}{5} = 6.8$ and $z = -5.4$ form a solution.

  % Hence

  % \begin{align}
  %   \cv{x_1;x_2}       & = A^{*}\cv{2.1;6.8;-5.4}  \\
  %                      & = 
  %                  \begin{pmatrix}
  %                    1 & 2 & 3                    \\
  %                    1 & 1 & 1
  %                  \end{pmatrix} \cv{2.1;6.8;-5.4} \\
  %                      & =\cv{-0.5;3.5}
  % \end{align}

  
  % Note that $f(-0.5, 3.5) = \abs{-0.5+3.5 - 3}^2 + \abs{-2\*0.5+3.5+1}^{2}+\abs{-3\*0.5+3.5-2}^2$
\end{soln}


\end{document}