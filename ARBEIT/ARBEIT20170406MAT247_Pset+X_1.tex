
%%% Local Variables:
%%% mode: latex
%%% TeX-master: t
%%% End:

\documentclass[11pt]{scrartcl}
\usepackage[beaue, pset, anon]{masty}
\pSet{\hw{MAT247}{X}{1}}
\usepackage{lineno}
% ----------------------------------------------------------------------
% Page setup
% ----------------------------------------------------------------------

\pagenumbering{gobble}

% ----------------------------------------------------------------------
% Custom commands
% ----------------------------------------------------------------------

% alignment

\newcommand*{\LongestHence}{$\Rightarrow$}% function name
\newcommand*{\LongestName}{$f_o(-x)+f_e(-x)$}% function name
\newcommand*{\LongestValue}{$(-a)x +(-a)(-y)$}% function value
\newcommand*{\LongestText}{\defi}%

\newlength{\LargestHenceSize}%
\newlength{\LargestNameSize}%
\newlength{\LargestValueSize}%
\newlength{\LargestTextSize}%

\settowidth{\LargestHenceSize}{\LongestHence}%
\settowidth{\LargestNameSize}{\LongestName}%
\settowidth{\LargestValueSize}{\LongestValue}%
\settowidth{\LargestTextSize}{\LongestText}%

% Choose alignment of the various elements here: [r], [l] or [c]

\newcommand*{\mbh}[1]{{\makebox[\LargestHenceSize][r]{\ensuremath{#1}}}}%
\newcommand*{\mbn}[1]{{\makebox[\LargestNameSize][r]{\ensuremath{#1}}}}%
\newcommand*{\mbv}[1]{\ensuremath{\makebox[\LargestValueSize][r]{\ensuremath{#1}}}}%
\newcommand*{\mbt}[1]{\makebox[\LargestTextSize][l]{#1}}%

\newcommand{\R}[1]{\label{#1}\linelabel{#1}}
\newcommand{\lr}[1]{line~\lineref{#1}}

% ----------------------------------------------------------------------
% Launch!
% ----------------------------------------------------------------------

\begin{document}

\begin{theorem*}
  \hfill

  Suppose that the characteristic polynomial of $T \in \End(V)$ splits. 

  Let $\lambda_1, \lambda_2, \dots, \lambda_k$ be the distinct
  eigenvectors of $T$ and let $p_i$ be the size of the largest Jordan
  block corresponding to $\lambda_i$ in a Jordan canonical form of
  $T$.

  The minimal polynomial of $T$ is

  \begin{equation*}
    p(t) = \prod_{i=1}^k(t-\lambda_i)^{p_i}.
  \end{equation*}

\end{theorem*}

\begin{proof}
  \hfill

  Since the characteristic polynomial of $T \in \End(V)$ splits, $V = \bigoplus_{i=1}^k K_{\lambda_i}$.

  For $i\in[1, k]\cap \NN$, consider $T|_{K_{\lambda_i}}$.

  % Since $K_{\lambda_i}$ is $T$-invariant, $T|_{K_{\lambda_i}} \in \End(K_{\lambda_i})$.

  Let $v$ be an initial vector such that a cycle
  $\gamma = \set{v, (T-\lambda_iI)v, \dots, (T-\lambda_iI)^{p_i-1}v}$
  corresponds to the longest column in the dot diagram for
  $K_{\lambda_i}$.

  Let $G_i = \spn \gamma$.  

  Since $(T-\lambda_i I)^{p_i}v = 0 \in G_i$, we deduce that $G_i$ is
  $(T-\lambda_iI)|_{G_i}$-invariant, since any linear transformation is defined
  uniquely by its action on a basis.

  Note that $[(T-\lambda_iI)|_{G_{i}}]_{\beta}$ is $
  \begin{pmatrix}
    0 & \dots & & 0\\
    1 & & & 0\\
    \vdots &\ddots & & \vdots\\
    0 &\dots & 1 & 0
  \end{pmatrix}
  $. 

  Repeatedly using Laplacian expansion along the last row, we see that
  the characteristic polynomial of $(T-\lambda_i I)|_{G_i}$, $g(t)$,
  is such that $g(t) = (-1)^{p_i}t^{p_i}$. By Cayley-Hamilton Theorem,
  $(T-\lambda_i I)^{p_i}|_{G_i} = 0$.

  We now show that, in fact, $(T-\lambda_i I)^{p_i} v= 0$ for any
  $v\in K_{\lambda_i}$.

  Indeed, since there exists a cycle basis $\beta_i$ of
  $K_{\lambda_i}$ and $\gamma$ is a cycle with the greatest length,
  all the other cycles of $\beta_i$ have the length of less than or
  equal to $p_i$. Thus, by definition of a cycle, for any vector
  $v\in K_{\lambda_i}$, since $(T-\lambda_i I)$ is linear and
  homogeneous, then there exists $m\in\ZZ^+$ such that $m \leq p_i$
  and $(T-\lambda_i I)^mv = 0$. In particular, since
  $(T-\lambda_i I)(0)=0$, we know that $(T-\lambda_i I)^{p_i}v = 0$
  for all $v\in K_{\lambda_i}$.

  Let $g_i(x) = (x-\lambda_i)^{p_i}$. Let $g(x) = \prod_{i=1}^kg_i$.

  Note that the absolute value of the leading coefficient of $g(x)$ is
  $1$.

  Note that $g(T) = 0$, from Theorem E.4, the fact that
  $V = \bigoplus_{i=1}^k K_{\lambda_i}$ and $g_i(x) = 0$ for all
  $x\in K_{\lambda_i}$.

  By Theorem 7.12, $p(t)$ divides $g(t)$.

  Therefore, there exists $c\in\FF$ such that $g(t) = cp(t)$. Since
  $g(t)$ and $p(t)$ are monic up to the sign, we see that $c = 1$.

  Hence,     $p(t) = \prod_{i=1}^k(t-\lambda_i)^{p_i}$.
\end{proof}
\end{document}
