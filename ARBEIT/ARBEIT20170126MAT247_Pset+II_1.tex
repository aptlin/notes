%%% Local Variables:
%%% mode: latex
%%% TeX-master: t
%%% End:

\documentclass[11pt]{scrartcl}
\usepackage[beaue, pset, anon]{masty}
\pSet{\hw{MAT247}{II}{1}}
\usepackage{lineno}
% ----------------------------------------------------------------------
% Page setup
% ----------------------------------------------------------------------

\pagenumbering{gobble}

% ----------------------------------------------------------------------
% Custom commands
% ----------------------------------------------------------------------

% alignment

\newcommand*{\LongestHence}{$\Rightarrow$}% function name
\newcommand*{\LongestName}{$f_o(-x)+f_e(-x)$}% function name
\newcommand*{\LongestValue}{$(-a)x +(-a)(-y)$}% function value
\newcommand*{\LongestText}{\defi}%

\newlength{\LargestHenceSize}%
\newlength{\LargestNameSize}%
\newlength{\LargestValueSize}%
\newlength{\LargestTextSize}%

\settowidth{\LargestHenceSize}{\LongestHence}%
\settowidth{\LargestNameSize}{\LongestName}%
\settowidth{\LargestValueSize}{\LongestValue}%
\settowidth{\LargestTextSize}{\LongestText}%

% Choose alignment of the various elements here: [r], [l] or [c]

\newcommand*{\mbh}[1]{{\makebox[\LargestHenceSize][r]{\ensuremath{#1}}}}%
\newcommand*{\mbn}[1]{{\makebox[\LargestNameSize][r]{\ensuremath{#1}}}}%
\newcommand*{\mbv}[1]{\ensuremath{\makebox[\LargestValueSize][r]{\ensuremath{#1}}}}%
\newcommand*{\mbt}[1]{\makebox[\LargestTextSize][l]{#1}}%

\newcommand{\R}[1]{\label{#1}\linelabel{#1}}
\newcommand{\lr}[1]{line~\lineref{#1}}

% ----------------------------------------------------------------------
% Launch!
% ----------------------------------------------------------------------

\begin{document}

% ----------------------------------------------------------------------
% Body
% ----------------------------------------------------------------------
\begin{linenumbers}

  \begin{problem*}
    Consider the matrix $A =
    \begin{pmatrix}
      -5 & 8 \\
      -4 & 7
    \end{pmatrix}$. Compute $A^n$ for any $n > 0$.
  \end{problem*}
  \begin{soln}
    Calculating the characteristic polynomial of $A$, we obtain
    \[f(\lambda) = (-5-\lambda)(7-\lambda) + 32.\]

    Therefore,
    \begin{align}
      & & (\lambda - 7)(\lambda + 5) + 32 = 0 \\
      \lra & &  \lambda^2-2\lambda - 3 =0          \\
      \lra & & ( \lambda = -1) \Or (\lambda = 3)
    \end{align}

    Thus, for $\lambda = -1$, the corresponding eigenvalues in the
    form $\cv{x;y}$ is such that:
    \begin{equation*}
      \begin{pmatrix}
        -4 & 8\\
        -4 & 8
      \end{pmatrix} \cv{x;y} =
      \begin{pmatrix}
        -4x + 8y\\
        -4x + 8y
      \end{pmatrix}=
      \begin{pmatrix}
        0 & 0\\
        0 & 0
      \end{pmatrix}
    \end{equation*}

    Therefore, $x = 2y$ and thus $\cv{2; 1}$ spans $E_{-1}$.

    For $\lambda = 3$, the corresponding eigenvalues in the
    form $\cv{x;y}$ is such that:


    \begin{equation*}
      \begin{pmatrix}
        -8 & 8\\
        -4 & 4
      \end{pmatrix} \cv{x;y} =
      \begin{pmatrix}
        -8x + 8y\\
        -4x + 4y
      \end{pmatrix}=
      \begin{pmatrix}
        0 & 0\\
        0 & 0
      \end{pmatrix}
    \end{equation*}

    Therefore, $x = y$ and thus $\cv{1; 1}$ spans $E_{3}$.

    Suppose $T \in \Hom(\RR^2,\RR^2)$ is a linear transformation corresponding
    to the matrix $A$.

    Since $\RR^2 = E_1\oplus E_{2}$ (vectors spanning $E_1$ and
    $E_{2}$ are linearly independent, then $T$ is
    diagonalisable. Thus, taking $\gamma = \set{\cv{2;1}, \cv{1;1}}$, we obtain

    \begin{equation*}
      [T]_{\beta} =
      \begin{pmatrix}
        -1 & 0\\
        0 & 3
      \end{pmatrix}.
    \end{equation*}

    Denote $\beta = \set{\cv{1;0}, \cv{0;1}}$.

    Note that

    $[I]_{\beta}^{\gamma} =
    \begin{pmatrix}
      2 & 1\\
      1 & 1
    \end{pmatrix}  = U
    $. Moreover,
    \begin{align}
      &\cv{1;0} = 1 \cv{2;1} - 1 \cv{1;1}\\
      &\cv{0;1} =  (-1)\*\cv{2;1} +2 \cv{1;1}
    \end{align}

    and hence
    $[I]_{\gamma}^{\beta} =
    \begin{pmatrix}
      1  & -1 \\
      -1 & 2
    \end{pmatrix} = U^{-1}$.

    Therefore,
    \begin{equation*}
      A =     \begin{pmatrix}
        1  & -1 \\
        -1 & 2
      \end{pmatrix}
      \begin{pmatrix}
        -1 & 0\\
        0 & 3
      \end{pmatrix}
      \begin{pmatrix}
        2 & 1\\
        1 & 1
      \end{pmatrix}
    \end{equation*}
    Since $UU^{-1}=I$, it follows that
    \begin{equation*}
      A^{n} =     \begin{pmatrix}
        1  & -1 \\
        -1 & 2
      \end{pmatrix}
      \begin{pmatrix}
        (-1)^{n} & 0\\
        0 & 3^{n}
      \end{pmatrix}
      \begin{pmatrix}
        2 & 1\\
        1 & 1
      \end{pmatrix}
    \end{equation*}






  \end{soln}



\end{linenumbers}
\end{document}