%%% Local Variables:
%%% mode: latex
%%% TeX-master: t
%%% End:

\documentclass[11pt]{scrartcl}
\usepackage[beaue, pset, anon]{masty}
\pSet{\hw{MAT247}{II}{4}}
\usepackage{lineno}
% ----------------------------------------------------------------------
% Page setup
% ----------------------------------------------------------------------

\pagenumbering{gobble}

% ----------------------------------------------------------------------
% Custom commands
% ----------------------------------------------------------------------

% alignment

\newcommand*{\LongestHence}{$\Rightarrow$}% function name
\newcommand*{\LongestName}{$f_o(-x)+f_e(-x)$}% function name
\newcommand*{\LongestValue}{$(-a)x +(-a)(-y)$}% function value
\newcommand*{\LongestText}{\defi}%

\newlength{\LargestHenceSize}%
\newlength{\LargestNameSize}%
\newlength{\LargestValueSize}%
\newlength{\LargestTextSize}%

\settowidth{\LargestHenceSize}{\LongestHence}%
\settowidth{\LargestNameSize}{\LongestName}%
\settowidth{\LargestValueSize}{\LongestValue}%
\settowidth{\LargestTextSize}{\LongestText}%

% Choose alignment of the various elements here: [r], [l] or [c]

\newcommand*{\mbh}[1]{{\makebox[\LargestHenceSize][r]{\ensuremath{#1}}}}%
\newcommand*{\mbn}[1]{{\makebox[\LargestNameSize][r]{\ensuremath{#1}}}}%
\newcommand*{\mbv}[1]{\ensuremath{\makebox[\LargestValueSize][r]{\ensuremath{#1}}}}%
\newcommand*{\mbt}[1]{\makebox[\LargestTextSize][l]{#1}}%

\newcommand{\R}[1]{\label{#1}\linelabel{#1}}
\newcommand{\lr}[1]{line~\lineref{#1}}

% ----------------------------------------------------------------------
% Launch!
% ----------------------------------------------------------------------

\begin{document}

Suppose that $W_1, \dots, W_{k}$ are subspaces of a finite-dimensional
vector space V such that $W_1+\cdots+W_k = V$.

\begin{claim*}

  \begin{equation*}
    \sum_{i=1}^k\dim W_i \geq \dim V
  \end{equation*}

\end{claim*}

\begin{proof}

  Consider $v\in V$.

  Let $\beta = \bigcup_{i=1}^k\beta_i$ be the union of the bases
  $\beta_i$ of all $W_i$. Let $w_{i}$ for $i\in[1,k]\cap \NN$ be
  vectors in $W_i$ such that
  \begin{equation*}
    v = \sum_{i=1}^kw_{i}
  \end{equation*}


  Since each $w_i$ can be represented as a linear combination of
  vectors in $\beta_i$, it follows that
  $v\in \spn\set{\bigcup_{i=1}^k\beta_i} = \spn \beta$.

  Hence, $\sum_{i=1}^k\dim W_i \geq \dim V$.
\end{proof}
\begin{claim*}
  $\sum_{i=1}^k \dim W_i = \dim V $ if and only if
  $ V = W_1 \oplus W_2 \oplus \cdots \oplus W_k$.
\end{claim*}
\begin{proof}
  Suppose first $ V = W_1 \oplus W_2 \oplus \cdots \oplus W_k$.
  Thus, $W_i\cap(\sum_{j\neq i}W_{j}) = \set{0}$.

  From the previous claim, $\sum_{i=1}^k\dim W_i \geq \dim V$, and
  thus it is sufficient to show that
  $\sum_{i=1}^k\dim W_i \leq \dim V$. Note that since all $W_i$ are
  subspaces of $V$, if $\gamma$ is a basis of $V$, then all $W_i$ are
  also subsets of the span of $\gamma$.

  Consider $\beta = \bigcup_{i=1}^k\beta_i$, where $\beta_{i}$ is a
  basis of $W_i$.

  Since, $W_i\cap(\sum_{j\neq i}W_{j}) = \set{0}$, then
  $\spn{\beta_i}\cap\bigcup_{i\neq j}(\spn W_{j}) = \set{0}$, and
  thus, since all $\beta_{i}$ are linearly independent, then the union
  of the bases is linearly independent as well. Hence, $\beta$ is the
  basis for $V$, which gives that $\dim V = \sum_{i=1}^k \dim W_{i}$,
  as required.
\end{proof}

\end{document}