% -*- coding: utf-8; -*-
%%% Local Variables:
%%% mode: latex
%%% TeX-engine: xetex
%%% TeX-master: t
%%% End:
\documentclass[11pt]{scrartcl}
\usepackage[fancy, beaue, pset, anon]{masty}
\pSet{\nt{Robert Guralnick}{}{Maps from the Generic Riemann Surface}}
\usepackage{lineno}
% ----------------------------------------------------------------------
% Page setup
% ----------------------------------------------------------------------

\pagenumbering{gobble}

% ----------------------------------------------------------------------
% Custom commands
% ----------------------------------------------------------------------

% alignment

\newcommand*{\LongestHence}{$\Rightarrow$}% function name
\newcommand*{\LongestName}{$f_o(-x)+f_e(-x)$}% function name
\newcommand*{\LongestValue}{$(-a)x +(-a)(-y)$}% function value
\newcommand*{\LongestText}{\defi}%

\newlength{\LargestHenceSize}%
\newlength{\LargestNameSize}%
\newlength{\LargestValueSize}%
\newlength{\LargestTextSize}%

\settowidth{\LargestHenceSize}{\LongestHence}%
\settowidth{\LargestNameSize}{\LongestName}%
\settowidth{\LargestValueSize}{\LongestValue}%
\settowidth{\LargestTextSize}{\LongestText}%

% Choose alignment of the various elements here: [r], [l] or [c]

\newcommand*{\mbh}[1]{{\makebox[\LargestHenceSize][r]{\ensuremath{#1}}}}%
\newcommand*{\mbn}[1]{{\makebox[\LargestNameSize][r]{\ensuremath{#1}}}}%
\newcommand*{\mbv}[1]{\ensuremath{\makebox[\LargestValueSize][r]{\ensuremath{#1}}}}%
\newcommand*{\mbt}[1]{\makebox[\LargestTextSize][l]{#1}}%

\newcommand{\R}[1]{\label{#1}\linelabel{#1}}
\newcommand{\lr}[1]{line~\lineref{#1}}

% ----------------------------------------------------------------------
% Launch!
% ----------------------------------------------------------------------

\begin{document}

\section{Maps from the Generic Riemann Surface}

\textbf{Key terms}: moduli spaces, smooth projective curves, least
ramification, primitive faithful group, simple vs finite groups

Consider maps from the generic Riemann surface.

Denote a Riemann surface of genus $g$ as $\chi_g$.

Suppose that $\chi_g$ is \textit{generic}. For example, for $g = 0$,
anything is generic. If $g = 1$, $y^2 = x(x-1)(x-t)$ is generic.

Let $\mm_g$ be a moduli space of genus $g$ of smooth projective curves
over $T_k$.

Then $\dim \mm_g = 
\begin{cases}
  3g - 3, g \geq 2\\
  4, g = 1\\
  0, g = 0
\end{cases}
$.

Suppose now $g > 2 $. Then a property is generic if it holds on a
non-empty open subset of $\mm_g$.

\begin{theorem}[Enrique's Conjeture]

Fix $n\in \ZZ^+$.

If $g > 6$, there is no solvable map from the \textit{generic}
Riemannian surface of genus $g$ to $\PP^1$ of degree $n$.

Consider now $\set{x \in \mm_j; \text{ there exists
    $\phi: x \to \PP^1$ of degree $n$ such that $\phi$ is solvable}}$.
\begin{remark}
  For every Riemann surface of genus $g$ and each $x$, there exists
  $\phi:x\to \PP^1$ of the degree similar to $\frac{g+1}{2}$.
\end{remark}

\end{theorem}




\end{document}
