% -*- coding: utf-8; -*-
%%% Local Variables:
%%% mode: latex
%%% TeX-engine: xetex
%%% TeX-master: t
%%% End:
\documentclass[11pt]{scrartcl}
\usepackage[fancy, beaue, pset, anon]{sdll}
\pSet{\nt{Bjorn Poonen}{}{Undecidability in Number Theory}}
\usepackage{lineno}
% ----------------------------------------------------------------------
% Page setup
% ----------------------------------------------------------------------

\pagenumbering{gobble}

% ----------------------------------------------------------------------
% Custom commands
% ----------------------------------------------------------------------

% alignment

\newcommand*{\LongestHence}{$\Rightarrow$}% function name
\newcommand*{\LongestName}{$f_o(-x)+f_e(-x)$}% function name
\newcommand*{\LongestValue}{$(-a)x +(-a)(-y)$}% function value
\newcommand*{\LongestText}{\defi}%

\newlength{\LargestHenceSize}%
\newlength{\LargestNameSize}%
\newlength{\LargestValueSize}%
\newlength{\LargestTextSize}%

\settowidth{\LargestHenceSize}{\LongestHence}%
\settowidth{\LargestNameSize}{\LongestName}%
\settowidth{\LargestValueSize}{\LongestValue}%
\settowidth{\LargestTextSize}{\LongestText}%

% Choose alignment of the various elements here: [r], [l] or [c]

\newcommand*{\mbh}[1]{{\makebox[\LargestHenceSize][r]{\ensuremath{#1}}}}%
\newcommand*{\mbn}[1]{{\makebox[\LargestNameSize][r]{\ensuremath{#1}}}}%
\newcommand*{\mbv}[1]{\ensuremath{\makebox[\LargestValueSize][r]{\ensuremath{#1}}}}%
\newcommand*{\mbt}[1]{\makebox[\LargestTextSize][l]{#1}}%

\newcommand{\R}[1]{\label{#1}\linelabel{#1}}
\newcommand{\lr}[1]{line~\lineref{#1}}

% ----------------------------------------------------------------------
% Launch!
% ----------------------------------------------------------------------

\begin{document}

% ----------------------------------------------------------------------
% Body
% ----------------------------------------------------------------------

\begin{itemize}
\item Diophantine sets (eg \(\NN\), cf. Lagrange's Theorem)
\item Listable sets (eg the set of integers expressible as a sum of
  three cubes) -- the source of undecidability

  Diophantine sets are also listable -- same approach with boxes!
  Even more, the converse is true: Davis-Putnam-Robinson-Matiyasevich
  proved that diophantine \(\lra\) listable. Thus, the theory of
  diophantine equations is rich enough to simulate any computer.

  The unsolvability of the Halting Problem provides a listable set for
  which no algorithm can decide membership.

  Thus, there exists a diophantine set for which no algorithm can
  decide membership -- Hilbert 10th problem has a negative answer.

\item Beyond H10, there are applications of diophantine \(\lra\)
  listable to the search of prime-producing polynomials and
  investigation of diophantine statement of the Riemann Hypothesis.

\item There is a trick allowing to reduce any diophantine equation to
  one of degree 4 at most.

\item Jones - Sato - Wada - Wiens: the set of primes equals the set of
  positive values assumed by the 26-variable polynomial.

\item The DPRM theorem gives an explicit polynomial equation that has
  a solution if and only if the Riemann Hypothesis is false.

  A computer program can be written such that, if left running
  forever, it will output a counterexample to the Riemann hypothesis
  if one exists -- this program can be simulated by a diophantine equation!

  \begin{ques*}

    What is the complexity of conversion between a computer program
    and a diophantine equation?

  \end{ques*}

\item Ring of integers of a number field \(k\):
  \(\OO_{k} = \set{\alpha\in k; f(\alpha) = 0 \text{ for some monic
      \(f\in\ZZ[x]\)}}\)
\item Conjecture: H10/\(\OO_k\) has a negative answer for every number field \(k\).

\item What about H10 over \(\QQ\)? The answer is not known! If \(\ZZ\)
  is diophantine over \(\QQ\), then the negative answer for \(\ZZ\)
  implies a negative answer for \(\QQ\). However, there is a
  conjecture (cf. Mazur 1992) that implies that \(\ZZ\) is not
  diophantine over \(\QQ\). There was no progress on the conjecture
  for the varieties beyond the dimension 1.

\item H10 is also about the truth of \textbf{positive existential sentences}:

  \begin{equation*}
    (\exists x_{1} \exists x_{2} \cdots \exists x_{n}) p(x_{1}, x_{2}, \dots, x_{n})
  \end{equation*}

\item Robinson, Poonen, Koenigsmann \(\ra\) Robinson: There is no
  algorithm to decide the truth of a first-order sentence over \(\QQ\).
\item Koenigsmann 2016: \(\QQ \setminus \ZZ\) is diophantine over \(\QQ\).


\end{itemize}
\end{document}