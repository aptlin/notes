% -*- coding: utf-8; -*-
%%% Local Variables:
%%% mode: latex
%%% TeX-engine: xetex
%%% TeX-master: t
%%% End:
\documentclass[11pt]{scrartcl}
\usepackage[fancy, beaue, pset, anon]{masty}
\pSet{\nt{Frank Thome - Perspectives in Arithmetic Statistics}{}{}}
\usepackage{lineno}
% ----------------------------------------------------------------------
% Page setup
% ----------------------------------------------------------------------

\pagenumbering{gobble}

% ----------------------------------------------------------------------
% Custom commands
% ----------------------------------------------------------------------

% alignment

\newcommand*{\LongestHence}{$\Rightarrow$}% function name
\newcommand*{\LongestName}{$f_o(-x)+f_e(-x)$}% function name
\newcommand*{\LongestValue}{$(-a)x +(-a)(-y)$}% function value
\newcommand*{\LongestText}{\defi}%

\newlength{\LargestHenceSize}%
\newlength{\LargestNameSize}%
\newlength{\LargestValueSize}%
\newlength{\LargestTextSize}%

\settowidth{\LargestHenceSize}{\LongestHence}%
\settowidth{\LargestNameSize}{\LongestName}%
\settowidth{\LargestValueSize}{\LongestValue}%
\settowidth{\LargestTextSize}{\LongestText}%

% Choose alignment of the various elements here: [r], [l] or [c]

\newcommand*{\mbh}[1]{{\makebox[\LargestHenceSize][r]{\ensuremath{#1}}}}%
\newcommand*{\mbn}[1]{{\makebox[\LargestNameSize][r]{\ensuremath{#1}}}}%
\newcommand*{\mbv}[1]{\ensuremath{\makebox[\LargestValueSize][r]{\ensuremath{#1}}}}%
\newcommand*{\mbt}[1]{\makebox[\LargestTextSize][l]{#1}}%

\newcommand{\R}[1]{\label{#1}\linelabel{#1}}
\newcommand{\lr}[1]{line~\lineref{#1}}

% ----------------------------------------------------------------------
% Launch!
% ----------------------------------------------------------------------

\begin{document}

\section{Analytic Perspectives in Arithmetic Statistics}

Let $V = V(\ZZ)$ be a latics of binary quadratic forms $au^2+buv+cv^2$ such that $a, b, c\in \ZZ$.

Note that there is an action of $\SL_2(\ZZ)$. 

We also define a discriminant $\Disc(f) = b^2-4ac$.

We know that $\Disc(g\circ f) = \Disc(f)$ for all $g\in\GL_2(\ZZ)$.

Let $V' = \set{v\in V; \Disc(v) \neq 0}$.

We also know that the action of $\GL_2(\CC)$ on $V'(\CC)$ has one orbit, $\GL_2(\RR)$ on $V(\FF)$ has two orbits, while actions of $\GL_{2}(\ZZ)$ and $\SL_2(\ZZ)$ on $V'(\ZZ)$ have infinitely many orbits!

Therefore, there is \textsl{more information stored} in $\ZZ$.

Let $h(D)$ be  the number of orbits of discriminant $D$. It can be shown that it is always finite and greater than one for $D \equiv 0, 1 \mod 4$.

\textbf{Gauss' Conjecture (now Theorem)}: If $D < 0$ and $h(D) = 1$,
then we have that $D\in \set{-3, -4, -7 , -8, -11, -19, -43, -67, -163}$.

\begin{theorem}

For each $D$, the set of equivalence classes of BQFs of $\Disc D$  forms an abelian group.
\end{theorem}

\begin{remark}
$\Cl (D) \to \Cl(\theta)$ is an isomorphism such that
$\theta = [1, \frac{D+\sqrt{D}}{2}]\suq
\theta_Q(\sqrt{\theta_Q}(\sqrt{D})$.
\end{remark}

\end{document}
