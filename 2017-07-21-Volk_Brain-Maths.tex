% -*- coding: utf-8; -*-
%%% Local Variables:
%%% mode: latex
%%% TeX-engine: xetex
%%% TeX-master: t
%%% End:
\documentclass[11pt]{scrartcl}
\usepackage[fancy, beaue, pset, anon]{masty}
\pSet{\nt{Denis Volk}{1}{Brain Maths}}
  \usepackage{lineno}
  % ----------------------------------------------------------------------
  % Page setup
  % ----------------------------------------------------------------------

  \pagenumbering{gobble}

  % ----------------------------------------------------------------------
  % Custom commands
  % ----------------------------------------------------------------------

  % alignment

  \newcommand*{\LongestHence}{$\Rightarrow$}% function name
  \newcommand*{\LongestName}{$f_o(-x)+f_e(-x)$}% function name
  \newcommand*{\LongestValue}{$(-a)x +(-a)(-y)$}% function value
  \newcommand*{\LongestText}{\defi}%

  \newlength{\LargestHenceSize}%
  \newlength{\LargestNameSize}%
  \newlength{\LargestValueSize}%
  \newlength{\LargestTextSize}%

  \settowidth{\LargestHenceSize}{\LongestHence}%
  \settowidth{\LargestNameSize}{\LongestName}%
  \settowidth{\LargestValueSize}{\LongestValue}%
  \settowidth{\LargestTextSize}{\LongestText}%

  % Choose alignment of the various elements here: [r], [l] or [c]

  \newcommand*{\mbh}[1]{{\makebox[\LargestHenceSize][r]{\ensuremath{#1}}}}%
  \newcommand*{\mbn}[1]{{\makebox[\LargestNameSize][r]{\ensuremath{#1}}}}%
  \newcommand*{\mbv}[1]{\ensuremath{\makebox[\LargestValueSize][r]{\ensuremath{#1}}}}%
  \newcommand*{\mbt}[1]{\makebox[\LargestTextSize][l]{#1}}%

  \newcommand{\R}[1]{\label{#1}\linelabel{#1}}
  \newcommand{\lr}[1]{line~\lineref{#1}}

  % ----------------------------------------------------------------------
  % Launch!
  % ----------------------------------------------------------------------

  \begin{document}

  \section{Brain Maths}

  Modern neuroscience focuses on neurons. Neurons generate
  electricity for communication.

  There are about $10^{11}$ neurons on average in a human brain. Each
  neuron consists of two main parts -- a body and an axon. The body
  also have synapses attached to its surface which serve asa
  communication device for other neurons. Synapses, in turn, connect
  to dendrites of other neurons.

  The body of a neuron has a membrane acting like a capacitor storing
  electricity for communication. We will denote the potential
  difference that the body stores as $\Delta V$.

  Neurons communicate by utilising spikes in the potential difference.
  An average neuron has a period of relaxation after each spike of 100
  mV over 5 ms.

  The usual method of measuring the output of a neuron is as follows:
  electrodes are inserted inside the body which then measures the
  potential difference.

  How can a neuron control the potential difference?

  Membrane contains proteins which act as channels migrating
  electrically charged ions like $K^{+}, Na^{+}, Ca^{2+} $ and
  $Cl^{-}$. There are two aspects affecting the flow of ions --
  electric gradient and osmosis. The flow would be unrestricted but
  for pumps also present in the membrane. These pumps utilise internal
  energy to manage the change in potential difference.

  Each pump or channel is affected by the potential difference between
  the inside and the outside of the cell, concentration of charged
  ions and neuromediators present in the surroundings.

  The spikes result from the sudden opening of channels and the action
  of pumps. It is worthwhile to note that the action of pumps is much
  slower than that of channels.

  There are several types of spikes. Some neurons give a one-off
  spike, another rattle a sequence and then sleep, while the other
  neurons give off spikes on the periodical basis.

  How can such a complicated machinery work with such efficiency? How
  can we predict what effect each spike would bring upon the other
  neurons?

  Each neuron has around $10^{4}$ synapses, which receive the signal
  along the axons from the body and then transfer the charge to
  dendrites communicating with other neurons.

  One of the first models assumed that each neuron processes all the
  received signals ${I_{i}}$ from the dendrites with some weights
  ${w_{i}}$ assigned, which then bring about post-synaptic effects.
  If, however, the signal was greater than some well-defined barrier
  $\wh{I}$, it was thought to give off a spike itself to communicate
  the message to other neurons:
  \begin{equation}
    \sum w_{i} I_{i} > \wh{I}
  \end{equation}

  This, however, turned out to be far from what really happens in a
  brain.

  First of all, the barrier $\wh{I}$ is fuzzy -- there is a middle
  range at which a neuron operates optimally.

  Secondly, there are latencies in spikes.

  Thirdly, spikes are all different, and dependent on the input
  signal.

  Finally, some neurons give off semi-spikes, while others do not.

  Experiments have shown that if we transfer a current meant to
  decrease the potential difference, neurons give off rebound spikes.

  If we give a fast periodic impulse, some neurons give off a spike
  which is a sum of the outputs that would occur for each individual
  pulse. Some neurons were responsive to the frequency of the received
  impulse.

  To explain this largely unpredictable behaviour, several tentative
  fixes were introduced. One example is classification of neurons into
  resonators and integrators.

  To build a model of a neuron, we need to identify main parameters.

  First, denote the potential difference as $V$, and the fraction of
  channels and pumps of particular type $i$ open as
  $x_{1}, \dots, x_{M}$, where
  $\dot{x_{i}} = g_{i}(V, x_{1}, \dots, x_{M})$. The current through
  the membrane would be $C\dot{V} = I + f(V, x_{1}, \dots, x_{M})$,
  where $C$ is the capacitance of the membrane, $I$ is the total
  magnitude of external currents, and $f$ represents the total
  magnitude of currents coming through the internal means.

  The main purpose of the model is to simulate the actions of a real
  neuron. However, if parameters are shuffled in without any
  consideration, obtained models are most often unrepresentative of
  any physical neuron. To account for this problem, we can study a toy
  model, which includes only one equation:
  
  \begin{equation*}
    C\dot{V} = I + f(V)
  \end{equation*}

  
\end{document}
