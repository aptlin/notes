% -*- coding: utf-8; -*-
%%% Local Variables:
%%% mode: latex
%%% TeX-engine: xetex
%%% TeX-master: t
%%% End:
\documentclass[11pt]{scrartcl}
\usepackage[fancy, beaue, pset, anon]{sdll}
\pSet{\nt{MAT157}{II.4}{Properties of Integrals}}
\usepackage{lineno}
% ----------------------------------------------------------------------
% Page setup
% ----------------------------------------------------------------------

\pagenumbering{gobble}

% ----------------------------------------------------------------------
% Custom commands
% ----------------------------------------------------------------------

% alignment

\newcommand*{\LongestHence}{$\Rightarrow$}% function name
\newcommand*{\LongestName}{$f_o(-x)+f_e(-x)$}% function name
\newcommand*{\LongestValue}{$(-a)x +(-a)(-y)$}% function value
\newcommand*{\LongestText}{\defi}%

\newlength{\LargestHenceSize}%
\newlength{\LargestNameSize}%
\newlength{\LargestValueSize}%
\newlength{\LargestTextSize}%

\settowidth{\LargestHenceSize}{\LongestHence}%
\settowidth{\LargestNameSize}{\LongestName}%
\settowidth{\LargestValueSize}{\LongestValue}%
\settowidth{\LargestTextSize}{\LongestText}%

% Choose alignment of the various elements here: [r], [l] or [c]

\newcommand*{\mbh}[1]{{\makebox[\LargestHenceSize][r]{\ensuremath{#1}}}}%
\newcommand*{\mbn}[1]{{\makebox[\LargestNameSize][r]{\ensuremath{#1}}}}%
\newcommand*{\mbv}[1]{\ensuremath{\makebox[\LargestValueSize][r]{\ensuremath{#1}}}}%
\newcommand*{\mbt}[1]{\makebox[\LargestTextSize][l]{#1}}%

\newcommand{\R}[1]{\label{#1}\linelabel{#1}}
\newcommand{\lr}[1]{line~\lineref{#1}}

% ----------------------------------------------------------------------
% Launch!
% ----------------------------------------------------------------------

\begin{document}

\begin{theorem}
  \label{sec:1}
  Suppose $f$ is integrable and $m\leq f(x) \leq M$ for all $x$ in $[a, b]$.
  Then, $m(b-a)\leq \int_a^b f \leq M(b-a)$.
\end{theorem}

\begin{proof}
  Note that $m(b-a) \leq L(f, P)$ and $U(f, P) \leq M(b-a)$ for any
  partition $P$. Then
  $m(b-a) \leq L(f, P) \leq \int_a^b f \leq U(f, P) \leq M(b-a)$.
\end{proof}
\section{Fundamental Theorem of Calculus}
\begin{theorem}
  Let $f$ be integrable on $[a, b]$ and define $F$ on $[a, b]$ by
  $F(x) = \int_a^xf $.

  Then $F$ is continuous on $[a, b]$.
\end{theorem}

\begin{proof}
  Take $c$ in $[a, b]$. Since $f$ is integrable on $[a, b]$, it is
  bounded. Define $M$ as a number such that $\abs{f(x)} \leq M$ for
  all $x\in[a, b]$. If $h > 0$, then

  \begin{equation*}
    F(c+h) - F(c) = \int_{a}^{c+h}f - \int_{a}^{c}f = \int_{c}^{c+h} f.
  \end{equation*}

  Since $-M \leq f(x) \leq M $ for all $x$, it follows from Theorem \ref{sec:1} that

  \begin{equation*}
    -M\* h \leq \int_c^{c+h} f \leq M\*h
  \end{equation*}

  Thus,
  \begin{equation*}
    -M\* h \leq F(c+h) - F(h) \leq M\*h
  \end{equation*}

  If $h < 0$, a similar inequalitiy can be derived, since
  $F(c+h) - f(h) = - \int^c_{c+h}f$. Therefore, for $[c+h, c]$,

  \begin{equation*}
    Mh \leq \int_{c+h}^cf \leq - Mh,
  \end{equation*}

  which gives


  \begin{equation*}
    M\* h \leq F(c+h) - F(h) \leq -M\*h
  \end{equation*}

  Thus,


  \begin{equation*}
    \abs{F(c+h) - F(h)} \leq M\abs{h}
  \end{equation*}

  Therefore, for any $\epsilon>0$, if $\abs{h} < \frac{\epsilon}{M}$,

  \begin{equation*}
    \abs{F(c+h) - F(h)} < \epsilon
  \end{equation*}
\end{proof}

\begin{theorem}
  Let $f$ be integrable on $[a, b]$ and define $F$ on $[a, b]$ by
  $F(x) = \int_a^xf $.

  If $f$ is continuous at $c$ in $[a,b]$, then $F$ is differentiable
  at $c$, and $F'(c) = f(c)$.
\end{theorem}

\begin{proof}
  Suppose first that $c$ is in $(a, b)$. By definition,
  \begin{equation*}
    F'(c) = \lim_{h\to 0}\frac{F(c+h) - F(c)}{h}.
  \end{equation*}

  For each $h$, define $m_h$ and $M_{h}$ as follows:

  \begin{align}
    m_h &= \inf \set{f(x); c \leq x \leq c+h}\\
    M_{h} &= \sup \set{f(x); c \leq x \leq c+h}
  \end{align}

  From Theorem \ref{sec:1},


  \begin{equation*}
    m_h\* h \leq \int_c^{c+h} f \leq M_h\*h,
  \end{equation*}

  and thus

  \begin{equation*}
    m_h\leq \frac{F(c+h) - F(c)}{h}\leq M_h
  \end{equation*}

  If $h\leq 0$, then

  \begin{align}
    m_h &= \inf \set{f(x); c+h \leq x \leq c}\\
    M_{h} &= \sup \set{f(x); c+h \leq x \leq c}
  \end{align}

  From Theorem \ref{sec:1},


  \begin{equation*}
    m_h\* (-h) \leq \int_{c+h}^c f \leq M_h\*(-h),
  \end{equation*}

  and thus

  \begin{equation*}
    m_h\* (-h) \geq F(c+h) - F(c) \geq M_h\*(-h).
  \end{equation*}

  For $h<0$,

  \begin{equation*}
    m_h\leq \frac{F(c+h) - F(c)}{h}\leq M_h.
  \end{equation*}

  This inequality holds for any integrable function. However, since $f$
  is continuous,


  \begin{equation*}
    \lim_{h\to 0}m_{h} = \lim_{h\to 0} M_{h} = f(c)
  \end{equation*}

  and thus

  $F'(c)= \lim_{h\to 0}\frac{F(c+h) - F(c)}{h} = f(c)$.
\end{proof}

\begin{note*}
  If $G$ is defined by $G(x) = \int_{x}^bf$, then
  $G(x) = \int_a^b f - \int_a^x f$. Therefore,


  \begin{equation*}
    G'(c)  = - f(c)
  \end{equation*}

\end{note*}

\end{document}