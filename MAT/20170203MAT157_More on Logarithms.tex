% -*- coding: utf-8; -*-
%%% Local Variables:
%%% mode: latex
%%% TeX-engine: xetex
%%% TeX-master: t
%%% End:
\documentclass[11pt]{scrartcl}
\usepackage[fancy, beaue, pset, anon]{masty}
\pSet{\nt{MAT157}{XII}{More on Logarithms}}
\usepackage{lineno}
% ----------------------------------------------------------------------
% Page setup
% ----------------------------------------------------------------------

\pagenumbering{gobble}

% ----------------------------------------------------------------------
% Custom commands
% ----------------------------------------------------------------------

% alignment

\newcommand*{\LongestHence}{$\Rightarrow$}% function name
\newcommand*{\LongestName}{$f_o(-x)+f_e(-x)$}% function name
\newcommand*{\LongestValue}{$(-a)x +(-a)(-y)$}% function value
\newcommand*{\LongestText}{\defi}%

\newlength{\LargestHenceSize}%
\newlength{\LargestNameSize}%
\newlength{\LargestValueSize}%
\newlength{\LargestTextSize}%

\settowidth{\LargestHenceSize}{\LongestHence}%
\settowidth{\LargestNameSize}{\LongestName}%
\settowidth{\LargestValueSize}{\LongestValue}%
\settowidth{\LargestTextSize}{\LongestText}%

% Choose alignment of the various elements here: [r], [l] or [c]

\newcommand*{\mbh}[1]{{\makebox[\LargestHenceSize][r]{\ensuremath{#1}}}}%
\newcommand*{\mbn}[1]{{\makebox[\LargestNameSize][r]{\ensuremath{#1}}}}%
\newcommand*{\mbv}[1]{\ensuremath{\makebox[\LargestValueSize][r]{\ensuremath{#1}}}}%
\newcommand*{\mbt}[1]{\makebox[\LargestTextSize][l]{#1}}%

\newcommand{\R}[1]{\label{#1}\linelabel{#1}}
\newcommand{\lr}[1]{line~\lineref{#1}}

% ----------------------------------------------------------------------
% Launch!
% ----------------------------------------------------------------------

\begin{document}

\section{More on Logarithms}

\begin{theorem}
If $x, y > 0$, then $\log(xy) = \log(x) + \log(y)$
\end{theorem}

\begin{proof}
  For some $y > 0$, let $f(x) = \log(yx)$. Therefore, $f'(x) = \frac{1}{x} = \log'(x)$.

  Thus, there exists a number $c$ such that $f(x) = \log(x) + c$. When
  $x=1$, $f(1) = \log y$ by definition of $f$ and $f(1)=c$ by the obtained relation.

  Hence, $\log xy = \log x + \log y$.
\end{proof}

\begin{corollary}
For all $n\in \NN$, $\log(x^n) = n\log(x)$.
\end{corollary}

\begin{corollary}
$\log(\frac{x}{y}) = \log x - \log y$.
\end{corollary}

Since for any $n\in \NN$, $\log 2^n = n\log 2$. Moreover,
$\log(\frac{1}{2^n}) = -n \log 2$, and thus $\log$ is neither bounded
above nor below. Since $\log$ is continuous, it follows that it takes
all the values in $\RR$.

\begin{definition}
$\forall x\in \RR.\exp x = \log ^{-1} x$ 
\end{definition}

\begin{theorem}
$\forall x\in\RR.\exp' x = \exp x$
\end{theorem}

\begin{proof}
  Observe that

  \begin{align}
    \exp'(x) = (\log^{-1}x)' & = \frac{1}{\log(\log^{-1} x)}    \\
                             & =\frac{1}{\frac{1}{\log^{-1} x}} \\
                             & =\log^{-1}x  = \exp x
  \end{align}
\end{proof}

\begin{theorem}
  \label{sec:more-logarithms}
For any $x, y\in\RR$, $\exp(x+y) = \exp(x)\exp(y)$
\end{theorem}
\begin{proof}
  Let $x' = \exp x$ and $y' = \exp y$.

  Then  $x+y = \log x' +\log y' = \log x'y'$, and thus $\exp(x+y) = \exp(x)\exp(y)$.
\end{proof}

\begin{definition}
$e = \exp(1)$
\end{definition}

From (\ref{sec:more-logarithms}) we obtain that for any $x\in\QQ$,
$\exp(1)^x=\exp(x)$. It is consistent with our earlier use of the
exponential notation to define $e^x$ as $\exp x$ for all $x\in\RR$.

\begin{definition}
$e^x= \exp x$.
\end{definition}

\begin{definition}
If $a>0$, for any real number $a^x= e^{x\log a}$.
\end{definition}


\end{document}