% -*- coding: utf-8; -*-
%%% Local Variables:
%%% mode: latex
%%% TeX-engine: xetex
%%% TeX-master: t
%%% End:
\documentclass[11pt]{scrartcl}
\usepackage[fancy, beaue, pset, anon]{masty}
\pSet{\nt{MAT157}{}{Approximating Functions}}
\usepackage{lineno}
% ----------------------------------------------------------------------
% Page setup
% ----------------------------------------------------------------------

\pagenumbering{gobble}

% ----------------------------------------------------------------------
% Custom commands
% ----------------------------------------------------------------------

% alignment

\newcommand*{\LongestHence}{$\Rightarrow$}% function name
\newcommand*{\LongestName}{$f_o(-x)+f_e(-x)$}% function name
\newcommand*{\LongestValue}{$(-a)x +(-a)(-y)$}% function value
\newcommand*{\LongestText}{\defi}%

\newlength{\LargestHenceSize}%
\newlength{\LargestNameSize}%
\newlength{\LargestValueSize}%
\newlength{\LargestTextSize}%

\settowidth{\LargestHenceSize}{\LongestHence}%
\settowidth{\LargestNameSize}{\LongestName}%
\settowidth{\LargestValueSize}{\LongestValue}%
\settowidth{\LargestTextSize}{\LongestText}%

% Choose alignment of the various elements here: [r], [l] or [c]

\newcommand*{\mbh}[1]{{\makebox[\LargestHenceSize][r]{\ensuremath{#1}}}}%
\newcommand*{\mbn}[1]{{\makebox[\LargestNameSize][r]{\ensuremath{#1}}}}%
\newcommand*{\mbv}[1]{\ensuremath{\makebox[\LargestValueSize][r]{\ensuremath{#1}}}}%
\newcommand*{\mbt}[1]{\makebox[\LargestTextSize][l]{#1}}%

\newcommand{\R}[1]{\label{#1}\linelabel{#1}}
\newcommand{\lr}[1]{line~\lineref{#1}}

% ----------------------------------------------------------------------
% Launch!
% ----------------------------------------------------------------------

\begin{document}

\section{Approximating Functions}

What do we mean when we say that a polynomial approximates $f(x)$?

Can we say what we definitely do not mean by polynomial approximation?
Yes! For example, in case of discontinuous functions, we cannot find a
suitable continuous function \textit{close enough} to be similar.

We have already shown that $f(x) = e^x$, a polynomial
$P_n(x) = \sum_{k=0}^n \frac{x^k}{k!}$ is a \textit{good}
approximation. What we mean by \textit{good} in a special case is, for
instance, that for greater $n$, $P_n(1)$ gets closer to $e$.

Suppose now $g(x) = \log x$.

Note that, for $n>0$, $g^{(n)}(x) = (-1)^{n-1}\frac{(n-1)!}{x^{n}}$.

Therefore, $Q_n(x) = f(a) + \sum_{k=1}^n \frac{-1^{k-1}}{k a^{k}}
(x-a)^k$. However, this is a \textit{very slow} approximation, which
is not \textit{good}.

For another example, take $h(x) = \arctan x$. Since
$h'(x) = \frac{1}{1+x^2}$, to find $h^{(k)}(x)$ is a nontrivial task.

Suppose now $f(x)$ is $n$-time differentiable at $x=a$.

Let $P_n(x) = \sum_{k=0}^n \frac{f^k(x)}{k!}(x-a)^k$.

To make an approximation \textit{good}, one of the methods is to
minimise $\abs{P_n(x) - f(x)}$. However, it does not account for where
the approximation is centered. Therefore, it makes intuitive sense to consider
$\frac{P_n(x) - f(x)}{(x-a)^n}$. Thus, for polynomials,

\begin{equation*}
\lim_{x\to a}\frac{P_n(x) - f(x)}{(x-a)^n} = 0.
\end{equation*}

The great news is that it is also true for any function $f$.

\begin{theorem}
$\lim_{x\to a}\frac{P_n(x) - f(x)}{(x-a)^n} = 0$ for all $f$ and corresponding Taylor polynomials $P_n(x)$.
\end{theorem}

\begin{proof}
  \hfill

  Note the following:

  \begin{align}
    \frac{P_n(x)-f(x)}{(x-a)^n} &= \frac{\sum_{k=0}^{n-1} \frac{f^k(x)}{k!}(x-a)^k - f(x) +\frac{f^n(x)}{n!}(x-a)^n}{(x-a)^{n}}\\
                                &= \frac{\sum_{k=0}^{n-1} \frac{f^k(x)}{k!}(x-a)^k - f(x)}{(x-a)^{n}}  +\frac{f^n(x)}{n!}.
  \end{align}

  We know that the consecutive derivatives of $\sum_{k=0}^{n-1} \frac{f^k(x)}{k!}(x-a)^k$ at $a$ are equal to the derivatives of $f$ at $a$ up to the $(n-1)$-degree.

  Therefore, by recursive application of the l'Hospital rule,

  \begin{align}
    \lim_{x\to a}\frac{\sum_{k=0}^{n-1} \frac{f^k(x)}{k!}(x-a)^k - f(x)}{(x-a)^n} &= \lim_{x\to a} \frac{f^{n-1}(a) - f^{n-1}(x)}{n! (x-a)}\\
                                                                                  &= - \frac{f^n(a)}{n!}.
  \end{align}
  
  Therefore, $\lim_{x\to a}\frac{P_n(x)-f(x)}{(x-a)^n} = 0$, and thus close to $x = a$, $f(x)$ behaves like $P_n(x)$.
\end{proof}

  % Suppose $f'(a) = f''(a) = \cdots = f^{(m-1)}(a) = 0$, while
  % $f^m\neq 0$, and suppose that we want to say that in this case
  % $f(x)$ has a \textit{good} approximation by the Taylor polynomial,
  % with the degree of the minimal nonconstant monomial being equal to
  % $m$.
\end{document}