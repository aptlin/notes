% -*- coding: utf-8; -*-
%%% Local Variables:
%%% mode: latex
%%% TeX-engine: xetex
%%% TeX-master: t
%%% End:
\documentclass[11pt]{scrartcl}
\usepackage[fancy, beaue, pset, anon]{sdll}
\pSet{\nt{MAT247}{V}{Inner Product Spaces}}
\usepackage{lineno}
% ----------------------------------------------------------------------
% Page setup
% ----------------------------------------------------------------------

\pagenumbering{gobble}

% ----------------------------------------------------------------------
% Custom commands
% ----------------------------------------------------------------------

% alignment

\newcommand*{\LongestHence}{$\Rightarrow$}% function name
\newcommand*{\LongestName}{$f_o(-x)+f_e(-x)$}% function name
\newcommand*{\LongestValue}{$(-a)x +(-a)(-y)$}% function value
\newcommand*{\LongestText}{\defi}%

\newlength{\LargestHenceSize}%
\newlength{\LargestNameSize}%
\newlength{\LargestValueSize}%
\newlength{\LargestTextSize}%

\settowidth{\LargestHenceSize}{\LongestHence}%
\settowidth{\LargestNameSize}{\LongestName}%
\settowidth{\LargestValueSize}{\LongestValue}%
\settowidth{\LargestTextSize}{\LongestText}%

% Choose alignment of the various elements here: [r], [l] or [c]

\newcommand*{\mbh}[1]{{\makebox[\LargestHenceSize][r]{\ensuremath{#1}}}}%
\newcommand*{\mbn}[1]{{\makebox[\LargestNameSize][r]{\ensuremath{#1}}}}%
\newcommand*{\mbv}[1]{\ensuremath{\makebox[\LargestValueSize][r]{\ensuremath{#1}}}}%
\newcommand*{\mbt}[1]{\makebox[\LargestTextSize][l]{#1}}%

\newcommand{\R}[1]{\label{#1}\linelabel{#1}}
\newcommand{\lr}[1]{line~\lineref{#1}}

% ----------------------------------------------------------------------
% Launch!
% ----------------------------------------------------------------------

\begin{document}

\section{Introduction to Inner Product Spaces}
Assume that $\FF = \RR$ or $\FF = \CC$.
\subsection{Inner Products}
\begin{definition}
  Suppose $V$ is a vector space over $\RR$ or $\CC$.

  An \textbf{inner product} on $V$ is a function
  $\ipr{\*}{\*}: V\times V\to \FF$ such that
  \begin{enumerate}[label=\alph*)]
  \item $\ipr{\*}{\*}$ is linear in the first component, i.e.
    $\ipr{cx+y}{ z} = c \ipr{x}{ z} + \ipr{y}{ z}$ for all
    $x, y, z \in V$ and $c\in \FF$.
  \item $\ipr{x}{ y} = \ipr{\ol{y}}{ x}$ for all $x, y \in V$, where
    $\ol{y}$ is a complex conjugate.
  \item $\ipr{x}{ x} > 0$ for all non-zero $x$ in $V$
\end{enumerate}
\end{definition}
\begin{remark}
\begin{itemize}
\item If $\FF = \RR$, then by (b) we obtain $\ipr{x}{ y} = \ipr{y}{ x}$.
\item By (b), $\ipr{x}{ x} = \ol{\ipr{x}{x}} \ra \ipr{x}{ x}\in \RR$
\item By (a), $\ipr{0}{x} = 0$ for all $x$ and thus by (b) $\ipr{x}{0} = 0$ for all $x$
\item By (b), we obtain that $\ipr{\*}{ \*}$ is \textbf{conjugate linear} in the second component, since
  \begin{align}
    \text{LHS} & = \ol{\ipr{cy+z}{ x}}             \\
               & = \ol{c \ipr{y}{ x} + \ipr{z}{ x}} \\
               & = \ol{c}\*\ol{\ipr{y}{ x}} + \ol{\ipr{z}{ x}}\\
               & = \ol{c} \ipr{x}{ y} + \ipr{x}{ z} = \text{RHS}
  \end{align}
\end{itemize}

\begin{example}

  If $V = \FF^{n}$, define
  $\ipr{a}{b} = \sum_{i=1}^na_{i}\cdot \ol{b_{i}}$, also known as a
  \textbf{standard inner product on $\FF^{n}$}.

  Check it!
\end{example}
\end{remark}



\end{document}