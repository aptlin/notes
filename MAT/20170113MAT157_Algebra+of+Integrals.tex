% -*- coding: utf-8; -*-
%%% Local Variables:
%%% mode: latex
%%% TeX-engine: xetex
%%% TeX-master: t
%%% End:
\documentclass[11pt]{scrartcl}
\usepackage[fancy, beaue, pset, anon]{sdll}
\pSet{\nt{MAT157}{II.4}{Algebra of Integrals}}
\usepackage{lineno}
% ----------------------------------------------------------------------
% Page setup
% ----------------------------------------------------------------------

\pagenumbering{gobble}

% ----------------------------------------------------------------------
% Custom commands
% ----------------------------------------------------------------------

% alignment

\newcommand*{\LongestHence}{$\Rightarrow$}% function name
\newcommand*{\LongestName}{$f_o(-x)+f_e(-x)$}% function name
\newcommand*{\LongestValue}{$(-a)x +(-a)(-y)$}% function value
\newcommand*{\LongestText}{\defi}%

\newlength{\LargestHenceSize}%
\newlength{\LargestNameSize}%
\newlength{\LargestValueSize}%
\newlength{\LargestTextSize}%

\settowidth{\LargestHenceSize}{\LongestHence}%
\settowidth{\LargestNameSize}{\LongestName}%
\settowidth{\LargestValueSize}{\LongestValue}%
\settowidth{\LargestTextSize}{\LongestText}%

% Choose alignment of the various elements here: [r], [l] or [c]

\newcommand*{\mbh}[1]{{\makebox[\LargestHenceSize][r]{\ensuremath{#1}}}}%
\newcommand*{\mbn}[1]{{\makebox[\LargestNameSize][r]{\ensuremath{#1}}}}%
\newcommand*{\mbv}[1]{\ensuremath{\makebox[\LargestValueSize][r]{\ensuremath{#1}}}}%
\newcommand*{\mbt}[1]{\makebox[\LargestTextSize][l]{#1}}%

\newcommand{\R}[1]{\label{#1}\linelabel{#1}}
\newcommand{\lr}[1]{line~\lineref{#1}}

% ----------------------------------------------------------------------
% Launch!
% ----------------------------------------------------------------------

\begin{document}

\begin{theorem}
  Let $a < c < b$. If $f$ is integrable on $[a, b]$, then for any
  $c\in[a, b]$ $f$ is integrable on $[a,c]$ and $[c, b]$. Conversely,
  if $f$ is integrable on $[a, c]$ and $[c, b]$, then $f$ is also
  integrable on $[a,b]$. Therefore, if $f$ is integrable on $[a,b]$,


  \begin{equation*}
    \int _{a}^{c}f+\int_{c}^{b}f = \int_{a}^{b}f
  \end{equation*}

\end{theorem}

\begin{proof}
  Suppose that $f$ is integrable on $[a,b]$.

  Since $f$ is integrable, there exists a partition such that

  \begin{equation*}
    U(f, P) - L(f, P) < \epsilon
  \end{equation*}

  Consider such a partition $P = \set{t_0, t_1, \dots, t_{n}}$ of
  $[a,b]$.

  Suppose first that $c$ is not of $t_{j}$. Then construct another
  partition $Q$ such that $P\su Q$ to obtain
  $U(f, Q) - L(f, Q) \leq U(f, P) - L(f, P) < \epsilon$. Thus, we may assume
  that $c$ is equal to one of $t_{j}$.

  Consider partitions $P' = [t_0, t_1, \dots, t_j]$ of $[a,c]$ and
  $P'' = [t_{j+1}, \dots, t_n]$ of $[c,b]$. Then by definition of
  $L(\cdot, \cdot)$ it follows that
  \begin{align}
    L(f, P) &= L(f, P') + L(f, P'')\\
    U(f, P) &= U(f, P') + U(f, P'')
  \end{align}

  Therefore,
  $[U(f, P'')-L(f, P'')] + [U(f, P') - L(f, P')] = U(f, P) - L(f, P) <
  \epsilon$.

  Since each term on LHS is nonnegative, it follows that $f$ is
  integrable on $[a, c]$ and $[c, b]$. Note also that
  \begin{align}
    L(f, P') &\leq \int_{a}^{c}f \leq U(f, P')\\
    L(f, P'') &\leq \int_c^bf \leq U(f, P''),
  \end{align}

  and thus $L(f, P) \leq \int_a^cf + \int_c^bf\leq U(f, P)$. Since $P$ was
  chosen arbitrarily, $\int_a^cf + \int_c^bf = \int_a^bf$.

  Conversely, if $f$ is integrable on $[a, c]$ and $[c, b]$, it follows that

  \begin{align}
    U(f, P') - L(f, P') &< \frac{\epsilon}{2}\\
    U(f, P'') - L(f, P'') &< \frac{\epsilon}{2}
  \end{align}

  Construct a partition $P$ containing both $P'$ and $P''$.  Then
  $L(f, P) = L(f, P') + L(f, P'')$ and
  $U(f, P) = U(f, P') + U(f, P'')$.

  Therefore, from inequalities above, $U(f, P) - L(f, P) < \epsilon$,
  and thus $f$ is integrable on $[a,b]$.

\end{proof}

\begin{definition}
  \begin{align}
    \int_a^af  &= 0\\
    \int_b^{a} &= - \int_{a}^bf\\
  \end{align}
\end{definition}

\begin{definition}
  Suppose $f(x)$ is integrable on $[a, b]$. Pick $x\in[a,b]$, and
  define the \textbf{indefinite integral} of $f$:
  \begin{equation*}
    F(x) = \int_{a}^{x}f(t)dt
  \end{equation*}
\end{definition}


\end{document}