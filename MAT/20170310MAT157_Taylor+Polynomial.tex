% -*- coding: utf-8; -*-
%%% Local Variables:
%%% mode: latex
%%% TeX-engine: xetex
%%% TeX-master: t
%%% End:
\documentclass[11pt]{scrartcl}
\usepackage[fancy, beaue, pset, anon]{masty}
\pSet{\nt{MAT157}{}{Taylor Polynomial}}
\usepackage{lineno}
% ----------------------------------------------------------------------
% Page setup
% ----------------------------------------------------------------------

\pagenumbering{gobble}

% ----------------------------------------------------------------------
% Custom commands
% ----------------------------------------------------------------------

% alignment

\newcommand*{\LongestHence}{$\Rightarrow$}% function name
\newcommand*{\LongestName}{$f_o(-x)+f_e(-x)$}% function name
\newcommand*{\LongestValue}{$(-a)x +(-a)(-y)$}% function value
\newcommand*{\LongestText}{\defi}%

\newlength{\LargestHenceSize}%
\newlength{\LargestNameSize}%
\newlength{\LargestValueSize}%
\newlength{\LargestTextSize}%

\settowidth{\LargestHenceSize}{\LongestHence}%
\settowidth{\LargestNameSize}{\LongestName}%
\settowidth{\LargestValueSize}{\LongestValue}%
\settowidth{\LargestTextSize}{\LongestText}%

% Choose alignment of the various elements here: [r], [l] or [c]

\newcommand*{\mbh}[1]{{\makebox[\LargestHenceSize][r]{\ensuremath{#1}}}}%
\newcommand*{\mbn}[1]{{\makebox[\LargestNameSize][r]{\ensuremath{#1}}}}%
\newcommand*{\mbv}[1]{\ensuremath{\makebox[\LargestValueSize][r]{\ensuremath{#1}}}}%
\newcommand*{\mbt}[1]{\makebox[\LargestTextSize][l]{#1}}%

\newcommand{\R}[1]{\label{#1}\linelabel{#1}}
\newcommand{\lr}[1]{line~\lineref{#1}}

% ----------------------------------------------------------------------
% Launch!
% ----------------------------------------------------------------------

\begin{document}

\section{Taylor Polynomial}

For $n\in\NN$, $a\in\SD(f)$ and an $n$-differentiable function $f$, we
define a Taylor polynomial as
\begin{equation*}
  P_{n, a, f} = \sum_{i=0}^n \frac{f^{(i)}(a)}{n!}(x-a)^i.
\end{equation*}

We have already shown that $\lim_{x\to a } \frac{f(x) - P_{n, a, f}(x)}{(x-a)^n} = 0$.

If $f$ is $n$-times differentiable at $x=a$, $P_{n,a, f}$ is well-defined.

However, even if $f$ is not $n$-times differentiable, a polynomial $Q(x)$ can
be found such that $\lim_{x\to a } \frac{f(x) - P(x)}{(x-a)^n} = 0$.

In this situation, $f$ is said to agree mith $P$ up to order $n$.

\begin{theorem}
  If $P, Q$ are polynomials of degree less that or equal to $n$, and $P$ agrees with $Q$ up to order $n$, then $P = Q$.
\end{theorem}

\begin{proof}
  \hfill

  Let $f = P - Q$ be a polynomial of degree less than or equal to $n$.

  Write $f(x) = a_0 + a_1(x-a) + \cdots +a_n(x-a)^n$. Note that
  $\lim_{x\to a} \frac{f(x)}{(x-a)^n} = 0$ , and thus for all
  $k\leq n$ we have $\lim_{x\to a} \frac{f(x)}{(x-a)^k} = 0$.

  Therefore, we obtain that $\lim_{x\to a} f(x) = 0$, and thus $a_0 =0$.

  By induction, $a_k= a_1$ for $k\in[0, n]\cap \NN$.
\end{proof}

\begin{corollary}
  Suppose that  $f(x)$ is a $n$-times differentiable at $x=a$ and $P$ is a polynomial of degree less than or equal to $n$, which agrees with $f$ up to order $n$. Then $P = P_{n, a, f}$.

  \begin{proof}
    \hfill

    Observe that $P(x)$ and  $P_{n, a, f}$ both agree with $f$ up to order $n$.

    Therefore, 
    \begin{equation*}
      \lim_{x\to a} \frac{P(x)-f(x)}{(x-a)^n} + \frac{f(x)-P_{n, a, f}(a)}{(x-a)^n} = 0.
    \end{equation*}

    Then $\lim_{x\to a} \frac{P-P_{n, a, y}}{(x-a)^n} = 0$, which proves the result.

  \end{proof}
\end{corollary}

Consider $\arctan x = \int_0^{k} \frac{1}{1+x^{2}}$ for $x\in(-1, 1)$.

Note that $\frac{1}{1+t^2} = 1-t^2+t^4s-\cdots +(-1)^2p^{2n}$.

Therefore,
$\arctan x = \int_0^x
\frac{1}{1+t^2}=\int_0^a(1-t^2+t^4+\dots+(-1)^{n}t^{2n} +
\frac{(-1)^{n+1}t^{2n+2}}{1+t^{2}} \dif t$.

Thus,

\[\arctan x = x- \frac{x^3}{3!}+\frac{x^5}{5!} + \dots + (-1)^n \frac{x^{2n+1}}{2n+1} + \int_0^x \frac{(-1)^{n+1}t^{2n+2}}{1+t^2}\dif t\].

The right hand side but for the last term \textit{should be} $P_{2n+1, 0, \arctan}$.

This would hold if it agrees with $\arctan x$ up to order $2n+1$.

Consider $\lim_{x\to 0} \frac{\frac{(-1)^{n+1}x^{2n+2}}{1+x^2}}{(2n+1)x^{2n}}$.

Thus, taking an absolute value, we obtain that


\begin{equation*}
  \lim_{x\to 0} \frac{\frac{\abs{x}^{2n+2}}{1+x^2}}{(2n+1)\abs{x}^{2n}} = \lim_{x\to 0} \frac{\frac{\abs{x}^2}{1+x^2}}{2n+1} < \lim_{x\to 0} \frac{\abs{x}^2}{2n+1} = 0.
\end{equation*}




\end{document}