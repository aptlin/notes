% -*- coding: utf-8; -*-
%%% Local Variables:
%%% mode: latex
%%% TeX-engine: xetex
%%% TeX-master: t
%%% End:
\documentclass[11pt]{scrartcl}
\usepackage[fancy, beaue, pset, anon]{masty}
\pSet{\nt{MAT157}{XVII}{Integration Tricks}}
\usepackage{lineno}
% ----------------------------------------------------------------------
% Page setup
% ----------------------------------------------------------------------

\pagenumbering{gobble}

% ----------------------------------------------------------------------
% Custom commands
% ----------------------------------------------------------------------

% alignment

\newcommand*{\LongestHence}{$\Rightarrow$}% function name
\newcommand*{\LongestName}{$f_o(-x)+f_e(-x)$}% function name
\newcommand*{\LongestValue}{$(-a)x +(-a)(-y)$}% function value
\newcommand*{\LongestText}{\defi}%

\newlength{\LargestHenceSize}%
\newlength{\LargestNameSize}%
\newlength{\LargestValueSize}%
\newlength{\LargestTextSize}%

\settowidth{\LargestHenceSize}{\LongestHence}%
\settowidth{\LargestNameSize}{\LongestName}%
\settowidth{\LargestValueSize}{\LongestValue}%
\settowidth{\LargestTextSize}{\LongestText}%

% Choose alignment of the various elements here: [r], [l] or [c]

\newcommand*{\mbh}[1]{{\makebox[\LargestHenceSize][r]{\ensuremath{#1}}}}%
\newcommand*{\mbn}[1]{{\makebox[\LargestNameSize][r]{\ensuremath{#1}}}}%
\newcommand*{\mbv}[1]{\ensuremath{\makebox[\LargestValueSize][r]{\ensuremath{#1}}}}%
\newcommand*{\mbt}[1]{\makebox[\LargestTextSize][l]{#1}}%

\newcommand{\R}[1]{\label{#1}\linelabel{#1}}
\newcommand{\lr}[1]{line~\lineref{#1}}

% ----------------------------------------------------------------------
% Launch!
% ----------------------------------------------------------------------

\begin{document}

\section{Integration Tricks}

For rational functions of $\sin$ and $\cos$, if all else fails, try the substitution $x = \tan \frac{\theta}{2}$.

Thus, $1+x^2 = \sec^2(\frac{\theta}{2})$, and hence $\frac{1}{1+x^2} = \cos^2(\frac{\theta}{2})$, and hence $\cos \theta = \frac{1-x^2}{1+x^2}$. Moreover, $\sin^2 \theta = \frac{4x^2}{(1+x^2)^2}$, which means that $\sin \theta = \frac{2x}{1+x^2}$.

In this way, this substitution converts any rational function of
$\cos$ and $\sin$ into a rational function of $x$. In principle, this
leads to an integral which always can be integrated. Generally,
however, this substitution results in an unwieldy equation, and thus
this method should be used as a last resort.

\section{Solids of Revolution}

Suppose a curve is given in the first quarter of the Cartesian
plane, defined over the interval $[a, b]$. Rotate this curve around the $x$ axis to obtain a \textit{solid of revolution}.

Informally, the volume of the obtained solid can be calculated by
summing the volumes of very thin slices. Suppose that $\dif x$ is the
thickness of the slice. Then its volume is $\pi f(x)^2 \dif x$, and
hence $V = \int_a^b \pi f(x)^2 \dif x.$

Consider a cone, of height $h$ and radius $r$, with the vertex at the
origin. Then $y=f(x) = \frac{r}{h} x$ is the corresponding
curve. Thus, $V = \frac{\pi}{3}r^2h$.
\end{document}