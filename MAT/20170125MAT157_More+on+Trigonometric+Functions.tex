% -*- coding: utf-8; -*-
%%% Local Variables:
%%% mode: latex
%%% TeX-engine: xetex
%%% TeX-master: t
%%% End:
\documentclass[11pt]{scrartcl}
\usepackage[fancy, beaue, pset, anon]{masty}
\pSet{\nt{MAT157}{VIII}{More on Trigonometric Functions}}
\usepackage{lineno}
% ----------------------------------------------------------------------
% Page setup
% ----------------------------------------------------------------------

\pagenumbering{gobble}

% ----------------------------------------------------------------------
% Custom commands
% ----------------------------------------------------------------------

% alignment

\newcommand*{\LongestHence}{$\Rightarrow$}% function name
\newcommand*{\LongestName}{$f_o(-x)+f_e(-x)$}% function name
\newcommand*{\LongestValue}{$(-a)x +(-a)(-y)$}% function value
\newcommand*{\LongestText}{\defi}%

\newlength{\LargestHenceSize}%
\newlength{\LargestNameSize}%
\newlength{\LargestValueSize}%
\newlength{\LargestTextSize}%

\settowidth{\LargestHenceSize}{\LongestHence}%
\settowidth{\LargestNameSize}{\LongestName}%
\settowidth{\LargestValueSize}{\LongestValue}%
\settowidth{\LargestTextSize}{\LongestText}%

% Choose alignment of the various elements here: [r], [l] or [c]

\newcommand*{\mbh}[1]{{\makebox[\LargestHenceSize][r]{\ensuremath{#1}}}}%
\newcommand*{\mbn}[1]{{\makebox[\LargestNameSize][r]{\ensuremath{#1}}}}%
\newcommand*{\mbv}[1]{\ensuremath{\makebox[\LargestValueSize][r]{\ensuremath{#1}}}}%
\newcommand*{\mbt}[1]{\makebox[\LargestTextSize][l]{#1}}%

\newcommand{\R}[1]{\label{#1}\linelabel{#1}}
\newcommand{\lr}[1]{line~\lineref{#1}}

% ----------------------------------------------------------------------
% Launch!
% ----------------------------------------------------------------------

\begin{document}

The values of $\sin x$ and $\cos x$ not in $[0, \pi]$ can be defined
by a two-step piecing process:

\begin{enumerate}
\item\label{item:1} If $\pi\leq x\leq 2\pi$, then
  \begin{align}
    \sin x &= - \sin(2\pi - x)\\
    \cos x &= \cos(2\pi - x)
  \end{align}
\item If $ x = 2\pi k + x'$ for some $k\in \ZZ$ and some
  $x'\in [0,2\pi]$, then

  \begin{align}
    \sin x &= \sin(x')\\
    \cos x &= \cos(x')
  \end{align}
\end{enumerate}

This extended definition is consistent with all the usual properties
we expect from the trigonometric functions:

\begin{enumerate}
\item\label{item:2} $\sin^2x + \cos^2x  = 1$, by the geometric argument
\item \begin{align}
        \sin'(x) &= \cos x\\
        \cos'(x) &= -\sin x
      \end{align}

      For example, if $\pi < x < 2\pi$, then
      $\sin x = -\sin (2\pi - x)$, and thus \[\sin'(x) = -\sin'(2\pi - x)(-1) = \cos x.\]

      If $x$ is a multiple of $\pi$, then considering the fact that
      $\sin$ is continuous in the $\epsilon$-neighbourhood of $x$ will
      give us a similar conclusion.
    \end{enumerate}

\begin{theorem}

  If $-1<x<1$, then

  \begin{align}
    \arcsin'(x) &= \frac{1}{\sqrt{1-x^2}}\\
    \arccos'(x) &= \frac{-1}{\sqrt{1-x^2}}
  \end{align}

  If $x \in \RR$, then
  \begin{equation*}
    \arctan'(x) = \frac{1}{1+x^2}
  \end{equation*}
\end{theorem}
\begin{proof}
  \begin{align}
    \arcsin'(x) & = (f^{-1})'(x)               \\
                & = \frac{1}{f'(f^{-1}(x))}    \\
                & = \frac{1}{\sin'(\arcsin x)} \\
                & = \frac{1}{\cos(\arcsin x)}
  \end{align}

  Note that $\sin^2(\arcsin x) + \cos^2(\arcsin x) = 1$, and thus

  \begin{equation*}
    \cos(\arcsin x)^2 = \sqrt{1-x^{2}}
  \end{equation*}

  The second formula can be derived from the fact that
  \begin{equation*}
    A(x) = \frac{x\sqrt{1-x^2}}{2} + \int_x^1\sqrt{1-t^2} \dif t.
  \end{equation*}
  and that $2A(\cos x) = x$.

  Finally, by Pythagoras's identity,

  \begin{align}
    \arctan'(x) & = (h^{-1})'(x)               \\
                & = \frac{1}{h'(h^{-1}(x))}    \\
                & = \frac{1}{\tan'(\arctan x)} \\
                & = \frac{1}{\sec^{2}(\arcsin x)}\\
                & = \frac{1}{x^1+1}
  \end{align}
\end{proof}

\begin{lemma}
  Suppose $f$ has a second derivative everywhere and that the following conditions are satisfied:

  \begin{align}
    \label{eq:1}
    f''   + f & = 0 \\
    f(0)      & = 0 \\
    f'(0)     & = 0
  \end{align}

  Then $f$ is a zero function.
\end{lemma}

\begin{proof}
  From the equation (\ref{eq:1}) given we obtain that
  
  \begin{equation*}
    f'f''+ff'=0.
  \end{equation*}

  Therefore, $(f')^2+f^2$ is a constant function, which by the other
  two conditions is equal to 0. Therefore, $f(x) = 0$ for all $x$.

\end{proof}

\begin{lemma}
  \label{sec:1}
  Suppose $f$ has a second derivative everywhere and that the following conditions are satisfied:

  \begin{align}
    \label{eq:2}
    f''   + f & = 0 \\
    f(0)      & = a \\
    f'(0)     & = b
  \end{align}

  Then $f$ is in the form $b\*\sin  + a \* \cos$.
\end{lemma}

\begin{proof}

  We use the result given by the previous lemma.
  
  Let $g(x) = f(x) - b \sin x - a \cos x$.

  Then
  \begin{align}
    & g'(x) = f'(x) - b\cos x + a \sin x\\
    & g''(x) = f''(x) + b\sin x + a \cos x
  \end{align}

  Note that $g''+g = 0$, $g(0)= 0$, and $g'(0) = 0$,
  which shows that $g = 0$.
\end{proof}

\begin{theorem}
  If $x, y \in \RR$, then
  \begin{align}
    &\sin(x+y) = \sin x\cos x + \cos x \sin y\\
    &\cos(x+y) = \cos x \cos y - \sin x \sin y
  \end{align}
\end{theorem}
\begin{proof}
  For any particular number $y$ we can shift $\sin$ so that
  $f(x) = \sin(x+y)$. Then $f'(x) = \cos(x+y)$ and
  $f''(x) = -\sin(x+y)$.

  Therefore,
  \begin{align}
    f''   + f & = 0 \\
    f(0)      & = \sin y \\
    f'(0)     & = \cos y,
  \end{align}

  which by Lemma \ref{sec:1} implies that
  
  \begin{equation*}
    f(x) = \sin x \cos y + \cos y \sin x,
  \end{equation*}
  and thus $\sin(x+y) = \sin x\cos x + \cos x \sin y$.

  The second formula can be proven similarly.

\end{proof}
    
\end{document}