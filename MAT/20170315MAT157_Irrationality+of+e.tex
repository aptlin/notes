% -*- coding: utf-8; -*-
%%% Local Variables:
%%% mode: latex
%%% TeX-engine: xetex
%%% TeX-master: t
%%% End:
\documentclass[11pt]{scrartcl}
\usepackage[fancy, beaue, pset, anon]{masty}
\pSet{\nt{MAT157}{}{Irrationality of $e$}}
\usepackage{lineno}
% ----------------------------------------------------------------------
% Page setup
% ----------------------------------------------------------------------

\pagenumbering{gobble}

% ----------------------------------------------------------------------
% Custom commands
% ----------------------------------------------------------------------

% alignment

\newcommand*{\LongestHence}{$\Rightarrow$}% function name
\newcommand*{\LongestName}{$f_o(-x)+f_e(-x)$}% function name
\newcommand*{\LongestValue}{$(-a)x +(-a)(-y)$}% function value
\newcommand*{\LongestText}{\defi}%

\newlength{\LargestHenceSize}%
\newlength{\LargestNameSize}%
\newlength{\LargestValueSize}%
\newlength{\LargestTextSize}%

\settowidth{\LargestHenceSize}{\LongestHence}%
\settowidth{\LargestNameSize}{\LongestName}%
\settowidth{\LargestValueSize}{\LongestValue}%
\settowidth{\LargestTextSize}{\LongestText}%

% Choose alignment of the various elements here: [r], [l] or [c]

\newcommand*{\mbh}[1]{{\makebox[\LargestHenceSize][r]{\ensuremath{#1}}}}%
\newcommand*{\mbn}[1]{{\makebox[\LargestNameSize][r]{\ensuremath{#1}}}}%
\newcommand*{\mbv}[1]{\ensuremath{\makebox[\LargestValueSize][r]{\ensuremath{#1}}}}%
\newcommand*{\mbt}[1]{\makebox[\LargestTextSize][l]{#1}}%

\newcommand{\R}[1]{\label{#1}\linelabel{#1}}
\newcommand{\lr}[1]{line~\lineref{#1}}

% ----------------------------------------------------------------------
% Launch!
% ----------------------------------------------------------------------

\begin{document}

\section{Irrationality of $e$}

Note that
$e = 1+ \sum_{i=1}^n\frac{1}{i!}+ R_n(1) = 1+
\sum_{i=1}^n\frac{1}{i!}+ \frac{e^t}{(n+1)!}$ for some $t\in[0, 1]$.

Note that $e^{x}$ is increasing, and thus $e^t\leq e < 3$.

Therefore, $R_n(1) \leq \frac{3}{(n+1)!}$.

\begin{theorem}
$e$ is irrational.
\end{theorem}

\begin{proof}
  \hfill

  Suppose that $e = \frac{a}{b}\in\QQ$.

  Choose $n> b$, $n \geq 4$. Then
  
  \begin{equation*}
    \frac{a}{b} = 1+ \sum_{i=1}^n{1}{i!} + R_n(1)
  \end{equation*}

  Therefore, $n! \frac{a}{b} = n! + n! \sum_{i=1}^{n} \frac{1}{i!}+ n! R_{n}$.

  Therefore, $n!R_n \in \ZZ$.

  But $R_n \leq \frac{3}{(n+1)!}$, and thus
  $n!R_n \leq \frac{3}{n+1} \leq \frac{3}{5}$ and hence it is not an
  integer.

  Therefore, by contradiction, $e$ is irrational.
\end{proof}

\begin{theorem}
$e$ is not algebraic.
\end{theorem}

\begin{proof}
  \hfill

Exercise.
\end{proof}


\end{document}