% -*- coding: utf-8; -*-
%%% Local Variables:
%%% mode: latex
%%% TeX-engine: xetex
%%% TeX-master: t
%%% End:
\documentclass[11pt]{scrartcl}
\usepackage[fancy, beaue, pset, anon]{masty}
\pSet{\nt{MAT247}{}{Minimal Polynomials}}
\usepackage{lineno}
% ----------------------------------------------------------------------
% Page setup
% ----------------------------------------------------------------------

\pagenumbering{gobble}

% ----------------------------------------------------------------------
% Custom commands
% ----------------------------------------------------------------------

% alignment

\newcommand*{\LongestHence}{$\Rightarrow$}% function name
\newcommand*{\LongestName}{$f_o(-x)+f_e(-x)$}% function name
\newcommand*{\LongestValue}{$(-a)x +(-a)(-y)$}% function value
\newcommand*{\LongestText}{\defi}%

\newlength{\LargestHenceSize}%
\newlength{\LargestNameSize}%
\newlength{\LargestValueSize}%
\newlength{\LargestTextSize}%

\settowidth{\LargestHenceSize}{\LongestHence}%
\settowidth{\LargestNameSize}{\LongestName}%
\settowidth{\LargestValueSize}{\LongestValue}%
\settowidth{\LargestTextSize}{\LongestText}%

% Choose alignment of the various elements here: [r], [l] or [c]

\newcommand*{\mbh}[1]{{\makebox[\LargestHenceSize][r]{\ensuremath{#1}}}}%
\newcommand*{\mbn}[1]{{\makebox[\LargestNameSize][r]{\ensuremath{#1}}}}%
\newcommand*{\mbv}[1]{\ensuremath{\makebox[\LargestValueSize][r]{\ensuremath{#1}}}}%
\newcommand*{\mbt}[1]{\makebox[\LargestTextSize][l]{#1}}%

\newcommand{\R}[1]{\label{#1}\linelabel{#1}}
\newcommand{\lr}[1]{line~\lineref{#1}}

% ----------------------------------------------------------------------
% Launch!
% ----------------------------------------------------------------------

\begin{document}

\section{Minimal Polynomials}

\subsection{Review}
\begin{itemize}
\item There is a factorisation of any non-zero polynomial into
  distinct monic irreducible polynomials unique up to the ordering of
  the factors.
\item 

Let $T \in \End(V)$ be arbitrary.

Note that a \textit{minimal polynomial} of $T$ is a monic polynomial $p(t)$ of the least degree such that $p(T) = 0$.

Cayley-Hamilton Theorem gives us an upper bound on the degree of a monic polynomial:

\begin{equation*}
\deg p(t) \leq \dim V
\end{equation*}
\end{itemize} 

\subsection{Uniqueness of the Minimal Polynomial}

\begin{theorem}
Assume $p(t)$ is a minimal polyonmial of $T$.

\begin{enumerate}[label=\alph*)]
\item \label{item:1} If $g(t) \in \SP(\FF)$ is a ny polynomial such that $g(T) = 0$, then $p(t) | g(t)$
\item $p(t)$ is the unique minimal polynomial
\end{enumerate}
\end{theorem}
\begin{remark}
  Note that \ref{item:1} implies by Cayley-Hamilton Theorem that $p(t) | f(t)$, where $f(t)$ is a characteristic polynomial.
\end{remark}
\begin{proof}
  \hfill

Using the division algorithm, we know that $g(t) = q(t)p(t) +r(t)$, where $\deg(r) < \deg (p)$.

Plugging in $T$, we know that $g(T) = q(T) p(T) + r(T)$.

Since $g(T) = 0$ and $p(T)$, then $r(T) = 0$.

If $r(t) \neq 0$, it can be rescaled to be monic, but
$\deg r < \deg p$, which contradicts that $p(t)$ is minimal, and hence
we can deduce that $r(t) = 0$.

Therefore, $r(t) = 0$, so $p(t) | g(t)$.

Suppose now that $p'(t)$ is another minimal polynomial. Therefore, $\deg p'(t) = \deg p(t)$.

Moreover, $p(T) = p'(T) = 0$, and thus $p(T) | p'(T)$. But $p$ and $p'$ have the same degree, and thus $p'(t) = cp(t)$ for $c\in \FF\setminus \set{0}$. Since $p$ and $p'$ are monic, then $c=1$, and hence $p(t) = p'(t)$.
\end{proof}

\begin{theorem}
\label{sec:uniq-minim-polyn}
The characteristic polynomial $f(t)$ and the minimal polynomial $p(t)$ have the same zeroes ni $\FF$.
\end{theorem}

\begin{proof}
  \hfill

  If a minimal polynomial $p(t)$ has a zero, since $p(t) | f(t)$, then $f(t)$ also has a zero.

  Suppose now that $\lambda$ is a zero of $f(t)$, so $\lambda$ is an eigenvector of $T$.

  Pick $x\neq 0$ such that $Tx = \lambda x$. Therefore, $p(T) x = 0$,
  since $p$ is a minimal polynomial.

  Since $x$ is an eigenvector, we get that $p(\lambda) x = 0$, and
  since $x\neq 0$, then $p(\lambda) = 0$.
\end{proof}

\begin{corollary}
  If the characteristic polynomial
  $f(t) = (-1)^n\prod_{i=1}^n(t-\lambda_i)^{n_i}$, where $\lambda_i$
  are the distinct eigenvalues, then the minimal polynomial is
  $p(t) = \prod_{i=1}^n(t-\lambda_i)^{d_i}$, where
  $1 \leq d_i \leq n_i$.
\end{corollary}

\begin{proof}
  \hfill

Use $p(t) | f(t)$ by Theorem \ref{sec:uniq-minim-polyn} and unique factorisation.

\end{proof}

\begin{example}

Let $A$ be equal to 
$
\begin{pmatrix}
2 & 2 & 3\\
1 & 3 & 3\\
-1 & -2 & -2
\end{pmatrix}
$ over $\FF= \QQ$. 

We have already seen that $f(t) = -(t-1)^3$.

Therefore, $p(t) = (t-1)^d$ for $1\leq d \leq 3$.

The dot diagram is $
\begin{pmatrix}
\bullet & \bullet\\
\bullet
\end{pmatrix}
$, which means that $A-I \neq 0$ and the nullity is 2.

Therefore, $(A-I)^2= 0$.

\end{example}

\begin{theorem}
  Suppose $\dim V = n$ and $V$ is the $T$-cyclic subspace generated by
  $x\in V$. Then the minimal polynomial has a degree of $n$ and
  $f(t) = (-1)^np(t)$.
\end{theorem}

\begin{proof}
  \hfill

We know that $\beta=\set{x, Tx, \dots, T^{n-1}x}$ is a basis of $V$.

Suppose that $g(t)$ is a polynomial of degree less that $n$, so
$g(t) = \sum_{i=0}^{n-1}a_{i}t^i$ for some $a_i \in \FF$.

Then $g(T) = a_0I+ \sum_{i=1}^{n-1}a_iT^i$, and hence $g(T) = 0$,
which means that $g(T)(x) = 0$, and thus $a_{n-1} = \cdots = a_0 = 0$,
because $\beta$ is a basis. 

Therefore, the minimal polynomial has a degree of $n$.
\end{proof}

\begin{theorem}
  $T$ is diagonalisable if and only if them minimal polynomial of $T$ is of the form $\prod_{i=1}^{s} (t-\lambda_i)$, where $\lambda_i\in\FF$ are distinct.
\end{theorem}
\begin{description}

\item[e.g.] $T = \lambda I$ if and only if the minimal polynomial is $(t-\lambda)$

\end{description}
\begin{proof}
  \hfill

  Suppose that $\lambda_1, \dots, \lambda_s\in\FF$ are  distinct eigenvalues of $T$.

  Therefore, $V = \bigoplus_{i=1}^{s} E_{\lambda_i}$.

  By Theorem \ref{sec:uniq-minim-polyn}, $g(t) = \prod_{i=1}^{s} (t-\lambda_i) | p(t)$.

  We need to show that $g(t)$ is indeed a minimal polynomial.

  It is enough to show that $g(T)(x) = 0$ for all $x\in E_{\lambda_i}$
  for all possible $\lambda_i$.

  But $g(T)(x) = g(\lambda_i)(x)$, since $x$ is an eigenvector. But
  $g(\lambda_i) = 0$, and therefore $g(T) = 0$.

  \textbf{Exercise:} Prove the other direction.
\end{proof}

\end{document}
