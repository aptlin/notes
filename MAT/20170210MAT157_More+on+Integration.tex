% -*- coding: utf-8; -*-
%%% Local Variables:
%%% mode: latex
%%% TeX-engine: xetex
%%% TeX-master: t
%%% End:
\documentclass[11pt]{scrartcl}
\usepackage[fancy, beaue, pset, anon]{masty}
\pSet{\nt{MAT157}{XV}{More on Integration}}
\usepackage{lineno}
% ----------------------------------------------------------------------
% Page setup
% ----------------------------------------------------------------------

\pagenumbering{gobble}

% ----------------------------------------------------------------------
% Custom commands
% ----------------------------------------------------------------------

% alignment

\newcommand*{\LongestHence}{$\Rightarrow$}% function name
\newcommand*{\LongestName}{$f_o(-x)+f_e(-x)$}% function name
\newcommand*{\LongestValue}{$(-a)x +(-a)(-y)$}% function value
\newcommand*{\LongestText}{\defi}%

\newlength{\LargestHenceSize}%
\newlength{\LargestNameSize}%
\newlength{\LargestValueSize}%
\newlength{\LargestTextSize}%

\settowidth{\LargestHenceSize}{\LongestHence}%
\settowidth{\LargestNameSize}{\LongestName}%
\settowidth{\LargestValueSize}{\LongestValue}%
\settowidth{\LargestTextSize}{\LongestText}%

% Choose alignment of the various elements here: [r], [l] or [c]

\newcommand*{\mbh}[1]{{\makebox[\LargestHenceSize][r]{\ensuremath{#1}}}}%
\newcommand*{\mbn}[1]{{\makebox[\LargestNameSize][r]{\ensuremath{#1}}}}%
\newcommand*{\mbv}[1]{\ensuremath{\makebox[\LargestValueSize][r]{\ensuremath{#1}}}}%
\newcommand*{\mbt}[1]{\makebox[\LargestTextSize][l]{#1}}%

\newcommand{\R}[1]{\label{#1}\linelabel{#1}}
\newcommand{\lr}[1]{line~\lineref{#1}}

% ----------------------------------------------------------------------
% Launch!
% ----------------------------------------------------------------------

\begin{document}

\begin{example}

  Let $u=\cos x$, and thus $\dif u = -\sin(x) dx$. Therefore,

  \begin{align}
    \int \sin(x)\cos^7(x) \dif x & = -\int u^7 \dif u \\
                                 & = -\frac{u^8}{8}   \\
                                 & = -\frac{1}{8}\cos^8
  \end{align}
\end{example}

\begin{example}

  If $m, n\in\NN$ are not both even, then
  $\int \cos^m x\sin^nx \dif x$ can be solved by using the substitution as above.

  If $m, n$ are both even, the equation can be reduced to something
  more manageable by using double angle formulas.
\end{example}

\section{Reduction Formula}

Suppose $\int \sin ^n x \dif x = \sin x \sin^{n-1}x \dif x$ is
given. Integrating it by parts, we obtain:

\begin{align}
  \int \sin^nx \dif x &= -\cos x \sin^{n-1} x + (n-1)\int \sin^{n-2}x\cos^2x \dif x\\
                      &= -\frac{1}{n}\cos x \sin^{n-1}x+\frac{n-1}{n}\int \sin^{n-2}\dif x
\end{align}

\section{Examples}

\begin{example}

  Note that $\sec \theta \tan \theta + \sec^2\theta = \sec \theta(\tan\theta + \sec\theta)$.

  Therefore, $\sec \theta = \frac{\sec \theta(\tan\theta + \sec\theta)}{\tan\theta + \sec\theta)} = \frac{f'}{f}$

  $\int \sec\theta \dif \theta = \log \abs{\sec \theta +\tan \theta}$.
\end{example}

\begin{example}

  \begin{align}
    \int \frac{1}{x^2 \sqrt{1-x^2}}\dif x &= \int \frac{1}{\cos^2\theta\sin\theta}\dif\theta\\
    &= -\int \sec^2\theta = -\tan \theta
  \end{align}
\end{example}

\begin{example}

  Consider now $\int x \sqrt{9-x^2}$.

  Let $x=3\sin x$, and $\dif x = 3\cos x$ .

  Therefore,

  \begin{align}
    \int x \sqrt{9-x^2} & = 27\int \sin \theta \cos^2\theta \\
                        & = -9 \cos ^3\theta                \\
                        & =-9(1-\frac{x^2}{9})^{\nicefrac{3}{2}}
  \end{align}

\end{example}
\end{document}