% -*- coding: utf-8; -*-
%%% Local Variables:
%%% mode: latex
%%% TeX-engine: xetex
%%% TeX-master: t
%%% End:
\documentclass[11pt]{scrartcl}
\usepackage[fancy, beaue, pset, anon]{masty}
\pSet{\nt{MAT157}{XIII}{More on the Exponential Functions}}
\usepackage{lineno}
% ----------------------------------------------------------------------
% Page setup
% ----------------------------------------------------------------------

\pagenumbering{gobble}

% ----------------------------------------------------------------------
% Custom commands
% ----------------------------------------------------------------------

% alignment

\newcommand*{\LongestHence}{$\Rightarrow$}% function name
\newcommand*{\LongestName}{$f_o(-x)+f_e(-x)$}% function name
\newcommand*{\LongestValue}{$(-a)x +(-a)(-y)$}% function value
\newcommand*{\LongestText}{\defi}%

\newlength{\LargestHenceSize}%
\newlength{\LargestNameSize}%
\newlength{\LargestValueSize}%
\newlength{\LargestTextSize}%

\settowidth{\LargestHenceSize}{\LongestHence}%
\settowidth{\LargestNameSize}{\LongestName}%
\settowidth{\LargestValueSize}{\LongestValue}%
\settowidth{\LargestTextSize}{\LongestText}%

% Choose alignment of the various elements here: [r], [l] or [c]

\newcommand*{\mbh}[1]{{\makebox[\LargestHenceSize][r]{\ensuremath{#1}}}}%
\newcommand*{\mbn}[1]{{\makebox[\LargestNameSize][r]{\ensuremath{#1}}}}%
\newcommand*{\mbv}[1]{\ensuremath{\makebox[\LargestValueSize][r]{\ensuremath{#1}}}}%
\newcommand*{\mbt}[1]{\makebox[\LargestTextSize][l]{#1}}%

\newcommand{\R}[1]{\label{#1}\linelabel{#1}}
\newcommand{\lr}[1]{line~\lineref{#1}}

% ----------------------------------------------------------------------
% Launch!
% ----------------------------------------------------------------------

\begin{document}

\section{More on the Exponential Functios}

Note that for $a>0$,
\begin{equation*}
  a^{r} = e^{r\log a}.
\end{equation*}

Therefore,

\begin{align}
  \frac{\dif }{\dif x}x^{r} & = \frac{\dif }{\dif x}e^{r\log x} \\
                            & = \frac{r}{x} e^{r\log x}         \\
                            & =rx^{r-1}
\end{align}

Define $f$ as follows:
\begin{equation*}
  f(x) = \begin{cases}
    e^{\frac{-1}{x^2}}, x\neq 0\\
    0, x = 0
  \end{cases}
\end{equation*}

It can be proven that $f(x)$ is continuous.

In fact, it is smooth and $f^{(k)}(0) = 0$ for all $k\in \NN$.

Consider $g(x) = \frac{(x+2)^4}{(x^2-x+1)^5}\arctan(x)$.

Even though finding $g'$ is straightforward, it is messy. Taking logs, however, is easier:

\begin{equation*}
\log g(x) = 4 \log(x+2) - 5 \log(x^2-x+1) + \log(\arctan(x)).
\end{equation*}

This makes differentiation cleaner.

By chain rule,
\begin{equation*}
  \frac{\dif }{\dif x}\log(g(x)) = \frac{g'(x)}{g(x)}.
\end{equation*}

So
\begin{align}
  g'(x) & = g(x) \frac{\dif}{\dif x}\log(g(x)) \\
        & = \frac{(x+2)^4}{(x^2-x+1)^5}\arctan(x)[\frac{4}{x+2}-\frac{5}{x^2-x +1}+\frac{1}{(1+x^2)\arctan x}]
\end{align}

\section{Hyperbolic Functions}

\begin{definition}
  \begin{align}
    \cosh(x) & = \frac{e^x+e^{-x}}{2} \\
    \sinh(x) & = \frac{e^x-e^{-x}}{2} \\
    \tanh(x) & = \frac{\sinh(x)}{\cosh(x)}
  \end{align}
\end{definition}

Note the following:
\begin{align}
  \frac{\dif }{\dif x}\cosh x & = \sinh x     \\
  \frac{\dif }{\dif x}\sinh x & = \cosh x     \\
  \frac{\dif }{\dif x}\sinh x & = \sech^{2} x \\
  \cosh^2 x -\sinh^2x         & =1
\end{align}

Note that the area of the hyperbolic sector for some $x$ is $\frac{x}{2}$.

The shape of the $\cosh x$ graph is \textit{catenary}, which is the
shape of a hanging rope with the uniform mass.


\section{Techniques of Integration}
\subsection{Integration by Parts}
Recall product rule:
\[(uv)' = u'v + uv'\]

Then
\[\int uv' = uv - \int u' v \]

\end{document}