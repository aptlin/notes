% -*- coding: utf-8; -*-
%%% Local Variables:
%%% mode: latex
%%% TeX-engine: xetex
%%% TeX-master: t
%%% End:
\documentclass[11pt]{scrartcl}
\usepackage[fancy, beaue, pset, anon]{masty}
\pSet{\nt{MAT247}{}{Canonical Forms}}
\usepackage{lineno}
% ----------------------------------------------------------------------
% Page setup
% ----------------------------------------------------------------------

\pagenumbering{gobble}

% ----------------------------------------------------------------------
% Custom commands
% ----------------------------------------------------------------------

% alignment

\newcommand*{\LongestHence}{$\Rightarrow$}% function name
\newcommand*{\LongestName}{$f_o(-x)+f_e(-x)$}% function name
\newcommand*{\LongestValue}{$(-a)x +(-a)(-y)$}% function value
\newcommand*{\LongestText}{\defi}%

\newlength{\LargestHenceSize}%
\newlength{\LargestNameSize}%
\newlength{\LargestValueSize}%
\newlength{\LargestTextSize}%

\settowidth{\LargestHenceSize}{\LongestHence}%
\settowidth{\LargestNameSize}{\LongestName}%
\settowidth{\LargestValueSize}{\LongestValue}%
\settowidth{\LargestTextSize}{\LongestText}%

% Choose alignment of the various elements here: [r], [l] or [c]

\newcommand*{\mbh}[1]{{\makebox[\LargestHenceSize][r]{\ensuremath{#1}}}}%
\newcommand*{\mbn}[1]{{\makebox[\LargestNameSize][r]{\ensuremath{#1}}}}%
\newcommand*{\mbv}[1]{\ensuremath{\makebox[\LargestValueSize][r]{\ensuremath{#1}}}}%
\newcommand*{\mbt}[1]{\makebox[\LargestTextSize][l]{#1}}%

\newcommand{\R}[1]{\label{#1}\linelabel{#1}}
\newcommand{\lr}[1]{line~\lineref{#1}}

% ----------------------------------------------------------------------
% Launch!
% ----------------------------------------------------------------------

\begin{document}

\section{Canonical Forms}

\subsection{Review}
\subsubsection{Strategy to Find the Intersection of Two Subspaces}

Suppose $W_1 = \spn \set{x_1, \dots, x_k}$ and $W_2= \spn{y_1,\dots, y_l}$.

We want to find all the solutions of 

\begin{equation*}
\sum_{i=1}^k\lambda_ix_i = \sum_{j=1}^l\mu_{j}y_j.
\end{equation*}

These form a homogeneous system of equations, and we know how to solve it!

\subsubsection{Strategy for Finding Jordan Canonical Basis}

Fix an eigenvalue $\lambda$ and find the dot diagram for $T|_{K_{\lambda}}$.

\begin{enumerate}
\item\label{item:1} First find $(T-\lambda I)^3v_1\in \ker(T-\lambda I)\cap (\img (T-\lambda I)^3)$.
\item Solve for $v_1$, thus obtaining the first cycle.
\item Extend to a basis $(T-\lambda I)^3v_1$, $(T-\lambda I)^2$,
  $(T-\lambda I)^2v_3, \dots \in \ker (T-\lambda I)\cap \img(T-\lambda
  I)^2$
\item Repeat the procedure for $v_2, v_3, \dots $
\end{enumerate}

\section{Canonical Forms}

\begin{theorem}
Let $A, B \in M_{n\times n}(\FF)$  be such that their characteristic polynomial split.

Then $A, B$ are similar if and only if they have the same JCF (up to the reordering of blocks).
\end{theorem}
\begin{remark}
This is a useful method to test the similarity of matrices.
\end{remark}

\begin{proof}
  \hfill

First note the following observations:
\begin{enumerate}
\item  $A$ is similar to its Jordan Canonical form, because $[L_A]_{\beta} = J_A$ for some basis $\beta$.
\item If $J_1$ and $J_2$ are matrices in JCF, then they are similar to each other.
\item If $J_1, J_2$ are matrices in JCF corresponding to the same
  linear transfromation, then they are similar to each other ( the basis can be reordered).
\end{enumerate}

Let $\sim:M_{n\times n}(\FF)\times M_{n\times n}(\FF)\to M_{n\times n}(\FF)$ denote the relation 

By the first observation, $A$ is similar to $J_A$ and $B$ is similar to $J_{B}$.

If $A, B$ are similar, then $A\sim B$, but $A\sim J_A$ and $B\sim J_{B}$, and hence $J_A\sim J_{B}$.

Suppose, on the other hand, that $J_A\sim J_{B}$ are the same up to
reordering of blocks, and thus $J_A$ and $J_{B}$ are similar to each other by the second observation.
\end{proof}

\section{Review of Polynomials}

Let $\FF$ be a field.

Let $\FF[x]$ denote the polynomials over $\FF$.

We define a polynomial as a formal exrpession
$a_nx^n+a_{n-1}x^{n-1}+\dots+a_1x+a_{0}$, where $a_i\in \FF$ and
$n \geq 0$.

Two polynomials are equal to each other if and only if all
coefficients of the terms with the same power are equal.

A polynomial $f(x)$ can be evaluated at any element $c\in\FF$ so that
$f(c) = \sum_{i=1}^na_ic^i \in \FF$.

Polynomials are not the same as polynomial-functions, because the
polynomials may be equal while the correspnodingq polynomial-functions
are not.

For example, $f(x) = x^3-x$ over $\FF = \ZZ_2$ gives the zero function.

\begin{definition}
If $a_n\neq 0$, the degree of a polynomial is defined as $\deg f = n$.
\end{definition}

\begin{description}

\item[e.g.] $\deg f = 0$ if and only if $f(x) = a_0\neq 0$.
\item[Convention:] $\deg 0 = - \infty$
\end{description}

The following properties hold:

\begin{itemize}
\item $\deg(fg) = \deg(f) + \deg(g)$
\item $\deg(f+g) \leq \max \set{\deg(f), \deg (g)}$
\end{itemize}

\begin{definition}
If the leading coefficient $a_n$ is such that  $a_n=1$, $f(x)$ is said to be \textit{monic}.
\end{definition}

\subsection{Division Algorithm }

If $f(x)$, $g(x) \in \SP(\FF)$ and $g(x)\neq 0$, then there exist $q(x)$ and $r(x) \in \SP(\FF)$ such that $f(x) = q(x)g(x) + r(x{}$ such that $\deg(r) < \deg(g)$.

These $q, r$ can be found by long division.

\begin{lemma}
If $a\in\FF$ and $f(a)=0$, then $(x-a) | f(x)$.
\end{lemma}
\begin{proof}
  \hfill

We use the Factor Theorem. 

Note that  $f(x)=f(x)(x-a)+r(a)$, where $\deg(r)\leq 0$, so $r$ is constant.

We evaluate it at $a$: $f(a) = 0+ r(a)$, and hence $r(a) = 0$., which
means that $r(x) = 0$.
\end{proof}

\begin{lemma}
  If $a_1, \dots, a_{s} \in \FF$ are distinct zeroes of $f(x)$, then $\prod_{i=1}^s(x-a_i)|f(x)$.

  Thus, $\deg f \geq s$, so $f(x)$ can have at most $\deg f$ zeroes.
\end{lemma}

\begin{definition}
We say that $f\in \FF[x]$ is irreducible if $\deg f > 0$ and it cannot be written as a product of two polynomials of lesser positive degree.
\end{definition}
\begin{description}
\item[e.g.] $x-a$ is irreducible for all $a\in\FF$. $x^2+1$ is irreducible over $\RR$.
\end{description}

More generally, a quadratic or cubic polynomial is irreducible if and only if there is no zero in $\FF$.

For the polynomial of degree greater than or equal to 4.

\begin{description}

\item[e.g.] $(x^p-p)\in\FF[\QQ]$ is irreducible for all $p$ prime.

\end{description}

\begin{example}

Over $\ZZ_2$, we know by plugging in $0$ and $1$ that $x^3+x+1$ is irreducible.

However, since over $\ZZ_2$ we have $(a+b)^2=a^2+b^2$, then $x^4+x^2+1 = (x^2+x+1)^2$.
\end{example}

\begin{definition}
Two nonzero polynomials are relatively prime if there is no polynomial of positive degree dividing both of them.
\end{definition}

\begin{example}

Over $\ZZ_2$, we know that $x^3+x+1$ and $x^4+x^2+1$ are relatively prime, because $x^3+x+1$ is irreducible, and thus the only factor of positive degree is $x^3+x+1$. However, $(x^3+x+1) | x^4+x^2+1$, since the quotient would be linear (but $x^4+x^2+1$ has no zeroes).
\end{example} 

\begin{remark}
In this way we see that distinct monic irreducible polynomials are relatively prime.
\end{remark}

\begin{theorem}
  If $f(x)$ , $g(x)$ are relative prime, three exists $u(x)$ and
  $v(x) \in\FF[x]$ such that $f(x)u(x) +g(x)v(x) = 1$.
\end{theorem}

\begin{lemma}
Suppose that $f$ and $g$ are polynomials that are relatively prime and $f|gh$. Then $f|h$.
\end{lemma}
\begin{proof}
  \hfill

We know that $1= fu + gv$, and therefore $h = fuh + ghv$, which means that $f|h$.
\end{proof}

\begin{theorem}
  If $\phi(x)$ is irreducible and $\phi(x) | f(x) g(x)$, then
  $\phi(x) | f(x)$ or $\phi(x) | g(x)$.
\end{theorem}

\begin{theorem}[Unique Factorisation]
  If $f(x)\neq 0$, we can write $f(x) = c\prod_{i=1}^{n_s}\phi_s(x)^{n_s}$, where $c\in \FF\setminus \set{0}$ and $\phi_i(x)$ are distinct monic irreducible polynomials with $n_i\geq 1$.

  The factorisation is unique up to the ordering of the factors.
\end{theorem}

\begin{example}

Factor into irreducible polynomials over $\FF = \ZZ_3$ $x^4+x^2+1 = x^4-2x^2+1 = (x^2-1)^2$.

\end{example}

\begin{remark}
  If $f(x)\neq 0$ and $f(x) = c\prod_{i=1}^{n_s}\phi_s(x)^{n_s}$, then
  the possible factors of $f(x)$ are $d\prod_{i=1}^{k_s}$ for
  $0\leq k_i\leq n_i$ and $d\in \FF\setminus \set{0}$.
\end{remark}

\section{Minimal Polynomials}

Let $V$ be a finite dimensional vector space over $\FF$.

Suppose that $T \in \End(V)$ has a characteristic polynomial $f(t)$
with $\deg(f) = \dim$.

Recall that by Cayley-Hamilton Theorem we have that $f(T) = 0$.

Note that, however, $f$ might not be monic, so we can rescale it in such a way that there exists a monic polynomial $g$ of degree $\dim V$ such that $g(T) = 0$.

\begin{definition}
  A minimal polynomial of $T$ is a monic polynomial of the least
  degree such that $p(T) = 0$.
\end{definition}
\begin{remark}
Note that $1 \leq \deg (p) \leq \dim V$.
\end{remark}

By Cayley-Hamilton Theorem, a minimal polynomial exists.

\begin{description}

\item[e.g.] If$\dim V = n$ and $T=I$, then $t-1$ is a minimal polynomial of $T$.

\end{description}

\begin{example}

  For $V = \FF^2$, $A \in M_{2\times 2}(\FF) $ such that $\begin{pmatrix}
1 &2 \\
3 & 4
\end{pmatrix}$, then the characteristic polynomial is a minimal
polynomial, because $A-\lambda I \neq 0$ for all $c\in\FF$.

\end{example}


\end{document}
