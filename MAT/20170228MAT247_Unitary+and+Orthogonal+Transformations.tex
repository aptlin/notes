% -*- coding: utf-8; -*-
%%% Local Variables:
%%% mode: latex
%%% TeX-engine: xetex
%%% TeX-master: t
%%% End:
\documentclass[11pt]{scrartcl}
\usepackage[fancy, beaue, pset, anon]{masty}
\pSet{\nt{MAT247}{}{Unitary/Orthogonal Transformations}}
\usepackage{lineno}
% ----------------------------------------------------------------------
% Page setup
% ----------------------------------------------------------------------

\pagenumbering{gobble}

% ----------------------------------------------------------------------
% Custom commands
% ----------------------------------------------------------------------

% alignment

\newcommand*{\LongestHence}{$\Rightarrow$}% function name
\newcommand*{\LongestName}{$f_o(-x)+f_e(-x)$}% function name
\newcommand*{\LongestValue}{$(-a)x +(-a)(-y)$}% function value
\newcommand*{\LongestText}{\defi}%

\newlength{\LargestHenceSize}%
\newlength{\LargestNameSize}%
\newlength{\LargestValueSize}%
\newlength{\LargestTextSize}%

\settowidth{\LargestHenceSize}{\LongestHence}%
\settowidth{\LargestNameSize}{\LongestName}%
\settowidth{\LargestValueSize}{\LongestValue}%
\settowidth{\LargestTextSize}{\LongestText}%

% Choose alignment of the various elements here: [r], [l] or [c]

\newcommand*{\mbh}[1]{{\makebox[\LargestHenceSize][r]{\ensuremath{#1}}}}%
\newcommand*{\mbn}[1]{{\makebox[\LargestNameSize][r]{\ensuremath{#1}}}}%
\newcommand*{\mbv}[1]{\ensuremath{\makebox[\LargestValueSize][r]{\ensuremath{#1}}}}%
\newcommand*{\mbt}[1]{\makebox[\LargestTextSize][l]{#1}}%

\newcommand{\R}[1]{\label{#1}\linelabel{#1}}
\newcommand{\lr}[1]{line~\lineref{#1}}

% ----------------------------------------------------------------------
% Launch!
% ----------------------------------------------------------------------

\begin{document}

\section{Review}

$T \in \Hom(V,V)$ is unitary/orthogonal if
\begin{itemize}
\item $\norm{Tx}=\norm{x}$
\item $\ipr{Tx}{Ty} = \ipr{x}{y}$
\item $TT^{*} = T^{*}T = I$
\item $T$ sends an orthonormal basis to an orthonormal basis.
\end{itemize}

\begin{definition}
  For $A, B \in M_{n\times n}(\FF)$, $A$ and $B$ are called
  unitarily/orthogonally equivalent if there exists $Q$
  unitary/orthogonalsuch that $A=Q^{-1}BQ$.
\end{definition}
\begin{remark}
  If $\beta$ is an orthonormal basis, then $T$ is unitary/orthogonal
  if and only if $[T]_{\beta}$ is unitary/orthogonal.
\end{remark}
\begin{remark}
  If $\beta, \beta'$ are orthonormal bases, then the change-of-basis
  matrix $[I]_{\beta}^{\beta'}$ is unitary/orthogonal.
\end{remark}
\begin{description}

\item[e.g.] If $\beta$ is a standard basis, which is orthonormal, then
  $[I]_{\beta}^{\beta'}$ is unitary/orthogonal, since colums are an
  orthonormal basis of $\FF^n$. Thus, if $\beta, \beta'$ are
  orthonormal bases, then $[T]_{\beta}$, $[T]_{\beta'}$ are
  unitarily/orthogonally equivalent.
\end{description}

\begin{lemma}
  \label{sec:review}
  Let $T \in \Hom(V,V)$. If $T$ has an eigenvalue $\lambda$, then
  $T^{*}$ has an eigenvalue $\ol{\lambda}$.
\end{lemma}

\begin{theorem}
  If the characteristic polynomial of $T$ splits, then there exists an
  orthonormal basis $\beta$ such that $[T]_{\beta}$ is
  upper-triangular.
\end{theorem}

\begin{proof}
  \hfill

  We proceed by induction on $\dim V$.

  If $n=1$, then the change-of-basis matrix is one by one, and hence
  upper-triangular.

  If $n> 1$, assume the claim holds for $n-1$.

  Since the characteristic polynomial $f(t)$ of $T$ splits, then $T$
  has an eigenvalue, say $\lambda \in \FF$.

  Then by Lemma \ref{sec:review}, $T^{*}$ has an eigenvalue
  $\ol{\lambda}$, say $T^{*}v = \ol{\lambda}v$, with $\norm{v} = 1$.

  Thus, $W = \spn{v}$ is $T^{*}$-invariant, and thus $W^{\bot}$ is
  $T$-invariant, with $\dim W^{\bot} = n-1$.

  By Theorem 5.21, the characteristic polynomial of $T_{W^{\bot}}$
  divides $f(t)$, so $f_{W^{\bot}}(t)$ splits.

  By inductive hypothesis, hdece exits an orthonormal basis $\gamma$
  of $W^{\bot}$ such that $[T_{W^{\bot}}]_{\gamma}$ is
  upper-triangular. Therefore, $\beta = \gamma\cup\set{v}$ is an
  orthonormal basis of $V$.

  Since $[T_{W^{\bot}}]_{\gamma}$ is upper-triangular, then
  $[T]_{\gamma}$ is upper-triangular.

  \begin{corollary}
    If $A \in M_{n\times n}(\FF)$ and the characteristic polynomial of
    $A$ splits, then $A$is unitarily/orthogonally equivalent to an
    upper-triangular matrix.
  \end{corollary}
\end{proof}
\section{Orthogonal Projections and The Spectral Theorem}

Let $V$ be a finite-dimensional inner product space.

Suppose $W\su V$. Thus, $V = W\oplus W^{\bot}$, and thus we can define
a linear map, called an orthogonal projection onto $W$,
$P_W \in \Hom(V,V)$ by $P_W(w+w^{\bot}) = w$.

\begin{theorem}
  \label{sec:orth-proj-spectr}
  \begin{enumerate}[label=\alph*)]
  \item $\img P_W = W$, $\ker P_W = W^{\bot}$.
  \item $P_W^2 = P_{W}$ and $P_{W}^{*} = P_{W}$
  \end{enumerate}
\end{theorem}

\begin{proof}
  \hfill

  We prove (b).

  By definiton, $P'(w+w^{\bot}) = P_W P_W(w+w^{\bot}) = P_w w = w$.

  To check that $P_W^{*} = P_{W}$, observe the following, for
  $v, w\in W$:

  \begin{align}
    \ipr{P_{W}(w+w^{\bot})}{v+v^{\bot}} &= \ipr{w}{v+v^{\bot}} = \ipr{w}{v}\\
    \ipr{w+w^{\bot}}{P_W(v+v^{\bot})} &= \ipr{w+w^{\bot}}{v} = \ipr{w}{v}
  \end{align}
\end{proof}

\begin{theorem}[6.24]
  Suppose $T \in \Hom(V,V)$. Then $T$ is an orthogonal projection if
  and only if $T^2=T$ and $T^{*}=T$.
\end{theorem}
\begin{proof}
  \hfill

  We have proved the first direction in Theorem
  \ref{sec:orth-proj-spectr}.

  Suppose now $T^2=T$ and $T^{*}=T$.
    
  Let $W=\img (T)$. We want to show that $T = P_{W}$.

  Since $\img (T^{*})^{\bot} = \ker T$, then $W^{\bot} = \ker(T)$.

  Since $T=T^{*}$, $\img T^{*} = \img T$.

  Therefore, $V = \ker T \oplus \img T$.

  Fix $v\in V$. Since $V = \ker T \oplus \img T$, there exists
  $w\in \img T$, $w' \in \ker T$ such that $v = w+w'$.

  Then $Tv = T(w+w') = Tw + Tw'$, and since $w'\in\ker T$, $Tv = Tw$.

  Note that $x = Tx + (x - Tx)$. Observe that $Tx = T^2x+ T(x-Tx)$,
  and since $T=T^{2}$, then $x-Tx \in \ker T$. Therefore,
  $w+w' - Tw - Tw' = w-Tw + w' \in \ker T$, which means that
  $w-Tw \in \ker T$. But $w-Tw\in\img T$, because $\img T$ is
  $T$-invariant, while $V = \img T \oplus \ker T$, and thus $w-Tw = 0$.

  Hence, $Tv = Tw = w$, and hence $T$ is an orthogonal projection.
\end{proof}

\end{document}