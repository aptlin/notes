% -*- coding: utf-8; -*-
%%% Local Variables:
%%% mode: latex
%%% TeX-engine: xetex
%%% TeX-master: t
%%% End:
\documentclass[11pt]{scrartcl}
\usepackage[fancy, beaue, pset, anon]{sdll}
\pSet{\nt{MAT157}{}{More on Integrability}}
\usepackage{lineno}
% ----------------------------------------------------------------------
% Page setup
% ----------------------------------------------------------------------

\pagenumbering{gobble}

% ----------------------------------------------------------------------
% Custom commands
% ----------------------------------------------------------------------

% alignment

\newcommand*{\LongestHence}{$\Rightarrow$}% function name
\newcommand*{\LongestName}{$f_o(-x)+f_e(-x)$}% function name
\newcommand*{\LongestValue}{$(-a)x +(-a)(-y)$}% function value
\newcommand*{\LongestText}{\defi}%

\newlength{\LargestHenceSize}%
\newlength{\LargestNameSize}%
\newlength{\LargestValueSize}%
\newlength{\LargestTextSize}%

\settowidth{\LargestHenceSize}{\LongestHence}%
\settowidth{\LargestNameSize}{\LongestName}%
\settowidth{\LargestValueSize}{\LongestValue}%
\settowidth{\LargestTextSize}{\LongestText}%

% Choose alignment of the various elements here: [r], [l] or [c]

\newcommand*{\mbh}[1]{{\makebox[\LargestHenceSize][r]{\ensuremath{#1}}}}%
\newcommand*{\mbn}[1]{{\makebox[\LargestNameSize][r]{\ensuremath{#1}}}}%
\newcommand*{\mbv}[1]{\ensuremath{\makebox[\LargestValueSize][r]{\ensuremath{#1}}}}%
\newcommand*{\mbt}[1]{\makebox[\LargestTextSize][l]{#1}}%

\newcommand{\R}[1]{\label{#1}\linelabel{#1}}
\newcommand{\lr}[1]{line~\lineref{#1}}

% ----------------------------------------------------------------------
% Launch!
% ----------------------------------------------------------------------

\begin{document}

\section{Integrability of Continuous Functions}

\begin{definition}
  Note that $L(f, P)$, $U(f, P)$ are defined for any $P$, so that
  $L(f, P) \leq U(f, P)$. Then for any bounded $f$
  \begin{align*}
    &L\int_a^bf(x)dx = \sup_PL(f, P)\\
    &U\int_a^bf(x)\dif x = \inf_PU(f, P)
  \end{align*}
\end{definition}

\begin{lemma}
  $L\int_a^bf(x)\dif x + L\int_c^bf(x)\dif x = L\int_a^bf(x)\dif x$
\end{lemma}
\begin{lemma}
  $U\int_a^bf(x)\dif x + U\int_c^bf(x)\dif x = U\int_a^bf(x)\dif x$
\end{lemma}

If $m< f(x) < M$, then
$m(b-a)\leq L\int_a^bf(x)\dif x \leq U\int_a^bf(x)\dif x \leq M(b-a)$.

Fix $x\in(a,b)$.

Define
\begin{align*}
 & L(x) = L\int_a^xf(x)\dif x\\
 & U(x) = U\int_a^xf(x)\dif x.
\end{align*}

Observe that both always exist.

To fund $L'(x)$, we need to find


\begin{equation*}
  \lim_{h\to 0 }\frac{L(x+h) - L(x)}{h} = \lim_{h\to 0 } \frac{L\int_a^{x+h}f(t)dt}{h}
\end{equation*}

If $h>0$, define
\begin{align}
  &m_{h} = \inf\set{f(t); x\leq t \leq x+h}\\
  &M_{h} = \sup\set{f(t); x\leq t \leq x+h}
\end{align}

Fix some $x\in (a, b)$.

Therefore,

\begin{align}
  & m_h(x+h - x) \leq L\int_x^{x+h}f(t)dt \leq U\int_x^{x+h}f(t)dt \leq M_h(x+h-x)\\
  \lra   & m_h \leq \frac{L\int_x^{x+h}f(t)dt }{h}\leq \frac{U\int_x^{x+h}f(t)dt }{h}\leq M_h\\
  \lra & m_h \leq \frac{L(x+h) - L(x)}{h}\leq \frac{U(x+h) - U(x) }{h}\leq M_h
\end{align}

If $h < 0$, the inequalities are similar.

If $f$ is continuous at $x$, then

\begin{equation*}
\lim_{h\to 0}m_h = \lim_{h\to 0}M_h = f(x),
\end{equation*}

so $L'(x) = f(x) = U'(x)$ and thus there exists a cotstant $c\in \RR$ such that $U(x) = L(x) + c$.

But $U(a) = L(a) = 0$, so $c = 0$ and hence $U(x) = L(x)$. In
particular, $L(b) = U(b)$, so $f$ is integrable.

\begin{ques*}

If $f$ is integrable and $F(x) = \int_a^xf(t)dt$, then $F'(x) = f(x)$. What about $G(x) = \int_{x}^bf(t)dt$?

\end{ques*}

\begin{answer*}

  Let $H(x) = \int_a^xf(t)dt +\int_x^bf(t)dt = \int_a^bf(t) dt$.

  So $H'(x) = 0$. Therefore,


  \begin{equation*}
    0 = \frac{\dif}{\dif x}\int_a^x + \frac{\dif}{\dif x}\int_x^bf = f(x) + G'(x),
  \end{equation*}

  and thus $G'(x) = - f(x)$.
\end{answer*}

\begin{example}
  \begin{align}
    \frac{\dif }{\dif x} \int_{\cos x}^{\sin x}\frac{t}{1+t^{2}}\dif t = \cos x \sin x \left( \frac{1}{1+\cos^2x} + \frac{1}{1+\sin^2x} \right)
  \end{align}
\end{example}
\end{document}
