% -*- coding: utf-8; -*-
%%% Local Variables:
%%% mode: latex
%%% TeX-engine: xetex
%%% TeX-master: t
%%% End:
\documentclass[11pt]{scrartcl}
\usepackage[fancy, beaue, pset, anon]{masty}
\pSet{\nt{MAT247}{XIII}{More on Isometries}}
\usepackage{lineno}
% ----------------------------------------------------------------------
% Page setup
% ----------------------------------------------------------------------

\pagenumbering{gobble}

% ----------------------------------------------------------------------
% Custom commands
% ----------------------------------------------------------------------

% alignment

\newcommand*{\LongestHence}{$\Rightarrow$}% function name
\newcommand*{\LongestName}{$f_o(-x)+f_e(-x)$}% function name
\newcommand*{\LongestValue}{$(-a)x +(-a)(-y)$}% function value
\newcommand*{\LongestText}{\defi}%

\newlength{\LargestHenceSize}%
\newlength{\LargestNameSize}%
\newlength{\LargestValueSize}%
\newlength{\LargestTextSize}%

\settowidth{\LargestHenceSize}{\LongestHence}%
\settowidth{\LargestNameSize}{\LongestName}%
\settowidth{\LargestValueSize}{\LongestValue}%
\settowidth{\LargestTextSize}{\LongestText}%

% Choose alignment of the various elements here: [r], [l] or [c]

\newcommand*{\mbh}[1]{{\makebox[\LargestHenceSize][r]{\ensuremath{#1}}}}%
\newcommand*{\mbn}[1]{{\makebox[\LargestNameSize][r]{\ensuremath{#1}}}}%
\newcommand*{\mbv}[1]{\ensuremath{\makebox[\LargestValueSize][r]{\ensuremath{#1}}}}%
\newcommand*{\mbt}[1]{\makebox[\LargestTextSize][l]{#1}}%

\newcommand{\R}[1]{\label{#1}\linelabel{#1}}
\newcommand{\lr}[1]{line~\lineref{#1}}

% ----------------------------------------------------------------------
% Launch!
% ----------------------------------------------------------------------

\begin{document}

\section{More on Isometries}
\begin{corollary}
  \label{sec:more-isometries}
  $T$ is unitary/orthonormal if and only if $T$ is normal and every eigenvalue $\lambda$ is such that $\abs{\lambda}=1$.
\end{corollary}
\begin{theorem}
  Let $T \in \Hom(V,V)$ be an operator on $V$ with $F = \RR$. Then $T$ is orthogonal and self-adjoint if and only if $V$ has an orthonormal basis of eigenvectors for $T$ with eigenvalues $\pm 1$. 
\end{theorem}
\begin{proof}
  \hfill
  From Corollary \ref{sec:more-isometries}, all eigenvalues are $\pm 1$. Then by Theorem 6.17 (Friedberg \textit{et al}), there exists an orthonormal basis of eigenvectors.

  Now, pick an orthonormal basis $\beta$ of eigenvectors with eigenvalues $\pm 1$.

  Then $[T]_{\beta} = 
  \begin{pmatrix}
    \lambda_1 &        & \\
              & \ddots & \\
              &        & \lambda_n
            \end{pmatrix}$, and
  $[T^{*}]_{\beta}\begin{pmatrix}
    \lambda_1 &        & \\
              & \ddots & \\
              &        & \lambda_n
            \end{pmatrix}$, and hence
            $[TT^{*}]_{\beta} = I$.
          \end{proof}

          \begin{definition}
            $A \in M_{n\times n}(\FF)$ is orthogonal ($\FF = \RR$)/unitary ($\FF=\CC$) if $AA^{*}=I = A^{*}A$.
          \end{definition}

\begin{remark}
    If $\beta$ is an orthonormal basis, then $T$ is orthogonal/unitary if and only if $[T]_{\beta}u$ is orthogonal/unitary.
  \end{remark}

  \begin{remark}
    $A$ is orthogonal/unitary if and only if rows or columns form an orthonormal basis of $\FF^n$ with a standard inner product.
  \end{remark}
  \begin{proof}
    \hfill

    \begin{align}
      (AA^{*})_{ij} & = \sum_k A_{ik}A^{*}        \\
                    & = \sum_{k} A_{ik}\ol{A_{jk}} \\
                    & = \ipr{ \text{$i^{\Th}$ row}}{\text{$j^{\Th}$ row}}
    \end{align}
    \begin{align}
      (A^* A)_{ij} & = \sum_kA_{ik}(A^{*})        \\
                    & = \sum_{k} A_{ik}\ol{A_{jk}} \\
                    & = \ipr{ \text{$i^{\Th}$ row}}{\text{$j^{\Th}$ row}}
    \end{align}
  \end{proof}
  \begin{definition}
    Two matrices $A, B \in M_{n\times n}(\FF)$ are unitarily/orthogonally equivalent there exists a unitary/orthogonal $Q$ usch that $Q^{-1}AQ = B$, which is equivalent to $Q^{*}AQ = B$.
  \end{definition}
  \begin{theorem}
    $A \in M_{n\times n}(\FF)$ is normal if and only if $A$ is unitarily equivalent to a diagonal matrix.
  \end{theorem}

  \begin{proof}
    \hfill

    If $A$ is normal, then $L_A:\CC^n\to\CC^n$ is normal, since (since $[L_A]_{\text{std}}=A$.

    Thus, by Theorem 6.16 there exists an orthonormal basis $\beta$ of eigenvectors for $L_{A}$. Thus, $Q^{-1}[L_A]_{\text{std}}Q$, where $Q=[I]_{\beta}^{\text{std}}$.

    On the other hand, if $A = Q^{-1}DQ=Q^{*}DQ$, where $Q$ is unitary and $D$ is diagonal. Then $A^{*}=Q^{*}D^{*}Q^{**}=Q^{*}DQ$, and hence \[AA^{*} = Q^{*}DQQ^{*}D^{*}Q = Q^{*}DD^{*}Q.\]
    Thus,  $A^{*}A = Q^{*}D^{*}DQ$.
  \end{proof}

  \begin{theorem}
    $A \in M_{n\times n}(\FF)$ is self-adjoint if and only if $A$ is orthogonally equivalent to a diagonal matrix.
  \end{theorem}

  \begin{theorem}[Schur]
    If the characteristic polynomial of $T \in \Hom(V,V)$ splits, then there exists $\beta$ such that $[T]_{\beta}$ is upper-triangular.
  \end{theorem}

  \begin{remark}
See also Exercise 5.4/32 in Friedberg \textit{et al}.
\end{remark}

\begin{lemma}
  If $T \in \Hom(V,V)$ has an eigenvalue $\lambda$, then $T^{*}$ has an eigenvalue $\ol{\lambda}$.
\end{lemma}

\begin{proof}
  \hfill

  Note that $\rank(T-\lambda I)^{*}=\rank(T-\lambda I)$. herefore, $\ker (T-\lambda I)^{*}= \ker (T-\lambda I) >0$.

  Thus $T^{*}$ has an eigenvalue $\ol{\lambda}$.
\end{proof}







\end{document}