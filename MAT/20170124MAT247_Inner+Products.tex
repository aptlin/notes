% -*- coding: utf-8; -*-
%%% Local Variables:
%%% mode: latex
%%% TeX-engine: xetex
%%% TeX-master: t
%%% End:
\documentclass[11pt]{scrartcl}
\usepackage[fancy, beaue, pset, anon]{masty}
\pSet{\nt{MAT247}{VII}{Inner Products}}
\usepackage{lineno}
% ----------------------------------------------------------------------
% Page setup
% ----------------------------------------------------------------------

\pagenumbering{gobble}

% ----------------------------------------------------------------------
% Custom commands
% ----------------------------------------------------------------------

% alignment

\newcommand*{\LongestHence}{$\Rightarrow$}% function name
\newcommand*{\LongestName}{$f_o(-x)+f_e(-x)$}% function name
\newcommand*{\LongestValue}{$(-a)x +(-a)(-y)$}% function value
\newcommand*{\LongestText}{\defi}%

\newlength{\LargestHenceSize}%
\newlength{\LargestNameSize}%
\newlength{\LargestValueSize}%
\newlength{\LargestTextSize}%

\settowidth{\LargestHenceSize}{\LongestHence}%
\settowidth{\LargestNameSize}{\LongestName}%
\settowidth{\LargestValueSize}{\LongestValue}%
\settowidth{\LargestTextSize}{\LongestText}%

% Choose alignment of the various elements here: [r], [l] or [c]

\newcommand*{\mbh}[1]{{\makebox[\LargestHenceSize][r]{\ensuremath{#1}}}}%
\newcommand*{\mbn}[1]{{\makebox[\LargestNameSize][r]{\ensuremath{#1}}}}%
\newcommand*{\mbv}[1]{\ensuremath{\makebox[\LargestValueSize][r]{\ensuremath{#1}}}}%
\newcommand*{\mbt}[1]{\makebox[\LargestTextSize][l]{#1}}%

\newcommand{\R}[1]{\label{#1}\linelabel{#1}}
\newcommand{\lr}[1]{line~\lineref{#1}}

% ----------------------------------------------------------------------
% Launch!
% ----------------------------------------------------------------------

\begin{document}

\begin{remark}
On a vector space $V$ there can be many inner products.
\end{remark}
\begin{example}
  \hfill
\begin{itemize}
\item if $\ipr{\*}{\*}$ is an inner product, then so is $c\ipr{\*}{\*}$ for $c>0$ in $\RR$
\item if $\phi:V\to V$ is an isomorphism, then also $\ipr{x}{y}'=\ipr{\phi(x)}{\phi{y}}$
\item if $V = C[a,b]$ is a vector space of continuous products
  ($a<b$), where $\FF = \RR$). Then $\ipr{f(x)}{f(y)} = \int_a^bf(t)g(t)\dif t$ is an inner product.
\item if $V = C[a,b]$ is a vector space of continuous products
  ($a<b$), where $\FF = \CC$). Then
  $\ipr{f(x)}{f(y)} = \int_a^bf(t)\ol{g(t)}\dif t$ is an inner
  product.

  Here, if $f(x) \in C[a,b]$, write $f(x) = f_1(x) + i f_2(x)$, where
  $f_1, f_2\in \RR$. Define
  \[\int_a^bf(t)\dif t = \int_a^bf_1 \dif t + i\int_a^bf_2 \dif t.\] Then
  \[\ol{\int f(t) \dif t} = \int \ol{f(t) \dif t }\] and
  \[\int(f(t)+cg(t))\dif t = \int f(t)\dif t + c \int g(t) \dif t.\]
\end{itemize}
\end{example}

\begin{definition}
  \hfill

  \textbf{H} is the inner product space $C[0,2\pi]$, $\FF = \CC$, with
  $\ipr{f, g} = \frac{1}{2\pi}\int_0^{2\pi}f(t)\ol{g(t)}\dif t$.
\end{definition}

\begin{theorem}
  Let $V$ be an inner product space with the inner product $\ipr{\*}{\*}$.

  \begin{enumerate}[label=\alph*)]
  \item $\ipr{x}{x} = 0 \lra x = 0$
  \item If for all $x\in V.\ipr{x}{y} = \ipr{x}{z}$, then $y = z$
  \end{enumerate}

\end{theorem}
\begin{proof}
  \hfill

  \begin{enumerate}[label=\alph*)]
  \item If $x = 0$, then $\ipr{x}{x} = 0$. Otherwise, $\ipr{x}{x}>0$
  \item If $\ipr{x}{y} = \ipr{x}{z}$ for all $x\in V$, then
    $\ipr{x}{y-z} = 0$. Therefore, taking $x = y-z$, we obtain $y - z =0$.
  \end{enumerate}
\end{proof}

\begin{definition}
  \hfill

The \textbf{norm} or \textbf{length} of $x\in V$ is $\norm{\sqrt{\ipr{x}{x}}} \geq 0$
\end{definition}
\begin{example}

  If $V = \FF^n$ with the standard inner product
  $\ipr{a}{b} = \sum a_i\ol{b_{i}}$, then

  \begin{equation*}
    \sqrt{\abs{a_1}^2+\abs{a_2}^2+\cdots + \abs{a_n}^2}
  \end{equation*}
\end{example}

\begin{theorem}
  Let $V$ be an inner product space. Then the following holds:

  \begin{enumerate}[label=\alph*)]
  \item $\norm{cx} = \abs{c}\norm{x}$ for all $c \in \FF$ and $x\ in V$
  \item $\norm{x} = 0 \lra x = 0$
  \item $\abs{\ipr{x}{y}}\leq\norm{x}\norm{y}$ (\textit{Cauchy-Schwarz})
  \item $\norm{x+y}\leq \norm{x} + \norm{y}$ (\textit{Triangle Inequality})
  \end{enumerate}
\end{theorem}

\begin{proof}
  \begin{enumerate}[label=\alph*)]
  \item$\norm{cx} = \sqrt{\ipr{cx}{cx}} = \sqrt{c\ol{c}\ipr{x}{x}} = \abs{c}\norm{x}$
  \item Exercise.
  \item Note that if $y=0$, the inequality holds. Suppose now
    $y\neq 0$. Note the following:
    \begin{align}
      0\leq \ipr{x+cy}{x+cy} &= \ipr{x}{x+cy}+ c \ipr{y}{x+cy}\\
      &= \ipr{x}{x} + \ol{c}\ipr{x}{y}+c\ipr{y}{x}+ c\ol{c}\ipr{y}{y}
    \end{align}

    Plugging in
    $c = -\frac{\ipr{x}{y}}{\ipr{y}{y}} = -\ipr{x}{(\norm{y})^{2}}$,
    we obtain
    \begin{equation*}
      0\leq \norm{x}^{2} - \frac{\abs{\ipr{x}{y}}}{\norm{y}^{2}},
    \end{equation*}
    and thus $\abs{\ipr{x}{y}\leq \norm{x}\norm{y}}$
  \end{enumerate}
\item Consider $\norm{x+y}^2 \leq (\norm{x}+\norm{y})^2$. Note that
  $\ipr{x+y}{x+y} = \ipr{x}{x}+2\Re{\ipr{x}{y}}+\ipr{y}{y}$
\end{proof}

For the standard innerp product on $\FF^n$, from Cauchy-Schwarz it
follows that
$\abs{\sum_{i=1}^na_ib_i} \leq \sqrt{\sum \abs{a_i}^2}\sqrt{\sum
  \abs{b_i}^2}$ and by Triangle Inequality $\abs{\sum_{i=1}^na_ib_i} \leq \sqrt{\sum \abs{a_i}^2}\sqrt{\sum
  \abs{b_i}^2}$


\end{document}