% -*- coding: utf-8; -*-
%%% Local Variables:
%%% mode: latex
%%% TeX-engine: xetex
%%% TeX-master: t
%%% End:
\documentclass[11pt]{scrartcl}
\usepackage[fancy, beaue, pset, anon]{masty}
\pSet{\nt{MAT247}{}{Rational Canonical Forms}}
\usepackage{lineno}
% ----------------------------------------------------------------------
% Page setup
% ----------------------------------------------------------------------

\pagenumbering{gobble}

% ----------------------------------------------------------------------
% Custom commands
% ----------------------------------------------------------------------

% alignment

\newcommand*{\LongestHence}{$\Rightarrow$}% function name
\newcommand*{\LongestName}{$f_o(-x)+f_e(-x)$}% function name
\newcommand*{\LongestValue}{$(-a)x +(-a)(-y)$}% function value
\newcommand*{\LongestText}{\defi}%

\newlength{\LargestHenceSize}%
\newlength{\LargestNameSize}%
\newlength{\LargestValueSize}%
\newlength{\LargestTextSize}%

\settowidth{\LargestHenceSize}{\LongestHence}%
\settowidth{\LargestNameSize}{\LongestName}%
\settowidth{\LargestValueSize}{\LongestValue}%
\settowidth{\LargestTextSize}{\LongestText}%

% Choose alignment of the various elements here: [r], [l] or [c]

\newcommand*{\mbh}[1]{{\makebox[\LargestHenceSize][r]{\ensuremath{#1}}}}%
\newcommand*{\mbn}[1]{{\makebox[\LargestNameSize][r]{\ensuremath{#1}}}}%
\newcommand*{\mbv}[1]{\ensuremath{\makebox[\LargestValueSize][r]{\ensuremath{#1}}}}%
\newcommand*{\mbt}[1]{\makebox[\LargestTextSize][l]{#1}}%

\newcommand{\R}[1]{\label{#1}\linelabel{#1}}
\newcommand{\lr}[1]{line~\lineref{#1}}

% ----------------------------------------------------------------------
% Launch!
% ----------------------------------------------------------------------

\begin{document}
\section{Review}

Suppose $T \in \End(V)$.

The minimal polynomial of $T$ is a unique monic polynomial $p(t)$ of the least degree such that
\begin{itemize}
\item $g(T) = 0$, which is equivalent to saying that $p(t)|g(t)$ and
  thus $p(t)$ divides the characteristic polynomial of $T$
\item $p(t)$ has the same roots as the characteristic polynomial.
\end{itemize}
\section{Rational Canonical Forms}

Let $T \in \End(V)$ be a transformation with an arbitrary characteristic polynomial. Recall that, if $x\in V\setminus \set{0}$, then let $W$ be a $T$-cyclic subspace generated by $x$.

Then $W$ has a basis $\set{x, Tx, \dots, T^{k-1}x}$ for some $k\geq 1$.

A companion matrix $C_{g(t)}$ of $g(t) = t^k + a_{k-1}t^{k-1}+\dots + a_0$ can be defined in the following form:
\begin{equation*}
[T]_{\beta} = 
\begin{pmatrix}
0 & & \dots & & -a_0\\
1 & 0 & \dots & & -a_{1}\\
0&\ddots & & &\\
\vdots&&&&\vdots\\
0 & \dots & & 1 & -a_{k-1}
\end{pmatrix},
\end{equation*}

where $a_i\in \FF$ are such that $a_0 + \sum_{i=1}^k a_{i}T^{i}(x) = 0$.

We want to show that there exists a basis such that the matrix is
representable as a multiblock matrix, with companion matrices
$C_{g_i}$as blocks such that $g_i = \phi_i^{n_i}$, where $\phi$ is an irreducible monic polynomial.
\begin{example}

Suppose $\FF = \RR$. Then 
\begin{equation*}
\begin{pmatrix}
0 & -1 & & \\
1 & 0 & & \\
& & 0 & -1\\
& & 1 & 2
\end{pmatrix}
\end{equation*}

is in Rational Canonical Form, since blocks are $C_{t^2+1}$ and $C_{(t-1)^2}$ are such that $t^2+1$ and $t-1$ are irreducible.
\end{example}
\begin{remark}
$
\begin{pmatrix}
0 & -1 \\
1 & 2
\end{pmatrix}
$ is in RCF with the characteristic polynomial $(t-1)^2$, while JCF is $
\begin{pmatrix}
1 & 1 \\
0 & 1
\end{pmatrix}
$.
\end{remark}

\begin{example}

  We can find all possible RCFs for $\FF = \ZZ_2$ and $M_{2\times 2}(\FF)$.

  First, we need to find all irreducible polynomials of degree 1 and
  2. It is easy to see that $t$ and $t-1$ are the only irreducible
  polynomials of degree 1, while $t^2+t+1$ is the only irreducible
  polynomial of degree 2.

The only two possibilities for $2\times2$ matrices are $
\begin{pmatrix}
C_{g_1} & \\
& C_{g_2}
\end{pmatrix}
$, where $g_1$ and $g_2$ have the degree of 1, and $
\begin{pmatrix}
C_g
\end{pmatrix}$, where $g$ has the degree of 2.

Therefore, there are three matrices of the first kind:

\begin{align}
  \begin{pmatrix}
C_t & \\
 & C_t
  \end{pmatrix} = 
   \begin{pmatrix}
0 & \\
& 0
   \end{pmatrix}, 
  \begin{pmatrix}
C_t & \\
& C_{t-1}
  \end{pmatrix} = 
  \begin{pmatrix}
0 & \\
& 1 
  \end{pmatrix},
  \begin{pmatrix}
C_{t-1} & \\
& C_{t-1}
  \end{pmatrix} = 
  \begin{pmatrix}
1 & \\
& 1
  \end{pmatrix}.
\end{align}

Similarly, there are three matrices of the second kind:

\begin{align}
  \begin{pmatrix}
C_{t^2}
  \end{pmatrix}=
  \begin{pmatrix}
0 & 0\\
1 & 0
  \end{pmatrix},
    \begin{pmatrix}
C_{(t-1)^2}
    \end{pmatrix} = 
    \begin{pmatrix}
0 & 1 \\
1 & 0
    \end{pmatrix}, 
    \begin{pmatrix}
C_{t^2+t+1}
    \end{pmatrix}=
    \begin{pmatrix}
0 & 1 \\
1 & 1
    \end{pmatrix}
\end{align}
\end{example}
\begin{definition}
\hfill

Suppose $T \in \End(V)$.

  If $\phi(t)$ is a monic irreducible polynomial, then $K_{\phi} = \set{x\in V; \phi(T)^n(x) = 0 \text{ for some $n\geq 1$}}$.
\end{definition}
\begin{description}

\item[e.g.] If $\phi(t) = t-\lambda$, then $K_{\phi} = K_{\lambda}$.

\end{description}

It can be shown that $V = \bigoplus_{i=1}^s K_{\phi_{i}}$, where the
characteristic polynomial $f$ of $T$ is such that
$f(t) =\pm \prod_{i=1}^s\phi^{m_i}_{i}$ for some $m_i\in \ZZ^+$ and
distinct monic irreducible $\phi_i$.

\begin{definition}
\hfill

Let $x\in V$ be arbitrary.

\textbf{$T$-annihilator of $x$} is a monic polynomial $p(t)$ of the
least degree such that $p(T)(x)=0$.
\end{definition}

\begin{theorem}
\hfill

  \begin{enumerate}[label=\alph*)]
  \item $T$-annihilator of $x$ is unique
  \item The $T$-annihilator of $x$ divides any $g(t)$ such that $g(T)(x) = 0$, and thus a $T$-annihilator divides the minimal polynomial.
  \item Let $W_x$ be a $T$-cyclic subspace generated by $x$. Then
    $T$-annihilator of $x$ is the minimal polynomial of $T|_{W_x}$.
  \end{enumerate}
\end{theorem}
\begin{description}

\item[Ex. 1] If $x\in E_{\lambda}$ and $x\neq 0$, then $T$-annihilator is $t-\lambda$.
\item[Ex. 2] If $x\neq 0$ and $x\in E_{\phi} = \ker(\phi(T))$, where $\phi$ is monic irreducible, then $T$-annihilator is $\phi(t)$.

\end{description}

\begin{theorem}
Let $\phi(t)$ be monic irreducible.

\begin{enumerate}[label=\alph*)]
\item $K_{\phi}$ is a $T$-invariant subspace.
\item $K_{\phi}  \neq 0$ if and only if $\phi(t)|p(t)$, where $p(t)$ is a minimal polynomial
\item $K_{\phi} = \ker(\phi(T)^d)$, where $\phi(t)^d$ is the largest power of $\phi$ dividing $p(t)$.
\end{enumerate}
\end{theorem}

\begin{proof}
  \hfill

  \begin{enumerate}[label=\alph*)]
  \item Exercise.
  \item Pick $x\in K_{\phi}\setminus \set{0}$.

    Then $\phi(T)^n(x) = 0$ for some $n \in \ZZ^+$ by definition of
    $K_{\phi}$.

    Thus, $T$-annihilator divides $\phi(T)^n$ and thus $T$-annihilator
    is equal to $\phi(T)^k$ for $1 \leq k \leq n$, since $\phi$ is
    irreducible.

    Note that $T$-annihilator divides the minimal polynomial $p(t)$,
    and hence $\phi(T)^k$ divides the minimal polynomial. In
    particular, we see that $\phi(T)$ divides the minimal polynomial.

    If $\phi(t)|p(t)$, then $p(t) = \phi(t)q(t)$ for some $q(t)$ such
    that $q(T) \neq 0$ (since $\deg q < \deg p$).

    Suppose that $q(T)(y) \neq 0$ for some $y\in V$.

    Then $\phi(T)(q(T)(y)) = p(T)(y) = 0$. Therefore,
    $q(T)y \in K_{\phi}$ and $q(T)y \neq 0$.
  \item From the discussion above, if
    $x\in K_{\phi} \setminus \set{0}$, then $\phi(T)^k(x) = 0$ for
    some $k\in \ZZ^+$ such that $\phi(T)^k | p(t)$, which means that
    $k \leq d$.

    Therefore, $\phi(T)^d(x) = 0$, and thus $x\in \ker (\phi(T)^d)$,
    which show that $K_{\phi} \suq \ker \phi(T)^d$. 

    The inclusion in the other direction follows from the definition of $K_{\phi}$.
  \end{enumerate}
\end{proof}

The following facts can be shown:

\begin{enumerate}[label=\alph*)]
\item If $f(t) = \pm \prod_{i=1}^s\phi_1(t)^{m_i}$, where $\phi_i$ are
  distinct monic irreducble polynomials and $n_1 \geq 1$.
\item $\dim K_{\phi_i} = m_i \def \phi_i$.
\item $K_{\phi_1}$ has a basis $\beta_i$ that is a disjoint union of $T$-cyclic bases. 

  Note that if $x\in K_{\phi_i}\setminus \set{0}$, we have a
  $T$-cyclic basis $\beta_x = \set{x, Tx, \dots, T^{k-1}x}$ if the
  $T$-cyclic basis is generated by $x$.

  Then $[T_{W_x}]_{\beta_x} = C_{g(t)}$, where $g(t)$ is the minimal
  polynomial of $T_{W_x}$. We have also seen that $T$-annihilator of
  $x$ is equal to $\phi_i(t)^k$ for some $k\in \ZZ^+$.

  Thus, we can conclude that
\begin{equation*}
[T_{K_{\phi_{i}}}]_{\beta_i} = 
\begin{pmatrix}
C_{\phi_1^{k_1}} & & \\
& \ddots &\\
& & C_{\phi_s}^{k_s}
\end{pmatrix}
\end{equation*}
\item The union $\beta = \bigcup_{i=1}^s \beta_{i}$ is a rational
  canonical basis of $V$ such that $[T]_{\beta}$ is in RCF.
\item The minimal polynomial is $\prod_{i=1}^s\phi_i(t)^{d_i}$, where
  $1 \leq d_i \leq n_i$ for all $i$, and thus the characteristic
  polynomial has the same irreducible factors.
\end{enumerate}




\end{document}
