% -*- coding: utf-8; -*-
%%% Local Variables:
%%% mode: latex
%%% TeX-engine: xetex
%%% TeX-master: t
%%% End:
\documentclass[11pt]{scrartcl}
\usepackage[fancy, beaue, pset, anon]{masty}
\pSet{\nt{MAT157}{IX}{Irrationality of $\pi$}}
\usepackage{lineno}
% ----------------------------------------------------------------------
% Page setup
% ----------------------------------------------------------------------

\pagenumbering{gobble}

% ----------------------------------------------------------------------
% Custom commands
% ----------------------------------------------------------------------

% alignment

\newcommand*{\LongestHence}{$\Rightarrow$}% function name
\newcommand*{\LongestName}{$f_o(-x)+f_e(-x)$}% function name
\newcommand*{\LongestValue}{$(-a)x +(-a)(-y)$}% function value
\newcommand*{\LongestText}{\defi}%

\newlength{\LargestHenceSize}%
\newlength{\LargestNameSize}%
\newlength{\LargestValueSize}%
\newlength{\LargestTextSize}%

\settowidth{\LargestHenceSize}{\LongestHence}%
\settowidth{\LargestNameSize}{\LongestName}%
\settowidth{\LargestValueSize}{\LongestValue}%
\settowidth{\LargestTextSize}{\LongestText}%

% Choose alignment of the various elements here: [r], [l] or [c]

\newcommand*{\mbh}[1]{{\makebox[\LargestHenceSize][r]{\ensuremath{#1}}}}%
\newcommand*{\mbn}[1]{{\makebox[\LargestNameSize][r]{\ensuremath{#1}}}}%
\newcommand*{\mbv}[1]{\ensuremath{\makebox[\LargestValueSize][r]{\ensuremath{#1}}}}%
\newcommand*{\mbt}[1]{\makebox[\LargestTextSize][l]{#1}}%

\newcommand{\R}[1]{\label{#1}\linelabel{#1}}
\newcommand{\lr}[1]{line~\lineref{#1}}

% ----------------------------------------------------------------------
% Launch!
% ----------------------------------------------------------------------

\begin{document}
\section{Irrationality of $\pi$}

\subsection{Observations}

\begin{enumerate}
\item\label{item:1} Consider the function $f_n(x) = \frac{x^n(1-x)^n}{n!}$.

  Note that it satisfies the following inequality:


  \begin{equation*}
0 < f_n(x) < \frac{1}{n!}\text{ for $0< x < 1$}
\end{equation*}

Observe also that $f_n$ can be written as follows:

\begin{equation}
f_n(x) = \frac{1}{n!}\sum_{i=n}^{2n}c_ix^i,
\end{equation}

where $c_i \in \ZZ$.

Therefore, $f_n^{(k)}(0) \in \ZZ$. Moreover, since
$f_n(x) = f_n(1-x)$, then $f_n^{(k)}(0) =
(-1)^kf_n^{(k)}(1-x)$. Therefore, $f_n^{k}(1) \in \ZZ$.

\item For any $a \in \RR$ and $\epsilon > 0$, then for sufficiently
  large $n$ we have $\frac{a^n}{n!}< \epsilon$.

  To see this, observe that if $n\geq 2a$, then

  \begin{equation}
    \label{eq:1}
    \frac{a^{n+1}}{(n+1)!} = \frac{a}{n+1}\frac{a^n}{n!} <
    \frac{1}{2}\frac{a^n}{n!}.
  \end{equation}


  Now let $n_0$ be any natural number with $n_0\geq 2a$.

  Therefore, applying the inequality (\ref{eq:1}), we obtain that
  there exists $k\in \NN$ such that


  \begin{equation*}
    \frac{a^{n_0+k}}{(n_0+k)!}< \epsilon.
  \end{equation*}

\end{enumerate}

Now we are ready to proceed with the proof.

\subsection{Proof}


\begin{theorem}
$\pi$ and $\pi^2$ are both irrational.
\end{theorem}

\begin{proof}
  Suppose, on the other hand, that $\pi = \frac{a}{b}$ for some
  $a, b\in \NN$. Consider the following function:

  \begin{equation}
    \label{eq:2}
    G(x) = b^n(\pi^{2n}f_n(x) - \pi^{2n-2}f_n''(x) + \pi^{2n-4}f_n^{(4)}(x) - \cdots + (-1)^nf_n^{(2n)}(x).
  \end{equation}

  Since $b^n\pi^{2n-k} = a^{n-k}b^k$ is an integer, while
  $f_n^{(k)}(0)$ and $f_n^{(k)}(1)$ are also integers, then $G(0)$ and
  $G(1)$ are integers.

  Notice that

  \begin{equation}
    \label{eq:3}
    G''(x) = b^n(\pi^{2n}f_n''(x) - \pi^{2n-2}f_n^{(4)}(x) + \cdots  + (-1)^nf_n^{(2n+2)}(x).
  \end{equation}

  Since $2n+2 > 2n$, the last term is zero. Therefore, adding
  (\ref{eq:2}) and (\ref{eq:3}), we obtain that

  \begin{equation}
G''(x) + \pi^2G(x) = b^n\pi^{2n+2}f_n(x) = \pi^2a^2f_n(x)
\end{equation}

Now let
\begin{equation}
H(x) = G'(x) \sin(\pi x)  - \pi G(x) \cos(\pi x).
\end{equation}

Therefore,

\begin{equation}
H'(x) = \pi^{2}a^nf_n\sin(\pi x)
\end{equation}

Thus, by the Second Theorem of Calculus,


\begin{align}
  \pi^2\int_0^1a^nf_n(x)\sin(\pi x) \dif x &= H(1) - H(0)\\
  &= \pi[G(1) + G(0)],
\end{align}

and therefore $  \pi\int_0^1a^nf_n(x)\sin(\pi x) \dif x $ is an integer.

On the other hand, since $0 < f_n(x) < \frac{1}{n!}$ for $0 < x < 1 $, it follows that for $0 < x < 1 $ $0<  \pi a^nf_n(x)\sin(\pi x) \dif x < \frac{\pi a^n}{n!}$, which means that :


\begin{equation*}
0<   \pi\int_0^1a^nf_n(x)\sin(\pi x) \dif x < \frac{\pi a^n}{n!} < 1.
\end{equation*}

This is a contradiction, since $\pi\int_0^1a^nf_n(x)\sin(\pi x) \dif x $ was shown to be an integer. Thus, the original assumption that $\pi^2$ is rational does not hold, and hence $\pi$ is irrational.

\end{proof}

\end{document}