% -*- coding: utf-8; -*-
%%% Local Variables:
%%% mode: latex
%%% TeX-engine: xetex
%%% TeX-master: t
%%% End:
\documentclass[11pt]{scrartcl}
\usepackage[fancy, beaue, pset, anon]{masty}
\pSet{\nt{MAT157}{}{Series}}
\usepackage{lineno}
% ----------------------------------------------------------------------
% Page setup
% ----------------------------------------------------------------------

\pagenumbering{gobble}

% ----------------------------------------------------------------------
% Custom commands
% ----------------------------------------------------------------------

% alignment

\newcommand*{\LongestHence}{$\Rightarrow$}% function name
\newcommand*{\LongestName}{$f_o(-x)+f_e(-x)$}% function name
\newcommand*{\LongestValue}{$(-a)x +(-a)(-y)$}% function value
\newcommand*{\LongestText}{\defi}%

\newlength{\LargestHenceSize}%
\newlength{\LargestNameSize}%
\newlength{\LargestValueSize}%
\newlength{\LargestTextSize}%

\settowidth{\LargestHenceSize}{\LongestHence}%
\settowidth{\LargestNameSize}{\LongestName}%
\settowidth{\LargestValueSize}{\LongestValue}%
\settowidth{\LargestTextSize}{\LongestText}%

% Choose alignment of the various elements here: [r], [l] or [c]

\newcommand*{\mbh}[1]{{\makebox[\LargestHenceSize][r]{\ensuremath{#1}}}}%
\newcommand*{\mbn}[1]{{\makebox[\LargestNameSize][r]{\ensuremath{#1}}}}%
\newcommand*{\mbv}[1]{\ensuremath{\makebox[\LargestValueSize][r]{\ensuremath{#1}}}}%
\newcommand*{\mbt}[1]{\makebox[\LargestTextSize][l]{#1}}%

\newcommand{\R}[1]{\label{#1}\linelabel{#1}}
\newcommand{\lr}[1]{line~\lineref{#1}}

% ----------------------------------------------------------------------
% Launch!
% ----------------------------------------------------------------------

\begin{document}
\section{Review}
Suppose that $a_n \geq 0$, and let $s_n = \sum _{k=1}^na_k$.

\begin{remark}
Note that $a_n= s_n-s_{n-1}$.

So if the series converges, then $\lim_{n\to \infty} a_n = \lim_{n\to\infty}(s_n- s_{n-1}) = 0$.
\end{remark}

\begin{theorem}
If $\sum_{k=1}^{\infty} a_{k}$ converges, then $\lim_{n\to \infty} a_n = 0$.
\end{theorem}
\begin{remark}
Note that the condition $\lim_{n\to\infty}a_n = 0$ is necessary but not sufficient.
\end{remark}
\section{Limit Comparison Test}

We have already shown that if $a_i\leq b_i$ for all $i\in I$, then
$\sum_{i\in I} a_{i}$ converges whenever $\sum_{i\in I}b_n$ converges.

\begin{theorem}
Suppose there exists a nonzero constant $c \in \RR$ and $N>0$ such
that $\forall n \in N.(a_n \leq c b_n)$, then if $\sum_{n\in I} b_{n}$
converges, then $\sum_{i\in I} a_{n}$ converges.

\end{theorem}

\begin{example}

Consider $\sum_{n=1}^{\infty}\frac{1}{2n+1}$. 

Since $\sum \frac{1}{n}$ diverges, so does $\sum \frac{1}{2n+1}$,
because $\frac{1}{2n+1}> 3 \* \frac{1}{n}$ for $n > 1$.
\end{example}

\begin{corollary}
 If $\lim_{in \to \infty} \frac{a_n}{b_n} = c \neq 0$, then $\sum a_n$ and $\sum b_{n}$ either both converge or both diverge.
\end{corollary}

\begin{proof}
  \hfill

If $\frac{a_n}{b_n} \to c$, then there exists $N\in\NN$ such that $\forall n\in N.\frac{a_n}{b_n} < c+1$ and hence the theorem applies.
\end{proof}

Consider $\sum_{n=1}^{\infty}\frac{1}{n^2}$. We can use the inerval test. Since $f'(x) = - \frac{2}{x^2} < 0$ , then $f$ is decreasing for $x > 0$.

Note that
$\int_{i=1}^{\infty} \frac{\dif x}{x^2} = \lim_{R\to \infty \int_1^R
  \frac{\dif x}{x^2}}$, which is then equal to $1$. Since the integral converges, then the series converges.

We call series in the form $\sum_{i=1}^{\infty}\frac{1}{n^{p}}$ a
\textbf{$p$-series}.

For $p>1$, almost exactly the same calulation show that $\int_{i=1}^{\infty} \frac{\dif x}{x^p}$ converges, then  $\sum \frac{1}{n^p}$ converges.

\begin{example}

Consider $a_n = \frac{1}{3^n+}$. 

Then $\frac{a_{n+1}}{a_n} = \frac{1+ \frac{1}{3^{n}}}{3 + \frac{1}{7+\frac{1}{3^{n+1}}}}$.

\end{example}

\section{Ratio Test}
\begin{theorem}
If $\lim_{n\to \infty} \frac{a_{n+1}}{a_n}=r$ and 
\begin{itemize}
\item $r< 1$, then $\sum a_n$ converges
\item $r> 1$, then $\sum a_n$ diverges
\item $r = 1$, then the test is inconclusive.
\end{itemize}

\end{theorem}

In the example above, $r=1$, and thus the sequence converges.

We have seen that $\log x = \sum_{i=1}^n (-1)^{i+1}\frac{x^i}{i!}$.

\end{document}
