% -*- coding: utf-8; -*-
%%% Local Variables:
%%% mode: latex
%%% TeX-engine: xetex
%%% TeX-master: t
%%% End:
\documentclass[11pt]{scrartcl}
\usepackage[fancy, beaue, pset, anon]{masty}
\pSet{\nt{MAT247}{IX}{Orthogonal Complements and Adjoints}}
\usepackage{lineno}
% ----------------------------------------------------------------------
% Page setup
% ----------------------------------------------------------------------

\pagenumbering{gobble}

% ----------------------------------------------------------------------
% Custom commands
% ----------------------------------------------------------------------

% alignment

\newcommand*{\LongestHence}{$\Rightarrow$}% function name
\newcommand*{\LongestName}{$f_o(-x)+f_e(-x)$}% function name
\newcommand*{\LongestValue}{$(-a)x +(-a)(-y)$}% function value
\newcommand*{\LongestText}{\defi}%

\newlength{\LargestHenceSize}%
\newlength{\LargestNameSize}%
\newlength{\LargestValueSize}%
\newlength{\LargestTextSize}%

\settowidth{\LargestHenceSize}{\LongestHence}%
\settowidth{\LargestNameSize}{\LongestName}%
\settowidth{\LargestValueSize}{\LongestValue}%
\settowidth{\LargestTextSize}{\LongestText}%

% Choose alignment of the various elements here: [r], [l] or [c]

\newcommand*{\mbh}[1]{{\makebox[\LargestHenceSize][r]{\ensuremath{#1}}}}%
\newcommand*{\mbn}[1]{{\makebox[\LargestNameSize][r]{\ensuremath{#1}}}}%
\newcommand*{\mbv}[1]{\ensuremath{\makebox[\LargestValueSize][r]{\ensuremath{#1}}}}%
\newcommand*{\mbt}[1]{\makebox[\LargestTextSize][l]{#1}}%

\newcommand{\R}[1]{\label{#1}\linelabel{#1}}
\newcommand{\lr}[1]{line~\lineref{#1}}

% ----------------------------------------------------------------------
% Launch!
% ----------------------------------------------------------------------

\begin{document}

\section{More on Orthogonal Complements}

\begin{remark}
  If $x = w+ z$ ($w\in W$, $z\in W^{\bot}$, then $w$ is the
  \textbf{orthogonal projection of $x$ on $W$}.  In fact, any
  orthogonal projection is a linear map.
\end{remark}

\begin{corollary}
  If $V$ is finite dimensional, then $\dim V = \dim W + \dim W^{\bot}$
  and $(W^{\bot})^{\bot} = W$.
\end{corollary}
\begin{proof}
  Since $V = W \oplus W^{\bot}$, then
  $\dim V = \dim W + \dim W^{\bot}$.

  Moreover, $\dim(W^{\bot}) = \dim V - \dim W^{\bot} = \dim W$.

  We prove now that $W\su (W^{\bot})^{\bot}$.

  If $x\in W$, then $\ipr{x}{y} = 0$ for all $y\in W^\bot$. Therefore,
  $x\in (W^{\bot})^{\bot}$.
\end{proof}

\begin{corollary}
  If $x = w+z$ for $w\in W, z \in W^{\bot}$, then $w$ is the unique vector closest to $x$ in $W$. Thus, for all $u\in W$ such that $u\neq w$:
  
  \begin{equation*}
    \norm{x-u} < \norm{x-w}
  \end{equation*}
\end{corollary}
\begin{proof}
  If $a, b \in V$ and $\ipr{a}{b} = 0$, then
  $\norm{a+b}^2 = \norm{a}^{2}+\norm{b}^{2}$.

  Therefore,
  $\norm{(w-u)+z}^2= \norm{w-u}^2 + \norm{z}^2 \geq \norm{z}^2$.
\end{proof}

\begin{theorem}
  Let $S = \set{v_1,\dots, v_k}$ be an orthonormal subset such that $\dim V = n$:
  \begin{enumerate}[label=\alph*)]
  \item Then $S$ can be extended to an orthonormal basis of $V$
  \item If $W = \spn\set{v_1, \dots, v_k}$, then $W^{\bot} = \spn\set{v_{k+1}, \dots, v_n}$.
  \end{enumerate}
\end{theorem}

\begin{proof}
  \begin{enumerate}[label=\alph*)]
  \item 
  
  First, extend $\set{v_1, \dots, v_k}$. Use the Gram-Schmidt procedure to make it orthonormal. Note that any orthonormal subset is linearly independent.

  The first $k$ elements of the new basis are unchanged. In this way, $v_1, \dots, v_n$ is an orthonormal basis of $V$ after normalisation.
\item If $V = \spn\set{v_k, \dots, v_n}$, then $\ipr{x}{y} = 0$ for all $x\in W$ and $g\in V$, which is logically equivalent to $V\su W^{\bot}$.
  Since $\dim W^{\bot} = \dim U$, then $V = W^{\bot}$.
    \end{enumerate}
\end{proof}

\section{Adjoints}

If $V$ is a finite dimensional inner product space and
$T \in \Hom(V,V)$, then there exists a unique $T^{*}\in \Hom(V,V)$,
called an \textbf{adjoint} such that
$\ipr{T(x)}{y} = \ipr{x}{T^{*}(y)}$ for all $x, y \in V$.

Note that $[T^{*}]_{\beta} = ([T]_{\beta})^{*}$. Tho show this, note
that if $y\in V$, the function $f_y: V\to \FF$ is linear.

\begin{theorem}
  Let $V$ be a finite dimensional vector space. If $f: V \to \FF$ is
  linear, then there exists a unique $y\in V$ such that $f = f_y$, i.e. $f(x) = \ipr{x}{y}$ for all $x \in V$.
\end{theorem}

\begin{proof}
  Pick an orthonormal basis $v_1,\dots, v_n$.

  Suppose $f = f_y$.

  Therefore, $f(v_1) = \ipr{v_1}{y}$. Hence,
  $y = \sum_{i=1}^n \ipr{y}{v_i}v_i = \sum_{i=1}^n \ol{f(v_i)}v_{i}$.

  Define $y = \sum_{i=1}^n\ol{f(v_i)}v_i$.

  To prove that $y$ is unique, it's enough to show that $f(v_i) = f_y(v_i)$.

  Observe that $f_y(v_i) = \ipr{v_i}{y} = \ipr{v_i}{\sum_{j=1}^n\ol{f(v_j)}v_j}$.

  Therefore, $f_y(v_i) = \sum_{j=1}^nf(v_j)\ipr{v_i}{v_j} = f(v_i)$.
\end{proof}



\end{document}