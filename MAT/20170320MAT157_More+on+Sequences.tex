% -*- coding: utf-8; -*-
%%% Local Variables:
%%% mode: latex
%%% TeX-engine: xetex
%%% TeX-master: t
%%% End:
\documentclass[11pt]{scrartcl}
\usepackage[fancy, beaue, pset, anon]{masty}
\pSet{\nt{MAT157}{}{More on Sequences}}
\usepackage{lineno}
% ----------------------------------------------------------------------
% Page setup
% ----------------------------------------------------------------------

\pagenumbering{gobble}

% ----------------------------------------------------------------------
% Custom commands
% ----------------------------------------------------------------------

% alignment

\newcommand*{\LongestHence}{$\Rightarrow$}% function name
\newcommand*{\LongestName}{$f_o(-x)+f_e(-x)$}% function name
\newcommand*{\LongestValue}{$(-a)x +(-a)(-y)$}% function value
\newcommand*{\LongestText}{\defi}%

\newlength{\LargestHenceSize}%
\newlength{\LargestNameSize}%
\newlength{\LargestValueSize}%
\newlength{\LargestTextSize}%

\settowidth{\LargestHenceSize}{\LongestHence}%
\settowidth{\LargestNameSize}{\LongestName}%
\settowidth{\LargestValueSize}{\LongestValue}%
\settowidth{\LargestTextSize}{\LongestText}%

% Choose alignment of the various elements here: [r], [l] or [c]

\newcommand*{\mbh}[1]{{\makebox[\LargestHenceSize][r]{\ensuremath{#1}}}}%
\newcommand*{\mbn}[1]{{\makebox[\LargestNameSize][r]{\ensuremath{#1}}}}%
\newcommand*{\mbv}[1]{\ensuremath{\makebox[\LargestValueSize][r]{\ensuremath{#1}}}}%
\newcommand*{\mbt}[1]{\makebox[\LargestTextSize][l]{#1}}%

\newcommand{\R}[1]{\label{#1}\linelabel{#1}}
\newcommand{\lr}[1]{line~\lineref{#1}}

% ----------------------------------------------------------------------
% Launch!
% ----------------------------------------------------------------------

\begin{document}

\section{More on Sequences}

\begin{definition}
  A sequence $\set{a_n}$ is a \textbf{Cauchy sequence} if for all
  $\epsilon > 0$ there exists $N\in \NN$ such that
  $\abs{a_n-a_m} < \epsilon$ whenever $m, n > N$.
\end{definition}

\begin{theorem}
A sequence is Cauchy if and only if it converges.
\end{theorem}

\begin{proof}
  \hfill

The converse follows easily from the definition of convergence and triangle inequality.

Suppose now that $\set{a_n}$ is Cauchy.

% Let $I_{m,n} = \begin{cases}
%   (a_m, a_n), \text{ if $a_m$}\\
% (a_n, a_m),\\
% \set{a_n},
% \end{cases}$ if $a_m $

Consider $\lim_{n\to \infty} \sup a_n$, $\lim_{n\to\infty}\inf
a_n$. We want to show that they exist and are equal to each other.

Given $\epsilon > 0$, choose $N$ as above. Then
$\abs{a_n-a_{n+1}} < \epsilon$ for any $n > N$. Thus
$\abs{a_n-a_m} < 2\epsilon$ for any $m, n > N$.

We know that $\abs{\lim_{n\to\infty} \sup a_n - a_{N+1}} < \epsilon$
and $\lim \inf_{n\to \infty} a_n - a_{N+1} < \epsilon$.

Therefore, $\abs{\lim \sup a_n - \lim \inf a_n} < 2\epsilon$, and thus
$\lim \sup a_n = \lim \inf a_n$ and hence there exists $\lim a_n$
exists.
\end{proof}

We now consider nonnegative series.

We have defined the $n$th partial sum of a sequence as $s_n = \sum_{k=1}^na_k$. Note that they are non-decreasing, since $a_k\geq 0$.

\begin{theorem}
  If a sequence $\set{a_i}$ is nonnegative, then
  $\sum_{n=1}^{\infty} a_n$ converges if and only if the partial sums
  are bounded.
\end{theorem}
\begin{remark}
There is no equivalent result for sequences that are not nonnegative. Take for example, the alternating sequence of $1$ and $-1$. Then $s_n$ are bounded, but there is no limit.
\end{remark}

\begin{theorem}
Suppose that $0\leq a_n \leq b_n$ for all $n\in\NN$.

If $\sum_{i=1}^{\infty}b_i$ converges, then $\sum_{i=1}^{\infty}a_i$ converges.
\end{theorem}
\begin{proof}
  \hfill

  Note that $S_k(\set{a_i}) \leq S_k(\set{b_i})$. 

  Suppose that $\sum_{i=1}^{\infty}b_i$ converges and thus
  $S_k(\set{b_i})$ is bounded.

  Therefore, $S(\set{a_i})$ is bounded and hence
  $\sum_{i=1}^{\infty} a_i$ converges.
\end{proof}

\begin{example}

  $\sum_{n=1}^{\infty} \frac{1}{6\* 2^n + \cos (n)} \leq
  \sum_{n=1}^{\infty} \frac{1}{5\*2^n}$ and thus
  $\set{a_i = \frac{1}{6\* 2^n + \cos (n)}}$ converges.
\end{example}

Consider the graph of $f(x) = \frac{1}{x}$ and the corresponding harmonic series.

Note that
\[\sum_{i=1}^{\infty} \frac{1}{n} > \int_{i=1}^{\infty} \frac{\dif t}{t} = \lim_{R\to \infty}\int_{i=1}^R \frac{\dif t}{t} = \lim_{R\to \infty} \log (R).\]

We can then define an \textbf{integral test}:

\begin{definition}
\hfill

  Suppose $f$ is a non-increasing punction and $a_n = f(n)$. Then
  $\sum_{i=1}^{\infty}a_n$ converges if and only if
  $\int_{i=1}^{\infty} f(x) \dif x$ converges.
\end{definition}

We can still use this test as long as the hypothesis is satisfied after some ponts.
\end{document}
