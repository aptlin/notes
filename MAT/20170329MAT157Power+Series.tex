% -*- coding: utf-8; -*-
%%% Local Variables:
%%% mode: latex
%%% TeX-engine: xetex
%%% TeX-master: t
%%% End:
\documentclass[11pt]{scrartcl}
\usepackage[fancy, beaue, pset, anon]{masty}
\pSet{\nt{MAT157}{}{Power Series}}
\usepackage{lineno}
% ----------------------------------------------------------------------
% Page setup
% ----------------------------------------------------------------------

\pagenumbering{gobble}

% ----------------------------------------------------------------------
% Custom commands
% ----------------------------------------------------------------------

% alignment

\newcommand*{\LongestHence}{$\Rightarrow$}% function name
\newcommand*{\LongestName}{$f_o(-x)+f_e(-x)$}% function name
\newcommand*{\LongestValue}{$(-a)x +(-a)(-y)$}% function value
\newcommand*{\LongestText}{\defi}%

\newlength{\LargestHenceSize}%
\newlength{\LargestNameSize}%
\newlength{\LargestValueSize}%
\newlength{\LargestTextSize}%

\settowidth{\LargestHenceSize}{\LongestHence}%
\settowidth{\LargestNameSize}{\LongestName}%
\settowidth{\LargestValueSize}{\LongestValue}%
\settowidth{\LargestTextSize}{\LongestText}%

% Choose alignment of the various elements here: [r], [l] or [c]

\newcommand*{\mbh}[1]{{\makebox[\LargestHenceSize][r]{\ensuremath{#1}}}}%
\newcommand*{\mbn}[1]{{\makebox[\LargestNameSize][r]{\ensuremath{#1}}}}%
\newcommand*{\mbv}[1]{\ensuremath{\makebox[\LargestValueSize][r]{\ensuremath{#1}}}}%
\newcommand*{\mbt}[1]{\makebox[\LargestTextSize][l]{#1}}%

\newcommand{\R}[1]{\label{#1}\linelabel{#1}}
\newcommand{\lr}[1]{line~\lineref{#1}}

% ----------------------------------------------------------------------
% Launch!
% ----------------------------------------------------------------------

\begin{document}

\section{Power Series}

\subsection{Review}

Consider $f(x) = \sum_{n=0}^{\infty}a_n(x-a)^n$.

We have shown that $f(x)$ converges on an open interval $(a-r, a+r)$. Moreover, $f(x)$ converges uniformly on any closed subintreval $[a-s, a+s]$ such that $0 < s < r$.

In particular, $f(x)$ is continuous on $(a-r, a+r)$.

Our guess is that $f'(x) = \sum_{n=0}^{\infty}a_nn(x-a)^n$.

\subsection{Weierstrass $M$-Test}

\begin{theorem}
Suppose that $\abs{f_n(x)} \leq M_n$ for all $x\in I$, where $I$ is some interval.

Suppose also that $\sum_{n=0}^{\infty}M_n$ converges.

Then $\sum_{n=0}^{\infty}f_n(x)$ converges absolutely and uniformly on $I$.
\end{theorem}

\begin{proof}
  \hfill

Proceed with the comparison test.
\end{proof}
\begin{lemma}
If $f_n(x)$ is continuous for all $x\in D(f)$, then
$\sum_{n=0}^{\infty}f_n(x)$ is continuous.
\end{lemma}
\begin{theorem}
  Suppose $f_n(x) \to f(x)$ for all $x\in I$, and suppose that
  $f_n(x)$ is differentiable for all $n\in\NN$ and $x\in I$. Suppose
  also that $f_n'(x)$ converges uniformly to $g(x)$ on $I$ and $g(x)$ is continuous.

  Then $g$ is differentiable and $g(x) = f'(x)$.
\end{theorem}

\begin{proof}
  \hfill

  \[\int_a^xg(t)\dif t = \int_a^x\lim_{n\to\infty}f_n'(t)\dif t = \lim_{n\to\infty}\int_a^xf_n'(t)\dif t,\] 

  where the second equality holds because of uniform convergence.

  Therefore, $\lim_{n\to\infty}(f_n(x) - f_n(a)) = f(x)-f(a)$, so
  $f'(x) = g(x)$.
\end{proof}

For $f(x) = \sum_{n=0}^{\infty}(x-a)^n$, let
$f_n(x) = \sum_{m=0}^na_m(x-a)^m$.

Then by the ratio test we can guarantee that $f_n(x)\to f(x)$
uniformly on $[a-s, a+s]$.

Note that $f_n'(x) = \sum_{m=0}^na_mm(x-a)^{m-1}$, and $f'_m(x)$ converges uniformly to $f'(x)$.

\begin{remark}
\hfill

  On the interval $(a-r, a+r)$, where $\sum_{n=0}^{\infty}a_n(x-a)^n$
  converges absolutely, its behaviour is
  $\sum_{n=1}^{\infty} n a_n(x-a)^{n-1}$ which also converges
  absolutely.

\end{remark}

\begin{example}

Consider $s(x) =\sum_{n=1}^{\infty} n^2x^{n}$.

Let $a_n = n^2x^n$.

Using the ratio test, we can see that
\[\abs{\frac{a_{n+1}}{a_n}} = \abs{\frac{(n+1)^2x^{n+1}}{n^2x^n}} = \abs{x}\] as $n\to \infty$.

Therefore, $s(x)$ converges for $\abs{x}< 1$.

Note that $\frac{x}{(1-x)^2} = \sum_{n=1}^{\infty}nx^n$.

Differentiating both sides, we obtain that
$\frac{1+x}{(1-x)^2}= \sum_{n=1}^{\infty}n^2x^{n-1}$, and thus
$\\frac{x+x^2}{(1-x)^2} = \sum_{n=1}^{\infty}n^2x^n$.

\end{example}

We can now construct a hierarchy of functions: $C^{\infty} $ $\su$ $\dots$ $\su$ Continuously Differentiable Functions $\su$ Differentiable Functions $\su$ Continuous Functions $\su$ Functions. 

There is also another class, the class of \textit{analytic functions},
with the corresponding power series convergent at each point.

We know that there are functions in $C^{\infty}$ that are not
analytic, for example, $e^{-\nicefrac{1}{x^{2}}}$.
\end{document}
