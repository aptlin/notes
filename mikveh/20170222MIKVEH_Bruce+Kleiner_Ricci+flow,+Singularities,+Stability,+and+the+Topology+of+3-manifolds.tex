% -*- coding: utf-8; -*-
%%% Local Variables:
%%% mode: latex
%%% TeX-engine: xetex
%%% TeX-master: t
%%% End:
\documentclass[11pt]{scrartcl}
\usepackage[fancy, beaue, pset, anon]{masty}
\pSet{\nt{Bruce Kleiner}{}{Ricci Flow and the Topology of 3-Manifolds }}
\usepackage{lineno}
% ----------------------------------------------------------------------
% Page setup
% ----------------------------------------------------------------------

\pagenumbering{gobble}

% ----------------------------------------------------------------------
% Custom commands
% ----------------------------------------------------------------------

% alignment

\newcommand*{\LongestHence}{$\Rightarrow$}% function name
\newcommand*{\LongestName}{$f_o(-x)+f_e(-x)$}% function name
\newcommand*{\LongestValue}{$(-a)x +(-a)(-y)$}% function value
\newcommand*{\LongestText}{\defi}%

\newlength{\LargestHenceSize}%
\newlength{\LargestNameSize}%
\newlength{\LargestValueSize}%
\newlength{\LargestTextSize}%

\settowidth{\LargestHenceSize}{\LongestHence}%
\settowidth{\LargestNameSize}{\LongestName}%
\settowidth{\LargestValueSize}{\LongestValue}%
\settowidth{\LargestTextSize}{\LongestText}%

% Choose alignment of the various elements here: [r], [l] or [c]

\newcommand*{\mbh}[1]{{\makebox[\LargestHenceSize][r]{\ensuremath{#1}}}}%
\newcommand*{\mbn}[1]{{\makebox[\LargestNameSize][r]{\ensuremath{#1}}}}%
\newcommand*{\mbv}[1]{\ensuremath{\makebox[\LargestValueSize][r]{\ensuremath{#1}}}}%
\newcommand*{\mbt}[1]{\makebox[\LargestTextSize][l]{#1}}%

\newcommand{\R}[1]{\label{#1}\linelabel{#1}}
\newcommand{\lr}[1]{line~\lineref{#1}}

% ----------------------------------------------------------------------
% Launch!
% ----------------------------------------------------------------------

\begin{document}

\section{Ricci Flow and the Topology of 3-Manifolds}

The following result was proved in XIX century:
\begin{theorem}
  Any closed orientable surface $N$ is diffeomorphic to a connected
  sum $T^2\#\cdots\#T^2$. Moreover, $N$ has a Riemannian metric $g$
  with a constant Gaussian curvature $K\in\set{-1, 0, 1}$.
\end{theorem}

\subsection{Topology and Geometry of 3-Manifolds}

Key words: quotient manifold, Poincare dodecahedral space, flat
manifolds, Thurston geometry, homogeneous Riemannian 3-manifolds,
fundamental Lie groups and their quotients, geometric manifold,
connected sum of manifolds, prime manifolds

\begin{definition}
  A smooth 3-manifold $M$ is geometrizable if it admits a Riemannian
  metric $g$ such that $(M, g)$ is geometric.
\end{definition}

It was proven by Kneeser and Milnor that decomposition of every
nontrivial 3-manifold into prime manifolds is always possible, and the
corresponding primes are unique up to order. However, there are prime
manifolds which are not geometrisable.

\subsection{Ricci Flow}

Intuitively, Ricci flow connects the change of a manifold with time to
its Ricci curvature.

Key Words: deformation retraction, homotopy equivalence, fibration, stabiliser, Sobolev space

\end{document}