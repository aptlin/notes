% -*- coding: utf-8; -*-
%%% Local Variables:
%%% mode: latex
%%% TeX-engine: xetex
%%% TeX-master: t
%%% End:
\documentclass[11pt]{scrartcl}
\usepackage[fancy, beaue, pset, anon]{masty}
\pSet{\nt{Hector Pastor}{}{On the abc Conjecture}}
\usepackage{lineno}
% ----------------------------------------------------------------------
% Page setup
% ----------------------------------------------------------------------

\pagenumbering{gobble}

% ----------------------------------------------------------------------
% Custom commands
% ----------------------------------------------------------------------

% alignment

\newcommand*{\LongestHence}{$\Rightarrow$}% function name
\newcommand*{\LongestName}{$f_o(-x)+f_e(-x)$}% function name
\newcommand*{\LongestValue}{$(-a)x +(-a)(-y)$}% function value
\newcommand*{\LongestText}{\defi}%

\newlength{\LargestHenceSize}%
\newlength{\LargestNameSize}%
\newlength{\LargestValueSize}%
\newlength{\LargestTextSize}%

\settowidth{\LargestHenceSize}{\LongestHence}%
\settowidth{\LargestNameSize}{\LongestName}%
\settowidth{\LargestValueSize}{\LongestValue}%
\settowidth{\LargestTextSize}{\LongestText}%

% Choose alignment of the various elements here: [r], [l] or [c]

\newcommand*{\mbh}[1]{{\makebox[\LargestHenceSize][r]{\ensuremath{#1}}}}%
\newcommand*{\mbn}[1]{{\makebox[\LargestNameSize][r]{\ensuremath{#1}}}}%
\newcommand*{\mbv}[1]{\ensuremath{\makebox[\LargestValueSize][r]{\ensuremath{#1}}}}%
\newcommand*{\mbt}[1]{\makebox[\LargestTextSize][l]{#1}}%

\newcommand{\R}[1]{\label{#1}\linelabel{#1}}
\newcommand{\lr}[1]{line~\lineref{#1}}

% ----------------------------------------------------------------------
% Launch!
% ----------------------------------------------------------------------

\begin{document}

\section{On the abc Conjecture}
\subsection{Introduction}

Problems that directly relate the additive and multiplicative structure in $\ZZ$ tend to look deceptively simple and harshly difficult. (Goldbach's conjecture, Twin prime conjecture, FLT)

The abc conjecture gives a lower bound for $\rad$ of integer sums.

In general, bounds of the form $c < f(\rad(abc))$ are far from trivial
(see Mahler's theorem).

More precisely, the abc conjecture states that for some $\epsilon >0$
there is a constant $K_{\epsilon}$ such that for all coprime positive
integers $a, b, c$ with $a+b = c$ we have
$c < K_{\epsilon}\rad(abc)^{1+\epsilon}$.

We know that the condition $\epsilon > 0$ is necessary.

% We also know that there is no known 

\subsection{Relevance}
If the abc conjecture is true, it
\begin{itemize}
\item guarantees there  a \textit{simple} proof
  of Falting's theorem (general equation $f(x, y) = 0$ with $f\in \QQ[x, y]$ with $\deg f \geq 4$ has only finitely many solutions in $\QQ$)
\item gives a master key for ternary Diophantine equations
\item gives many results in the theory of elliptic curves
\item implies FLT for large exponents
\item gives an asymptotic formula for counting squarefree values of polynomials $f(t)\in \ZZ[t]$ as we evaluate at $t=1, 2, \dots$, as shown by Granville (1998)
\item warrants the infinitude of non-Wieferich primes ($p$ such that $p^2\nmid 2^{p-1} - 1$)
\item power values imply power factoriation:

  If abc($\QQ^{\leq k}$) holds, then for $k\geq 2$ there is a constant
  $M = M(k)$ such that if a \textbf{monic} polynomial $f(t\in \ZZ[t])$
  of \textbf{degree} $k$ satisfies the condition that $f(1)$, $f(2)$,
  \dots, $f(M)$ are all powers of integers, then $f(x) = (t+b)^k$ for
  some $b\in\ZZ$.
\item gives applications to Erd\"os-Ulam problem about rational
  distance sets:

  A \textbf{rational distance set} $U\suq \RR^2$ is a set such that
  for all $x, y \in U$ we have $\norm{x-y}_2\in \QQ$. For instance,
  $\QQ$ in the $X$-axis is a rational distance set.


\end{itemize}

\subsection{Is the field of meromorphic functions \textit{easy}?}

For $k$ a field with an absolute value, we write $\SM_k$ for the field of (possibly transcendental) meromorphic functions on $k$.

Vojta formulated a conjecture in the 1980 by which number fields have arithmetic analogous to that of the fields.q

For example, in case of curves, Faltings theorem is analogous to the
Picard-Berkovich theorem.

Vojta's dictionary is intimately related to the abc conjecture.

\subsection{Elliptic Curves}

\begin{definition}
  An elliptic curve over $\QQ$ is a smooth, geometrically connected,
  projective curve over $\QQ$ of genus $1$ with a distinguished
  $\QQ$-rational point.
\end{definition}

There are two integers attached to every elliptic curve: $\Delta_E$,
the absolute value of the minimal discriminant, and $N_E$, the
conductor of $E$, where $N_E$ divides $\Delta_E$ and they have the
same prime factors.

It can be shown that all elliptic curves over $\QQ$ are modular (see
Wiles et al).  The are some curves $X_0(N)$ over $\QQ$ such that each
curve can be reconstructed from the given integer $N$. They are
well-understood and support the modular theory.
\end{document}
