% -*- coding: utf-8; -*-
%%% Local Variables:
%%% mode: latex
%%% TeX-engine: xetex
%%% TeX-master: t
%%% End:
\documentclass[11pt]{scrartcl}
\usepackage[fancy, beaue, pset, anon]{sdll}
\pSet{\nt{Curtis McMullen}{}{A Skeptical History of Numbers}}
\usepackage{lineno}
% ----------------------------------------------------------------------
% Page setup
% ----------------------------------------------------------------------

\pagenumbering{gobble}

% ----------------------------------------------------------------------
% Custom commands
% ----------------------------------------------------------------------

% alignment

\newcommand*{\LongestHence}{$\Rightarrow$}% function name
\newcommand*{\LongestName}{$f_o(-x)+f_e(-x)$}% function name
\newcommand*{\LongestValue}{$(-a)x +(-a)(-y)$}% function value
\newcommand*{\LongestText}{\defi}%

\newlength{\LargestHenceSize}%
\newlength{\LargestNameSize}%
\newlength{\LargestValueSize}%
\newlength{\LargestTextSize}%

\settowidth{\LargestHenceSize}{\LongestHence}%
\settowidth{\LargestNameSize}{\LongestName}%
\settowidth{\LargestValueSize}{\LongestValue}%
\settowidth{\LargestTextSize}{\LongestText}%

% Choose alignment of the various elements here: [r], [l] or [c]

\newcommand*{\mbh}[1]{{\makebox[\LargestHenceSize][r]{\ensuremath{#1}}}}%
\newcommand*{\mbn}[1]{{\makebox[\LargestNameSize][r]{\ensuremath{#1}}}}%
\newcommand*{\mbv}[1]{\ensuremath{\makebox[\LargestValueSize][r]{\ensuremath{#1}}}}%
\newcommand*{\mbt}[1]{\makebox[\LargestTextSize][l]{#1}}%

\newcommand{\R}[1]{\label{#1}\linelabel{#1}}
\newcommand{\lr}[1]{line~\lineref{#1}}

% ----------------------------------------------------------------------
% Launch!
% ----------------------------------------------------------------------

\begin{document}

\begin{itemize}
\item Doyle-McMullen,  solving the quintic (2000)
\item Representation of the proof of the fundamental theorem of
  algebra with the syrup spreading out over the complex plane
\item Skewes' number is the bound for when first $\pi(x) < li(x)$ $<<$
  wowsers $<<$ Graham's number (party theorem on $n$-dim cube for 4
  friends/enemies on a hyperplane. Note that $12 < n < G$)
\item Busy beaver function: $B(n)=$ larget possible output of a rogue
  but mortal computer program of length $n$
\item Precision is required --
  $N = [\text{the smallest positive integer not definable in fewer
    than twelve words}]$
\item Solutions of crises: type theory (do not define $A$ in terms of
  $A$), intuitionism (deal with constructable things), formalism
  (agree on axioms, and only admit conclusions from them)
\item Edward Nelson: length of the shortest proof such that $0=1$?
\item Radically elementary theory of probability
\end{itemize}
\end{document}