% -*- coding: utf-8; -*-
%%% Local Variables:
%%% mode: latex
%%% TeX-engine: xetex
%%% TeX-master: t
%%% End:
\documentclass[11pt]{scrartcl}
\usepackage[fancy, beaue, pset, anon]{masty}
\pSet{\nt{Fedor Manin}{}{Why Should You Care About Model Categories?}}
\usepackage{lineno}
% ----------------------------------------------------------------------
% Page setup
% ----------------------------------------------------------------------

\pagenumbering{gobble}

% ----------------------------------------------------------------------
% Custom commands
% ----------------------------------------------------------------------

% alignment

\newcommand*{\LongestHence}{$\Rightarrow$}% function name
\newcommand*{\LongestName}{$f_o(-x)+f_e(-x)$}% function name
\newcommand*{\LongestValue}{$(-a)x +(-a)(-y)$}% function value
\newcommand*{\LongestText}{\defi}%

\newlength{\LargestHenceSize}%
\newlength{\LargestNameSize}%
\newlength{\LargestValueSize}%
\newlength{\LargestTextSize}%

\settowidth{\LargestHenceSize}{\LongestHence}%
\settowidth{\LargestNameSize}{\LongestName}%
\settowidth{\LargestValueSize}{\LongestValue}%
\settowidth{\LargestTextSize}{\LongestText}%

% Choose alignment of the various elements here: [r], [l] or [c]

\newcommand*{\mbh}[1]{{\makebox[\LargestHenceSize][r]{\ensuremath{#1}}}}%
\newcommand*{\mbn}[1]{{\makebox[\LargestNameSize][r]{\ensuremath{#1}}}}%
\newcommand*{\mbv}[1]{\ensuremath{\makebox[\LargestValueSize][r]{\ensuremath{#1}}}}%
\newcommand*{\mbt}[1]{\makebox[\LargestTextSize][l]{#1}}%

\newcommand{\R}[1]{\label{#1}\linelabel{#1}}
\newcommand{\lr}[1]{line~\lineref{#1}}

% ----------------------------------------------------------------------
% Launch!
% ----------------------------------------------------------------------

\begin{document}

\begin{theorem}
  If a map between CW complexes is a \textit{weak homotopy
    equivalence}, then it is a homotopy equivalence.
\end{theorem}

\begin{remark}
  This does not hold for more general spaces. For example, Cantor's
  set is not a homotopy equivalence to discrete uncountable
  space. Similarly, the Warsaw circle, \textit{connected topologist sine curve} is
  not homotopy equivalent to a point, because the length of the
  topologist sine curve has an infinitely large length.
\end{remark}

Let $G$ be a finite group. What can we say about spaces with the
action of G and $G$-equivalent maps? What can we say about
$G$-isovariant maps, maps which are equivariant and such that for any
$x$, the stabiliser of $f(x)$ is the same as the stabiliser of $x$?

What if, provided a map between $G$-CW complexes induces equivariant
isomorphisms on homotopy groups, then it's an equivariant homotopy
equivalence? Unfortunately, this does not hold. However, we still want
to something about equivariant homotopy equivalences.

\begin{definition}
  A \textit{model structure} on a category $\SC$, which has all limits
  and colimits, consists of $3$ distinguished classes of morphims:
  fibrations (F), cofibrations (C), and weak equivalences (W).
\end{definition}

We provide the following axioms, along with the property of
\textit{lifting}:

\begin{itemize}
\item if $g$is a fibration, cofibration or a weak equivalence, and $f$
  is a retract of $g$, then $f$ is also.
\item If $g\circ f$ can be composed and 2 out $f, g, g\circ f$ are one
  of F, C or W, then so is the third.
\item every morphism $f$ can be written as $p\circ i$, where $p$ is a
  fibration, while $i$ is a cofibration, and either one can also be a
  weak fibration.
\end{itemize}
\begin{remark}
  What is a cofibration? A cofibration can be thought as an inclusion
  of a subcomplex of a CW complex.
\end{remark}

We can now define a standard model structure on Top:

\begin{itemize}
\item W are weak homotopy equivalences.
\item F are Serre fibrations
\item C are whatever that has the lifing property with respect to acyclic fibration
\end{itemize}

We need some more definitions.

\begin{definition}
  An object $X$ is fibrant if $ X \to *$ is a fibration.
\end{definition}

\begin{definition}
  An object $X$ is cofibrant if $\emptyset \to X$is a cofibration.
\end{definition}

Now, we have enough to prove the following lemma.

\begin{lemma}
Every object is weak equivalent to one that is both fibrant and cofibrant.
\end{lemma}

Having defined a cylinder object, we can define left and right
homotopies, so that the following can be proven.

\begin{theorem}
  If $X, Y$ are both fibrant and cofibrant, then $f, g: X\to Y$ are
  left homotopy equivalent if and only if they are right homotopy
  equivalent.
\end{theorem}

\begin{theorem}
Fibrant-cofibrant objects are weak equivalent if and only if they are homotopy equivalent.
\end{theorem}

Do equivariant spaces and maps model structure for Top$^G$?

We can make the following choices:

\begin{itemize}
\item F are still equivariant Serre fibrations.
\item A choice of W determines C
\end{itemize}

It turns out that cofibrant spaces are free $G$-CW complexes.
\end{document}