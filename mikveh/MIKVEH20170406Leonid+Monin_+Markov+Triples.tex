% -*- coding: utf-8; -*-
%%% Local Variables:
%%% mode: latex
%%% TeX-engine: xetex
%%% TeX-master: t
%%% End:
\documentclass[11pt]{scrartcl}
\usepackage[fancy, beaue, pset, anon]{masty}
\pSet{\nt{Leonid Monin}{}{Markov Triples}}
\usepackage{lineno}
% ----------------------------------------------------------------------
% Page setup
% ----------------------------------------------------------------------

\pagenumbering{gobble}

% ----------------------------------------------------------------------
% Custom commands
% ----------------------------------------------------------------------

% alignment

\newcommand*{\LongestHence}{$\Rightarrow$}% function name
\newcommand*{\LongestName}{$f_o(-x)+f_e(-x)$}% function name
\newcommand*{\LongestValue}{$(-a)x +(-a)(-y)$}% function value
\newcommand*{\LongestText}{\defi}%

\newlength{\LargestHenceSize}%
\newlength{\LargestNameSize}%
\newlength{\LargestValueSize}%
\newlength{\LargestTextSize}%

\settowidth{\LargestHenceSize}{\LongestHence}%
\settowidth{\LargestNameSize}{\LongestName}%
\settowidth{\LargestValueSize}{\LongestValue}%
\settowidth{\LargestTextSize}{\LongestText}%

% Choose alignment of the various elements here: [r], [l] or [c]

\newcommand*{\mbh}[1]{{\makebox[\LargestHenceSize][r]{\ensuremath{#1}}}}%
\newcommand*{\mbn}[1]{{\makebox[\LargestNameSize][r]{\ensuremath{#1}}}}%
\newcommand*{\mbv}[1]{\ensuremath{\makebox[\LargestValueSize][r]{\ensuremath{#1}}}}%
\newcommand*{\mbt}[1]{\makebox[\LargestTextSize][l]{#1}}%

\newcommand{\R}[1]{\label{#1}\linelabel{#1}}
\newcommand{\lr}[1]{line~\lineref{#1}}

% ----------------------------------------------------------------------
% Launch!
% ----------------------------------------------------------------------

\begin{document}

\section{Markov Triples}

\begin{definition}
A \textbf{Markov triple} is an integer solution of the quation $x^2+y^2+z^2 = 3xyz$.
\end{definition}

Note that the equation $x^2+y^2+z^2 = 3xyz$ is quadratic in each variable.

In this sence, it can be rearranged, for example, to $x^2-(3bc)x+b^2+c^2=0$.

Using the Vieta formula, we know that, if $a$ and $a'$ are two
solutions, then $a+a' = 3bc$.

Note that we can construct a tree of solutions starting with a trivial
solution $(1, 1, 1)$: $(1, 1, 1) \to (1, 1, 2)\to(1, 2, 5)\to \dots$.

\begin{ques*}

  Are Markov triples determined by its maximal element?

\end{ques*}

\subsection{Approximations of Irrational Numbers $\alpha \in \RR\setminus \QQ$}

\begin{theorem}
  For any $\alpha \in \RR \setminus \QQ$ and $N\in\NN$, there exists
  $\frac{p}{q}\in \QQ$ such that $\gcd(p, q)=1$ and $q < N$ and
  $\abs{\alpha - \frac{p}{q}} < \frac{1}{q\* N} \leq \frac{1}{q^c}$
  for some $ c\in\NN$.
\end{theorem}
\begin{corollary}
  For each $\alpha\in \RR\setminus\QQ$, there exist infinitely many
  $\frac{p}{q}$ such that
  $\abs{\alpha - \frac{p}{q}} < \frac{1}{q^2}$.
\end{corollary}
\begin{definition}
  $\alpha$ is \textbf{approximable} of degree $t$ if and only if there
  exist infinitely many $\frac{p}{q}$ such that
  $\abs{\alpha - \frac{p}{q}} < \frac{1}{q^t}$.
\end{definition}

\begin{theorem}
  If $\alpha \in \RR$, $\epsilon > 0$ and there exist infinitely many
  $\frac{p}{q}$ such that
  $\abs{\alpha - \frac{p}{q}} < \frac{1}{q^{2+\epsilon}}$, then
  $\alpha$ is transcendental.
\end{theorem}
\begin{remark}
See also the works of Liouville, Thue, Siegel and Dyson.
\end{remark}

Let $L(\alpha)$ be a supremum of $\set{c; \text{ there exist infinitely many $\frac{p}{q}\in \QQ$ such that $\abs{\alpha- \frac{p}{q}} < \frac{1}{cq^2}$}}$.

\begin{definition}
  A \textbf{Lagrange spectrum} is the set
  $\LL = \set{L(\alpha); \alpha \in \RR\setminus\QQ}$.
\end{definition}

Denote the set of integers which appear in Markov triples as $\SM$.

\begin{theorem}
  Consider $\LL_{<3} = \set{\frac{\sqrt{9m^2-4}}{m}; m\in \SM}$, and
  let $\gamma_m = \frac{a_m+\sqrt{9m^2-4}}{b_m}$ for some integer
  $a_m$ and $b_m$.
\end{theorem}

Thus, if 

\[\inf Q = \inf_{\shortstack[l]{$x,y \in
    \ZZ$\\$Q(x, y)\neq 0$}}\abs{Q(x, y)} =m,\]

then 
\begin{equation*}
  \LL_{< 3} = \set{\frac{\Delta(Q)}{\inf Q}}_{< 3}\Delta(Q) = 9m^2-4.
\end{equation*}

\textit{See further}: multiplicative commutator, cluster algebra

\end{document}
