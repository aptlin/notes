% -*- coding: utf-8; -*-
%%% Local Variables:
%%% mode: latex
%%% TeX-engine: xetex
%%% TeX-master: t
%%% End:
\documentclass[11pt]{scrartcl}
\usepackage[fancy, beaue, pset, anon]{masty}
\pSet{\nt{Arthur Jaffe}{}{A New Look at the Mathematics of Quantum Information}}
\usepackage{lineno}
% ----------------------------------------------------------------------
% Page setup
% ----------------------------------------------------------------------

\pagenumbering{gobble}

% ----------------------------------------------------------------------
% Custom commands
% ----------------------------------------------------------------------

% alignment

\newcommand*{\LongestHence}{$\Rightarrow$}% function name
\newcommand*{\LongestName}{$f_o(-x)+f_e(-x)$}% function name
\newcommand*{\LongestValue}{$(-a)x +(-a)(-y)$}% function value
\newcommand*{\LongestText}{\defi}%

\newlength{\LargestHenceSize}%
\newlength{\LargestNameSize}%
\newlength{\LargestValueSize}%
\newlength{\LargestTextSize}%

\settowidth{\LargestHenceSize}{\LongestHence}%
\settowidth{\LargestNameSize}{\LongestName}%
\settowidth{\LargestValueSize}{\LongestValue}%
\settowidth{\LargestTextSize}{\LongestText}%

% Choose alignment of the various elements here: [r], [l] or [c]

\newcommand*{\mbh}[1]{{\makebox[\LargestHenceSize][r]{\ensuremath{#1}}}}%
\newcommand*{\mbn}[1]{{\makebox[\LargestNameSize][r]{\ensuremath{#1}}}}%
\newcommand*{\mbv}[1]{\ensuremath{\makebox[\LargestValueSize][r]{\ensuremath{#1}}}}%
\newcommand*{\mbt}[1]{\makebox[\LargestTextSize][l]{#1}}%

\newcommand{\R}[1]{\label{#1}\linelabel{#1}}
\newcommand{\lr}[1]{line~\lineref{#1}}

% ----------------------------------------------------------------------
% Launch!
% ----------------------------------------------------------------------

\begin{document}

\section{A New Look at the Mathematics of Quantum Information}

\textit{see Zhengwei Liu, Alex Wozniakowski, $PA+PF \ra PAPPA \ra QI \ra M\cup \Phi$}

We know that a language of algebra can be defined for $\dim = 1$,
planar algebra for $\dim = 2$ and quon for $\dim = 3$. On the other
hand, there is a connection between 1-string theory and category
theory, 2-string theory and planar algebra and 4-string theory and
quon.

To study quantum information, we can use a diagrammatic
language. Thus, we can prove theorems with pictures! In this way, a
\textquote{charge} can be denoted as $\cap_k$ for $k\in \ZZ_d$ and its
adjoint, a \textit{vertical reflection}, as $\cap_k^{*} = \cup_{-k}$.

\textit{Key concepts}: twisted product, reflection positivity
condition, string Fourier transform (a generalisation of the Fourier
transform), Pauli matrices, Reidemeister moves
\end{document}
