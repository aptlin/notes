% -*- coding: utf-8; -*-
%%% Local Variables:
%%% mode: latex
%%% TeX-engine: xetex
%%% TeX-master: t
%%% End:
\documentclass[11pt]{scrartcl}
\usepackage[fancy, beaue, pset, anon]{masty}
\pSet{\nt{Peter Marbach}{}{Modeling and Analysis of Social, Economic and Financial Networks}}
\usepackage{lineno}
% ----------------------------------------------------------------------
% Page setup
% ----------------------------------------------------------------------

\pagenumbering{gobble}

% ----------------------------------------------------------------------
% Custom commands
% ----------------------------------------------------------------------

% alignment

\newcommand*{\LongestHence}{$\Rightarrow$}% function name
\newcommand*{\LongestName}{$f_o(-x)+f_e(-x)$}% function name
\newcommand*{\LongestValue}{$(-a)x +(-a)(-y)$}% function value
\newcommand*{\LongestText}{\defi}%

\newlength{\LargestHenceSize}%
\newlength{\LargestNameSize}%
\newlength{\LargestValueSize}%
\newlength{\LargestTextSize}%

\settowidth{\LargestHenceSize}{\LongestHence}%
\settowidth{\LargestNameSize}{\LongestName}%
\settowidth{\LargestValueSize}{\LongestValue}%
\settowidth{\LargestTextSize}{\LongestText}%

% Choose alignment of the various elements here: [r], [l] or [c]

\newcommand*{\mbh}[1]{{\makebox[\LargestHenceSize][r]{\ensuremath{#1}}}}%
\newcommand*{\mbn}[1]{{\makebox[\LargestNameSize][r]{\ensuremath{#1}}}}%
\newcommand*{\mbv}[1]{\ensuremath{\makebox[\LargestValueSize][r]{\ensuremath{#1}}}}%
\newcommand*{\mbt}[1]{\makebox[\LargestTextSize][l]{#1}}%

\newcommand{\R}[1]{\label{#1}\linelabel{#1}}
\newcommand{\lr}[1]{line~\lineref{#1}}

% ----------------------------------------------------------------------
% Launch!
% ----------------------------------------------------------------------

\begin{document}

\section{Modeling and Analysis of Social, Economic and Financial Networks}

\begin{problem*}
Social, economic and financial networks are important. How can we model them? How can we reach deeper understanding and develop more efficient algorithms?
\end{problem*}

\subsection{First Model}

Information was deemed as a concept too complex to formalise. Then
information theory came along, and the progress in the fields of
communication and computing followed.

Can networks be formalised?

Take social networks, for example. Influence is one of the
characteristic notions used to describe social relations. How can we
come up with the precise definition of influence and derive implied
properties?

All networks seem to have distinct similarities: there exist agents
making decisions according to their decision algorithms (utility
functions) and there are flows of objects (information, goods or
funds).

So we can form the following hypothesis: networks solve an
optimisation problem in a distributed manner by deciding on
information/good/funds to offer and consume, there are links that are
being created on which flow of information/goods/funds occurs. In this
way, communities, influence, and reputation emerge as \textit{solution
  concepts}: information/communities are easy to find, information
spreads fast and efficiently, decision-making on which links ot create
is easy and accurate.

In information networks, agents produce and consume content. Agent
have different interests in the content they want to consume, and
there are different abilities to produce content. There is a cost for
consuming content (agent independent), and there are rewards for
getting interesting content (agent-dependent). Thus, the model of such a network must include:

\begin{itemize}
\item interests of agents (ring $R$ with the ring metric $\norm{x_1-x_2}$ for $x_1, x_2\in R$)
\item distance between interests
\item association between agents and interests (content is characterised by its type $x\in R$)
\item interaction between agents (either agents decide to interact
  selfishly on their own, or there are other factors involved)
\begin{itemize}
\item Which content to produce?
\item Which agents to connect to?
\item What content to forward on each link?
\end{itemize}
\end{itemize}

This model predicts that communities emerge as a Nash equilibrium.  Producers/consumers in a community have similar interests. Agents specialise on producing a unique content. Produced/consumed content in a community is focused/concentrated on \textit{main} interest of a community. There is a lso a differentiation between the content produced in different communities. There is a dichotomy between homophily and influence. In this way, reputation and influence become \textit{optimality concepts}.

\subsection{Future Research}

There are several questions crying for an answer:
\begin{itemize}
\item How dos information spread in a community?
\item How do communities form?
\item Why are communities easy to find?
\item How do hierarchies form?
\end{itemize}
\end{document}