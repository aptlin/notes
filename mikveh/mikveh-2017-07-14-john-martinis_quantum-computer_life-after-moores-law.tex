% Created 2017-07-14 Fri 22:59
% Intended LaTeX compiler: pdflatex
\documentclass[11pt]{article}
\usepackage[utf8]{inputenc}
\usepackage{lmodern}
\usepackage[T1]{fontenc}
\usepackage{fixltx2e}
\usepackage{graphicx}
\usepackage{longtable}
\usepackage{float}
\usepackage{wrapfig}
\usepackage{rotating}
\usepackage[normalem]{ulem}
\usepackage{amsmath}
\usepackage{textcomp}
\usepackage{marvosym}
\usepackage{wasysym}
\usepackage{amssymb}
\usepackage{amsmath}
\usepackage[numbers,super,sort&compress]{natbib}
\usepackage{natmove}
\usepackage{url}
\usepackage{attachfile}
\author{John Martinis (UCSB \& Google)}
\date{\today}
\title{Quantum Computing: Life after Moore's Law}
\begin{document}
\maketitle
\label{sec:org950691b}
\begin{itemize}
\item There are no guarantees that we can build a quantum computer
\item Competition is against the unbelievably large computing power of already existing technologies
\item Deep learning requires a lot -- how can we keep up with it?
\item Crash Course into Quantum Mechanics
\begin{itemize}
\item The direct consequence of quantum mechanics is the fact that objects have size
\item Seeming randomness of quantum mechanics is not a good way to look at quantum mechanics
\item There are excited atom states stable enough to allow computation
\item Instead of conducting computations over the separate values of 0 and 1, superposition of quantum states allows the computation over a range of \emph{fuzzy} values.
\item The processing power of a quantum computer grows exponentially with the number of qubits
\item If there are 300 qubits, the amount of parallelism is bigger than the number of atoms in the universe
\end{itemize}
\item Qubit Systems
\begin{itemize}
\item Light is much bigger than an average molecule -- how can we control it?
\item We can make special molecules, of size comparable to the wavelength of light
\item Another way is to build electrical circuits that can behave as quantum mechanical objects, which are easier to control due to the bigger size
\item Most of these circuits are made of aluminium alloys
\item One of the aspects of the strange behavior these circuits show is the bidirectionality of currents in the same conductor
\item This technology is hard to make work
\end{itemize}
\item Qubit Operations
\begin{itemize}
\item A qubit oscillates and changes its state
\item A qubit is projected into the state, and if the conditions are right, we can utilise its probability distribution
\item The oscillations of qubits can be used to construct gates (it is relatively easy to obtain a not-gate, for example)
\end{itemize}
\item Practical Challenges
\begin{itemize}
\item The most difficult part of building a quantum computer nowadays is in cryogenics
\item The cycle of designing and producing a qubit system takes less than 4 weeks
\end{itemize}
\item Achieving Supremacy
\begin{itemize}
\item Can be viewed as a big guy vs a little guy struggle -- verifying the operation of a 49-qubit computer designed at Google would require the most powerful supercomputer
\item The plan is to verify the execution of a random algorithm
\item Supremacy is possibble with some margin
\item From 3 qubits to 9 qubits, the error of prediction per cycle ratio grows from 0.6\% to 2.9\%
\end{itemize}
\item Theory of Quantum Matrials
\begin{itemize}
\item A 9-qubit device can be easily modelled with existing technology
\item Quantum simulators allow the computational modelling of experiments which cannot be usually conducted in the lab -- for example, theoretical prediction of the effect of 10000 T magnetic field is feasible
\end{itemize}
\item Toward a Useful Quantum Computer
\begin{itemize}
\item Feynman have proposed a quantum simulator to study quantum chemistry
\item The advances in quantum computing would help us design more efficient methods of producing ammonia
\item There is a huge progress in algorithms from 2005 to 2017, with the complexity decreasing from \(O(poly(N))\) to \(O(N)\)
\item See the research of Ryan Babbush, McClean, Wecker, Poulin, Hastings, Toloni, Perruzo, Seeley, Whitfield, Aspuru-Guzik
\end{itemize}
\item Is Useful QC now possible?
\begin{itemize}
\item We do not know yet.
\end{itemize}
\item People
\begin{itemize}
\item Mikhail Lukin (Harvard, RQC)
\end{itemize}
\end{itemize}
\end{document}
