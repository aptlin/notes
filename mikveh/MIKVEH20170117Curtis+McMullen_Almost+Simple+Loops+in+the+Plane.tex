% -*- coding: utf-8; -*-
%%% Local Variables:
%%% mode: latex
%%% TeX-engine: xetex
%%% TeX-master: t
%%% End:
\documentclass[11pt]{scrartcl}
\usepackage[fancy, beaue, pset, anon]{sdll}
\pSet{\nt{Curtis McMullen}{}{Simple Loops in the Plane}}
\usepackage{lineno}
% ----------------------------------------------------------------------
% Page setup
% ----------------------------------------------------------------------

\pagenumbering{gobble}

% ----------------------------------------------------------------------
% Custom commands
% ----------------------------------------------------------------------

% alignment

\newcommand*{\LongestHence}{$\Rightarrow$}% function name
\newcommand*{\LongestName}{$f_o(-x)+f_e(-x)$}% function name
\newcommand*{\LongestValue}{$(-a)x +(-a)(-y)$}% function value
\newcommand*{\LongestText}{\defi}%

\newlength{\LargestHenceSize}%
\newlength{\LargestNameSize}%
\newlength{\LargestValueSize}%
\newlength{\LargestTextSize}%

\settowidth{\LargestHenceSize}{\LongestHence}%
\settowidth{\LargestNameSize}{\LongestName}%
\settowidth{\LargestValueSize}{\LongestValue}%
\settowidth{\LargestTextSize}{\LongestText}%

% Choose alignment of the various elements here: [r], [l] or [c]

\newcommand*{\mbh}[1]{{\makebox[\LargestHenceSize][r]{\ensuremath{#1}}}}%
\newcommand*{\mbn}[1]{{\makebox[\LargestNameSize][r]{\ensuremath{#1}}}}%
\newcommand*{\mbv}[1]{\ensuremath{\makebox[\LargestValueSize][r]{\ensuremath{#1}}}}%
\newcommand*{\mbt}[1]{\makebox[\LargestTextSize][l]{#1}}%

\newcommand{\R}[1]{\label{#1}\linelabel{#1}}
\newcommand{\lr}[1]{line~\lineref{#1}}

% ----------------------------------------------------------------------
% Launch!
% ----------------------------------------------------------------------

\begin{document}

\section{Simple Curves}

\begin{itemize}
\item Α simple loop with exponentially increasing complexity can be
  generated by introducing three pins on the plane and then
  \textit{blender} them by interchanging continuously two points (the
  second with the third, then the new first with the second, etc). Note that
\item Space of simple closed curves on the surface is the space of
  integral points in the larger space isomorphic to $\RR ^{log -
    6}$. Thus $\ZZ^{log -6} \ssq \RR ^{log -6}$

\item The number of geodesics grows like a polynomial (cf. Maryam Mirzakhani)
\end{itemize}

\section{Random Paths in the Plane}

\begin{itemize}
\item Take a very fine grid and follow a random walk, which gives a
  combinatorial path in a canonical way such that beyond some point
  the properties of the curve depend only on the underlying surface
\item Brownian motion and picking curves at random give the same
  asymptotic results.
\item Selberg trace formula can give, however, something more.
\item A random walk is Markov, while construction of simple curves
  requires avoidance of \textit{cornering}.
\item There is a way around it. Take an ordinary random walk on a lattice and, if the walk comes
  to the place where it was, make the loop and erase it.
\item cf. Sheffield and Viklund: Random Simple Curves

\end{itemize}

\section{Three Punctured Sphere}

The plane has no topology, however it is not an obstacle -- we can,
for example, puncture it. Consider the triply punctured Riemann sphere
with $0$, $1$ and $\infty$ removed. How does the hyperbolic geometry
of this object look like? There is a metric corresponding to it. What
do we want to study? We may study geodesics -- but there are no simple
loops on the triply punctured spheres!

We can change the rules of the game and still count the nonexisting
geodesics.

Consider non-simple closed geodesics. There are two invariants which
can be associated with them. One of them is the combinatorial length,
number of times it passes through the upper half-plane, which is half
the number of times the curve crosses $\RR$. Another one is the number
of times it crosses itself.

For any loop the self-intersection number is at least the
combinatorial length minus 1.

Define a \textit{defect} as the difference of the self-intersection
number and combinatorial length.

Almost simple loops with a fixed defect have quadratic growth.

The number of loops of a given type can be given using binomial
coefficients and by describing the motifs of a curve.

This formula can be generalized for a greater number of punctures, but
the sum in the formula becomes infinite.

To prove the theorem, geodesics must be described in the topological
and combinatorial terms. Choose a base point in either half-plane and
draw loops around the infinity, which defines a free groop generated
by three terms such that their product is identity.

\textbf{Key terms}: Coxeter group, orbifold fundamental group, mirrors,
generators, orientation preserving subgroup of the orientation group

Combinatorial itinerary can be described by words, which tie nicely
with the describing groups. This, in turns, can help us prove the
theorem on the defect bounds.
\end{document}