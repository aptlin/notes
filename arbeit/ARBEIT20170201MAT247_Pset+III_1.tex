%%% Local Variables:
%%% mode: latex
%%% TeX-master: t
%%% End:

\documentclass[11pt]{scrartcl}
\usepackage[beaue, pset, anon]{masty}
\pSet{\hw{MAT247}{III}{1}}
\usepackage{lineno}
% ----------------------------------------------------------------------
% Page setup
% ----------------------------------------------------------------------

\pagenumbering{gobble}

% ----------------------------------------------------------------------
% Custom commands
% ----------------------------------------------------------------------

% alignment

\newcommand*{\LongestHence}{$\Rightarrow$}% function name
\newcommand*{\LongestName}{$f_o(-x)+f_e(-x)$}% function name
\newcommand*{\LongestValue}{$(-a)x +(-a)(-y)$}% function value
\newcommand*{\LongestText}{\defi}%

\newlength{\LargestHenceSize}%
\newlength{\LargestNameSize}%
\newlength{\LargestValueSize}%
\newlength{\LargestTextSize}%

\settowidth{\LargestHenceSize}{\LongestHence}%
\settowidth{\LargestNameSize}{\LongestName}%
\settowidth{\LargestValueSize}{\LongestValue}%
\settowidth{\LargestTextSize}{\LongestText}%

% Choose alignment of the various elements here: [r], [l] or [c]

\newcommand*{\mbh}[1]{{\makebox[\LargestHenceSize][r]{\ensuremath{#1}}}}%
\newcommand*{\mbn}[1]{{\makebox[\LargestNameSize][r]{\ensuremath{#1}}}}%
\newcommand*{\mbv}[1]{\ensuremath{\makebox[\LargestValueSize][r]{\ensuremath{#1}}}}%
\newcommand*{\mbt}[1]{\makebox[\LargestTextSize][l]{#1}}%

\newcommand{\R}[1]{\label{#1}\linelabel{#1}}
\newcommand{\lr}[1]{line~\lineref{#1}}

% ----------------------------------------------------------------------
% Launch!
% ----------------------------------------------------------------------

\begin{document}
\section*{Problem I}

Consider an operation defined for $x= (x_1, x_2)$ and $y = (y_1, y_2)$, both in $\CC^2$, as

\begin{equation}
  \ipr{x}{y} =
  \begin{pmatrix}
    x_1 & x_2
  \end{pmatrix}
\begin{pmatrix}
  1 & 1+i\\
  1-i & 1
\end{pmatrix}
\begin{pmatrix}
  \ol{y_1}\\
  \ol{y_{2}}
\end{pmatrix}
\end{equation}

Note that $\ipr{x}{y}$ is not an inner product, since if
$x =
\begin{pmatrix}
-1 & i
\end{pmatrix}$, then

\begin{align}
  \ipr{x}{x}                       & =
               \begin{pmatrix}
                 -1                 & i
               \end{pmatrix}
                     \begin{pmatrix}
                       1           & 1+i   \\
                       1-i         & 1
                     \end{pmatrix}
                             \begin{pmatrix}
                               -1           \\
                               -i
                             \end{pmatrix} \\
                                   & =
                             \begin{pmatrix}
                               i & -1
                             \end{pmatrix}
                             \begin{pmatrix}
                               -1           \\
                               -i
                             \end{pmatrix} \\
                                   & = -i +i = 0,
\end{align}

and thus $\ipr{x}{x} = 0$ even though $x =
\begin{pmatrix}
-1 & i
\end{pmatrix}\neq \bm{0}
$.

Consider now an operation defined for $x= (x_1, x_2)$ and
$y = (y_1, y_2)$, both in $\CC^2$, as

\begin{equation}
  [x, y] =
  \begin{pmatrix}
    x_1 & x_2
  \end{pmatrix}
  \begin{pmatrix}
    4 & i\\
    -i & 1
  \end{pmatrix}
  \begin{pmatrix}
    \ol{y_1}\\
    \ol{y_{2}}
  \end{pmatrix}.
\end{equation}

By definition of the matrix summation and the distributive law for
matrices, for any $x, y, z \in \CC^2$, $[x+y, z] = [x, z] + [y, z]$,
and thus $[\*, \*]$ is additive.

Similarly, by definition of the matrix multiplication by a scalar,
$[\lambda x, y] = \lambda [x, y]$, and thus $[\*, \*]$ is homogeneous.

Observe the following:
\begin{align}
  [x, y]          & =
  \begin{pmatrix}
    4x_1 - ix_{2} & x_2+ix_1
  \end{pmatrix}
  \begin{pmatrix}
    \ol{y_1}    \\
    \ol{y_{2}}
  \end{pmatrix} \\
                  & = (4x_1 - ix_{2})\ol{y_1} +  (x_2+ix_1)\ol{y_{2}}
\end{align}

Therefore,
\begin{align}
  \ol{[y, x]} & = \ol{(4y_1 - iy_{2})\ol{x_1} +  (y_2+iy_1)\ol{x_{2}}}      \\
              & = (4\ol{y_1} + i \ol{y_2})x_1 + (\ol{y_2} - i\ol{y_1})x_{2} \\
              & = (4x_1 - ix_{2})\ol{y_1} +  (x_2+ix_1)\ol{y_{2}}           \\
              & = [x,y]
\end{align}

Consider $[x, x]$.
\begin{align}
  [x,x] & = (4x_1 - ix_{2})\ol{x_1} + (x_2+ix_1)\ol{x_{2}} \\
        & = 4 \abs{x_1}^2 + \abs{x_2}^2 + i(x_1\ol{x_2}-\ol{x_1}x_2)\\
        & = 4 \abs{x_1}^2 + \abs{x_2}^2 + i(x_1\ol{x_2}-\ol{x_1\ol{x_2}})\\
  \label{eq:1}
        & = 4 \abs{x_1}^2 + \abs{x_2}^2 - 2\Im(x_{1}\ol{x_2}).
\end{align}

Since
$\Im(x_{1}\ol{x_2}) = \Re(x_1)\Im(\ol{x_2})+\Im(x_1)\Re(\ol{x_2})$,
while
\[\abs{x_1}^2 = \Re(x_1)^2 + \Im(x_1)^2\] and
\[\abs{x_2}^2 = \Re(x_2)^2 + \Im(x_2)^2,\]
it follows from (\ref{eq:1}) that

\begin{align}
  [x,x] & = 4\Re(x_1)^2 + 4\Im(x_1)^2 + \Re(x_2)^2 + \Im(x_2)^2 - 2 (\Re(x_1)\Im(\ol{x_2})+\Im(x_1)\Re(\ol{x_2})) \\
  \label{eq:2}
        & = 3(\Re(x_1)^2 + \Im(x_1)^2) + (\Re(x_1) + \Im(x_2))^2 +(\Im(x_1)-\Re(x_2))^{2} \geq 0
\end{align}

Note that, from (\ref{eq:2}), $[x, x] = 0$ if and only if
$x_1 = \bm{0}$ and $x_2= \bm{0}$, because $\Re(x_1) = 0$ and
$\Im(x_1) = 0$ from the first expression in parentheses, and the rest
follows from the second and third expressions in parentheses.

Therefore, $[\*, \*]$ is indeed an inner product.
\end{document}