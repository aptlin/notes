
%%% Local Variables:
%%% mode: latex
%%% TeX-master: t
%%% End:

\documentclass[11pt]{scrartcl}
\usepackage[beaue, pset, anon]{masty}
\pSet{\hw{MAT247}{VI}{2}}
\usepackage{lineno}
% ----------------------------------------------------------------------
% Page setup
% ----------------------------------------------------------------------

\pagenumbering{gobble}

% ----------------------------------------------------------------------
% Custom commands
% ----------------------------------------------------------------------

% alignment

\newcommand*{\LongestHence}{$\Rightarrow$}% function name
\newcommand*{\LongestName}{$f_o(-x)+f_e(-x)$}% function name
\newcommand*{\LongestValue}{$(-a)x +(-a)(-y)$}% function value
\newcommand*{\LongestText}{\defi}%

\newlength{\LargestHenceSize}%
\newlength{\LargestNameSize}%
\newlength{\LargestValueSize}%
\newlength{\LargestTextSize}%

\settowidth{\LargestHenceSize}{\LongestHence}%
\settowidth{\LargestNameSize}{\LongestName}%
\settowidth{\LargestValueSize}{\LongestValue}%
\settowidth{\LargestTextSize}{\LongestText}%

% Choose alignment of the various elements here: [r], [l] or [c]

\newcommand*{\mbh}[1]{{\makebox[\LargestHenceSize][r]{\ensuremath{#1}}}}%
\newcommand*{\mbn}[1]{{\makebox[\LargestNameSize][r]{\ensuremath{#1}}}}%
\newcommand*{\mbv}[1]{\ensuremath{\makebox[\LargestValueSize][r]{\ensuremath{#1}}}}%
\newcommand*{\mbt}[1]{\makebox[\LargestTextSize][l]{#1}}%

\newcommand{\R}[1]{\label{#1}\linelabel{#1}}
\newcommand{\lr}[1]{line~\lineref{#1}}

% ----------------------------------------------------------------------
% Launch!
% ----------------------------------------------------------------------

\begin{document}
\section{Problem II}
Recall that two matrices $A,B \in M_{n \times n}(\FF)$ are unitarily/orthogonally equivalent if there exists a unitary/orthogonal matrix $Q$ such that $A = Q^{-1} B Q$ (or equivalently, $A = Q^* B Q$). Let us write $A \sim B$ if this is the case.

Suppose now that $A,B,C \in M_{n \times n}(\FF)$.

\begin{lemma}
$A\sim A$
\end{lemma}

\begin{proof}
  \hfill

  Since for all $x\in V$ $\norm{Ix} = \norm{x}$, $I$ is
  unitary/orthogonal. Moreover, $I^{-1} = I$, and hence
  $A = I^{-1}A I$, which means that $A\sim A$.
\end{proof}



\begin{lemma}
  \label{sec:problem-ii}
If $Q$ is unitary/orthogonal, then $Q^{*}$ is also unitary/orthogonal.
\end{lemma}

\begin{proof}
  \hfill

  For any $x\in V$, $\ipr{Q^{*}x}{Q^{*}x} = \ipr{x}{QQ^{*}x}$, and
  since $QQ^{*}=Q^{*}Q = I$, then
  $\ipr{x}{QQ^{*}x} = \ipr{x}{Q^{*}Qx} = \ipr{Qx}{Qx}$.

  Because $Q$ is unitary/orthogonal, $\ipr{Q^{*}}{Q^{*}} = \ipr{Qx}{Qx} = x$.
\end{proof}
\begin{lemma}
 $A \sim B$ is equivalent to $B \sim A$.
\end{lemma}

\begin{proof}
  \hfill

  Suppose there exists a unitary/orthogonal matrix $Q$ such that
  $A = Q^{*} B Q$.

  Therefore, since $Q$ is invertible,
  $A Q^{*} = Q^{*}BQQ^{*} = Q^{*}B$.

  Moreover, $QAQ^{*} = QQ^{*}B = B$.

  Let $P = Q^{*}$. Therefore, $P^{*}= Q^{**} = Q$, and thus
  $B = P^{*} A P$.

  Note that $P$ is unitary/orthogonal from Lemma \ref{sec:problem-ii}.

  Relabelling $A$ as $B$ and $B$ as $A$, we obtain the conclusion in the other direction.
\end{proof}

\begin{lemma}
If $A \sim B$ and $B \sim C$, then $A \sim C$.
\end{lemma}

\begin{proof}
  \hfill

  Suppose there exists a unitary/orthogonal matrix $Q$ such
  that $A = Q^{*} B Q$, and suppose there exists a unitary/orthogonal
  matrix $P$ such that $B = P^{*} C P$.

  Therefore, $A = (Q^{*}P^{*})C (PQ)$.

  Note that $(PQ)^{*} = Q^{*}P^{*}$. Moreover, for any $x\in V$,
  $ \ipr{PQx}{PQx} = \ipr{Qx}{Qx}$, because $P$ is unitary/orthogonal,
  and since $Q$ is unitary/orthogonal, then
  $\ipr{Qx}{Qx} = \ipr{x}{x}$. Therefore,
  $\ipr{PQx}{PQx} = \ipr{x}{x}$, and thus $PQ$ is unitary/orthogonal.

  From Lemma \ref{sec:problem-ii}, $(PQ)^{*} = Q^{*}P^{*}$ is also
  unitary/orthogonal.

  If $R=PQ$, then $A = R^{*}C R$ and $R$ is
  unitary/orthogonal, and thus $A\sim C$.
\end{proof}

\end{document}