%%% Local Variables:
%%% mode: latex
%%% TeX-master: t
%%% End:

\documentclass[11pt]{scrartcl}
\usepackage[beaue, pset, anon]{masty}
\pSet{\hw{MAT247}{II}{3}}
\usepackage{lineno}
% ----------------------------------------------------------------------
% Page setup
% ----------------------------------------------------------------------

\pagenumbering{gobble}

% ----------------------------------------------------------------------
% Custom commands
% ----------------------------------------------------------------------

% alignment

\newcommand*{\LongestHence}{$\Rightarrow$}% function name
\newcommand*{\LongestName}{$f_o(-x)+f_e(-x)$}% function name
\newcommand*{\LongestValue}{$(-a)x +(-a)(-y)$}% function value
\newcommand*{\LongestText}{\defi}%

\newlength{\LargestHenceSize}%
\newlength{\LargestNameSize}%
\newlength{\LargestValueSize}%
\newlength{\LargestTextSize}%

\settowidth{\LargestHenceSize}{\LongestHence}%
\settowidth{\LargestNameSize}{\LongestName}%
\settowidth{\LargestValueSize}{\LongestValue}%
\settowidth{\LargestTextSize}{\LongestText}%

% Choose alignment of the various elements here: [r], [l] or [c]

\newcommand*{\mbh}[1]{{\makebox[\LargestHenceSize][r]{\ensuremath{#1}}}}%
\newcommand*{\mbn}[1]{{\makebox[\LargestNameSize][r]{\ensuremath{#1}}}}%
\newcommand*{\mbv}[1]{\ensuremath{\makebox[\LargestValueSize][r]{\ensuremath{#1}}}}%
\newcommand*{\mbt}[1]{\makebox[\LargestTextSize][l]{#1}}%

\newcommand{\R}[1]{\label{#1}\linelabel{#1}}
\newcommand{\lr}[1]{line~\lineref{#1}}

% ----------------------------------------------------------------------
% Launch!
% ----------------------------------------------------------------------

\begin{document}

% ----------------------------------------------------------------------
% Body
% ----------------------------------------------------------------------
Let $B =
\begin{pmatrix}
  1 & 2\\
  -3 & 4
\end{pmatrix}$
  \begin{enumerate}[label=\alph*)]
  \item

    Suppose $A = (a_{ij})$ for $i,j \in \set{1,2}$. Therefore,

    \begin{equation*}
      T=
      \begin{pmatrix}
        a_{11}+2a_{21}   & a_{12}+2a_{22} \\
        -3a_{11}+4a_{21} & -3a_{12}+4a_{44}
      \end{pmatrix}
    \end{equation*}


    Take $v = \cv{1;0} = \frac{4}{10}\cv{1;-3} + \frac{3}{10}\cv{2;4} $.

    Then $Tv = a_{11}\cv{1;-3} + a_{21}\cv{2;4} $.

    If, for any $\sigma\in \CC$, $a_{11}\neq \sigma\frac{4}{10}$ and
    $a_{21}\neq \sigma \frac{3}{10}$, then $v$ and $Tv$ are linearly
    independent, and thus $\spn{\set{v, Tv}}$ is 2-dimensional.
    % \begin{align}
    %   T^2v & = T \left(  a_{11}\cv{1;-3} + a_{21}\cv{2;4} \right) \\
    %        & = a_{11}T\cv{1;-3} + a_{21}T\cv{2;4}                 \\
    %        & =
    % \end{align}
    % and thus $W = \spn{\cv{1;0}, \cv{1;3}}$ is a $T$-cyclic subspace
    % of dimension 2, since $\cv{1;0}$ and $\cv{1;3}$ form a maximal
    % $T$-cyclical linearly independent set.

\item
\label{item:1}
Consider the characteristic polynomial of $B$,
$g(\lambda) = (1-\lambda)(4-\lambda) + 6$:

  \begin{equation*}
    g(\lambda) = \lambda^2-5\lambda +10
  \end{equation*}

  By Cayley-Hamilton Theorem, $g(B) = 0$.
\item

  Suppose $T \in \Hom(V,V)$ such that $T(A) = BA$. Suppose also $h$ is
  a polynomial such that $h(T) = 0$. Observe that $\det B = 10 > 0$,
  and thus $B$ is invertible.%  Note also that $h(AB) = 0$, since
  % $AB = B^{-1}(BA)B$ and thus $T$ is similar to $AB$.

  From \ref{item:1},
  \begin{equation}
    \label{eq:2}
    B^2-5B + 10I = \mathbf{0},
  \end{equation}
  and then
  \begin{equation}
    B^2A - 5 BA + 10A = \mathbf{0}.
  \end{equation}
  Thus,
  \begin{equation*}
    BT-5T = -10A,
  \end{equation*}
  and hence

  \begin{equation}
    \label{eq:3}
    A = \frac{-1}{10}(B-5I)T
  \end{equation}


  Moreover,
  \begin{equation}
    B^2A^2-5BA^2+10A^2=\mathbf{0}.
  \end{equation}
  Therefore,
  \begin{align}
    \label{eq:1}
    T^2=5TA-10A^2 = (5T-10A)A
  \end{align}


  which, when combined with the equation (\ref{eq:3}),
  implies that

  \begin{align}
    T^2 & = (5T +BT-5T)A \\
        & =BTA           \\
        & =B^{2}A^{2}
  \end{align}

  Therefore, $(BA)^2 = B^2A^2$, and hence $ABA = BAA$, which gives
  $TA = AT$. Thus, from (\ref{eq:1}),
  \begin{equation*}
    T^2-5AT+10A^2=0,
  \end{equation*}

  which means that $h(A)(t) = t^2I-5At+10A^{2}$.
\item
  \begin{claim*} Any $T$-cyclic subspace of $V$ has a dimension of
    at most 2.
  \end{claim*}
  \begin{proof}

    Consider $v\in V$. Let $W$ be a $T$-cyclic subspace generated by
    $v$.

    If $v$ is zero, then $T^n(v) = 0$ for all $n\in \NN$, and thus they
    are all linearly dependent, which means that $\dim W < 2$.

    Suppose now that $v$ is non-zero.

    Let $d \geq 1$ be the largest integer such that
    $v, T(v), \dots, T^{d-1}(v)$ are linearly independent.  The largest
    $d$ exists, since $\dim V$ is finite.

    Let $U = \spn(v, T(v), \dots, T^{d-1}(v))\suq W$.

    \begin{lemma*}
      $U$ is $T$-invariant.
    \end{lemma*}

    \begin{proof}
      \hfill

      Suppose $u = c_0v + c_1T(v)+\dots+c_{d-1}T^{d-1}(v)$ and
      $T(u) = 0$.

      Note that
      $T(c_0v + c_1T(v)+\dots+c_{d-1}T^{d-1}(v)) = c_0Tv + c_1T^2v
      +\dots + c_{d-1}T^dv = 0$.

      Since $d$ is the largest integer such that
      $v, T(v), \dots, T^{d-1}(v)$ are linearly independent, then
      $c_{d-1}$ is non-zero, and thus $T^d(w) \in U$.
    \end{proof}

    $U$ is $T$-invariant, and thus if $v\in U$, then $W\suq U$, because
    $W$ is the smallest $T$-invariant subspace containing $v$. By
    definition of $U$, $U \suq W$, and thus $U = W$.

    Therefore, $\dim W = d$. Note that since a maximally linearly
    independent set in $V$ has a dimension of 2, then $d\leq 2$.
  \end{proof}
\end{enumerate}
\end{document}