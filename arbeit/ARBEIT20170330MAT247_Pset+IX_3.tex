
%%% Local Variables:
%%% mode: latex
%%% TeX-master: t
%%% End:

\documentclass[11pt]{scrartcl}
\usepackage[beaue, pset, anon]{masty}
\pSet{\hw{MAT247}{IX}{3}}
\usepackage{lineno}
% ----------------------------------------------------------------------
% Page setup
% ----------------------------------------------------------------------

\pagenumbering{gobble}

% ----------------------------------------------------------------------
% Custom commands
% ----------------------------------------------------------------------

% alignment

\newcommand*{\LongestHence}{$\Rightarrow$}% function name
\newcommand*{\LongestName}{$f_o(-x)+f_e(-x)$}% function name
\newcommand*{\LongestValue}{$(-a)x +(-a)(-y)$}% function value
\newcommand*{\LongestText}{\defi}%

\newlength{\LargestHenceSize}%
\newlength{\LargestNameSize}%
\newlength{\LargestValueSize}%
\newlength{\LargestTextSize}%

\settowidth{\LargestHenceSize}{\LongestHence}%
\settowidth{\LargestNameSize}{\LongestName}%
\settowidth{\LargestValueSize}{\LongestValue}%
\settowidth{\LargestTextSize}{\LongestText}%

% Choose alignment of the various elements here: [r], [l] or [c]

\newcommand*{\mbh}[1]{{\makebox[\LargestHenceSize][r]{\ensuremath{#1}}}}%
\newcommand*{\mbn}[1]{{\makebox[\LargestNameSize][r]{\ensuremath{#1}}}}%
\newcommand*{\mbv}[1]{\ensuremath{\makebox[\LargestValueSize][r]{\ensuremath{#1}}}}%
\newcommand*{\mbt}[1]{\makebox[\LargestTextSize][l]{#1}}%

\newcommand{\R}[1]{\label{#1}\linelabel{#1}}
\newcommand{\lr}[1]{line~\lineref{#1}}

% ----------------------------------------------------------------------
% Launch!
% ----------------------------------------------------------------------
\allowdisplaybreaks
\begin{document}

\begin{problem*}
\hfill

Suppose that $\lambda\in F$ and that
$J = \left(\begin{matrix} \lambda & 1 \\ 0 &
    \lambda\end{matrix}\right)$.  

Prove by induction that
$J^n = \left(\begin{matrix} \lambda^n & n \lambda^{n-1} \\ 0 &
    \lambda^n\end{matrix}\right)$ for all $n \ge 1$.
\end{problem*}

\begin{proof}
  \hfill

  Let $P(n)=$\textquote{$
    J^n = \begin{pmatrix} 
      \lambda^n & n\lambda^{n-1} \\
      0         & \lambda^n
    \end{pmatrix}
$}.

In case $n=1$, the claim holds, since 
$    J = \begin{pmatrix} 
      \lambda^1 & 1\*\lambda^{1-1} \\
      0         & \lambda^1
    \end{pmatrix}
$. Thus, $P(1)$ is true.

Suppose now $P(k)$ holds for some $k\in\ZZ^+$.

Therefore, 
$
    J^k = \begin{pmatrix} 
      \lambda^k & k\lambda^{k-1} \\
      0         & \lambda^k
    \end{pmatrix}
$.

Note that, by inductive hyptothesis,
\begin{align}
J J^k                                         & = 
        \begin{pmatrix} 
      \lambda^k                               & k\lambda^{k-1}                     \\
      0                                       & \lambda^k
    \end{pmatrix}\left(\begin{matrix} \lambda & 1                                  \\ 0 & 
    \lambda\end{matrix}\right)                                                     \\
                                              & = 
  \begin{pmatrix}
    \lambda^k\lambda +k\lambda^{k-1}\* 0      & 1\*\lambda^k+k\lambda^{k-1}\lambda \\
    0\* \lambda+\lambda^k\* 0                 & 0\*1 + \lambda^k\lambda
  \end{pmatrix}\\
&=
  \begin{pmatrix}
\lambda^{k+1} & (k+1)\lambda^{k}\\
0 & \lambda^{k+1}
  \end{pmatrix},
\end{align}
which is exactly the claim in case $n=k+1$. Therefore, $P(k+1)$ holds.

Since the claim is also true in case $n=1$, then $\forall n \in \ZZ^+. P(n)$ is true by induction.
\end{proof}

\begin{problem*}
\hfill

Suppose
$A = \left(\begin{matrix} 0 & 1 & 1 \\ 2 & 1 & -1 \\ -6 & -5 & -3\end{matrix}\right)$
over $F = \mathbb Q$. 

Find an invertible matrix $Q$ such that $Q^{-1} A Q$ is in Jordan canonical form.
\end{problem*}

\begin{soln}
  \hfill

Note that $A-tI = 
\begin{pmatrix}
-t & 1 & 1\\
2 & 1-t & -1\\
-6 & -5 & -3 - t
\end{pmatrix}
$.

Expanding along the first row, we see that

\begin{align}
\det(A-tI) &= 
-t 
             \begin{pmatrix}
1-t & -1\\
-5 & -3-t
             \end{pmatrix}
-
     \begin{pmatrix}
2 & -1\\
-6 & -3-t
     \end{pmatrix}
+
     \begin{pmatrix}
2 & 1-t\\
-6 & -5
\end{pmatrix}\\
           &= -t \left(   (1-t) (-3-t) - 5\right)-\left( -6-2t-6 \right)+\left( -10+6-6t \right)\\
           &= t(1-t)(3+t)+5t+2(t+3)+ 6 -4 -6t\\
           &= -\left( (t^2-t)(t+3)-6-7t-6+6t+4 \right)\\
           &= - \left( t^3+2t^2-3t-8-t \right)\\
           &= - \left( t^3 + 2t^2 -4t -8\right)\\
           &=-\left(   t^ 2 (t+2) -4 (t+2)\right)\\
           &=-(t-2)(t+2)^2
\end{align}

Thus, $\lambda = 2$ and $\lambda= -2$ are the only eigenvalues.

Consider $A-2I$:

\begin{align}
A-2I  & =
 \begin{pmatrix}
   -2 & 1   & 1  \\
   2  & -1 & -1 \\
   -6 & -5  & -5
 \end{pmatrix}\\
\end{align}

Now we solve $(A-2I)| 0$:

\begin{align}
R_1\to R_1+R_2, R_3\to -\frac{1}{8}(R_{3}+3R_2)\ |      &\ras
        \begin{matreq}{ccc|c}
0 & 0 & 0 & 0\\
2 & -1 & -1 & 0\\
0& 1 & 1 & 0
        \end{matreq}\\
R_2\to \frac{1}{2}(R_2+R_3) \ &\ras
        \begin{matreq}{ccc|c}
0 & 0 & 0 & 0\\
1 & 0 & 0 & 0\\
0 & 1 & 1 & 0
        \end{matreq}
\end{align}

Therefore, $\cv{0;1;-1}$ spans $E_2$, and there is one column in the dot diagram
corresponding to $K_2$.

Consider $A+2I$:

\begin{align}
A+2I  & =
 \begin{pmatrix}
   2 & 1   & 1  \\
   2  & 3 & -1 \\
   -6 & -5  & -1
 \end{pmatrix}\\
\end{align}

Now we solve $(A+2I)| 0$:

\begin{align}
R_2\to \frac{1}{4}(R_1+R_2), R_3\to \frac{1}{2}(R_{3}+3R_1)\ |      &\ras
        \begin{matreq}{ccc|c}
   2 & 1   & 1 & 0 \\
   1  & 1 & 0  & 0\\
   0 & -1  & 1 & 0
        \end{matreq}\\
R_1\to (R_1-R_2) \ &\ras
        \begin{matreq}{ccc|c}
   0 & -1   & 1 & 0 \\
   1  & 1 & 0  & 0\\
   0 & -1  & 1 & 0
        \end{matreq}\\
R_3\to (R_3+R_1) \ &\ras
        \begin{matreq}{ccc|c}
   0 & -1   & 1 & 0 \\
   1  & 1 & 0  & 0\\
   0 & 0  & 0 & 0
        \end{matreq}
\end{align}

Therefore, $\cv{1;-1;-1}$ spans $E_{-2}$, and there is one column in the dot diagram
corresponding to $K_{-2}$. 

Since the algebraic multiplicity of $-2$ is $2$ and it is equal to
$\dim K_{-2}$, we know there must be a cycle of length two containing
an element $v\in V$ such that $(A+2I)v= \cv{1;-1;-1}$ and $v$ is a
generalised eigenvector in $K_{-2}$.

We now solve $A+2I\ | \cv{1;-1;-1}$:

\begin{align}
A+2I\  | \cv{1;-1;-1}  & =
  \begin{matreq}{ccc|c}
   2 & 1   & 1  & 1\\
   2  & 3 & -1 & -1\\
   -6 & -5  & -1 & -1
  \end{matreq}\\
R_2\to \frac{1}{4}(R_1+R_2), R_3\to \frac{1}{2}(R_{3}+3R_1)\ &\ras
  \begin{matreq}{ccc|c}
   2 & 1   & 1  & 1\\
   1  & 1 & 0 & 0\\
   0 & -1  & 1 & 1
  \end{matreq}\\
R_1\to \frac{1}{2}(R_1-R_3) \ &\ras
        \begin{matreq}{ccc|c}
   1 & 1   & 0  & 0\\
   1  & 1 & 0 & 0\\
   0 & -1  & 1 & 1
        \end{matreq}\\
R_1\to \frac{1}{2}(R_1-R_2) \ &\ras
        \begin{matreq}{ccc|c}
   1 & 1   & 0  & 0\\
   0  & 0 & 0 & 0\\
   0 & -1  & 1 & 1
        \end{matreq}
\end{align}

Therefore, if $v = \cv{x;y;z}$, then $x+y = 0$ and $-y+z = 1$.

Therefore, if $x= \tau$, then $v = \cv{\tau;-\tau;1-\tau} = \cv{0;0;1}+\tau\cv{1;-1;-1}$.

Now consider $(A+2I)^2$:

\begin{align}
(A+2I)^2 &= 
 \begin{pmatrix}
   2 & 1   & 1  \\
   2  & 3 & -1 \\
   -6 & -5  & -1
 \end{pmatrix}
 \begin{pmatrix}
   2 & 1   & 1  \\
   2  & 3 & -1 \\
   -6 & -5  & -1
 \end{pmatrix}\\
&= 
  \begin{pmatrix}
0 & 0 & 0\\
16 & 14 & 0\\
-20 & -16 & 0
  \end{pmatrix}
\end{align}

Note that $(A+2I)^2 \cv{0;0;1}= 0$, which means that $\cv{0;0;1}$ is a generalised eigenvector.

Since $\dim K_{-2} = 2$, $(T+2I)\cv{0;0;1} = \cv{1;-1;-1}$ and
$\beta = \set{\cv{0;0;1},\cv{1;-1;-1}}$ is linearly independent
(if there exist $a_1\in \FF$, $a_2\in\FF$ such that 

\begin{equation*}
a_1\cv{0;0;1}+a_2\cv{1;-1;-1} = 0,
\end{equation*}

then from the first row $a_2= 0$ and from the third $a_1=a_2= 0$), then $\beta$ spans $K_{-2}$.

Since the characteristic polynomial of $A$ splits, then
$V = K_2\oplus K_{-2}$.

Let $\gamma = \set{\cv{0;1;-1}, \cv{1;-1;-1}, \cv{0;0;1}}$. Since
$\set{\cv{0;1;-1}}$ is a basis of $K_2$, while $\beta$ is a basis of
$K_{-2}$, since $V = K_2\oplus K_{-2}$, we have that $\gamma$ is a
cycle basis of $V$.

Therefore, $[A]_{\gamma}$ is in Jordan Canonical Form, and therefore,
by the change-of-basis formula, $Q = 
\begin{pmatrix}
0 & 1 & 0\\
1 & -1 & 0\\
-1 & -1 & 1
\end{pmatrix}$.

Expanding along the first row, we obtain that $\det [A]_{\gamma} = -\det 
\begin{pmatrix}
1 & 0\\
-1  & 1
\end{pmatrix} = -1
$. 

Therefore, $Q$ is invertible.

We find $Q^{-1}$ by row-reduction:

\begin{align}
  \begin{matreq}{ccc|ccc}
0 & 1 & 0 & 1 & 0 & 0\\
1 & -1 & 0 & 0 & 1 & 0\\
-1 & -1 & 1 & 0 & 0 & 1
  \end{matreq} &\\
R_1\to R_1 + R_2\ |\ &\ras
  \begin{matreq}{ccc|ccc}
1 & 0 & 0 & 1 & 1 & 0\\
1 & -1 & 0 & 0 & 1 & 0\\
-1 & -1 & 1 & 0 & 0 & 1
  \end{matreq} &\\
\shortstack{$R_2\to -(-R_1 + R_2)$\\
$R_3\to R_{1}+R_{3}$}\ |\ &\ras
  \begin{matreq}{ccc|ccc}
1 & 0 & 0 & 1 & 1 & 0\\
0 & 1 & 0 & 1 & 0 & 0\\
0 & -1 & 1 & 1 & 1 & 1
  \end{matreq} &\\
R_2\to R_2 + R_3\ |\ &\ras
  \begin{matreq}{ccc|ccc}
1 & 0 & 0 & 1 & 1 & 0\\
0 & 1 & 0 & 1 & 0 & 0\\
0 & 0 & 1 & 2 & 1 & 1
  \end{matreq} &\\
\end{align}

Therefore, $Q^{-1} = 
\begin{pmatrix}
1 & 1 & 0\\
1 & 0 & 0\\
2 & 1 & 1 
\end{pmatrix}$.
\end{soln}

\begin{problem*}
  \hfill

Using the previous parts, compute $A^n$ for any $n \ge 1$. 
\end{problem*}

\begin{soln}
  \hfill

  From the previous discussion of the dot diagram, the Jordan
  Canonical Form for a basis $\gamma$ is as follows:

  \begin{align}
[A]_{\gamma} &= 
               \begin{pmatrix}
                 2 & 0 & 0\\
                 0 & -2 & 1\\
                 0 & 0 & -2
               \end{pmatrix}
  \end{align}


  Let $P = 
\begin{pmatrix}
2 & 0 & 0\\
0 & 0 & 0\\
0 & 0 & 0
  \end{pmatrix}$ and $R=
  \begin{pmatrix}
0 & 0 & 0\\
0 & -2 & 1\\
0 & 0 & -2
\end{pmatrix}$.

Note that $P^n = 
\begin{pmatrix}
2^n & 0 & 0\\
0 & 0 & 0\\
0 & 0 & 0
\end{pmatrix}
$, because the only nonzero entry is obtained by multiplying the first
row with the first column.

Moreover, $PR = 
\begin{pmatrix}
2 & 0 & 0\\
0 & 0 & 0\\
0 & 0 & 0
  \end{pmatrix}
  \begin{pmatrix}
0 & 0 & 0\\
0 & -2 & 1\\
0 & 0 & -2
\end{pmatrix} = 0
$ 

and $RP = 
  \begin{pmatrix}
0 & 0 & 0\\
0 & -2 & 1\\
0 & 0 & -2
\end{pmatrix}
\begin{pmatrix}
2 & 0 & 0\\
0 & 0 & 0\\
0 & 0 & 0
  \end{pmatrix}
= 0$. Therefore, $P$ and $R$ commute.

Since $[A]_{\gamma} = P+R$, then $A^n = P^n +  \sum_{i=1}^{n-1}(P^{n-i}R^{i}) + R^{n}$ by Binomial Theorem, which is applicable since $P$ and $R$ commute.

The middle sum is equal to $0$, because $PR = RP = 0$. 

We now calculate $R^n$.

We claim that  $R^n = 
\begin{pmatrix}
0 & 0 & 0\\
0 & (-2)^{n} & n(-2)^{n-1}\\
0 & 0 &(-2)^n
\end{pmatrix}$.

The claim holds in case $n = 1$.

Suppose that the claim holds in case $n = k$.

Then $R^k = 
\begin{pmatrix}
0 & 0 & 0\\
0 & (-2)^{k} & k(-2)^{k-1}\\
0 & 0 &(-2)^k
\end{pmatrix}$ and thus

\begin{align}
R^{k+1} = R^kR &= 
\begin{pmatrix}
0 & 0 & 0\\
0 & (-2)^{k} & k(-2)^{k-1}\\
0 & 0 &(-2)^k
\end{pmatrix}
  \begin{pmatrix}
0 & 0 & 0\\
0 & -2 & 1\\
0 & 0 & -2
\end{pmatrix}\\
& = 
  \begin{pmatrix}
0 & 0 & 0\\
0 & (-2)^{k+1} & (-2)^k+k(-2)^{k}\\
0 & 0 & (-2)^{k+1}
  \end{pmatrix}\\
&=
  \begin{pmatrix}
0 & 0 & 0\\
0 & (-2)^{k+1} & k+1(-2)^{k+1 - 1}\\
0 & 0 &(-2)^k+1
  \end{pmatrix},
\end{align}

which is exactly the claim in case $n = k+1$, similarly to the result in the problem 1.

Let $m_{21} = \cv{0;0}$, $m_{12}= (0,0)$. Then $R = \begin{pmatrix}
0 & m_{12}\\
m_{21} & J
\end{pmatrix}$ in the notation of the first problem.

From the discussion above, 

$R^n =
 \begin{pmatrix}
0 & m_{12}\\
m_{21} & J^{n}
\end{pmatrix}$ by induction.

Therefore, $[A]_{\gamma}^n = 
\begin{pmatrix}
2^n & 0 & 0\\
0 & (-2)^n & n(-2)^{n-1}\\
0 & 0 & (-2)^n
\end{pmatrix}$ and thus, since by the change-of-basis formula  we have $[A]_{\gamma} = Q^{-1}A Q$ , and thus
\begin{equation*}
A^{n} = \begin{pmatrix}
0 & 1 & 0\\
1 & -1 & 0\\
-1 & -1 & 1
\end{pmatrix}
\begin{pmatrix}
2^n & 0 & 0\\
0 & (-2)^n & n(-2)^{n-1}\\
0 & 0 & (-2)^n
\end{pmatrix}
\begin{pmatrix}
1 & 1 & 0\\
1 & 0 & 0\\
2 & 1 & 1 
\end{pmatrix}.
\end{equation*}
\end{soln}

\end{document}
