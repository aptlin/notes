
%%% Local Variables:
%%% mode: latex
%%% TeX-master: t
%%% End:

\documentclass[11pt]{scrartcl}
\usepackage[beaue, pset, anon]{masty}
\pSet{\hw{MAT247}{IV}{1}}
\usepackage{lineno}
% ----------------------------------------------------------------------
% Page setup
% ----------------------------------------------------------------------

\pagenumbering{gobble}

% ----------------------------------------------------------------------
% Custom commands
% ----------------------------------------------------------------------

% alignment

\newcommand*{\LongestHence}{$\Rightarrow$}% function name
\newcommand*{\LongestName}{$f_o(-x)+f_e(-x)$}% function name
\newcommand*{\LongestValue}{$(-a)x +(-a)(-y)$}% function value
\newcommand*{\LongestText}{\defi}%

\newlength{\LargestHenceSize}%
\newlength{\LargestNameSize}%
\newlength{\LargestValueSize}%
\newlength{\LargestTextSize}%

\settowidth{\LargestHenceSize}{\LongestHence}%
\settowidth{\LargestNameSize}{\LongestName}%
\settowidth{\LargestValueSize}{\LongestValue}%
\settowidth{\LargestTextSize}{\LongestText}%

% Choose alignment of the various elements here: [r], [l] or [c]

\newcommand*{\mbh}[1]{{\makebox[\LargestHenceSize][r]{\ensuremath{#1}}}}%
\newcommand*{\mbn}[1]{{\makebox[\LargestNameSize][r]{\ensuremath{#1}}}}%
\newcommand*{\mbv}[1]{\ensuremath{\makebox[\LargestValueSize][r]{\ensuremath{#1}}}}%
\newcommand*{\mbt}[1]{\makebox[\LargestTextSize][l]{#1}}%

\newcommand{\R}[1]{\label{#1}\linelabel{#1}}
\newcommand{\lr}[1]{line~\lineref{#1}}
\usepackage{units}
% ----------------------------------------------------------------------
% Launch!
% ----------------------------------------------------------------------

\begin{document}
  Consider $V=\CC^3$ with the standard inner product. Let $W$ be the
  subspace of $V$ spanned by $(2, i, 0)$ and $(0, 1, i)$.
\begin{problem*}
  Find an orthogonal basis $\beta$ of $W$.
\end{problem*}
\begin{soln}
  \hfill
    \begin{center}
    \renewcommand{\arraystretch}{1.5}
    \begin{tabular}{>{\centering\bfseries}m{0.1in}| >{\centering}m{0.5in}| >{\centering}m{1in}| >{\centering}m{1in}|  >{\centering\arraybackslash}m{1in}}%{|c|c|c|c|c|c|}
      \hline
      $i$                                                                                         & $v_i$       & $\sum_{j=1}^{i-1} \frac{ \ipr{v_i}{u_j}}{\ipr{u_j}{u_j}}u_j$ & $u_i$        & $\norm{u_i}^2$ \\
      \hline
      1                                                                                           & $(2, i, 0)$ & --                                                           & $ (2, i, 0)$ & $5$            \\
      \hline
      2                                                                                           & $(0, 1, i)$ & \shortstack[l]{ $\frac{2\*0+(-i)\*1+0\*i }{5}u_1 $                                           \\$=(-\nicefrac{2}{5}\ 
      i, \nicefrac{1}{5}
      , 0)$ }
                                                                                                  & $(0, 1, i)-(-\nicefrac{2}{5}
                                                                                                    \ i, \nicefrac{1}{5}
                                                                                                    , 0) = (\nicefrac{2}{5}
                                                                                                    \ i, \nicefrac{4}{5}
                                                                                                    , i)$ & $\nicefrac{9}{5}
                                                                                                            $ \\
      \hline
    \end{tabular}
  \end{center}
    
  Therefore,
  \begin{equation*}
    \beta = \set{(2,i, 0),(\nicefrac{2}{5}\ i,\nicefrac{4}{5}, i)}
  \end{equation*}
\end{soln}

\begin{problem*}
Find an orthogonal basis of $V$ that contains the basis $\beta$.
\end{problem*}

\begin{soln}
  Suppose $(1, 0, 0)\in\spn{\beta}$. Therefore, there exist
  $a, b\in\CC$ such that

  \begin{align}
a (2,i, 0) + b (\nicefrac{2}{5}\ i,\nicefrac{4}{5}, i) = (1, 0, 0)
  \end{align}

  Thus, $a\*0 + bi =0$ and hence $b = 0$. But then $ai = 0$, and thus
  $a=0$. Therefore, $(1, 0, 0)$ is not in the span of $\beta$.

  Let $v_3=(1, 0, 0)$. Observe that $\beta \cup \set{v_3}$ is linearly
  independent.

  Using the values in the table and applying the Gram-Shmidt procedure, we obtain

  
  \begin{align*}
    u_3 & = v_3 - \frac{\ipr{v_3}{u_1}}{\norm{u_{1}}^{2}}u_1 - \frac{\ipr{v_{3}}{u_2}}{\norm{u_{2}}^2}u_2                         \\
        & = (1, 0, 0) - \frac{2}{5}(2, i, 0) - \frac{- \nicefrac{2}{5}i }{\nicefrac{9}{5}}(\nicefrac{2}{5}\ i,\nicefrac{4}{5}, i) \\
        & =(\nicefrac{1}{5}, -\nicefrac{2}{5}i, 0) + \frac{2}{9}i(\nicefrac{2}{5}\ i,\nicefrac{4}{5}, i)                          \\
        & =(\nicefrac{1}{9}, -\nicefrac{2}{9}i, -\nicefrac{2}{9})
  \end{align*}

  Therefore, $\gamma = \beta \cup \set{(\nicefrac{1}{9}, -\nicefrac{2}{9}i, -\nicefrac{2}{9})}$ is an orthogonal linearly independent set of length 3 which contains $\beta$.

  Thus, since $\gamma$ has the right length, it is a basis of $\CC^3$.
\end{soln}




\end{document}