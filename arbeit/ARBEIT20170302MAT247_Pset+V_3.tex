
%%% Local Variables:
%%% mode: latex
%%% TeX-master: t
%%% End:

\documentclass[11pt]{scrartcl}
\usepackage[beaue, pset, anon]{masty}
\pSet{\hw{MAT247}{V}{3}}
\usepackage{lineno}
% ----------------------------------------------------------------------
% Page setup
% ----------------------------------------------------------------------

\pagenumbering{gobble}

% ----------------------------------------------------------------------
% Custom commands
% ----------------------------------------------------------------------

% alignment

\newcommand*{\LongestHence}{$\Rightarrow$}% function name
\newcommand*{\LongestName}{$f_o(-x)+f_e(-x)$}% function name
\newcommand*{\LongestValue}{$(-a)x +(-a)(-y)$}% function value
\newcommand*{\LongestText}{\defi}%

\newlength{\LargestHenceSize}%
\newlength{\LargestNameSize}%
\newlength{\LargestValueSize}%
\newlength{\LargestTextSize}%

\settowidth{\LargestHenceSize}{\LongestHence}%
\settowidth{\LargestNameSize}{\LongestName}%
\settowidth{\LargestValueSize}{\LongestValue}%
\settowidth{\LargestTextSize}{\LongestText}%

% Choose alignment of the various elements here: [r], [l] or [c]

\newcommand*{\mbh}[1]{{\makebox[\LargestHenceSize][r]{\ensuremath{#1}}}}%
\newcommand*{\mbn}[1]{{\makebox[\LargestNameSize][r]{\ensuremath{#1}}}}%
\newcommand*{\mbv}[1]{\ensuremath{\makebox[\LargestValueSize][r]{\ensuremath{#1}}}}%
\newcommand*{\mbt}[1]{\makebox[\LargestTextSize][l]{#1}}%

\newcommand{\R}[1]{\label{#1}\linelabel{#1}}
\newcommand{\lr}[1]{line~\lineref{#1}}

% ----------------------------------------------------------------------
% Launch!
% ----------------------------------------------------------------------

\begin{document}

Let $T \in \Hom(V,V)$, where $V$ is a finite-dimensional inner product
space over $\FF$. Define $T$ as positive semi-definite if $T=T^{*}$ and
$\ipr{T(x)}{x}\geq 0$ for all $x\in V$.

\begin{lemma}
  \label{sec:1}
  If $T = T^{*}$, then $T$ is positive semi-definite if and only if
  all eigenvalues of $T$ are in $\RR^+\cup\set{0}$.
\end{lemma}

\begin{proof}
  \hfill

  Assume $T = T^{*}$.

  Suppose first $T$ is positive semi-definite, i.e. $\ipr{T(x)}{x}\geq0$ for all $x\in V$.

  Let $\lambda$ be an arbitrary eigenvalue of $T$. Thus, for a corresponding eigenvector $x\in V$, $Tx = \lambda x$.

  Since $T$ is self-adjoint, then all the eigenvalues are real.

  Since $T$ is positive semi-definite, then
  $\ipr{\lambda x}{x} = \lambda \ipr{x}{x} \geq 0$.

  Thus, since $x$ is an eigenvector, $x\neq 0$, then $\ipr{x}{x}>0$
  and hence $\lambda \geq 0$.
  Suppose now any eigenvalue $\lambda$ is such that $\lambda\in \RR^+\cup\set{0}$.

  Since $T$ is self-adjoint and $\FF=\CC$ or $\FF = \RR$, then by the
  Spectral Theorem there is an orthonormal basis $\beta$ of $V$,
  consisting of eigenvectors of $T$. Let $v$ be an arbitrary vector of $\beta$.

  Note that $Tv = \lambda v$ for some  $\lambda\in \RR^+\cup\set{0}$ by assumption.

  Take an arbitrary $x\in V$. Since $\beta$ is a basis, there exist
  $a_1, a_2,\dots,a_n\in \FF$, where $n=\dim V$, such that, for
  $v_i\in \beta$ and $\lambda_i$ being a corresponding eigenvalue,
  \[v  = \sum_{i=1}^na_iv_i.
  \]
  

  Consider $\ipr{Tx}{x}$:
\allowdisplaybreaks
  \begin{align}
    \ipr{Tx}{x} & = \ipr{T(\sum_{i=1}^na_iv_i)}{\sum_{j=1}^na_jv_j}       \\
                & = \ipr{\sum_{i=1}^na_iT(v_i)}{\sum_{j=1}^na_jv_j}       \\
                & = \ipr{\sum_{i=1}^na_i\lambda_iv_i}{\sum_{j=1}^na_jv_j} \\
                & = \sum_{i=1}^na_i\lambda_i\ipr{v_i}{\sum_{j=1}^na_jv_j} \\
                & = \sum_{i=1}^na_i\lambda_i\sum_{j=1}^n\ol{a_j}\ipr{v_i}{v_j} \\
                & = \sum_{i=1}^na_i\lambda_i\sum_{j=1}^n\ol{a_j}\ipr{v_i}{v_j} \\
                & = \sum_{i=1}^na_i\lambda_i\sum_{j=1}^n\ol{a_j}\delta_{ij} \\
    \label{eq:1}
                & = \sum_{i=1}^n\abs{a^2_i}\lambda_i 
  \end{align}

  Since for all $i\in [1, n]\cap\NN$ $\lambda_i \geq 0$ by assumption and $\abs{a^2_i} \geq 0$ by definition of $\abs{\*}$, then $\ipr{Tx}{x}\geq 0$ for all $x\in V$ by generalisation of (\ref{eq:1}).
  % Thus, for any eigenvector $x$, $Tx = \lambda x$, and hence $\ipr{x}{(T-\lambda I)x}$
  % By definition of an adjoint and since $T=T^{*}$, $\ipr{Tx}{x} = \ipr{x}{Tx}$.
\end{proof}

\begin{lemma}
  \label{sec:2}
  $T$ is positive semi-definite if and only if there exists a linear transformation

  $S \in \Hom(V,V)$ such that $T=S^{*}S$.
\end{lemma}

\begin{proof}
  \hfill

  Suppose first $T$ is positive semi-definite.

  Thus, $T = T^{*}$ and $\ipr{T(x)}{x}\geq0$ for all $x\in V$.

  Since $T$ is self-adjoint, then by the Spectral Theorem there is an
  orthonormal basis $\beta = \set{v_1, \dots, v_n}$ of eigenvectors,
  with the corresponding eigenvalues $\lambda_1, \dots, \lambda_n$.

  Define a linear transformation $S$, $S \in \Hom(V,V)$, so that
  $Sv_i = \sqrt{\lambda_i}v_i$. Since any linear transformation is
  defined by its action on a basis, $S$ is well-defined.

  By Lemma \ref{sec:1}, any $\lambda_i$ is real and nonnegative. Hence, by
  definition of $S$, all eigenvalues of $S$ are real and
  nonnegative. Moreover, since $\beta$ is a basis and also a set of
  all eigenvectors of $S$, then, by Lemma \ref{sec:1} again, $T$ is
  self-adjoint. Therefore, by definition, $S$ is positive semi-definite.

  Note that for any $v_i\in\beta$ $S(Sv_i) = \lambda_iv_i = Tv_i$, by
  construction of $S$ and since $v_i$ is an eigenvector of $T$. Since
  $v_i\in\beta$, which is a basis of $V$, and $T$ and $S^2$ are
  defined uniquely by its action on a basis, then $S^2 = T$.

  Since $S$ is positive-definite, $S = S^{*}$, and thus $T = S^{*}S$.

  Suppose now there exists a linear transformation
  $S \in \Hom(V,V)$ such that $T=S^{*}S$.

  Therefore, $T^{*}=(S^{*}S)^{*} = S^{*}S = T$, and hence $T$ is
  self-adjoint. By the Spectral Theorem there is an orthonormal basis
  $\beta = \set{v_1, \dots, v_n}$ of $V$, consisting of eigenvectors
  of $T$, with the corresponding eigenvalues
  $\lambda_1, \dots, \lambda_n$.

  For any $v\in V$, consider $\ipr{Tv}{v}$:
  \begin{align}
    \ipr{Tv}{v} & = \ipr{S^{*}Sv}{v} \\
                & =\ipr{Sv}{Sv}\\
                &=\norm{Sv}\geq 0
  \end{align}

  Therefore, for all $v\in V$, $\ipr{Tv}{v}\geq 0$ and thus $T$ is
  positive semi-definite.
\end{proof}

\begin{ques}
  Is it true that $T$ is positive semi-definite if and only if there exists a linear transformation
  $S \in \Hom(V,V)$ such that $T=S^2$?
\end{ques}

\begin{answer*}

  From the proof of Lemma \ref{sec:2}, it is true that if $T$ is
  positive semi-definite, then there exists a linear transformation
  $S \in \Hom(V,V)$ such that $T=S^2$.

  However, suppose $T=
  \begin{pmatrix}
    1  & -2 \\
    0  & 1
  \end{pmatrix}
  $. It is easy to check that if $S = 
  \begin{pmatrix}
    -1 & 1  \\
    0  & -1
  \end{pmatrix}
  $, then $T = S^2$. However, since $T^{*}= 
  \begin{pmatrix}
    1  & 0  \\
    -2 & 1
  \end{pmatrix}\neq T$, then $T$ is not self-adjoint and hence not positive semi-definite.

\end{answer*}

\end{document}