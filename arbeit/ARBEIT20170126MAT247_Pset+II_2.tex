%%% Local Variables:
%%% mode: latex
%%% TeX-master: t
%%% End:

\documentclass[11pt]{scrartcl}
\usepackage[beaue, pset, anon]{masty}
\pSet{\hw{MAT247}{II}{2}}
\usepackage{lineno}
% ----------------------------------------------------------------------
% Page setup
% ----------------------------------------------------------------------

\pagenumbering{gobble}

% ----------------------------------------------------------------------
% Custom commands
% ----------------------------------------------------------------------

% alignment

\newcommand*{\LongestHence}{$\Rightarrow$}% function name
\newcommand*{\LongestName}{$f_o(-x)+f_e(-x)$}% function name
\newcommand*{\LongestValue}{$(-a)x +(-a)(-y)$}% function value
\newcommand*{\LongestText}{\defi}%

\newlength{\LargestHenceSize}%
\newlength{\LargestNameSize}%
\newlength{\LargestValueSize}%
\newlength{\LargestTextSize}%

\settowidth{\LargestHenceSize}{\LongestHence}%
\settowidth{\LargestNameSize}{\LongestName}%
\settowidth{\LargestValueSize}{\LongestValue}%
\settowidth{\LargestTextSize}{\LongestText}%

% Choose alignment of the various elements here: [r], [l] or [c]

\newcommand*{\mbh}[1]{{\makebox[\LargestHenceSize][r]{\ensuremath{#1}}}}%
\newcommand*{\mbn}[1]{{\makebox[\LargestNameSize][r]{\ensuremath{#1}}}}%
\newcommand*{\mbv}[1]{\ensuremath{\makebox[\LargestValueSize][r]{\ensuremath{#1}}}}%
\newcommand*{\mbt}[1]{\makebox[\LargestTextSize][l]{#1}}%

\newcommand{\R}[1]{\label{#1}\linelabel{#1}}
\newcommand{\lr}[1]{line~\lineref{#1}}

% ----------------------------------------------------------------------
% Launch!
% ----------------------------------------------------------------------

\begin{document}

% ----------------------------------------------------------------------
% Body
% ----------------------------------------------------------------------
\begin{linenumbers}
  Let $A\in M_{n\times n}(\FF)$. Recall that $A$ and its transpose
  $A^{t}$ have the same characteristic polynomial, hence have the same
  eigenvalues. For any eigenvalue $\lambda$, let $E_{\lambda}$ denote the $\lambda$-eigenspace of
  $A$ and $E'_{\lambda}$ the λ-eigenspace of $A^{t}$.

  Note that we can have $E_{\lambda}\neq E'_{\lambda}$

  \begin{example*}

    Take $A =
    \begin{pmatrix}
      1 & 4 \\
      1 & 1
    \end{pmatrix}
    $. Then
    $f_A(\lambda) = (1-\lambda)^2 - 4 = \lambda^2-2\lambda - 3 $ and
    hence $\lambda = -1$ or $\lambda = 3$.

    Thus, for $\lambda = 3$,

    \begin{align}
      \begin{pmatrix}
        -2 & 4 \\
        1 & -2
      \end{pmatrix}\cv{x;y} = \cv{-2x + 4y; x - 2y} = 0
    \end{align}

    Thus, $ x = 2y$, and $\cv{2;1}$ spans $E_{3}$.

    Consider now $E'_{3}$.   $A^{t} =
    \begin{pmatrix}
      -2 & 1 \\
      4 & -2
    \end{pmatrix} \cv{x;y} = \cv{-2x + y; 4x - 2y} = 0$, and thus
    $\cv{1;2}$ spans $E'_{\lambda}$. This means that
    $E'_{\lambda} \neq E_{\lambda}$, since $\cv{1; 2}$ and $\cv{2; 1}$
    are linearly independent.
  \end{example*}

  Suppose $T \in \Hom(V,V)$ is a linear transformation corresponding
  to the matrix $A$, where $V$ is a vector space over $\FF$.

  Observe that $\rank (A - \lambda I) = \rank (A^{t} - \lambda I)$,
  since $(A- \lambda I)^{t} = A^{t} - \lambda I$. Therefore, by
  Rank-Nullity Theorem,
  $\dim(E_{\lambda})  = \nll (A-\lambda I)  = \nll (A^{t} - \lambda I) = \dim(E'_{\lambda})$.
  \begin{lemma*}
    For a finite-dimensional vector space $V$ and $T \in \Hom(V,V)$
    with the distinct eigenvalues denoted as
    $\lambda_1, \lambda_2, \dots, \lambda_{k}$, $T$ is
    diagonalisable if and only if $\dim V = \sum_{i=1}^m \dim E_{\lambda_{m}}$.
  \end{lemma*}
  \begin{proof}
    Since $T$ is diagonalisable, it has a basis consisting of
    eigenvectors of $T$. Since all $\lambda_i$ are distinct,
    $E_{\lambda_i}\cap E_{\lambda_j} = \set{0}$ for any
    $i, j \in [1, k]\cap \NN$. Therefore, all the eigenvectors are in
    one and only one eigenspace, and thus
    $\dim V = \sum_{i=1}^m \dim E_{\lambda_{m}}$.

    Suppose now $\dim V = \sum_{i=1}^m \dim E_{\lambda_{m}}$.

    Choose a basis for each $E_{\lambda_{i}}$ and take their union,
    obtaining a set of eigenvectors
    \[\beta = \set{v_1, v_2,\dots, v_{n}}.\] There are $n$ of them by
    assumption. It has been proven earlier that it is linearly
    independent, and thus $\beta$ is a basis of $V$. Therefore, $T$ is
    diagonalisable.
  \end{proof}

  Since for all $\lambda$ $\dim(E_{\lambda}) = \dim(E'_{\lambda})$, it
  follows that $\dim V = \sum_{i=1}^m \dim E'_{\lambda_{m}}$. Thus, by
  the Lemma, $A^{t}$ is also diagonalisable.


\end{linenumbers}
\end{document}