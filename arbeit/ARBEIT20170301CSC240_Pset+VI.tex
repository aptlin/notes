%%% Local Variables:
%%% mode: latex
%%% TeX-engine: xetex
%%% TeX-master: t
%%% End:
\documentclass[11pt]{scrartcl}
\usepackage[fancy, beaue, pset, anon]{masty}
\usepackage{lineno}
% ----------------------------------------------------------------------
% Page setup
% ----------------------------------------------------------------------

\pagenumbering{gobble}

% ----------------------------------------------------------------------
% Custom commands
% ----------------------------------------------------------------------

% alignment

\newcommand*{\LongestHence}{$\Rightarrow$}% function name
\newcommand*{\LongestName}{$f_o(-x)+f_e(-x)$}% function name
\newcommand*{\LongestValue}{$(-a)x +(-a)(-y)$}% function value
\newcommand*{\LongestText}{\defi}%

\newlength{\LargestHenceSize}%
\newlength{\LargestNameSize}%
\newlength{\LargestValueSize}%
\newlength{\LargestTextSize}%

\settowidth{\LargestHenceSize}{\LongestHence}%
\settowidth{\LargestNameSize}{\LongestName}%
\settowidth{\LargestValueSize}{\LongestValue}%
\settowidth{\LargestTextSize}{\LongestText}%

% Choose alignment of the various elements here: [r], [l] or [c]

\newcommand*{\mbh}[1]{{\makebox[\LargestHenceSize][r]{\ensuremath{#1}}}}%
\newcommand*{\mbn}[1]{{\makebox[\LargestNameSize][r]{\ensuremath{#1}}}}%
\newcommand*{\mbv}[1]{\ensuremath{\makebox[\LargestValueSize][r]{\ensuremath{#1}}}}%
\newcommand*{\mbt}[1]{\makebox[\LargestTextSize][l]{#1}}%
\usepackage{csquotes}
% ----------------------------------------------------------------------
% Launch!
% ----------------------------------------------------------------------
\pSet{Alexander Illarionov - 1003590937 - CSC240 Pset VI}
\usepackage{program}
\begin{document}

% ----------------------------------------------------------------------
% Body
% ----------------------------------------------------------------------
\NumberProgramstrue
\textbf{Administrativia}: no discussions, no extra material consulted
\section*{Problem I}
Consider the following algorithm:

ISUM$(A)$
\begin{program}
  best \gets 0
  \IF A[1] > best \THEN best \gets A[1] \FI
    \FOR i\gets 1 \TO n-1 \DO
    b\gets A[i+1]
    \IF b > best \THEN best \gets b \FI
    \FOR j\gets i | down| \TO 1 \DO
    b\gets b+A[j]
    \IF b > best \THEN best \gets b \FI
    \OD
    \OD
    |return|(best)
\end{program}

What is the worst number of assignments performed by this algorithm?

\section*{Solution}

Let $T:\ZZ^+ \to \NN$ be the map such that for any $n\in\ZZ^{+}$
$T(n)$ is the maximum number of assignments performed by the algorithm
above on arrays $A$ of length $n$.

Let $n$ be arbitrary positive integer.

Let $A$ be an arbitrary array of length $n$.

From the algorithm, there is one assignment in the line (1), and at
most one assignment in the line (2).

From the line (3), there are $n-1$ iterations of the for-loop, and
thus there are $n-1$ assignments performed in the line (4).

There is at most one assignment for each iteration of the for-loop in
the line (5), and thus at most $n-1$ assignments executed on line (5)
in total.

For each $(i)$th iteration initiated in the line (3), there are $i$
iterations initiated by the line (6). Thus, for each $(i)$th iteration
there are $i$ assignments given by the line (7). There is at most
1 assignment on line $(8)$, and hence for each $(i)$th iteration
there are at most $i$ assignments executed on line (8).

Therefore,
\begin{align}
  T(n) &\leq 1 + 1 + (n-1)+(n-1)+ \sum_{i=1}^{n-1}\big(i+i\big)\\
       &=2n +n(n-1) = n^2+n
\end{align}

Thus, $T(n)\leq n^2+n$.

Consider now an array $G$ of length $n\in\NN$ such that, for all
$i\in[1, n]\cap\NN$, $G[i] = F[i]$, where $F(1) = 1$ and $F(2) = 2$,
and, for $j \geq 3$, $F(j) = F(j-1) + F(j-2)$.

\begin{lemma}
  \label{sec:solution}
For all $j\in\ZZ^+$, $0 < F(j) < F(j+1)$.
\end{lemma}

\begin{proof}
  \hfill

  We proceed by induction.

  For any $i\in \ZZ^+$, let $P(i)$ denote the predicate \enquote{$F(i) < F(i+1)$ and $F(i)> 0$}.

  \begin{description}
  \item[Base Case] \hfill

    Note that by definition $F(1) = 1 > 0$, $F(2) = 2 > 0$, and thus $0< F(1) < F(2)$. Thus, $P(1)$ holds.

    Moreover, $F(3) = F(1) + F(2) = 3 > 0$, and thus $0< F(2) < F(3)$.
  \item[Inductive Step] \hfill

    Assume the hypothesis holds for some $k\in \ZZ^{+}$ such that $k\geq 3$.

    Thus, $ 0 < F(k) < F(k+1)$. By definition of $F$, since $k\geq 3$,
    $F(k+2) = F(k) + F(k+1)$.

    By inductive hypothesis, $F(k) > 0$, and thus $F(k+2) > F(k+1)$.

    Since $F(k+1) > 0$, then $F(k+2) > F(k+1) > 0$.

    Therefore, $P(k+1)$.

  \item[Conclusion]
    \hfill

    For all $j\in\ZZ^+$, $0 <F(j) < F(j+1)$ by induction. $\qedsymbol$
  \end{description}

  For the array $G$, ISUM($G$) executes one assignment in the line (1), and since $F(1) = 1 > 0$, there is a second assignment on line (2).

  Since there are $n-1$ iterations of the for-loop initiated in the line (3), there are $n-1$ unconditional assignments in the line (4). Since for each $i$ from $1$ to $n-1$ $G[i+1] = F(i+1) > F(i) = G[i]$ by Lemma \ref{sec:solution}, then there are $n-1$ assignments in line (5).

  For each $(i)$th iteration initiated in the line (3), there are $i$
  iterations initiated by the line (6).

  For each $(i)$th iteration there are $i$ unconditional assignments
  given by the line (7).

  Since for each $(i)$th iteration $G[i]$ is positive by Lemma
  \ref{sec:solution}, then the if-condition in the line (8) is
  satisfied, and hence there are $i$ assignments executed from the
  line (8).

  The maximum number of assignments in the worst case is no better
  than the number of assignments for the array $G$, and hence

  \begin{align}
  T(n) &\geq 1 + 1 + (n-1)+(n-1)+ \sum_{i=1}^{n-1}\big(i+i\big)\\
       &=2n +n(n-1) = n^2+n
\end{align}.

Since $T(n)$ is also less than or equal to $n^2+n$, $T(n) = n^2 + n$.

Thus, $\forall n\in\ZZ^+.T(n) = n^2+n$ by generalisation.
\end{proof}

\section*{Problem II}

For any given $A$, define $R:\ZZ^+\to \NN$ as $R(n) =$ \textquote{the worst case number of assignments of sums of elements to $best, best'$ and $b$ performed by RSUM($A, 1, n$)}.

If $n=1$, then there is one assignment on line 1 and at worst one assignment on line 4.

If $n > 1$, there is one assignment on line 1, and there is one assignment on line 6.

There are $n-\floor{\frac{n+1}{2}}$ iterations initiated in the line 7, and thus there are $n-\floor{\frac{n+1}{2}}$ assignments on line 8 and at worst $n-\floor{\frac{n+1}{2}}$ assignments on line 9.

Then there are 2 assignments from lines 10 and 11.

In the line 12, a for-loop is initiated with $\floor{\frac{n+1}{2}}$ iterations, and hence there are $\floor{\frac{n+1}{2}}$ assignments on line 13 and at most $\floor{\frac{n+1}{2}}$ assignments on line 14.

Then there is one assignment on line 15.

On line 16, there are $R(\floor{\frac{n+1}{2}})$ assignments.

On line 17, there is at most one assignment.

On line 18, there are $R(n-\floor{\frac{n+1}{2}})$ assignments.

On line 19, there is at most one assignment.

Therefore, in the worst case, 
\begin{equation*}
  R(n) = \begin{cases}
    2 , \text{if }  n =1\\
    2n+7 + R(\floor{\frac{n+1}{2}}) + R(n-\floor{\frac{n+1}{2}}), \text{if } n > 1
  \end{cases}
\end{equation*}

Note that, by definition of a floor function, if $z\in\ZZ^+$, then $\floor{z+\nicefrac{1}{2}} = z$.

Consider $R(m)$ for $m= 2^{k} $ such that  $k\in\ZZ^+$:

\begin{align*}
  R(2^{k}) &=
             \begin{cases}
               2 , \text{if }  k =0\\
               2^{k+1}+7 + R(2^{k-1}) + R(2^{k-1}), \text{if } k > 0
         \end{cases}\\
       &=
         \begin{cases}
           2 , \text{if }  k =0\\
           7+2^{k+1}+ 2R(2^{k-1}), \text{if } k > 0
         \end{cases}\\
\end{align*}

\begin{claim*}
  For any $k\in\NN^+$, $R(2^k) = (2^k-1)7+(k+1)2^{k+1}$.
\end{claim*}
\begin{proof}
  \hfill

  For any $n\in\NN^+$, let $Q(n) =$\textquote{$R(2^n) = (2^n-1)7+(n+1)2^{n+1}$}.

  We proceed by induction on $n$.

  \begin{description}
  \item[Base Case] \hfill

    If $n=0$, then $R(2^0) = R(1) = 2 = (2^0-1)7+2^{0+1}$ and thus $Q(0)$ holds.
  \item[Inductive Step] \hfill

    For $n>0$, suppose $Q(n-1)$ holds.

    By the recurrent definition of $R(n)$, we obtain that
    $R(n) = 7+2^{n+1}+ 2R(2^{n-1})$, which is, by inductive
    hypothesis, equivalent to

    \begin{align}
      R(n) &= 7+ 2^{n+1}+2((2^{n-1}-1)7+n2^{n})\\
           &= 7+(n+1)2^{n+1}+(2^n-2)7\\
           &=(2^n-1)7+(n+1)2^{n+1}.
    \end{align}

    Thus, $Q(n)$ holds, and hence $\forall n\in\NN.P(n)$ by induction.
  \end{description}
\end{proof}




\end{document}