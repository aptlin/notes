
%%% Local Variables:
%%% mode: latex
%%% TeX-master: t
%%% End:

\documentclass[11pt]{scrartcl}
\usepackage[beaue, pset, anon]{masty}
\pSet{\hw{MAT247}{VIII}{3}}
\usepackage{lineno}
% ----------------------------------------------------------------------
% Page setup
% ----------------------------------------------------------------------

\pagenumbering{gobble}

% ----------------------------------------------------------------------
% Custom commands
% ----------------------------------------------------------------------

% alignment

\newcommand*{\LongestHence}{$\Rightarrow$}% function name
\newcommand*{\LongestName}{$f_o(-x)+f_e(-x)$}% function name
\newcommand*{\LongestValue}{$(-a)x +(-a)(-y)$}% function value
\newcommand*{\LongestText}{\defi}%

\newlength{\LargestHenceSize}%
\newlength{\LargestNameSize}%
\newlength{\LargestValueSize}%
\newlength{\LargestTextSize}%

\settowidth{\LargestHenceSize}{\LongestHence}%
\settowidth{\LargestNameSize}{\LongestName}%
\settowidth{\LargestValueSize}{\LongestValue}%
\settowidth{\LargestTextSize}{\LongestText}%

% Choose alignment of the various elements here: [r], [l] or [c]

\newcommand*{\mbh}[1]{{\makebox[\LargestHenceSize][r]{\ensuremath{#1}}}}%
\newcommand*{\mbn}[1]{{\makebox[\LargestNameSize][r]{\ensuremath{#1}}}}%
\newcommand*{\mbv}[1]{\ensuremath{\makebox[\LargestValueSize][r]{\ensuremath{#1}}}}%
\newcommand*{\mbt}[1]{\makebox[\LargestTextSize][l]{#1}}%

\newcommand{\R}[1]{\label{#1}\linelabel{#1}}
\newcommand{\lr}[1]{line~\lineref{#1}}

% ----------------------------------------------------------------------
% Launch!
% ----------------------------------------------------------------------

\begin{document}

\begin{problem*}
Suppose that $V = M_{2\times 2}(\FF)$ with $\FF = \ZZ_{5}$.

Let $T \in \End(V)$ be such that $TA = 
A \begin{pmatrix}
1 & 0\\
2 & 1
\end{pmatrix}$.

Find a basis of $V$ that consists of a disjoint union of cycles of generalized eigenvalues. Find a Jordan canonical form.
\end{problem*}

\begin{soln}
  \hfill

Let $B = 
\begin{pmatrix}
1 & 0\\
2 & 1
\end{pmatrix}
$ and let $A = 
\begin{pmatrix}
a & b\\
c & d
\end{pmatrix}$ for some $a, b, c, d \in \ZZ_{5}$.

Note that 
\begin{align}
\det(B-\lambda I) &= (1-\lambda)^2.
\end{align}

Let $f(t)=(1-t)^2$. 

By Cayley-Hamilton Theorem, we know that $f(B) = 0$.

Consider $f(T)$:
\begin{align}
  f(T)(A) &= (I - 2T+T^2)A\\
          &= A - 2T(A) +T^2(A)\\
          &= A - 2AB + T(AB)\\
          &= A - 2AB + AB^2\\
          &= A(1-2B+B^2)\\
          &= Af(B)\\
          &= 0.
\end{align}

Since (2)-(7) holds for any $A$, then $f(T)(A)$ is the zero homomorphism.

% Note that $T^2A = T(TA) = AB^2$.

% $B^2= \begin{pmatrix}
% 1 & 0\\
% 2 & 1
% \end{pmatrix}
% \begin{pmatrix}
% 1 & 0\\
% 2 & 1
% \end{pmatrix}=
% \begin{pmatrix}
% 1 & 0\\
% 4 & 1
% \end{pmatrix}
% $.

Let $A = 
\begin{pmatrix}
a & b\\
c & d
\end{pmatrix}$ for some $a, b, c, d\in \FF$.

Therefore, if $T(A) = AB=\lambda A$, then
\begin{equation*}
  \begin{cases}
a+2b &= \lambda a\\
c+2d &= \lambda c\\
b & = \lambda b\\
d & = \lambda d.
  \end{cases}
\end{equation*}

Therefore, from the equation 4, since $\lambda \neq 0$ (because  then $a=b=c=d = 0$), we have $\lambda = 1$.

Hence, $b = d = 0$, and thus 
$\begin{pmatrix}
1 & 0\\
0 & 0
\end{pmatrix}$ and
$\begin{pmatrix}
0 & 0\\
1 & 0
\end{pmatrix}$ span $\ker (T-I)$.



We have shown that $T^2-2T+I = 0$, and therefore $T^2= 2T - I$ and $(T-I)^2=0$ for any $A$, which means that $\ker (T-I)^2 = V$.

Therefore, $
\begin{pmatrix}
0 & 1\\
0 & 0
\end{pmatrix}$ and $
\begin{pmatrix}
0 & 0\\
0 & 1
\end{pmatrix}$ are generalised eigenvectors.

Since $\nll (T-I) = 2$, we know that there are exactly two Jordan blocks in the corresponding Jordan canonical form.

Now we find a cycle basis.

Note that $T \begin{pmatrix}
a & b\\
c & d
\end{pmatrix}= 
\begin{pmatrix}
a+2b & b\\
c+2d & d
\end{pmatrix}
$.

Take $v = 
\begin{pmatrix}
0 & 1 \\
0 & 0
\end{pmatrix}$.

Then $(T- I)v = Tv - Iv = 
\begin{pmatrix}
2 & 1\\
0 & 0
\end{pmatrix} - \begin{pmatrix}
0 & 1\\
0 & 0
\end{pmatrix}= 
  \begin{pmatrix}
2 & 0\\
0 & 0
\end{pmatrix}$, which is in $E_{1}$, and thus
$p=\begin{pmatrix}
2 & 0\\
0 & 0
\end{pmatrix}$ and 
$v=\begin{pmatrix}
0 & 1\\
0 & 0
\end{pmatrix}$ form a cycle.

Take now $w =
\begin{pmatrix}
0 & 0\\
0 & 1
\end{pmatrix}$.

Then $(T- I)w = Tw - Iw = 
\begin{pmatrix}
0 & 0\\
2 & 1
\end{pmatrix} - \begin{pmatrix}
0 & 0\\
0 & 1
\end{pmatrix}= 
  \begin{pmatrix}
0 & 0\\
2 & 0
\end{pmatrix}$, which is in $E_{1}$, and thus
$q=\begin{pmatrix}
0 & 0\\
2 & 0
\end{pmatrix}$ and 
$ w= \begin{pmatrix}
0 & 0\\
0 & 1
\end{pmatrix}$ form a cycle.

It is easy to see that $\beta = \set{p, v, q, w}$ is linearly
independent and has the length of 4, which is equal to $\dim
W$. Therefore, $\beta$ is a cycle basis and hence


\begin{equation*}
  [A]_{\beta} = 
  \begin{pmatrix}
    1 & 1 & 0 & 0\\
    0 & 1 & 0 & 0\\
    0 & 0 & 1 & 1\\
    0 & 0 & 0 & 1
  \end{pmatrix}.
\end{equation*}

\end{soln}



\end{document}
