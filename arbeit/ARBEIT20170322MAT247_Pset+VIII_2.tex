
%%% Local Variables:
%%% mode: latex
%%% TeX-master: t
%%% End:

\documentclass[11pt]{scrartcl}
\usepackage[beaue, pset, anon]{masty}
\pSet{\hw{MAT247}{VIII}{2}}
\usepackage{lineno}
% ----------------------------------------------------------------------
% Page setup
% ----------------------------------------------------------------------

\pagenumbering{gobble}

% ----------------------------------------------------------------------
% Custom commands
% ----------------------------------------------------------------------

% alignment

\newcommand*{\LongestHence}{$\Rightarrow$}% function name
\newcommand*{\LongestName}{$f_o(-x)+f_e(-x)$}% function name
\newcommand*{\LongestValue}{$(-a)x +(-a)(-y)$}% function value
\newcommand*{\LongestText}{\defi}%

\newlength{\LargestHenceSize}%
\newlength{\LargestNameSize}%
\newlength{\LargestValueSize}%
\newlength{\LargestTextSize}%

\settowidth{\LargestHenceSize}{\LongestHence}%
\settowidth{\LargestNameSize}{\LongestName}%
\settowidth{\LargestValueSize}{\LongestValue}%
\settowidth{\LargestTextSize}{\LongestText}%

% Choose alignment of the various elements here: [r], [l] or [c]

\newcommand*{\mbh}[1]{{\makebox[\LargestHenceSize][r]{\ensuremath{#1}}}}%
\newcommand*{\mbn}[1]{{\makebox[\LargestNameSize][r]{\ensuremath{#1}}}}%
\newcommand*{\mbv}[1]{\ensuremath{\makebox[\LargestValueSize][r]{\ensuremath{#1}}}}%
\newcommand*{\mbt}[1]{\makebox[\LargestTextSize][l]{#1}}%

\newcommand{\R}[1]{\label{#1}\linelabel{#1}}
\newcommand{\lr}[1]{line~\lineref{#1}}

% ----------------------------------------------------------------------
% Launch!
% ----------------------------------------------------------------------

\begin{document}
\section{Problem}
\begin{problem*}
Suppose a matrix $A$ is given:

\begin{equation*}
  A =
  \begin{pmatrix}
    6 & 3 & -4\\
    -1 & 1 & 1\\
    3 & 2 & -1
  \end{pmatrix}
\end{equation*}

Find a cycle basis (as defined in the lectures) and a Jordan canonical
form of $A$.
\end{problem*}

\begin{soln}
  \hfill

First we find a characteristic polynomial:

\begin{align}
\det(A-\lambda I) &= \det 
  \begin{pmatrix}
    6-\lambda & 3 & -4\\
    -1 & 1-\lambda & 1\\
    3 & 2 & -1 - \lambda
  \end{pmatrix}\\
&= 
(6-\lambda)((1-\lambda)(-1-\lambda)-2)\\
&\ + (3(-1 - \lambda) +8)\\
&\ + 3(3+4(1-\lambda))\\
&= (6-\lambda)((\lambda-1)(\lambda+1)-2)\\
&+ \ (-3\lambda+5)\\
&+ \ 3(-4\lambda+7)\\
&= (6-\lambda)(\lambda^2-3) -3\lambda+5-12\lambda+21\\
&=-\lambda^3 +6\lambda^2+3\lambda-18-15\lambda+26\\
&=-\lambda^3 +6\lambda^2 - 12\lambda+8\\
&=-\lambda^3+3\*2\lambda^{2}-3\*4\lambda+2^3\\
&=-(\lambda-2)^3
\end{align}

Thus, 

\begin{align}
\ker(A-2I) &=
             \ker\begin{pmatrix}
               4 & 3 & -4\\
               -1 & -1 & 1\\
               3 & 2 & -3
             \end{pmatrix}.
\end{align}

Suppose that $v=\cv{x;y;z}$ is an eigenvector. Hence, 
\begin{equation*}
  \begin{cases}
    4x+3y-4z &= 0\\
    -x-y+z &= 0\\
    3x+2y-3z &= 0
  \end{cases}
\end{equation*}

Note that 
\begin{align}
\begin{matreq}{ccc|c}
4 & 3 & - 4 & 0\\
-1 & -1 & 1 & 0\\
3 & 2 & -3 & 0
\end{matreq} &\\
  \text{$R_3-R_2 \to R_{3}$ }&\ras
             \begin{matreq}{ccc|c}
               4 & 3 & - 4 & 0\\
               -1 & -1 & 1 & 0\\
               4 & 3 & - 4 & 0\\
             \end{matreq}\\
  \text{$R_1-R_3 \to R_3$ }&\ras
             \begin{matreq}{ccc|c}
               4 & 3 & - 4 & 0\\
               -1 & -1 & 1 & 0\\
               0 & 0 & 0 & 0
             \end{matreq}\\
  \text{$R_1+4R_2\to R_{2}$ }&\ras
             \begin{matreq}{ccc|c}
               1 & 0 & - 1 & 0\\
               0 & -1 & 0 & 0\\
               0 & 0 & 0 & 0
             \end{matreq}
\end{align}

Hence, $y=0$ and $x=z$, which means that $\cv{1;0;1}$ spans $E_2$. Since $E_2$ is one-dimensional, there is only one Jordan block corresponding to $\lambda = 2$ by Corollary to Theorem 7.9, and the dot diagram has only one column.

Note that 

\begin{align}
  (A-2I)^2 &=             
           \begin{pmatrix}
             4 & 3 & -4\\
             -1 & -1 & 1\\
             3 & 2 & -3
           \end{pmatrix}
           \begin{pmatrix}
             4 & 3 & -4\\
             -1 & -1 & 1\\
             3 & 2 & -3
           \end{pmatrix}\\
           &=
           \begin{pmatrix}
             1 & 1 & -1\\
             0 & 0 & 0\\
             1 & 1 & -1
           \end{pmatrix}
\end{align}



% Note that $\rank (A-2I)^2 = 1$, because there is only one nonzero row
% in the row-reduced matrix of $(A-2I)^2$, while $\rank (A-2I) = 2$ (see
% eq. (18)). Therefore there is $2-1 = 1$ dot in the second row of the
% dot diagram corresponding to $A$.

Hence, if $u = \cv{x;y;z}$ is an eigenvector in $\ker(A-2I)^2$, then
$x+y = z$. 

We are trying to find $x, y, z$ such that $(A-2 I)u = \cv{1;0;1}$.

Therefore, $4x+3y-4z = 1$, $-x-y+z = 0$, $3x+2y-3z = 1$.

Thus, $12x+9y-12z = 3$ nd $12x+8y-12z = 4$. 

Therefore, $y=-1$, and thus $z= x-1$. 

Take $u = \cv{1;-1;0}$.

% Since there is only one dot in the second row of the dot diagram, there must be an eigenvector which is in $\img (A-2I)^2$ and not in $\ker(A-2I)$ or $\ker(A-2I)^2$. Take $u =\cv{0;0;1}$, and note that it is not in $E_2$ and $(A-2I)^2u = \cv{-1;0;1}$.

Now take an orthogonal vector to $u$, for example, $w = \cv{1;1;0}$.

Let $\beta = \set{\cv{1;0;1}, \cv{0;1;1}, \cv{1;1;0}}$. If $\beta$ is linearly independent, then it is a cycle basis, because $\dim V = 3$. We prove now that it is linearly independent. 

Suppose that there exist $a_1$, $a_2$ and $a_3$ such that the linear combination of the corresponding vectors in $\beta$ is 0. Then
\begin{equation*}
  \begin{cases}
a_1+a_2+0a_3 &= 0\\
0a_1+a_2+a_3 &= 0\\
a_1+0a_2+a_3&=0
  \end{cases}.
\end{equation*}

From the second and third equation we obtain that $a_1 = -a_3$. From
the third and first equation we get that $a_2=a_3$. From the second
equation we get that $a_3=0$, hence $a_1=a_2=a_3=0$, and thus $\beta$
is linearly independent.

Moreover, 

\begin{align}
(A-2I)^3 &=
           \begin{pmatrix}
             4 & 3 & -4\\
             -1 & -1 & 1\\
             3 & 2 & -3
           \end{pmatrix}
           \begin{pmatrix}
             1 & 1 & -1\\
             0 & 0 & 0\\
             1 & 1 & -1
           \end{pmatrix}\\
           &=
             \begin{pmatrix}
               0 & 0 & 0\\
               0 & 0 & 0\\
               0 & 0 & 0
             \end{pmatrix},
\end{align}

and thus $w$ is an generalised eigenvector of $A$.

Therefore, $\beta$ is a cycle basis.
\end{soln}
\end{document}
