%%% Local Variables:
%%% mode: latex
%%% TeX-master: t
%%% End:

\documentclass[11pt]{scrartcl}
\usepackage[beaue, pset, anon]{sdll}
\pSet{\hw{MAT247}{I}{2}}
\usepackage{lineno}
% ----------------------------------------------------------------------
% Page setup
% ----------------------------------------------------------------------

\pagenumbering{gobble}

% ----------------------------------------------------------------------
% Custom commands
% ----------------------------------------------------------------------

% alignment

\newcommand*{\LongestHence}{$\Rightarrow$}% function name
\newcommand*{\LongestName}{$f_o(-x)+f_e(-x)$}% function name
\newcommand*{\LongestValue}{$(-a)x +(-a)(-y)$}% function value
\newcommand*{\LongestText}{\defi}%

\newlength{\LargestHenceSize}%
\newlength{\LargestNameSize}%
\newlength{\LargestValueSize}%
\newlength{\LargestTextSize}%

\settowidth{\LargestHenceSize}{\LongestHence}%
\settowidth{\LargestNameSize}{\LongestName}%
\settowidth{\LargestValueSize}{\LongestValue}%
\settowidth{\LargestTextSize}{\LongestText}%

% Choose alignment of the various elements here: [r], [l] or [c]

\newcommand*{\mbh}[1]{{\makebox[\LargestHenceSize][r]{\ensuremath{#1}}}}%
\newcommand*{\mbn}[1]{{\makebox[\LargestNameSize][r]{\ensuremath{#1}}}}%
\newcommand*{\mbv}[1]{\ensuremath{\makebox[\LargestValueSize][r]{\ensuremath{#1}}}}%
\newcommand*{\mbt}[1]{\makebox[\LargestTextSize][l]{#1}}%

\newcommand{\R}[1]{\label{#1}\linelabel{#1}}
\newcommand{\lr}[1]{line~\lineref{#1}}

% ----------------------------------------------------------------------
% Launch!
% ----------------------------------------------------------------------

\begin{document}

% ----------------------------------------------------------------------
% Body
% ----------------------------------------------------------------------
\begin{linenumbers}
  Consider $V = \SP_2(\CC)$ and the linear transformation $T\in \Hom(V,V)$ given by

  \begin{equation*}
    T(f(x)) = xf'(x) + xf(1) + f(2).
  \end{equation*}

  \begin{problem*}
    Find the eigenvalues of $T$.
  \end{problem*}
  \begin{soln}
    Note that $\gamma = \set{1, x, x^{2}}$ is an ordered basis of
    $\SP(\CC)$. Note that a linear transformation is completely
    determined by its action on a basis. Thus,

    \begin{align}
      T(1)   & = x + 1               \\
      T(x)   & = x + x + 2 =  2x + 2 \\
      T(x^2) & = 2x^2 + x + 4
    \end{align}

    Therefore, $[T]_{\gamma} =
    \begin{pmatrix}
      1 & 2 & 4 \\
      1 & 2 & 1 \\
      0 & 0 & 2
    \end{pmatrix}
    $.

    Consider $\det([T]_{\gamma}-\lambda I) = 0$.


    \begin{align*}
      \det([T]_{\gamma}-\lambda I)                                         & =
      \det\begin{pmatrix}
        1 - \lambda                                                        & 2         & 4 \\
        1                                                                  & 2-\lambda & 1 \\
        0                                                                  & 0         & 2-\lambda
      \end{pmatrix}                                                                        \\
                                                                           & =
                                                               \det\begin{pmatrix}
                                                                 - \lambda & \lambda   & 3 \\
                                                                 1         & 2-\lambda & 1 \\
                                                                 0         & 0
                                                                           &
                                                                 2-\lambda
                                                               \end{pmatrix}               \\
                                                                           & =
                                                               \det\begin{pmatrix}
                                                                 0 & 3\lambda - \lambda^{2}   & 3 + \lambda \\
                                                                 1         & 2-\lambda & 1 \\
                                                                 0         & 0
                                                                           &
                                                                 2-\lambda
                                                               \end{pmatrix},
    \end{align*}
    which, expanding along the first column, becomes

    \begin{align}
      \det([T]_{\gamma}-\lambda I) & = -\lambda(3-\lambda)(2-\lambda) = 0
    \end{align}
    Therefore, the possible eigenvalues are
    $\lambda=0, \lambda = 3, \lambda = 2$.
  \end{soln}
  \begin{problem*}
    Find a basis $\beta$ for which $[T]_{\beta}$ is a diagonal matrix.
  \end{problem*}
  \begin{soln}
    First, we find eigenvectors corresponding to the eigenvalues found
    above.

    For $\lambda=0$, $
    \begin{pmatrix}
      1 & 2 & 4 \\
      1 & 2 & 1 \\
      0 & 0 & 2
    \end{pmatrix}\cv{x; y; z} = \bm{0}$ if and only if

    \begin{align}
      & x + 2y + 4z = 0 \\
      & x + 2y + z = 0  \\
      & 0 + 0 + 2z = 0.
    \end{align}

    Therefore, $\cv{2; -1; 0}$ spans $E_0$.

    For $\lambda=2$, $
    \begin{pmatrix}
      -1 & 2 & 4 \\
      1 & 0 & 1 \\
      0 & 0 & 0
    \end{pmatrix}\cv{x; y; z} = \bm{0}$ if and only if

    \begin{align}
      -& x + 2y + 4z = 0 \\
       & x  + 0 + z = 0     \\
       & 0 + 0 + 0 = 0.
    \end{align}

    Therefore, $\cv{2; 5; -2}$ spans $E_2$.

    For $\lambda=3$, $
    \begin{pmatrix}
      -2 & 2 & 4 \\
      1 & -1 & 1 \\
      0 & 0 & -1
    \end{pmatrix}\cv{x; y; z} = \bm{0}$ if and only if

    \begin{align}
      -& 2x + 2y + 4z = 0 \\
      & x - y + z = 0     \\
      & 0 + 0 + -z = 0.
    \end{align}

    Therefore, $\cv{1; 1; 0}$ spans $E_3$.

    Since the eigenvalues found are distinct, the corresponding
    eigenvalues are linearly independent, and thus, since there are
    three of them and the dimension of $\SP(\CC)$ is $3$, they form a basis.

    Take $\beta = \set{\cv{2;-1;0}, \cv{2; 5; -2}, \cv{1;1;0}}$. Then
    the corresponding diagonal matrix is


    \begin{equation*}
      [T]_{\beta} =
      \begin{pmatrix}
        0 & 0 & 0\\
        0 & 2 & 0\\
        0 & 0 & 3
      \end{pmatrix},
    \end{equation*}
    as required.
  \end{soln}

    % Note the following:
    % \begin{align}
    %   & \cv{1;0;0} = \frac{1}{3}\cv{2;-1;0} + 0 \cv{2;5;-2} +
    %     \frac{1}{3}\cv{1;1;0} \\
    %   & \cv{0;1;0} = -\frac{1}{3}\cv{2;-1;0} + 0 \cv{2;5;-2} +
    %     \frac{2}{3}\cv{1;1;0} \\
    %   & \cv{0;0;1} = -\frac{1}{2}\cv{2;-1;0} -\frac{1}{2} \cv{2;5;-2} +
    %     2\cv{1;1;0} \\
    % \end{align}

    % Therefore,
    % $[I]_{\gamma}^{\beta} =
    % \begin{pmatrix}
    %   \frac{1}{3} & - \frac{1}{3} & -\frac{1}{2} \\
    %   0           & 0             & -\frac{1}{2} \\
    %   \frac{1}{3} & \frac{2}{3}   & 2
    % \end{pmatrix}
    % $, while $[I]^{\gamma}_{\beta} =
    % \begin{pmatrix}
    %   2           & 2             & 1 \\
    %   -1          & 5             & 1 \\
    %   0           & -2            & 0
    % \end{pmatrix}
    % $.

    % Thus, since $[T]_{\beta} = [I]_{\gamma}^{\beta}[T]_{\gamma}[I]_{\beta}^{\gamma}$,


    % \begin{equation*}
    %   [T]_{\beta} =
    %   \begin{pmatrix}
    %     \frac{1}{3} & - \frac{1}{3} & -\frac{1}{2} \\
    %     0           & 0             & -\frac{1}{2} \\
    %     \frac{1}{3} & \frac{2}{3}   & 2
    %   \end{pmatrix}
    %   \begin{pmatrix}
    %     1 & 2 & 4 \\
    %     1 & 2 & 1 \\
    %     0 & 0 & 2
    %   \end{pmatrix}    \begin{pmatrix}
    %     2           & 2             & 1 \\
    %     -1          & 5             & 1 \\
    %     0           & -2            & 0
    %   \end{pmatrix} =

    %   \begin{pmatrix}
    %     \frac{1}{3} & - \frac{1}{3} & -\frac{1}{2} \\
    %     0           & 0             & -\frac{1}{2} \\
    %     \frac{1}{3} & \frac{2}{3}   & 2
    %   \end{pmatrix}
    %   \begin{pmatrix}

    %   \end{pmatrix}

    % \end{equation*}

\end{linenumbers}
\end{document}