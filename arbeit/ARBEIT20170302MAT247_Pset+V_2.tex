
%%% Local Variables:
%%% mode: latex
%%% TeX-master: t
%%% End:

\documentclass[11pt]{scrartcl}
\usepackage[beaue, pset, anon]{masty}
\pSet{\hw{MAT247}{V}{2}}
\usepackage{lineno}
% ----------------------------------------------------------------------
% Page setup
% ----------------------------------------------------------------------

\pagenumbering{gobble}

% ----------------------------------------------------------------------
% Custom commands
% ----------------------------------------------------------------------

% alignment

\newcommand*{\LongestHence}{$\Rightarrow$}% function name
\newcommand*{\LongestName}{$f_o(-x)+f_e(-x)$}% function name
\newcommand*{\LongestValue}{$(-a)x +(-a)(-y)$}% function value
\newcommand*{\LongestText}{\defi}%

\newlength{\LargestHenceSize}%
\newlength{\LargestNameSize}%
\newlength{\LargestValueSize}%
\newlength{\LargestTextSize}%

\settowidth{\LargestHenceSize}{\LongestHence}%
\settowidth{\LargestNameSize}{\LongestName}%
\settowidth{\LargestValueSize}{\LongestValue}%
\settowidth{\LargestTextSize}{\LongestText}%

% Choose alignment of the various elements here: [r], [l] or [c]

\newcommand*{\mbh}[1]{{\makebox[\LargestHenceSize][r]{\ensuremath{#1}}}}%
\newcommand*{\mbn}[1]{{\makebox[\LargestNameSize][r]{\ensuremath{#1}}}}%
\newcommand*{\mbv}[1]{\ensuremath{\makebox[\LargestValueSize][r]{\ensuremath{#1}}}}%
\newcommand*{\mbt}[1]{\makebox[\LargestTextSize][l]{#1}}%

\newcommand{\R}[1]{\label{#1}\linelabel{#1}}
\newcommand{\lr}[1]{line~\lineref{#1}}

% ----------------------------------------------------------------------
% Launch!
% ----------------------------------------------------------------------

\begin{document}

Let $T \in \Hom(V,V)$ be a linear transformation, where $V$ is a
finite-dimensional inner product space over $\FF$.

\begin{lemma}
  \label{sec:2}
  Suppose that $T = T^{*}$. Show that $\ipr{Tx}{x}\in\RR$ for all $x\in V$.
\end{lemma}

\begin{proof}
  \hfill

  Take $x\in V$.
  
  Since $\ipr{Tx}{x} = \ipr{x}{T^{*}x}$ by definition of an adjoint,
  then $\ipr{Tx}{x} = \ipr{x}{Tx}$ by assumption.

  Moreover, $\ipr{x}{Tx}= \ol{\ipr{Tx}{x}}$ from properties of an
  inner product, and thus $\ipr{Tx}{x} = \ol{\ipr{Tx}{x}}$, which
  means that $\ipr{Tx}{x}$ is real for all $x\in V$.
\end{proof}

\begin{lemma}
  \label{sec:1}
  If $\FF = \CC$ and $\ipr{Tx}{x}=0$ for all $x\in V$, then $T = \bm{0}$.
\end{lemma}

\begin{proof}
  \hfill

  Let $x, y\in V$.

  Note the following:

  \begin{align}
    T(x+y)  & = T(x) + T(y) \\
    T(x+iy) & = T(x) + iT(y)
  \end{align}

  Since $\ipr{T(x+y)}{x+y}= 0$, while $\ipr{Tx}{x} = 0$ and
  $\ipr{Ty}{y}=0$ then
  \begin{align}
    \ipr{T(x+y)}{x+y} & = \ipr{Tx}{x+y}+\ipr{Ty}{x+y}                     \\
                      & = \ipr{Tx}{x}+\ipr{Tx}{y}+\ipr{Ty}{x}+\ipr{Ty}{y} \\
                      & = \ipr{Tx}{y}+\ipr{Ty}{x} = 0
  \end{align}

  Therefore, $\ipr{Tx}{y}=-\ipr{Ty}{x}$

  Similarly,
  \begin{align}
    \ipr{T(x+iy)}{x+iy} & = \ipr{Tx}{x+iy}+i\ipr{Ty}{x+iy}                         \\
                        & = \ipr{Tx}{x}+\ipr{Tx}{iy}+i\ipr{Ty}{x}+i(-i)\ipr{Ty}{y} \\
                        & = \ipr{Tx}{x}-i\ipr{Tx}{y}+i\ipr{Ty}{x}+\ipr{Ty}{y}      \\
                        & = -i\ipr{Tx}{y}+i\ipr{Ty}{x} = 0
  \end{align}

  Therefore, $\ipr{Tx}{y} = \ipr{Ty}{x}$.

  Hence, by combining two equations above, we obtain that
  $\ipr{Tx}{y} = 0$, which holds for all $y\in V$. Note that
  $Tx\in V$, and thus $\ipr{Tx}{Tx} = 0$, which holds if and only if
  $Tx=0$ for all $x\in V$. Therefore, $T = \bm{0}$.
\end{proof}

\begin{lemma}
  If $\FF = \CC$ and $\ipr{Tx}{x}\in\RR$ for all $x\in V$, then $T = T^{*}$.
\end{lemma}
\begin{proof}
  \hfill

  Take $x\in V$.
  
  Note that, by definition of an adjoint,
  \begin{equation}
    \label{eq:1}
    \ipr{Tx}{x}=\ipr{x}{T^{*}x}.
  \end{equation}

  Moreover, since $\ipr{Tx}{x}\in \RR$, then

  \begin{equation}
    \label{eq:2}
  \ipr{Tx}{x}=\ol{\ipr{x}{Tx}} = \ipr{x}{Tx}.
  \end{equation}

  Therefore, subtracting (\ref{eq:2}) from (\ref{eq:1}), we obtain that
  \begin{equation}
    \ipr{x}{T^{*}x-Tx} = \ipr{x}{(T^{*}-T)x} = \ipr{(T^{*}-T)^{*}x}{x}=0.
  \end{equation}

  Therefore, since $\FF = \CC$, we get by \ref{sec:1},
  $(T^{*}-T)^{*} =\bm{0}$.

  Then $T - T^{*}=\bm{0}$, and hence $T = T^{*}$.
\end{proof}

\begin{lemma}
  If $\FF = \RR$, then $\ipr{Tx}{x}=0$ for all $x\in V$ if and only if $T^{*} = -T$.
\end{lemma}

\begin{proof}
  \hfill

  Since $\FF = \RR$, then
  $\ipr{Tx}{x}=\ol{\ipr{x}{Tx}} = \ipr{x}{Tx}$, and by definition of
  an adjoint,

  $\ipr{Tx}{x}= \ipr{x}{T^{*}x}$. Thus, $\ipr{x}{Tx} = \ipr{x}{T^{*}x}$.

  Suppose first $\ipr{Tx}{x}=0$. Then
  $\ipr{x}{Tx} = \ipr{x}{T^{*}x} = 0$, and thus, summing two
  equations, we obtain that, for all $x\in V$,

  \begin{align}
    \label{eq:3}
    \ipr{x}{(T+T^{*})x} = 0.
  \end{align}

  Note that this holds for an arbitrary $x\in V$. In particular,
  (\ref{eq:3}) holds for $x=(T+T^{*}x)$, and hence $(T+T^{*})x = 0$ for
  all $x\in\RR$, which means that $T +T^{*}= \bm{0}$, and thus $T^{*} = -T$.

  Suppose now $T^{*}=-T$. Therefore, $T^{*}+T = \bm{0}$, which means
  that for all $x\in V$, $(T^{*}+T)x = 0$. Therefore, since
  $V^{\bot} = \set{0}$, then for all $x\in V$
  $\ipr{(T^{*}+T)x}{x} = 0$. Therefore, $\ipr{T^{*}x}{x}=-\ipr{Tx}{x}$.

  Note that $\ipr{Tx}{x}= \ipr{x}{T^{*}x}$ by the derivation above,
  and thus $\ipr{Tx}{x} = -\ipr{Tx}{x}$, which means that
  $2\ipr{Tx}{x}=0$ and hence $\ipr{Tx}{x}=0$.
\end{proof}
\end{document}