
%%% Local Variables:
%%% mode: latex
%%% TeX-master: t
%%% End:

\documentclass[11pt]{scrartcl}
\usepackage[beaue, pset, anon]{masty}
\pSet{\hw{MAT247}{V}{4}}
\usepackage{lineno}
% ----------------------------------------------------------------------
% Page setup
% ----------------------------------------------------------------------

\pagenumbering{gobble}

% ----------------------------------------------------------------------
% Custom commands
% ----------------------------------------------------------------------

% alignment

\newcommand*{\LongestHence}{$\Rightarrow$}% function name
\newcommand*{\LongestName}{$f_o(-x)+f_e(-x)$}% function name
\newcommand*{\LongestValue}{$(-a)x +(-a)(-y)$}% function value
\newcommand*{\LongestText}{\defi}%

\newlength{\LargestHenceSize}%
\newlength{\LargestNameSize}%
\newlength{\LargestValueSize}%
\newlength{\LargestTextSize}%

\settowidth{\LargestHenceSize}{\LongestHence}%
\settowidth{\LargestNameSize}{\LongestName}%
\settowidth{\LargestValueSize}{\LongestValue}%
\settowidth{\LargestTextSize}{\LongestText}%

% Choose alignment of the various elements here: [r], [l] or [c]

\newcommand*{\mbh}[1]{{\makebox[\LargestHenceSize][r]{\ensuremath{#1}}}}%
\newcommand*{\mbn}[1]{{\makebox[\LargestNameSize][r]{\ensuremath{#1}}}}%
\newcommand*{\mbv}[1]{\ensuremath{\makebox[\LargestValueSize][r]{\ensuremath{#1}}}}%
\newcommand*{\mbt}[1]{\makebox[\LargestTextSize][l]{#1}}%

\newcommand{\R}[1]{\label{#1}\linelabel{#1}}
\newcommand{\lr}[1]{line~\lineref{#1}}

% ----------------------------------------------------------------------
% Launch!
% ----------------------------------------------------------------------

\begin{document}

\begin{problem*}
  Find an orthogonal matrix in $M_{3\times 3}(\RR)$ with first row $(\nicefrac{2}{3},−\nicefrac{1}{3},−\nicefrac{2}{3})$.
\end{problem*}
\begin{soln}
  \hfill

  Suppose $A = 
\frac{1}{3}  \begin{pmatrix}
  2 & -1 & -2 \\
  a & b  & c  \\
  d & e  & f
\end{pmatrix}$, with $a, b, c, d, e, f\in \RR$, is such that $A$ is an
orthogonal matrix. Therefore, $A^{*}A = AA^{*} =I$.

Note that $A^{*} = 
\frac{1}{3}  \begin{pmatrix}
  2 & a & d \\
  -1 & b  & e  \\
  -2 & c  & f
\end{pmatrix}$

Therefore,

\begin{align}
  AA^{*}= \frac{1}{9}
  \begin{pmatrix}
    9           & 2a-b-2c     & 2d-e-2f  \\
    2a-b-2c     & a^2+b^2+c^2 & ad+be+cf \\
    2d-e-2f     & ad+be+cf    & d^2+e^2+f^2
  \end{pmatrix}
\end{align}
Moreover,
\begin{align}
  A^{*}A= \frac{1}{9}
  \begin{pmatrix}
    4+a^2+d^2   & -2+ab+de    & -4+ac+df \\
    -2+ab+de    & 1+b^2+e^2   & 2+bc+ef  \\
    -4+ac+df    & 2+bc+ef     & 4+c^2+f^2
  \end{pmatrix}
\end{align}
Since $A^{*}A = AA^{*}=I$, then, from the diagonals of $A$ and $A^{*}$ we obtain
\begin{equation}
  \begin{cases}
    4+a^2+d^2   & =9                     \\
    a^2+b^2+c^2 & =1+b^2+e^2             \\
    d^2+e^2+f^2 & =4+c^2+f^2
  \end{cases},
\end{equation}
and therefore
\begin{equation}
  \begin{cases}
    a^2+d^2     & =5                     \\
    a^2+c^2     & =1+e^2                 \\
    d^2+e^2     & =4+c^2
  \end{cases}
\end{equation}

Suppose $a=1$, $b=-2$, $c=2$, $d=2$, $e = 2$, $f = 1$.

Therefore,
\begin{align}
  AA^{*} &= 
  \frac{1}{9}\begin{pmatrix}
    9 & 2 + 2 - 4 & 4 -2 -2\\
    2+2-4 & 1 + 4 +4 & 2-4+2\\
    4-2-2 & 2-4+2 & 4+4+1
  \end{pmatrix}\\
         &=
   \frac{1}{9}\begin{pmatrix}
    9 & 0 & 0\\
    0 & 9 & 0\\
    0 & 0 & 9
  \end{pmatrix}\\
         &= I\\
  &=\frac{1}{9}\begin{pmatrix}
    4+1+4 & -2 - 2 + 4 & -4 +2+2\\
    -2-2+4 & 1 + 4 +4 & 2-4+2\\
    -4+2+2 & 2-4+2 & 4+4+1
  \end{pmatrix}\\
  &= A^{*}A
\end{align}
% \begin{equation}
%   \begin{cases}
%     1^2+2^2&=5\\
%     1^2+3^2&=1+3^2\\
%     2^2+3^2=4+3^2
%   \end{cases}
% \end{equation}
Therefore, the matrix $A$ is orthogonal if
\begin{equation*}
  A=\frac{1}{3}
  \begin{pmatrix}
    2 & - 1 & 2\\
    1 & -2 & 2\\
    2 & 2 & 1
  \end{pmatrix}
\end{equation*}.




\end{soln}

\end{document}