
%%% Local Variables:
%%% mode: latex
%%% TeX-master: t
%%% End:

\documentclass[11pt]{scrartcl}
\usepackage[beaue, pset, anon]{masty}
\pSet{\hw{MAT247}{X}{3}}
\usepackage{lineno}
% ----------------------------------------------------------------------
% Page setup
% ----------------------------------------------------------------------

\pagenumbering{gobble}

% ----------------------------------------------------------------------
% Custom commands
% ----------------------------------------------------------------------

% alignment

\newcommand*{\LongestHence}{$\Rightarrow$}% function name
\newcommand*{\LongestName}{$f_o(-x)+f_e(-x)$}% function name
\newcommand*{\LongestValue}{$(-a)x +(-a)(-y)$}% function value
\newcommand*{\LongestText}{\defi}%

\newlength{\LargestHenceSize}%
\newlength{\LargestNameSize}%
\newlength{\LargestValueSize}%
\newlength{\LargestTextSize}%

\settowidth{\LargestHenceSize}{\LongestHence}%
\settowidth{\LargestNameSize}{\LongestName}%
\settowidth{\LargestValueSize}{\LongestValue}%
\settowidth{\LargestTextSize}{\LongestText}%

% Choose alignment of the various elements here: [r], [l] or [c]

\newcommand*{\mbh}[1]{{\makebox[\LargestHenceSize][r]{\ensuremath{#1}}}}%
\newcommand*{\mbn}[1]{{\makebox[\LargestNameSize][r]{\ensuremath{#1}}}}%
\newcommand*{\mbv}[1]{\ensuremath{\makebox[\LargestValueSize][r]{\ensuremath{#1}}}}%
\newcommand*{\mbt}[1]{\makebox[\LargestTextSize][l]{#1}}%

\newcommand{\R}[1]{\label{#1}\linelabel{#1}}
\newcommand{\lr}[1]{line~\lineref{#1}}

% ----------------------------------------------------------------------
% Launch!
% ----------------------------------------------------------------------

\begin{document}
  Suppose that $\FF = \ZZ_2$ and $A=
  \begin{pmatrix}
    0 & 1 & 0 & 1\\
    1 & 0 & 1 & 0\\
    0 & 1 & 1 & 1\\
    1 & 0 & 1 & 1
  \end{pmatrix}
$.
\begin{problem*}
  \hfill

Find a rational canonical form $R$ of $A$.

\end{problem*}

\begin{soln}
  \hfill

  First we find a characteristic polynomial of $A$.

  Note that 
  \begin{equation*}
A- tI = 
  \begin{pmatrix}
    -t & 1 & 0 & 1\\
    1 & -t & 1 & 0\\
    0 & 1 & 1-t & 1\\
    1 & 0 & 1 & 1-t
  \end{pmatrix}.
  \end{equation*}

  Using the Laplacian expansion along the first column, we see that
  \begin{align}
\det(A-tI) =& -t \det
             \begin{pmatrix}
               -t & 1 & 0\\
               1 & 1-t & 1\\
               0 & 1 & 1-t
             \end{pmatrix}\\
                       &- \det
                       \begin{pmatrix}
                         1 & 0 & 1\\
                         1 & 1-t & 1\\
                         0 & 1 & 1-t
                       \end{pmatrix}- \det
                                 \begin{pmatrix}
                                   1 & 0 & 1\\
                                   -t & 1 & 0\\
                                   1 & 1-t & 1
                                 \end{pmatrix}\\
    =& -t \left( -t[(1-t)^2-1] - [1-t-0] \right)\\
            & - \left( (1-t)^2-1 -1(0-1)\right)\\
            &- \left( 1 + (-t(1-t)-1) \right)\\
    =& -t(-t(-t)(2-t) +t-1) -\left( (-t) (2-t) +1\right)+t-t^2\\
    =& -t(2t^2-t^3+t-1) -(-2t+t^2+1)+t-t^2\\
    =&\ t^4-2t^3-t^2+t+2t-t^2-1+t-t^2\\
    =&\ t^4-2t^3-3t^2+4t-1\\
    =&\ t^4-3t^3+t^2+t^3-4t^2+4t-1\\
    =&\ t^2(t^2-3t+1)+ t^3-3t^2+t -t^2+3t-1\\
    =&\ t^2(t^2-3t+1)+t(t^2-3t+1) - (t^2-3t+1)\\
    =&\ (t^2-3t+1)(t^2+t-1)
  \end{align}

  Since $\FF = \ZZ_2$, then $f(t) = \det(A-tI) = (t^2+t+1)^2$.

  Note that 

  \begin{align}
    A^2 = 
  \begin{pmatrix}
    0 & 1 & 0 & 1\\
    1 & 0 & 1 & 0\\
    0 & 1 & 1 & 1\\
    1 & 0 & 1 & 1
  \end{pmatrix}
  \begin{pmatrix}
    0 & 1 & 0 & 1\\
    1 & 0 & 1 & 0\\
    0 & 1 & 1 & 1\\
    1 & 0 & 1 & 1
  \end{pmatrix}\\
=
\begin{pmatrix}
1 + 1 & 0 & 1+1 & 1\\
0 & 1+1 & 1 & 1 + 1\\
1 + 1 & 1 & 1+1+1 & 1+1\\
1 +1 & 1 +1 & 1 +1 & 1 + 1 +1
\end{pmatrix}\\
=
\begin{pmatrix}
0 & 0 & 0 & 1\\
0 & 0 & 1 & 0\\
0 & 1 & 1 & 0\\
1 & 0 & 0 & 1
\end{pmatrix},
  \end{align}
and thus 
\begin{align}
A^2 + A + I &= 
\begin{pmatrix}
0 & 0 & 0 & 1\\
0 & 0 & 1 & 0\\
0 & 1 & 1 & 0\\
1 & 0 & 0 & 1
\end{pmatrix}
+
  \begin{pmatrix}
    0 & 1 & 0 & 1\\
    1 & 0 & 1 & 0\\
    0 & 1 & 1 & 1\\
    1 & 0 & 1 & 1
  \end{pmatrix}
+
                \begin{pmatrix}
                  1 & 0 & 0 & 0\\
                  0 & 1 & 0 & 0\\
                  0 & 0 & 1 & 0\\
                  0 & 0 & 0 & 1
                \end{pmatrix}\\
            &=
 \begin{pmatrix}
   1 & 1 & 0 & 0\\
   1 & 1 & 0 & 0\\
   0 & 0 & 1 & 1\\
   0 & 0 & 1 & 1
 \end{pmatrix}.
\end{align}

Since $A^2+A+I \neq 0$, while the only divisor of the the minimal
polynomial $p(t)$ is $t^2+t+1$, we deduce that
$p(t)=f(t) = (t^2+t+1)^2 = t^4+t^2+1$.

Therefore, there exists a canonical basis $\beta$ such that
\begin{equation*}
[A]_{\beta} = R =
\begin{pmatrix}
0 & 0 & 0 & 1\\
1 & 0 & 0 & 0\\
0 & 1 & 0 & 1\\
0 & 0 & 1 & 0
\end{pmatrix},
\end{equation*}
where the signs were omitted since $-1 = 1$.

\end{soln}

\begin{problem*}
  \hfill

Find an invertible matrix $Q$ such that $R = Q^{-1}AQ$.
\end{problem*}

\begin{soln}
  \hfill

First we find the basis of $K_{\phi}$, where $\phi = t^2+t+1$.

Note that, by Theorem 7.18, since $p(t) = (t^2+t+1)^2$ and $t^2+t+1$
is irreducible, then $K_{\phi} = \ker \phi(A)^2$.
Since $\phi A = 
 \begin{pmatrix}
   1 & 1 & 0 & 0\\
   1 & 1 & 0 & 0\\
   0 & 0 & 1 & 1\\
   0 & 0 & 1 & 1
 \end{pmatrix}
 $, we know that $phi(A)^2 = 0$ and thus
 $\ker \phi(A)^2 = V = K_{\phi}$.

 We now look for the cycle basis of $K_{\phi}= V$, which has a length
 of $\dim V = 4$.

 Take $v = \cv{1;0;0;0}$.

 Therefore, $Av = \cv{0;1;0;1}$, $A(Av) = A^2v = \cv{0;0;0;1}$ and
 $A^3v = \cv{1;0;1;1}$. 

 Let
 $\beta = \set{\cv{1;0;0;0}, \cv{0;1;0;1}, \cv{0;0;0;1},
   \cv{1;0;1;1}}$.

 Note that, since $\phi(A)^2x = 0$ for any $x\in \beta$, then
 $\spn \beta \suq K_{\phi}$.

 We now prove that $\beta$ is linearly independent:
 \begin{align}
   \begin{matreq}{cccc|c}
1 & 0 & 0 & 1 & 0\\
0 & 1 & 0 & 0 & 0\\
0 & 0 & 0 & 1 & 0\\
0 & 1 & 1 & 1 & 0
   \end{matreq}
  &\ras\\
   R_1\to R_1-R_3, R_4\to R_4-R_{2}-R_{3}  \ &\ras
   \begin{matreq}{cccc|c}
1 & 0 & 0 & 0 & 0\\
0 & 1 & 0 & 0 & 0\\
0 & 0 & 0 & 1 & 0\\
0 & 0 & 1 & 0 & 0
   \end{matreq}.
 \end{align}

 Therefore, $\beta$ is linearly independent. 

 Since $\abs{\beta} = 4$ and $\spn \beta \suq K_{\phi}$, then $\beta$
 is a cyclic basis of $K_{\phi} = V$.

Let $Q = 
\begin{pmatrix}
1 & 0 & 0 & 1\\
0 & 1 & 0 & 0\\
0 & 0 & 0 & 1\\
0 & 1 & 1 & 1
\end{pmatrix}$.

We invert $Q$:
\begin{align}
  \begin{matreq}{cccc|cccc}
1 & 0 & 0 & 1 & 1 & 0 & 0 & 0\\
0 & 1 & 0 & 0 & 0 & 1 & 0 & 0\\
0 & 0 & 0 & 1 & 0 & 0 & 1 & 0\\
0 & 1 & 1 & 1 & 0 & 0 & 0 & 1
  \end{matreq}
 &\ras\\
  \shortstack[l]{$R_1\to R_1-R_3$, \\
$R_4\to R_4-R_{2}-R_{3}$  }
  \ &\ras
  \begin{matreq}{cccc|cccc}
1 & 0 & 0 & 0 & 1 & 0 & 1 & 0\\
0 & 1 & 0 & 0 & 0 & 1 & 0 & 0\\
0 & 0 & 0 & 1 & 0 & 0 & 1 & 0\\
0 & 0 & 1 & 0 & 0 & 1 & 1 & 1
  \end{matreq}\\
R_3 \lra R_4 \ &\ras 
  \begin{matreq}{cccc|cccc}
1 & 0 & 0 & 0 & 1 & 0 & 1 & 0\\
0 & 1 & 0 & 0 & 0 & 1 & 0 & 0\\
0 & 0 & 1 & 0 & 0 & 1 & 1 & 1\\
0 & 0 & 0 & 1 & 0 & 0 & 1 & 0
  \end{matreq}
\end{align}

Thus, $Q^{-1} = 
\begin{pmatrix}
1 & 0 & 1 & 0\\
0 & 1 & 0 & 0\\
0 & 1 & 1 & 1\\
0 & 0 & 1 & 0
\end{pmatrix}
$.

By the change-of-matrix formula, we know that $R = Q^{-1}A Q$, since
$\beta$ is a cycle basis of $K_{\phi}$ and thus a rational canonical
basis.
\end{soln}
\begin{problem*}
  \hfill

Find an $L_A$-invariant subspace $W\suq \FF^4$ of dimension 2.
\end{problem*}
\begin{soln}
  \hfill

  Let $\gamma = \set{\cv{0;0;0;1}, \cv{1;0;1;1}}\suq \beta$, and let $T = L_A$.

  Note that $\gamma$ is linearly independent, since $\beta$ is linearly independent.

  Let $W = \spn \gamma$.

  Since $T\cv{0;0;0;1} = \cv{1;0;1;1} \in W$ and
  $T \cv{1;0;1;1} = \cv{0;0;0;0} \in W$, then $W$ is $T$-invariant
  ($T$ is defined by its action on a basis).
\end{soln}

\end{document}
