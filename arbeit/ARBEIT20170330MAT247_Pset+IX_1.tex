
%%% Local Variables:
%%% mode: latex
%%% TeX-master: t
%%% End:

\documentclass[11pt]{scrartcl}
\usepackage[beaue, pset, anon]{masty}
\pSet{\hw{MAT247}{IX}{1}}
\usepackage{lineno}
% ----------------------------------------------------------------------
% Page setup
% ----------------------------------------------------------------------

\pagenumbering{gobble}

% ----------------------------------------------------------------------
% Custom commands
% ----------------------------------------------------------------------

% alignment

\newcommand*{\LongestHence}{$\Rightarrow$}% function name
\newcommand*{\LongestName}{$f_o(-x)+f_e(-x)$}% function name
\newcommand*{\LongestValue}{$(-a)x +(-a)(-y)$}% function value
\newcommand*{\LongestText}{\defi}%

\newlength{\LargestHenceSize}%
\newlength{\LargestNameSize}%
\newlength{\LargestValueSize}%
\newlength{\LargestTextSize}%

\settowidth{\LargestHenceSize}{\LongestHence}%
\settowidth{\LargestNameSize}{\LongestName}%
\settowidth{\LargestValueSize}{\LongestValue}%
\settowidth{\LargestTextSize}{\LongestText}%

% Choose alignment of the various elements here: [r], [l] or [c]

\newcommand*{\mbh}[1]{{\makebox[\LargestHenceSize][r]{\ensuremath{#1}}}}%
\newcommand*{\mbn}[1]{{\makebox[\LargestNameSize][r]{\ensuremath{#1}}}}%
\newcommand*{\mbv}[1]{\ensuremath{\makebox[\LargestValueSize][r]{\ensuremath{#1}}}}%
\newcommand*{\mbt}[1]{\makebox[\LargestTextSize][l]{#1}}%

\newcommand{\R}[1]{\label{#1}\linelabel{#1}}
\newcommand{\lr}[1]{line~\lineref{#1}}

% ----------------------------------------------------------------------
% Launch!
% ----------------------------------------------------------------------

\begin{document}

\section{Problem}

\begin{problem*}
\hfill

Suppose that $\FF = \CC$ and $A = \left(\begin{matrix}
    3 \\
    1&4 \\
    -1 & 1 & 3 \\
    2&0&1&3 \\
    &&&&4\\ &&&&2&3
\end{matrix}\right)$, 

  where all empty matrix entries are zeros.

  Find a Jordan canonical form for $A$ and find a basis of of $K_3$
  that is a disjoint union of cycles of generalised eigenvectors of
  $L_A$.
\end{problem*}
\begin{soln}
  \hfill

  Since $(A-\lambda I)^T = A^T - \lambda I^T = A^T - \lambda I$ and $\det(A-\lambda I) = \det(A-\lambda I)^T$, we have that $A^T$ and $A$ have the same eigenvalues.

  Since $A^T$ is upper-triangular while $\FF = \CC$, all the
  eigenvalues are on the eigenvalues.

  Using the Laplacian expansion on $A$ to obtain the characteristic
  polynomial of $A^T$ and $A$, we obtain that $f(t) = (t-3)^4(t-4)^2$.

  Note that 
  \begin{align}
(A-3I) &= 
  \begin{pmatrix}
    0  & 0 & 0 & 0 & 0 & 0 \\
    1  & 1 & 0 & 0 & 0 & 0 \\
    -1 & 1 & 0 & 0 & 0 & 0 \\
    2  & 0 & 1 & 0 & 0 & 0 \\
    0  & 0 & 0 & 0 & 1 & 0 \\
    0  & 0 & 0 & 0 & 2 & 0
  \end{pmatrix}\\
    \shortstack[l]{
$R_3 \to \frac{1}{2}(R_3+R_2)$ \\
$R_6 \to R_6-2R_{5}$}
\ &\ras 
  \begin{pmatrix}
    0  & 0 & 0 & 0 & 0 & 0 \\
    1  & 1 & 0 & 0 & 0 & 0 \\
    0 & 1 & 0 & 0 & 0 & 0 \\
    2  & 0 & 1 & 0 & 0 & 0 \\
    0  & 0 & 0 & 0 & 1 & 0 \\
    0  & 0 & 0 & 0 & 0 & 0
  \end{pmatrix}\\
R_2\to R_2-R_{3}
\ &\ras 
  \begin{pmatrix}
    0  & 0 & 0 & 0 & 0 & 0 \\
    1  & 0 & 0 & 0 & 0 & 0 \\
    0 & 1 & 0 & 0 & 0 & 0 \\
    2  & 0 & 1 & 0 & 0 & 0 \\
    0  & 0 & 0 & 0 & 1 & 0 \\
    0  & 0 & 0 & 0 & 0 & 0
  \end{pmatrix}\\
R_4\to R_4-R_{2}
\ &\ras 
  \begin{pmatrix}
    0  & 0 & 0 & 0 & 0 & 0 \\
    1  & 0 & 0 & 0 & 0 & 0 \\
    0 & 1 & 0 & 0 & 0 & 0 \\
    0  & 0 & 1 & 0 & 0 & 0 \\
    0  & 0 & 0 & 0 & 1 & 0 \\
    0  & 0 & 0 & 0 & 0 & 0
  \end{pmatrix}
\label{eq:1}
  \end{align}

Therefore, $\rank(A-3I) = 4$, and thus $\nll (A-3I) = 2$, which means that there are 2 columns in the dot diagram.

Moreover, 

  \begin{align}
(A-3I)^2 &= 
  \begin{pmatrix}
    0  & 0 & 0 & 0 & 0 & 0 \\
    1  & 1 & 0 & 0 & 0 & 0 \\
    -1 & 1 & 0 & 0 & 0 & 0 \\
    2  & 0 & 1 & 0 & 0 & 0 \\
    0  & 0 & 0 & 0 & 1 & 0 \\
    0  & 0 & 0 & 0 & 2 & 0
  \end{pmatrix}
  \begin{pmatrix}
    0  & 0 & 0 & 0 & 0 & 0 \\
    1  & 1 & 0 & 0 & 0 & 0 \\
    -1 & 1 & 0 & 0 & 0 & 0 \\
    2  & 0 & 1 & 0 & 0 & 0 \\
    0  & 0 & 0 & 0 & 1 & 0 \\
    0  & 0 & 0 & 0 & 2 & 0
  \end{pmatrix}\\
  &= \label{eq:2}
  \begin{pmatrix}
    0  & 0 & 0 & 0 & 0 & 0 \\
    1  & 1 & 0 & 0 & 0 & 0 \\
    1 & 1 & 0 & 0 & 0 & 0 \\
    -1  & 1 & 0 & 0 & 0 & 0 \\
    0  & 0 & 0 & 0 & 1 & 0 \\
    0  & 0 & 0 & 0 & 2 & 0
  \end{pmatrix}\\
    \shortstack[l]{
$R_3\to R_3-R_2$\\
$$ \\
$R_4 \to \frac{1}{2}(R_4+R_2)$\\
$R_6 \to R_6-2R_{5}$}
\ &\ras 
  \begin{pmatrix}
    0  & 0 & 0 & 0 & 0 & 0 \\
    1  & 1 & 0 & 0 & 0 & 0 \\
    0 & 0 & 0 & 0 & 0 & 0 \\
    0  & 1 & 0 & 0 & 0 & 0 \\
    0  & 0 & 0 & 0 & 1 & 0 \\
    0  & 0 & 0 & 0 & 0 & 0
  \end{pmatrix}
  \end{align}

which means that $\nll (A-3I)^2= 3$.

Since $3-2 = 1$, there is only one dot in the second row of the dot diagram, which means that the second column must have only one dot.

Therefore, the blocks are $3\times 3$ and $1\times 1$.

Consider now 

\begin{align}
(A- 4I) &=
  \begin{pmatrix}
    -1  & 0 & 0 & 0 & 0 & 0 \\
    1  & 0 & 0 & 0 & 0 & 0 \\
    -1 & 1 & -1 & 0 & 0 & 0 \\
    2  & 0 & 1 & -1 & 0 & 0 \\
    0  & 0 & 0 & 0 & 0 & 0 \\
    0  & 0 & 0 & 0 & 2 & -1
  \end{pmatrix}\\
  \shortstack[l]{ 
$R_1\to R_1+R_2$\\
$R_3\to R_3+R_2$\\
$R_4\to R_4-2R_2$ }
\ &\ras 
  \begin{pmatrix}
    0  & 0 & 0 & 0 & 0 & 0 \\
    1  & 0 & 0 & 0 & 0 & 0 \\
    0 & 1 & -1 & 0 & 0 & 0 \\
    0  & 0 & 1 & -1 & 0 & 0 \\
    0  & 0 & 0 & 0 & 0 & 0 \\
    0  & 0 & 0 & 0 & 2 & -1
  \end{pmatrix}\\
\end{align}

Thus, $\nll(A-4I) = 2$, and therefore there are two columns in the corresponding dot diagram.

\begin{align}
(A- 4I)^{2} &=
  \begin{pmatrix}
    -1  & 0 & 0 & 0 & 0 & 0 \\
    1  & 0 & 0 & 0 & 0 & 0 \\
    -1 & 1 & -1 & 0 & 0 & 0 \\
    2  & 0 & 1 & -1 & 0 & 0 \\
    0  & 0 & 0 & 0 & 0 & 0 \\
    0  & 0 & 0 & 0 & 2 & -1
  \end{pmatrix}
  \begin{pmatrix}
    -1  & 0 & 0 & 0 & 0 & 0 \\
    1  & 0 & 0 & 0 & 0 & 0 \\
    -1 & 1 & -1 & 0 & 0 & 0 \\
    2  & 0 & 1 & -1 & 0 & 0 \\
    0  & 0 & 0 & 0 & 0 & 0 \\
    0  & 0 & 0 & 0 & 2 & -1
  \end{pmatrix}\\
&=
  \begin{pmatrix}
    1  & 0 & 0 & 0 & 0 & 0 \\
    -1  & 0 & 0 & 0 & 0 & 0 \\
    3 & -1 & 1 & 0 & 0 & 0 \\
    -5  & 1 & -2 & 1 & 0 & 0 \\
    0  & 0 & 0 & 0 & 0 & 0 \\
    0  & 0 & 0 & 0 & -2 & 1
  \end{pmatrix}\\
  \shortstack[l]{ 
$R_2\to R_1+R_2$\\
$R_3\to R_3-3R_1$\\
$R_4\to R_4+5R_1$ }
\ &\ras 
  \begin{pmatrix}
    1  & 0 & 0 & 0 & 0 & 0 \\
    0  & 0 & 0 & 0 & 0 & 0 \\
    0 & -1 & 1 & 0 & 0 & 0 \\
    0  & 1 & -2 & 1 & 0 & 0 \\
    0  & 0 & 0 & 0 & 0 & 0 \\
    0  & 0 & 0 & 0 & -2 & 1
  \end{pmatrix}\\
R_4 \to R_4 + R_3\ &\ras
  \begin{pmatrix}
    1  & 0 & 0 & 0 & 0 & 0 \\
    0  & 0 & 0 & 0 & 0 & 0 \\
    0 & -1 & 1 & 0 & 0 & 0 \\
    0  & 0 & -1 & 1 & 0 & 0 \\
    0  & 0 & 0 & 0 & 0 & 0 \\
    0  & 0 & 0 & 0 & -2 & 1
  \end{pmatrix}
\end{align}

Thus, $\nll(A-4I)^2 = 2$, and there are two $2-2=0$ dots in the second row of the dot diagram, which means that there are two $1\times1$ Jordan blocks corresponding to $\lambda = 4$.

Hence, $[A]_{\beta} = 
\begin{pmatrix}
    4  & 0 & 0 & 0 & 0 & 0 \\
    0  & 4 & 0 & 0 & 0 & 0 \\
    0  & 0 & 3 & 1 & 0 & 0 \\
    0  & 0 & 0 & 3 & 1 & 0 \\
    0  & 0 & 0 & 0 & 3 & 0 \\
    0  & 0 & 0 & 0 & 0 & 3 \\
\end{pmatrix}$.

To find a cycle basis for $K_3$, note that

\begin{align}
\label{eq:3}
(A-3I)^3 & =
 \begin{pmatrix}
    0    & 0 & 0 & 0 & 0 & 0 \\
    1    & 1 & 0 & 0 & 0 & 0 \\
    -1   & 1 & 0 & 0 & 0 & 0 \\
    2    & 0 & 1 & 0 & 0 & 0 \\
    0    & 0 & 0 & 0 & 1 & 0 \\
    0    & 0 & 0 & 0 & 2 & 0
  \end{pmatrix}
  \begin{pmatrix}
    0    & 0 & 0 & 0 & 0 & 0 \\
    1    & 1 & 0 & 0 & 0 & 0 \\
    1    & 1 & 0 & 0 & 0 & 0 \\
    -1   & 1 & 0 & 0 & 0 & 0 \\
    0    & 0 & 0 & 0 & 1 & 0 \\
    0    & 0 & 0 & 0 & 2 & 0
  \end{pmatrix}              \\
         & =
  \begin{pmatrix}
    0    & 0 & 0 & 0 & 0 & 0 \\
    1    & 1 & 0 & 0 & 0 & 0 \\
    1    & 1 & 0 & 0 & 0 & 0 \\
    1   & 1 & 0 & 0 & 0 & 0 \\
    0    & 0 & 0 & 0 & 1 & 0 \\
    0    & 0 & 0 & 0 & 2 & 0
  \end{pmatrix}.
\end{align}

By the equation (\ref{eq:1}) we have that $u=v=w=y=0$, and
$\ker (A-3I)$ is spanned by $\cv{0;0;0;1;0;0}$ and $\cv{0;0;0;0;0;1}$.

We know that there are two cycles, one of length 1 and the other of length 3.

Therefore, there exists $p = \cv{u;v;w;x;y;z}\in V$ such that $(T-3I)^3=0$ but $(T-3I)^2\neq 0$.

In this way, from (\ref{eq:2}), at least one of $u+v$, $-u+v$ or $y$ is
nonzero, while from (\ref{eq:3}) $u+v = 0$ and $y = 0$. Therefore, $v\neq 0$, $y=0$ and $u=-v$.

Hence, $p = \cv{u;-u;w;x;0;z}$.

Take $p = \cv{-1;1;2;0;0}$.

Then $(A-3I)p = \cv{0;0;2;0;0;0}$, which is an eigenvalue of $(A-3I)^2$, because $(A-3I)^3p = 0$.

$(A-3I)^2p = \cv{0;0;0;2;0;0}$, which is an eigenvalue of $(A-3I)$, because $(A-3I)^3p =0$. 

If there exist $a_1, a_2, a_3$ such that
\begin{equation*}
  a_1\cv{-1;1;2;0;0;0} +a_2\cv{0;0;2;0;0;0} +a_3\cv{0;0;0;2;0;0} = 0,
\end{equation*}

then from the first row we have that $a_1 = 0$, from the fourth we obtain that $a_3=0$, and thus $a_2=0$, which means that $p$ generates a cycle basis of length $3$.

Now, take $q = \cv{0;0;0;0;0;1}$. We have already shown that it is an
eigenvector of $(A-3I)$. Since it is also not in the span of the cycle
basis generated by $p$, because each element of such a basis has the
last row equal to zero, while $\dim K_3 = 4$ from the characteristic
polynomial, then we have an independent list of the right length, and thus a Jordan canonical basis $\beta$ for $\dim K_3$:
\begin{equation*}
  \beta =\set{\cv{-1;1;2;0;0;0}, \cv{0;0;2;0;0;0}, \cv{0;0;0;2;0;0}, \cv{0;0;0;0;0;1}}
\end{equation*}





% Now we find $(A-3I)^3p\in \ker(T-3I)^2\cap \img(T-3I)^3$.

% Both of these vectors, if  $(A-3I)^3$ is applied, yield 0.

% Take $p = \cv{1;-1;1;1;0;1}$. Then $p$ is an eigenvector of $(A-3I)^3$ such that $(A-3I)^3p = 0$.

% We check $(A-3I)p = \cv{0;0;-2;3;0;0}$



% Now we want to find $p$ such that $p \in$
\end{soln}


\end{document}
