
%%% Local Variables:
%%% mode: latex
%%% TeX-master: t
%%% End:

\documentclass[11pt]{scrartcl}
\usepackage[beaue, pset, anon]{masty}
\pSet{\hw{MAT247}{IX}{4}}
\usepackage{lineno}
% ----------------------------------------------------------------------
% Page setup
% ----------------------------------------------------------------------

\pagenumbering{gobble}

% ----------------------------------------------------------------------
% Custom commands
% ----------------------------------------------------------------------

% alignment

\newcommand*{\LongestHence}{$\Rightarrow$}% function name
\newcommand*{\LongestName}{$f_o(-x)+f_e(-x)$}% function name
\newcommand*{\LongestValue}{$(-a)x +(-a)(-y)$}% function value
\newcommand*{\LongestText}{\defi}%

\newlength{\LargestHenceSize}%
\newlength{\LargestNameSize}%
\newlength{\LargestValueSize}%
\newlength{\LargestTextSize}%

\settowidth{\LargestHenceSize}{\LongestHence}%
\settowidth{\LargestNameSize}{\LongestName}%
\settowidth{\LargestValueSize}{\LongestValue}%
\settowidth{\LargestTextSize}{\LongestText}%

% Choose alignment of the various elements here: [r], [l] or [c]

\newcommand*{\mbh}[1]{{\makebox[\LargestHenceSize][r]{\ensuremath{#1}}}}%
\newcommand*{\mbn}[1]{{\makebox[\LargestNameSize][r]{\ensuremath{#1}}}}%
\newcommand*{\mbv}[1]{\ensuremath{\makebox[\LargestValueSize][r]{\ensuremath{#1}}}}%
\newcommand*{\mbt}[1]{\makebox[\LargestTextSize][l]{#1}}%

\newcommand{\R}[1]{\label{#1}\linelabel{#1}}
\newcommand{\lr}[1]{line~\lineref{#1}}

% ----------------------------------------------------------------------
% Launch!
% ----------------------------------------------------------------------

\begin{document}

\begin{problem*}
  \hfill

  Suppose that $A \in M_{n\times n}(\FF)$ is such that its
  characteristic polynomial splits.

  Then $A$ and $A^t$ have the same Jordan canonical form and
  $A\sim A^t$.


\end{problem*}

\begin{soln}
  \hfill

  Let $A \in M_{n\times n}(\FF)$ be such that its characteristic
  polynomial $f(t)$ splits.

  Note that for any $\lambda\in \FF$,
  $(A-\lambda I)^t = A^t -\lambda I^t = A^t - \lambda I$.

  Therefore, $\det(A^t- \lambda I) = \det (A - \lambda I)$, and thus
  the characteristic polynomial of $A^t$ is $f(t)$.

  Thus, $A^t$ has the same eigenvalues as $A$.

  % Let $\Lambda$ be the set of distinct eigenvalues of $A$ and $A^t$.

  Let $K_{\lambda}$ be a generalised eigenspace of $A$ corresponding
  to an eigenvalue $\lambda$ and $K'_{\lambda_{i}}$ be a generalised
  eigenspace of $A^t$ corresponding to $\lambda$.

  % Since $A$ and $A^t$ have the same characteristic polynomial, by
  % Theorem 7.4 we know that $\dim K_{\lambda} = \dim K'_{\lambda}$.

  Note that
  $((A- \lambda I)^r)^t = ((A-\lambda I)^t)^r = (A^t-\lambda I)^r$,
  where the first equality holds since for any $P$ and $Q$ in
  $M_{n\times n}(\FF)$ we know that $(PQ)^t = Q^tP^t$ and the second
  follows from the discussion above.

  For any matrix $C \in M_{n\times n}(\FF)$, we know that
  $\rank C = \rank C^t$.

  Therefore,
  $\rank (A-\lambda I)^r = \rank ((A-\lambda)^{r})^t = \rank
  (A^t-\lambda I)^r$ for any $r\in\NN$.

  In particular, $\rank (A^t-\lambda I) = \rank (A-\lambda I)$, and
  thus from the rank-nullity theorem we obtain that
  $\nll (A^t-\lambda I) = \nll (A-\lambda I)$, which means that the
  dot diagrams of $A^t-\lambda I$ and $A-\lambda I$ have the same
  number of columns. Moreover, we can deduce that for $r \geq 2$, we
  have 
  \[\rank(A-\lambda I)^{r-1} - \rank (A-\lambda I)^r = \rank (A^t-\lambda I)^{r-1} - \rank (A^t- \lambda I)^r,\] 

  and thus each row in the dot diagrams of $(A-\lambda I)$ and
  $(A^t-\lambda I)$ have the same number of dots, which means that
  $(A-\lambda I)$ and $(A^t-\lambda I)$ have the same dot diagram.

  Since this dot diagram corresponds to a unique Jordan canonical form
  (up to reordering of blocks), we infer that $A$ and $A^t$
  have the same JCF. 

  Denote this JCF as $J$.

  Since by the change of basis formula there exist invertible matrices
  $B$ and $B'$ such that $A = B^{-1}JB $ and $A^t = B'^{-1}JB'$, we
  deduce that $J = BAB^{-1} = B'A^tB'^{-1}$. 



  

  % From Theorem 2

  % We also know that $\rank(A-\lambda I) = \rank (A- \lambda I)^t$, and
  % thus $\rank (A-\lambda I) = \rank(A^t - \lambda I)$.

  % For $r\in[1, n]\cap \NN$, we prove by induction on $r$ that
  % $\rank (A-\lambda I)^r = \rank (A^t -\lambda I)^r$.

  % We have already shown that the claim holds in case $r = 1$.

  % Suppose now that it holds for all $1 \leq r \leq k-1$.

  % In particular,
  % $\rank (A-\lambda I)^{k-1} = \rank (A^t-\lambda
  % I)^{k-1}$. Therefore, by the rank-nullity theorem we know that $\nll $


%   Since the characteristic polynomial of $A$ and $A^t$ splits, by
%   Theorem 7.3 there exists a basis $\alpha$ of $V$ consisting of the
%   generalised eigenvectors of $A$ and a basis $\beta$ of $V$ consisting of the
%   generalised eigenvectors of $A^t$, corresponding to the distinct eigenvalues in
%   $\Lambda$ so that 
%   \begin{equation*}
% V= \bigoplus_{\lambda_i\in \Lambda} K_{\lambda_i} = \bigoplus_{\lambda_i\in \Lambda} K'_{\lambda_i},
%   \end{equation*}
%   where $K_{\lambda_i}$ are the generalised eigenspaces corresponding
%   to $A$ and $K'_{\lambda_{i}}$ are the generalised eigenspaces
%   corresponding to $A^t$.





  % We now show that for each eigenvalue $\lambda$ we have
  % $\ker(A-\lambda I)^n = \ker (A^t-\lambda I)^n$.

  % and $r\in[1, n]$ we have
  % $\ker (A^t-\lambda I)^r = \ker (A-\lambda I)^r$.

  % Consider the case when $r = 1$.

  % Suppose first that $v\in \ker (A-\lambda I)$. Therefore,
  % $(A-\lambda I) v= 0$, and $(A-\lambda I)|_{\ker (A-\lambda I)} = 0$.

  % Therefore,
  % $((A-\lambda I)|_{\ker (A-\lambda I)})^t = (A-\lambda I)^t|_{\ker
  %   (A-\lambda I)} = 0^t = 0$, which, from the previous discussion,
  % means that $(A^t-\lambda I)|_{\ker (A-\lambda I)} =0$ and thus
  % $\ker(A-\lambda I)\suq \ker (A^t-\lambda I)$.

  % Suppose now that $v\in \ker (A^t-\lambda I) = \ker ((A-\lambda I)^t)$.

  % Hence, $(A-\lambda I)^tv = 0$

  Thus, $A = B^{-1}B'A^tB'^{-1}B$. Since $(B'^{-1}B)^{-1} = B^{-1}B'$,
  taking $Q = B'^{-1}B$ we see that $A = Q^{-1}A^tQ$, and thus $A\sim A^t$.
\end{soln}

\end{document}
