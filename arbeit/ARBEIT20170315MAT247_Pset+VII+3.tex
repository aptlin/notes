%%% Local Variables:
%%% mode: latex
%%% TeX-master: t
%%% End:

\documentclass[11pt]{scrartcl}
\usepackage[beaue, pset, anon]{masty}
\pSet{\hw{MAT247}{VII}{3}}
\usepackage{lineno}
% ----------------------------------------------------------------------
% Page setup
% ----------------------------------------------------------------------

\pagenumbering{gobble}

% ----------------------------------------------------------------------
% Custom commands
% ----------------------------------------------------------------------

% alignment

\newcommand*{\LongestHence}{$\Rightarrow$}% function name
\newcommand*{\LongestName}{$f_o(-x)+f_e(-x)$}% function name
\newcommand*{\LongestValue}{$(-a)x +(-a)(-y)$}% function value
\newcommand*{\LongestText}{\defi}%

\newlength{\LargestHenceSize}%
\newlength{\LargestNameSize}%
\newlength{\LargestValueSize}%
\newlength{\LargestTextSize}%

\settowidth{\LargestHenceSize}{\LongestHence}%
\settowidth{\LargestNameSize}{\LongestName}%
\settowidth{\LargestValueSize}{\LongestValue}%
\settowidth{\LargestTextSize}{\LongestText}%

% Choose alignment of the various elements here: [r], [l] or [c]

\newcommand*{\mbh}[1]{{\makebox[\LargestHenceSize][r]{\ensuremath{#1}}}}%
\newcommand*{\mbn}[1]{{\makebox[\LargestNameSize][r]{\ensuremath{#1}}}}%
\newcommand*{\mbv}[1]{\ensuremath{\makebox[\LargestValueSize][r]{\ensuremath{#1}}}}%
\newcommand*{\mbt}[1]{\makebox[\LargestTextSize][l]{#1}}%

\newcommand{\R}[1]{\label{#1}\linelabel{#1}}
\newcommand{\lr}[1]{line~\lineref{#1}}

% ----------------------------------------------------------------------
% Launch!
% ----------------------------------------------------------------------

\begin{document}

\section{Problem III}

Suppose $T \in \End(V)$.

\begin{lemma}
  \label{sec:problem-iii}
  $\ker T \suq \ker T^2 \suq \cdots \suq \ker T^k \suq \ker T^{k+1} \suq \cdots$
\end{lemma}

\begin{proof}
  \hfill

  Suppose $v \in \ker T$. Therefore, $Tv = 0$, and hence
  $T^2v = T(0) = 0$, an thus $Tv\in\ker T^2$.

  Now, assume $\ker T^{k-1} \suq \ker T^k$ for some $k\in\ZZ^+$, and
  let $v\in \ker T^k$ be arbirtrary. Thus,
  $T^{k+1}v = TT^kv = T(0) = 0$, and hence $v\in \ker
  T^{k+1}$. Therefore, $T^k\suq T^{k+1}$.

  Hence,
  $\ker T \suq \ker T^2 \suq \cdots \suq \ker T^k \suq \ker T^{k+1}
  \suq \cdots$ by induction.

\end{proof}

\begin{lemma}
  \label{sec:problem-iii-2}
  If $\rank T^m = \rank T^{m+1}$ for some $m\geq 0$,

  then $\rank T^m = \rank T^k$ and $\ker T^m = \ker T^k$ for any $k \geq m$.
\end{lemma}

\begin{proof}
  \hfill

  Suppose $\rank T^m = \rank T^{m+1}$ for some $m\geq 0$.
  

  We want to prove that for any $k\in\ZZ^+$,
  $\rank T^{m+k} = \rank T^{m+k+1}$.

  Since $\img T^{m+k}$ is $T$-invariant, because $\img T$ is
  $T$-invariant and $\img T^{m+k}\suq \img T$, then
  $\img T^{m+k+1}\suq \img T^{m+k}$.

  Similarly, since $\img T^{m+1}\suq \img T^m$ and
  $\rank T^m = \rank T^{m+1}$, then $\img T^m = \img T^{m+1}$.

  Suppose now that $u \in \img T^{m+k}$. Therefore, there exists
  $x\in V$ such that $T^{m+k}x = u$.

  Hence, $T^m(T^kx) = u$, and then $u \in \img T^m=\img T^{m+1}$.

  Thus, there exists $w\in V$ such that $T^{m+1}w = u = T^{m+k}x$, so
  that $T^m(T^kx-w) = 0$ and $T^kx-w \in \ker T^m$.

  From Lemma \ref{sec:problem-iii}, we have
  $\ker T^m\suq \ker T^{m+1}$, and thus $T^{m+1}(T^kx-w) = 0$. Hence,
  $T^{m+k+1}x=T^{m+1}w = u$. Therefore, $u\in \img T^{m+k+1}$, and
  thus $\img T^{m+k+1}=\img T^{m+k}$. Since $\img T^m=\img T^{m+1}$,
  by transitive law we obtain that for any $n\geq m$ we have
  $\img T^m = \img T^n$ and $\rank T^m = \rank T^n$.

  % Since $\img T^m$ is $T^m$-invariant, then for any $k\geq m$ we have
  % $\img T^k \suq \img T^m$.

  % Let $v \in \img T^m$ be arbitrary. We prove now that $v$ is also in
  % $\img T^k$.

  % We proceed by induction on $l = k-m$ to prove that $v\in\img T^{k-m}$.

  % Suppose first $l =0$, and thus $k=m$. Then $v\in\img T^k$ and thus
  % $\img T^k = \img T^m$.

  % Suppose now $l \geq 1$.

  % Since $T^{k-m}v \in \img T^{k-m}$, 

  % By Lemma \ref{sec:problem-iii}, we have that
  % $\ker T^m\suq \ker T^{m+1}$.

  Now we prove that for any $k\in\ZZ^+$, $\ker T^{m+k+1} = \ker T^{m+k}$.

  Take arbirtrary $k\in \ZZ^+$.

  By the rank-nullity theorem, $\dim V = \rank T^{m+k}+\nll
  T^{m+k}$. Since we have already shown that
  $\img T^{m+k+1} = \img T^{m+k}$, while
  $\dim V = \rank T^{m+k+1} +\nll T^{m+k+1}$ , we see that
  $\nll T^{m+k} = \nll T^{m+k+1}$. Since also
  $\ker T^{m+k}\suq \ker T^{m+k+1}$ by Lemma \ref{sec:problem-iii}, we
  see that $\ker T^{m+k} = \ker T^{m+k+1}$, which means that for any
  $n\geq m$ we have $\ker T^m =\ker T^{n}$.
\end{proof}

\begin{lemma}
  \label{sec:problem-iii-5}
  $\rank (T-\lambda I)^m = \rank (T-\lambda I)^{m+1}$ for some $m \geq 0$ if and only if $K_{\lambda}= \ker (T-\lambda I)^m$.
\end{lemma}

\begin{proof}
  \hfill

  Note that, by definition of $K_{\lambda}$, for any $m \in \ZZ^+$,
  $\ker (T-\lambda I)^m \suq K_{\lambda}$.

  Suppose first that
  $\rank (T-\lambda I)^m = \rank (T-\lambda I)^{m+1}$.

  Let $v\in K_{\lambda}$ be arbitrary. Therefore, there exists
  $k\in \ZZ^+$ such that $x\in \ker(T-\lambda I)^k $.

  If $k \leq m$, by Lemma \ref{sec:problem-iii} we have that
  $\ker (T-\lambda I)^k\suq \ker (T-\lambda I)^m$, and therefore
  $v\in \ker (T-\lambda I)^m$.

  If $k > m$, by Lemma \ref{sec:problem-iii-2} we have that $\ker (T-\lambda I)^m = \ker (T-\lambda I)^k$, and thus $v\in \ker (T-\lambda I)^m$.

  Therefore, $\ker (T-\lambda I)^m = K_{\lambda}$.

  Suppose now that $K_{\lambda} = \ker(T-\lambda I)^m$.

  From Lemma \ref{sec:problem-iii},
  $\ker (T-\lambda I)^m = K_{\lambda}\suq \ker(T-\lambda I)^{m+1}$. By
  definition of $K_{\lambda}$, we have
  $\ker(T\lambda I)^{m+1}\suq K_{\lambda}$. Therefore,
  $\ker (T-\lambda I)^m = \ker (T-\lambda I)^{m+1}$, and thus we
  obtain $\nll (T-\lambda I)^m = \nll (T-\lambda I)^{m+1}$.

  By the rank-nullity theorem, we also know that
  $\rank (T-\lambda I)^m = \dim V - \nll (T-\lambda I)^m$ and
  $\rank (T-\lambda I)^{m+1} = \dim V - \nll (T-\lambda
  I)^{m+1}$.

  Thus, $\rank (T-\lambda I)^m = \rank (T-\lambda I)^{m+1}$, proving
  the implication to the left.
\end{proof}

\begin{lemma}
  $T$ is diagonalisable if and only if the characteristic polynomial
  of $T$ splits and $\rank (T-\lambda I) = \rank (T-\lambda I)^2$ for
  all eigenvalues $\lambda$.
\end{lemma}

\begin{proof}
  \hfill

  Suppose first that $T$ is diagonalisable. Therefore, we know by
  Theorem 5.6 that the characteristic polynomial of $T$
  splits.

  Moreover, by Corollary to Theorem 7.4, $E_{\lambda} =
  K_{\lambda}$.

  Thus, $K_{\lambda} = E_{\lambda}$, and hence
  $K_{\lambda} = \ker (T-\lambda)$ for any eigenvalue $\lambda$.

  By Lemma \ref{sec:problem-iii-5}, we therefore obtain that
  $\rank(T-\lambda I) = \rank(T-\lambda I)^2$ for any eigenvalue
  $\lambda$.

  Suppose now that the characteristic polynomial
  of $T$ splits and $\rank (T-\lambda I) = \rank (T-\lambda I)^2$ for
  all eigenvalues $\lambda$.

  By Lemma \ref{sec:problem-iii-5},
  $K_{\lambda} = \ker(T-\lambda I) = E_{\lambda}$ for any eigenvalue
  $\lambda$. Also, since the characteristic polynomial splits, by
  Corollary to Theorem 7.4 we obtain that $T$ is diagonalisable.
\end{proof}


\end{document}