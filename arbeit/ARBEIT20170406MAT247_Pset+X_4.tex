
%%% Local Variables:
%%% mode: latex
%%% TeX-master: t
%%% End:

\documentclass[11pt]{scrartcl}
\usepackage[beaue, pset, anon]{masty}
\pSet{\hw{MAT247}{X}{4}}
\usepackage{lineno}
% ----------------------------------------------------------------------
% Page setup
% ----------------------------------------------------------------------

\pagenumbering{gobble}

% ----------------------------------------------------------------------
% Custom commands
% ----------------------------------------------------------------------

% alignment

\newcommand*{\LongestHence}{$\Rightarrow$}% function name
\newcommand*{\LongestName}{$f_o(-x)+f_e(-x)$}% function name
\newcommand*{\LongestValue}{$(-a)x +(-a)(-y)$}% function value
\newcommand*{\LongestText}{\defi}%

\newlength{\LargestHenceSize}%
\newlength{\LargestNameSize}%
\newlength{\LargestValueSize}%
\newlength{\LargestTextSize}%

\settowidth{\LargestHenceSize}{\LongestHence}%
\settowidth{\LargestNameSize}{\LongestName}%
\settowidth{\LargestValueSize}{\LongestValue}%
\settowidth{\LargestTextSize}{\LongestText}%

% Choose alignment of the various elements here: [r], [l] or [c]

\newcommand*{\mbh}[1]{{\makebox[\LargestHenceSize][r]{\ensuremath{#1}}}}%
\newcommand*{\mbn}[1]{{\makebox[\LargestNameSize][r]{\ensuremath{#1}}}}%
\newcommand*{\mbv}[1]{\ensuremath{\makebox[\LargestValueSize][r]{\ensuremath{#1}}}}%
\newcommand*{\mbt}[1]{\makebox[\LargestTextSize][l]{#1}}%

\newcommand{\R}[1]{\label{#1}\linelabel{#1}}
\newcommand{\lr}[1]{line~\lineref{#1}}

% ----------------------------------------------------------------------
% Launch!
% ----------------------------------------------------------------------

\begin{document}

\begin{problem*}
  \hfill
Find all possible rational canonical forms with characteristic polynomial $t^6-1$ over $\FF = \ZZ^2$, up to similarity.
\end{problem*}

\begin{soln}
  \hfill

  Note that
  $t^6-1 = (t^3-1)(t^3+1) = (t-1)^2(t^2+t+1)^2 = (t+1)^2(t^2+t+1)^2$.

  Let $\phi_1 = t+1$ and $\phi_2=t^2+t+1$. Note that $\phi_1$ and
  $\phi_2$ are monic irreducible.

Note that $\phi^2_1 = t^2+1$ and $\phi_2^2=  t^4+t^2+1$.

  Since the characteristic polynomial and minimal polynomial have the
  same zeroes, there are four possibilities for a minimal polynomial $p(t)$:
  \begin{enumerate}
  \item $\phi_1 \phi_2$
  \item $\phi_1^2\phi_2$
  \item $\phi_1\phi_2^2$
  \item $\phi_1^2\phi_2^2$
  \end{enumerate}
  Hence, there are four possible rational canonical forms built from
  the following companion matrices:

  \begin{align}
    A(\phi_1) = \begin{pmatrix}
      1
    \end{pmatrix}, \; 
A(\phi_1^2) = 
          \begin{pmatrix}
            0 & 1\\
            1 & 0
          \end{pmatrix}\\
A(\phi_2) = 
    \begin{pmatrix}
      0 & 1\\
      1 & 1
    \end{pmatrix}, \;
A(\phi_2^2) = 
          \begin{pmatrix}
            0 & 0 & 0 & 1\\
            1 & 0 & 0 & 0\\
            0 & 1 & 0 & 1\\
            0 & 0 & 1 & 0
          \end{pmatrix},
  \end{align}

  which are thus
  \begin{align}
    1.\begin{pmatrix}
      1 & 0 & 0 & 0 & 0 & 0\\
      0 & 1 & 0 & 0 & 0 & 0\\
      0 & 0 & 0 & 1 & 0 & 0\\
      0 & 0 & 1 & 1 & 0 & 0\\
      0 & 0 & 0 & 0 & 0 & 1\\
      0 & 0 & 0 & 0 & 1 & 1
    \end{pmatrix}
                          2. 
    \begin{pmatrix}
      0 & 1 & 0 & 0 & 0 & 0\\
      1 & 0 & 0 & 0 & 0 & 0\\
      0 & 0 & 0 & 1 & 0 & 0\\
      0 & 0 & 1 & 1 & 0 & 0\\
      0 & 0 & 0 & 0 & 0 & 1\\
      0 & 0 & 0 & 0 & 1 & 1
    \end{pmatrix}\\
3.
    \begin{pmatrix}
      1 & 0 & 0 & 0 & 0 & 0\\
      0 & 1 & 0 & 0 & 0 & 0\\
      0 & 0 & 0 & 0 & 0 & 1\\
      0 & 0 & 1 & 0 & 0 & 0\\
      0 & 0 & 0 & 1 & 0 & 1\\
      0 & 0 & 0 & 0 & 1 & 0
    \end{pmatrix}
4.
    \begin{pmatrix}
      0 & 1 & 0 & 0 & 0 & 0\\
      1 & 0 & 0 & 0 & 0 & 0\\
      0 & 0 & 0 & 0 & 0 & 1\\
      0 & 0 & 1 & 0 & 0 & 0\\
      0 & 0 & 0 & 1 & 0 & 1\\
      0 & 0 & 0 & 0 & 1 & 0
    \end{pmatrix}
  \end{align}
\end{soln}

\begin{problem*}
  \hfill

  Show that any two matrices in $M_{6\times 6}(\RR)$ with the
  characteristic polynomial $f(t)=t(t^2+1)(t^2+2t+5)(t+1)$ are similar
  to each other.
\end{problem*}

\begin{soln}
  \hfill

  Let $\phi_1(t) = t$, $\phi_2(t) = t^2+1$, $\phi_3(t) = t^2+2t+5$,
  $\phi_4 = t+1$.

  Note that the discriminant of $\phi_3$ is $-21$, and thus $\phi_3$
  is monic irreducible. Similarly, since the same argument applies to
  $\phi_2$, while $t$ is 0 if and only if $t = 0$ and $t+1 = 0$ if and
  only if $t=-1$, which are not the roots of any other $\phi_i$, all
  $\phi_i$ for $i\in[1, 4]\cap \NN$ are monic irreducible.

  Note that, since there exists a rational canonical basis which is a
  union of disjoint unions of $K_{\phi_i}$ for $i\in[1, 4]\cap \NN$,
  we know that $V = \bigoplus_{i=1}^4 K_{\phi_i}$.

  Let $A, B$ be arbitrary matrices in $M_{6\times 6}(\RR)$ such that
  their characteristic polynomial is $f(t)$.

  Note that for $A$ and $B$ we have
  $\dim K_{\phi_1}= 1 = \dim K_{\phi_4}$ and
  $\dim K_{\phi_2} = 2 = \dim K_{\phi_3}$ by Theorem 7.23.

  Since minimal polynomials $p(t)$ of $A$ and $q(t)$ of $B$ have
  $\phi_i$ for $i\in[1, 4]\cap \NN$ as monic irreducible factors,
  while their multiplicity is 1 in the characteristic polynomial, then
  $p(t) = f(t) = q(t)$.

  Since the dimension of each $K_{\phi_i}$ for $i\in[1, 4]\cap \NN$ is
  1, there is only one dot in the corresponding dot diagram, and thus
  the dot diagrams for each $K_{\phi_i}$ of $A$ and $B$ are the
  same. Denote this rational canonical form as $R$:

  \begin{align}
R=  
    \begin{pmatrix}
0 &  &  &  &  &  \\
  & 0&-1&  &  &  \\
  & 1& 0&  &  &  \\
  &  &  &0 &-5&  \\
  &  &  &1 &-2&  \\
  &  &  &  &  & -1
    \end{pmatrix}
  \end{align}

  From the discussion above, $A\sim R$ and $B\sim R$. Since similarity
  of matrices induces an equivalence relation, we have $A\sim B$, as
  required.
\end{soln}


\end{document}
