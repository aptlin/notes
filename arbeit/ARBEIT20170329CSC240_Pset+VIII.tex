
%%% Local Variables:
%%% mode: latex
%%% TeX-master: t
%%% End:

\documentclass[11pt]{scrartcl}
\usepackage[beaue, pset, anon]{masty}
\usepackage{lineno}
% ----------------------------------------------------------------------
% Page setup
% ----------------------------------------------------------------------

\pagenumbering{gobble}

% ----------------------------------------------------------------------
% Custom commands
% ----------------------------------------------------------------------

% alignment

\newcommand*{\LongestHence}{$\Rightarrow$}% function name
\newcommand*{\LongestName}{$f_o(-x)+f_e(-x)$}% function name
\newcommand*{\LongestValue}{$(-a)x +(-a)(-y)$}% function value
\newcommand*{\LongestText}{\defi}%

\newlength{\LargestHenceSize}%
\newlength{\LargestNameSize}%
\newlength{\LargestValueSize}%
\newlength{\LargestTextSize}%

\settowidth{\LargestHenceSize}{\LongestHence}%
\settowidth{\LargestNameSize}{\LongestName}%
\settowidth{\LargestValueSize}{\LongestValue}%
\settowidth{\LargestTextSize}{\LongestText}%

% Choose alignment of the various elements here: [r], [l] or [c]

\newcommand*{\mbh}[1]{{\makebox[\LargestHenceSize][r]{\ensuremath{#1}}}}%
\newcommand*{\mbn}[1]{{\makebox[\LargestNameSize][r]{\ensuremath{#1}}}}%
\newcommand*{\mbv}[1]{\ensuremath{\makebox[\LargestValueSize][r]{\ensuremath{#1}}}}%
\newcommand*{\mbt}[1]{\makebox[\LargestTextSize][l]{#1}}%

\newcommand{\R}[1]{\label{#1}\linelabel{#1}}
\newcommand{\lr}[1]{line~\lineref{#1}}

% ----------------------------------------------------------------------
% Launch!
% ----------------------------------------------------------------------
% \usepackage[active,tightpage]{preview}
% \PreviewEnvironment{tikzpicture}
% \setlength\PreviewBorder{0pt}%
	%
\usepackage{pdfpages}
\pSet{Alexander Illarionov -- 1003590937 -- CSC240 Pset VIII}

\begin{document}
\textbf{Administrativia}: no discussions, no extra material consulted

\section{Problem}

\begin{figure}[h]
 \centering 
\includegraphics[scale=1.0]{4.pdf},
\end{figure}

Consider the two-state DFA $M$ shown in the figure above.

Let $L = \set{\set{a,b}^{*}; \text{ there is no run of length 2 }}$.

Let 

\begin{equation}
P(x): \delta^{*}(q_0, x) = 
\begin{cases}
  q_1 \THEN & x \in \LL( b\set{ab}^{*}a + a \set{ba}^{*} = P_1)\\
  q_{3} \THEN & x \in \LL( (b\set{ab}^{*}a + a \set{ba}^{*})aa\set{a}^{*}+\\
  &\ \left(   (a\set{ba}^{*}b + b \set{ba}^{*}b)b \set{b}^ {*} a \set{ba}^ {*} aa \set{a}^ {*}\right) \*\\
  &\ (\lambda + b \set{ab}^{*}b \set{b}^ {*} a \set{ba}^ {*} aa \set{a}^{*}) = P_{3})\\
  q_4 \THEN & x \in \LL( P_{3}b \set{ba}^{*} = P_{4})\\
  q_5 \THEN & x \in \LL( a\set{ba}^{*}b + b \set{ba}^{*} = P_5)\\
  q_7 \THEN & x \in \LL( (P_5 + P_4)(bb \set{b}^{*})\\
&\ (\lambda + \set{ab}^{*}(\lambda + aa \set{a}^{*}b \set{ab}^{*}bb \set{b}^{*})) = P_{7})\\
  q_{8} \THEN & x \in \LL( P_7 a \set{ba}^{*})

\end{cases}
\end{equation}

\begin{lemma}
$M$ accepts the language $L$.
\end{lemma}

\begin{proof}
  \hfill

  We use structural induction on $x\in L$.

\begin{description}
\item[Base Case] \hfill

  If $x = \lambda$, then there are zero characters in $x$ and hen   there are no ru. Thus, the claim holds in case $x= \lambda$.
\item[Inductive Step] \hfill

  Suppose the claim holds for $x=y$ and consider the case when $x= yt$
  for some $\Sigma^{*}$ and $t\in \Sigma$, where
  $\Sigma = \set{ a, b}$.

  There are two cases: $t = a$ and $t=b$.

  If $t = a$, then $\delta^{*}(q_0, ya) = \delta(\delta^{*}(q_0, y), a)$ by definition of $\delta^{*}$.

  Looking at the mess of 8 cases and the diagram and considering the
  inductive hypothesis, we see that none of the transitions introduces
  a run of length 2, and hence the claim holds if $t = a$.

  If $t = b$, then
  $\delta^{*}(q_0, yb) = \delta(\delta^{*}(q_0, y), b)$ by definition
  of $\delta^{*}$, and the similar reasoning holds.

\item[Conclusion]
  \hfill

  By strong induction, $M$ accepts the language $L$.
\end{description}
\end{proof}

To see why the regular expressions which define the languages in the
equation (1) are indeed correct, the reader is referred to the
diagram, even though the author fully admits that this reasoning in no
way constitutes a proof.

The main intuition is that the nodes $q_0$, $q_1$, $q_2$, $q_5$ and
$q_6$ \textit{sort} the input string in such a way that it does not
have a run of length 2 at the points defined by the first few regex
terms. Then the process \textit{goes into the loop} by first allowing
the runs of length greater than three and then ensuring that there are
no runs of length 2, which is achieved by linking from $q_{4}$ and
$q_8$ to the respective \textit{run-2 trimmers} ($q_{6}$ and $q_{2}$)
as defined by the edges after the nodes $q_2$ and $q_6$, which then
continue into the loop which consists of allowed states.
\end{document}
