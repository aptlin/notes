%%% Local Variables:
%%% mode: latex
%%% TeX-master: t
%%% End:

\documentclass[12pt]{article}
\usepackage[utf8]{inputenc}
%-----------------------------------------------------------
\usepackage{fullpage}
\usepackage{hyperref}
\usepackage{graphicx}
\usepackage{color}

\definecolor{mygrey}{gray}{0.90}
\raggedbottom
\raggedright
\setlength{\tabcolsep}{0in}

\usepackage{amsthm, amsmath, amssymb}
\usepackage{chngcntr}


% Adjust margins to 0.5in on all sides
\addtolength{\oddsidemargin}{-0.5in}
\addtolength{\evensidemargin}{-0.5in}
\addtolength{\textwidth}{1.0in}
\addtolength{\topmargin}{-0.5in}
\addtolength{\textheight}{1.0in}
% ----------------------------------------------------------------------
% Custom definitions
\def\Re{\mathbb{R}}
\def\P{\mathbb{P}}
\def\F{\mathbb{F}}
\def\defi{Definition of }
\def\mclo{Multiplicative Closure of }
\def\aclo{Additive Closure of }
\def\dist{Distributive Law}
\def\ainv{Existence of an Additive Inverse }
\def\minv{Existence of a Multiplicative Inverse }
\def\uainv{Uniqueness of an Additive Inverse }
\def\uminv{Uniqueness of a Multiplicative Inverse }
\def\comm{Commutative Law }
\def\tric{Trichotomy Law }
\def\assoc{Associative Law }
\def\aid{Existence of an Additive Identity }
\def\mid{Existence of a Multiplicative Identity }
\def\canc{Cancellation Property }
\def\die{Distinctness of an Additive Identity and Multiplicative Identity}

\def\ra{\Rightarrow}
\def\equ{\Leftrightarrow}
\def\v{\vspace{0.1in}}
\def\*{\cdot}
% -----------------------------------------------------------
%Custom commands
\newcommand{\resitem}[1]{\item #1 \vspace{-2pt}}
\newcommand{\resheading}[1]{{\large \colorbox{mygrey}{\begin{minipage}{\textwidth}{\textbf{#1 \vphantom{p\^{E}}}}\end{minipage}}}}
\newcommand{\ressubheading}[4]{
\begin{tabular*}{7.0in}{l@{\extracolsep{\fill}}r}
                \textbf{#1} & #2 \\
                \textit{#3} & \textit{#4} \\
\end{tabular*}\vspace{-6pt}}


\newcommand*{\LongestName}{$\Rightarrow\ (-a)(x-y)$}% function name
\newcommand*{\LongestValue}{$(-a)x +(-a)(-y)$}% function value
\newcommand*{\LongestText}{\defi - and \dist }%

\newlength{\LargestNameSize}%
\newlength{\LargestValueSize}%
\newlength{\LargestTextSize}%

\settowidth{\LargestNameSize}{\LongestName}%
\settowidth{\LargestValueSize}{\LongestValue}%
\settowidth{\LargestTextSize}{\LongestText}%

% Choose alignment of the various elements here: [r], [l] or [c]
\newcommand*{\mbn}[1]{{\makebox[\LargestNameSize][r]{\ensuremath{#1}}}}%
\newcommand*{\mbv}[1]{\ensuremath{\makebox[\LargestValueSize][l]{\ensuremath{#1}}}}%
\newcommand*{\mbt}[1]{\makebox[\LargestTextSize][l]{#1}}%

\newtheorem{theorem}{Theorem}[section]
\newtheorem*{theorem*}{Theorem}
\newtheorem{corollary}{Corollary}[theorem]
\newtheorem{lemma}[theorem]{Lemma}
\newtheorem*{lemma*}{Lemma}
\newtheorem{subtheorem}{Lemma}[theorem]
\theoremstyle{definition}
\newtheorem{definition}{Definition}[section]
\theoremstyle{remark}
\newtheorem*{remark}{Remark}
% -----------------------------------------------------------

\pagenumbering{gobble}

\counterwithin*{equation}{theorem}
\counterwithin*{equation}{corollary}
\counterwithin*{equation}{subtheorem}
\begin{document}
\begin{lemma*}{\canc}
  \begin{align}
    \mbn{\forall a,b,c \in \mathbb{F}: a+c=b+c}\ & \mbv{\Leftrightarrow a=b} & \mbt{}\\
    \mbn{\forall a,b,c \in \mathbb{F}, c\neq0: ac=bc}\ & \mbv{\Leftrightarrow a=b} & \mbt{}
  \end{align}
\end{lemma*}
\begin{proof}
  Suppose $a+c=b+c$.
  \begin{align}
    \mbn{\exists\ (-c): c + (-c)}\ & \mbv{=0} & \mbt{\ainv}\\
    \mbn{\Rightarrow\ (a+c)+(-c)}\ & \mbv{=(b+c)+(-c)} & \mbt{\defi =}\\
    \mbn{\Rightarrow\ a+(c+(-c))}\ & \mbv{=b+(c+(-c))} & \mbt{\assoc}\\
    \mbn{\Rightarrow\ a+0}\ & \mbv{=b+0} & \mbt{\ainv}\\
    \mbn{\Rightarrow\ a}\ & \mbv{=b} & \mbt{\aid}
  \end{align}
  Suppose now $ac=bc$.
  \begin{align}
    \mbn{\exists\ c^{-1}: c  c^{-1}}\ & \mbv{=1} & \mbt{\ainv}\\
    \mbn{\Rightarrow\ (ac)c^{-1}}\ & \mbv{=(bc)c^{-1}} & \mbt{\defi =}\\
    \mbn{\Rightarrow\ a(cc^{-1})}\ & \mbv{=b(cc^{-1})} & \mbt{\assoc}\\
    \mbn{\Rightarrow\ a\cdot1}\ & \mbv{=b\cdot1} & \mbt{\minv}\\
    \mbn{\Rightarrow\ a}\ & \mbv{=b} & \mbt{\aid}
  \end{align}
\end{proof}

\begin{lemma}
  \label{eq:negp}
  $\forall a,b \in \F : (-a)b=-ab$
\end{lemma}
\begin{proof}

  \begin{align}
    \mbn{ ab+(-a)b}\ & \mbv{=ba+b(-a)} & \mbt{\comm}\\
    \mbn{a+(-a)}\ & \mbv{=0} & \mbt{\ainv}\\
    \mbn{\Rightarrow b(a+(-a))}\ & \mbv{=b\*0} & \mbt{\dist}\\
    \mbn{}\ & \mbv{} & \mbt{and \ainv}\\
    \mbn{}\ & \mbv{=0} & \mbt{Lemma \ref{eq:zero}}\\
    \mbn{\Rightarrow ab+(-a)b}\ & \mbv{=0} & \mbt{\defi =}\\
    \mbn{\Rightarrow(-a)b+ab}\ & \mbv{=0} & \mbt{\comm}\\
    \mbn{(-a)b+ab-ab}\ & \mbv{=0-ab} & \mbt{\defi =}\\
    \mbn{\Rightarrow(-a)b+0}\ & \mbv{=-ab} & \mbt{\ainv}\\
    \mbn{}\ & \mbv{} & \mbt{and \aid}\nonumber\\
    \mbn{}\ & \mbv{=(-a)b} & \mbt{\aid}
  \end{align}
\end{proof}
\begin{corollary}
\label{eq:negm}
  $\forall a \in \F: -b=(-1)b$
\end{corollary}
\begin{proof}
  From Lemma \ref{eq:negp}, if $a=1$, then $(-1)b=-1 \cdot b$
  \begin{align}
    \mbn{-1\cdot b}\ & \mbv{=-b\cdot1} & \mbt{\comm}\\
    \mbn{\ra (-1)b}\ & \mbv{=-b} & \mbt{\defi =}\\
    \mbn{}\ & \mbv{} & \mbt{and \mid}\nonumber
  \end{align}
\end{proof}

\begin{lemma}
  $\forall a \in \mathbb{F}: a\cdot0=0$
  \label{eq:zero}
\end{lemma}

\begin{proof}
\begin{align}
  \mbn{0+0}\ & \mbv{=0} & \mbt{\aid}\\
  \mbn{\Rightarrow a\cdot(0+0)}\ & \mbv{=a\cdot 0 + a \cdot 0} & \mbt{\dist}\\
  \mbn{}\ & \mbv{=a \cdot 0} & \mbt{\defi =}\\
  \mbn{(a\cdot 0 + a \cdot 0)-(a\cdot 0)}\ & \mbv{=a \cdot 0 -(a \cdot 0)} & \mbt{\defi =}\\
  \mbn{\Rightarrow a\cdot0 +(a\cdot0-a\cdot0)}\ & \mbv{=0} & \mbt{\assoc}\\
  \mbn{}\ & \mbv{} & \mbt{and \ainv}\nonumber\\
  \mbn{\Rightarrow a\cdot0 +0}\ & \mbv{=0} & \mbt{\ainv}\\
  \mbn{\Rightarrow a\cdot0}\ & \mbv{=0} & \mbt{\aid}\nonumber
\end{align}
\end{proof}

\begin{lemma}
  \label{eq:negneg}
  $-(-a)=a$
\end{lemma}
\begin{proof}
  \begin{align}
    \mbn{a+(-a)}\ & \mbv{=0} & \mbt{\ainv}\\
    \mbn{(-1)(a+(-a))}\ & \mbv{=(-1)0} & \mbt{\defi =}\\
    \mbn{(-1)a+(-1)(-a)}\ & \mbv{=0} & \mbt{\dist}\\
    \mbn{}\ & \mbv{} & \mbt{and Lemma \ref{eq:zero}}\nonumber\\
    \mbn{\equ -a-(-a)}\ & \mbv{=0} & \mbt{Corollary \ref{eq:negm}}\\
    \mbn{a+(-a-(-a))}\ & \mbv{=a+0} & \mbt{\defi =}\\
    \mbn{(a-a)-(-a)}\ & \mbv{=a} & \mbt{\assoc}\\
    \mbn{}\ & \mbv{} & \mbt{and \aid}\\
    \mbn{0-(-a)}\ & \mbv{=a} & \mbt{\ainv}\\
    \mbn{-(-a)}\ & \mbv{=a} & \mbt{\aid}
  \end{align}
\end{proof}
\begin{lemma}
  \label{eq:zerzer}
  $\forall a,b \in \F: ab=0 \equ a=0 \vee b=0$
\end{lemma}
\begin{proof}
  \vspace{0.1in}
  By \comm and Lemma \ref{eq:zero}, $a=0 \ra ab=ba=b\cdot 0=0$.\\
  \vspace{0.1in}
  Similarly, $b=0 \ra ab=a\cdot 0=0$.
  \vspace{0.1in}
  If $ab=0$ and $b\neq 0$, $\exists\ b^{-1}: abb^{-1}=0\cdot b^{-1}$,
  hence by \comm and \minv $a\cdot 1=b^{-1}\cdot 0$, then by \mid
  and Lemma \ref{eq:zero} $a=0$.\\
  \vspace{0.1in}
  If $ab=0$ and $a\neq 0$, $\exists\ a^{-1}: a^{-1}ab=a^{-1}\cdot 0$,
  hence by \comm and Lemma \ref{eq:zero} $aa^{-1}b=0$, then by \minv
  $1\cdot b=0$, and by \comm and \mid $b\cdot 1=b=0$.\\
  \vspace{0.1in}
  If $a=0 \wedge b=0$, then by Lemma \ref{eq:zero} $ab=0\cdot0=0$
\end{proof}
\begin{theorem*}
  Let $\F$ be a field with $3$ elements ${0, 1, a}$. Then the following is true:
  \begin{enumerate}
  \item $1+1=a$
  \item $a+1=0$
  \item $a \cdot a =1$
  \end{enumerate}
\end{theorem*}
\begin{proof}
  Consider $a\cdot a$. By \mclo $\F$, there are three cases:

  \begin{enumerate}
  \item $a\cdot a=a$
  \item $a\cdot a=0$
  \item $a\cdot a=1$
  \end{enumerate}
  We argue by repetitive \textit{reductio ad absurdum} that $a\cdot a=1$.\\\v

  Suppose that $a\cdot a=a$. By distinctness of elements, $a\neq
  0$. Therefore by \canc $a=1$, which contradicts the distinctness of
  elements.\\\v

  Suppose now that $a\cdot a=0$. Since $a=a$ and Lemma
  \ref{eq:zerzer}, $a=0$, which again contradicts the distinctness of
  elements.\\\v

  Hence, $a\cdot a=1$.\\\v

  Therefore, $a+a\cdot a=a+1$. From \dist, $a(a+1)=(a+1)$. By
  \canc, $a(a+1)-(a+1)=0$. By \comm and \dist, $(a+1)(a-1)$=0. From Lemma \ref{eq:zerzer},
  \canc and distinctness of elements, $a=-1\ \veebar\ a=1$. Since
  $a\neq 1$ by definition, $a=-1$.\\\v

  Therefore, $a+1=-1+1=1+(-1) [\text{ by \comm}]=0$ [ by \ainv].\\\v

  We now prove that $1+1=0$.\\\v

  Suppose on the contrary that $1+1=1$.\\\v

  Then by cancellation property $1=0$, which is a
  contradiction to \die\ $\ra (1+1=0) \vee (1+1=a)$.\\ \v

  If $1+1 = 0 $, then $(1+1)+ a = 0 + a = a$ by \aid. But then by
  \assoc, \comm and \aid $1+(1+a)=1+(a+1)=1+0=1$ and hence $1=a$,
  which is a contradiction, since $a$ and $1$ are distinct by
  definition. Therefore, $1+1=a=-1$.

\end{proof}

\end{document}