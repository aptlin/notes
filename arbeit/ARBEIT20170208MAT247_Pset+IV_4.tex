% Created 2017-02-05 Sun 18:58
% Intended LaTeX compiler: pdflatex
\documentclass[11pt]{scrartcl}
  \usepackage[beaue, pset, anon]{masty}
\pSet{\hw{MAT247}{IV}{4}}
\pagenumbering{gobble}
\author{Alexander Illarionov}
\date{\today}
\title{}
\begin{document}

Suppose \(V\) is the infinite-dimensional vector space of sequences \(\sigma : \NN \to F\) that have only a finite number of non-zero terms. In other words, \(\sigma(k) \ne 0\) for only finitely many positive integers k. 

Define \(\langle \sigma,\tau\rangle = \sum_{k=1}^\infty \sigma(k) \overline{\tau(k)}\).

\textbf{Problem}

Show that this is an inner product on V.

\textbf{Solution}

\begin{description}

\item[Positivity]\hfill

Since the sum of products of nonnegative integers is nonnegative, then
for any $\sigma, \tau \in V$ we have $\ipr{\sigma}{\tau} \geq 0$.

\item[Definiteness] \hfill

Let $\sigma \in V$ be a sequence ${a_i}$.

Suppose $\ipr{v}{v} = 0$. Therefore, $\sum_{i=1}^n \abs{a_i}^2 =
  0$. Hence, $\ipr{v}{v}$ is $0$ if and only if $a_i = 0$ for all
  $i$, and thus $v = 0$.
  
\item[Additivity in the First Slot] \hfill

  Note that for all $\sigma, \tau, \upsilon \in V$
  \begin{align}
    \ipr{\sigma+\tau}{\upsilon} &= \sum_{i=1}^{\infty}(\sigma(i)+\tau(i))\ol{\upsilon(i)}\\
    &=\sum_{i=1}^{\infty}\sigma(i)\ol{\upsilon(i)} + \sum_{i=1}^{\infty}\sigma(i)\ol{\upsilon(i)}\\
    &=\ipr{\sigma}{\upsilon} + \ipr{\tau}{\upsilon}
  \end{align}

  
\item[Homogeneity in the First Slot]\hfill

  For all $\sigma, \tau \in V$ and $\lambda \in \FF$,
  \[    \ipr{\lambda \sigma}{\tau} = \sum_{i=1}^\infty\lambda \sigma(i)\ol{\tau(i)} = \lambda \sum_{i=1}^\infty\sigma(i)\ol{\tau(i)} = \lambda \ipr{\sigma}{\tau}
  \]
  
\item[Conjugate Symmetry] \hfill

  For all $\sigma, \tau \in V$

  \[\ipr{\sigma}{\tau} = \sum_{i=1}^\infty \sigma(i)\ol{\tau(i)} = \sum_{i=1}^\infty \ol{\ol{\sigma(i)}\tau(i)} =  \sum_{i=1}^\infty \ol{\tau(i)\ol{\sigma(i)}}} = \ol{\ipr{\tau}{\sigma}}
  \]
  
  Therefore, $\ipr{\*}{\*}$ is an inner product.
\end{description}
\textbf{Problem}

For \(n \ge 1\) define \(e_n \in V\) by \(e_n(k) = 1\) if \(k=n\) and \(e_n(k)
= 0\) if \(k\ne n\). Show that the set \(\{e_n\}\) is an orthonormal basis
of V

\textbf{Solution}

Since for each \(k\in \NN\), \(\sigma \in V\) and any \(i\in \NN\) there
exists \(\lambda \in \NN\) such that \(\sigma(i) = \lambda k\) by the
fundamental theorem of arithmetic, while each \(\sigma\) contains only
finitely many nonzero elements, then \(V\) is spanned by \(e_i\). By
definition, the set of all \(e_i\) is linearly independent.

We prove now that it is also orthonormal.

Take \(e_i, e_j\) for any \(i, j \in \NN\) such that \(i\neq j\)

Note that \(\ipr{e_i}{e_j}=\sum_{k=1}^{\infty} e_i(k)\ol{e_j(k)} = 0\),
since \(e_i\) is nonzero only at the \(i^{\text{th}}\) position and \(e_j\)
is nonzero at the \(j^{\text{th}}\) position.

If \(i=j\), however, we obtain \(\ipr{e_i}{e_i}=1\), since the value of
\(e_i\) at the \$i\$th position is \(1\), and thus \(e_i\ol{e_i} = 1\).

Therefore, \(\set{e_n}\) is an orthonormal basis.

\textbf{Problem} 

Let W be the subspace spanned by the elements \(e_1 + e_n\) for all \(n \ge 2\).

Show that \(e_1 \not\in W\) and that \(W^\perp = \{0\}\). 

Deduce that \((W^\perp)^\perp \ne W\).

\textbf{Solution}

Note that for any sequence \(w\in W\), \(w = a_1e_1 +
\sum_{i=2}^{\infty}a_ie_{i}\) by definition of \(W\). Moreover, if \(a_1=
0\), then \(w = 0\), since \(W\) is spanned by \(e_1+e_i\) for \(i\geq 2\). 

If, however, \(a_1\neq 0\), then there exists at least one \(k\geq 2\)
such that 
\[w = a_1e_1 + a_ke_k + \sum_{i=2, i\neq k}^{\infty}a_ie_i\] and \(a_{k}\neq 0\).

Since \(e_1\) and \(e_k\) are linearly independent, the first element and
the \(k^{\Th}\) in the sequence of \(w\) are nonzero.

Therefore, there does not exist an element \(v\in V\) such that the
first element in \(v\) is nonzero and the rest are zero. Therefore, \(e_1 \not\in W\).

Consider now \(W^{\bot}\).

Let \(w\in W\) be a sequence in \(W^{\bot}\).

Since \(\set{e_i}\) is a basis of \(V\), 
\[w = \sum_{i=1}^{\infty}a_ie_{i},\]
where there are only finitely many nonzero \(a_i\).

Suppose that there exists \(k\in \NN\) such that some \(a_k\) is nonzero.

By definition, \(\forall y\in W.\ipr{w}{y} = \sum_{i=1}^{\infty}a_i\ol{b_{i}}=0\), where \(b_i\) are such that \(y = \sum_{i=1}^{\infty}b_ie_i\).

Take some \(y\in W\) such that \(y\neq 0\) and the \(k^{\Th}\) element in \(y\) is nonzero. By the argument above and definition of \(W\) as a span of \(e_1+e_i\) for \(i\geq 2\), such an element exists. 

Therefore, \(\ipr{w}{y} = \sum_{i=1}^{\infty}a_i\ol{b_{i}}\), and thus
\(\ipr{w}{y} = a_k b_{k}\), since \(b_k\in \NN\) and thus \(b_k=\ol{b_k}\). 

Since \(a_k\neq 0\) and \(b_k\neq 0\), while \(a_k, b_k\in\NN\), we get that
\(\ipr{w}{y} > 0\). But \(w\in W^{\bot}\) by assumption, so our assumption must be false and hence there exists no such \(k\in \NN\) such that some \(a_k\) is nonzero. Therefore, \(W^{\bot}=\set{0}\).

Since \(\set{0}^{\bot} = V\), because each vector in \(V\) is orthogonal to \(0\), while \(e_1\not\in W\) and thus \(V\neq W\), it follows that \(W^{\bot\bot}\neq W\).
\end{document}