% Created 2017-02-04 Sat 22:38
% Intended LaTeX compiler: pdflatex
\documentclass[11pt]{scrartcl}
  \usepackage[beaue, pset, anon]{masty}
\pSet{\hw{MAT247}{IV}{2}}
\pagenumbering{gobble}
\author{Alexander Illarionov}
\date{\today}
\title{}
\begin{document}

Consider \(V = \SP_2(\FF)\) with the inner product \(\langle f(x),g(x)\rangle = \int_{-1}^1 f(t)\overline{g(t)} dt\).

Let \(\beta = \set{1+x, x^2}\) and let \(W\) be the subspace of \(V\) spanned by \(\beta\).

\begin{problem*}
Find $W^{\bot}$.
\end{problem*}

\textbf{Solution}

Let \(v_1  = 1 + x\) and \(v_2 = x^2\). 
Therefore, \(\ipr{1+x}{1+x} = \int_{-1}^1 (1+x)^2 = [x+x^2+\nicefrac{1}{3}x^3]_{-1}^1\), 
and thus \(\norm{1+x}^2 = 2 + \nicefrac{2}{3} = \frac{8}{3}\).

\begin{adjustbox}{center}
\begin{tabulary}{1.25\textwidth}{|C|C|C|C|C|C|}
\hline
\(i\) & \(v_i\) & \(\sum_{j=1}^{i-1} \frac{ \ipr{v_i}{u_j}}{\ipr{u_j}{u_j}}u_j\) & \(u_i\) & \(\norm{u_i}^2\) & \(e_i\)\\
\hline
1 & \(1 + x\) & -- & \(1+x\) & \(\int_{-1}^1(1+x)^2 = [x+x^2+\nicefrac{1}{3}x^3]_{-1}^1\) & \(\nicefrac{\sqrt{6}}{4}(1+x)\)\\
 &  &  &  & \(\norm{1+x}^2 = 2 + \nicefrac{2}{3} = \frac{8}{3}\). & \\
\hline
2 & \(x^2\) & \(\frac{\ipr{v_2}{u_1}}{\ipr{u_1}{u_1}}u_1=\) & \(x^2-\frac{1}{4}(1+x)\) &  & \\
 &  & \(\frac{\int_{-1}^1 x^2 + x^3 dx}{\nicefrac{8}{3}}(1+x) =\) & \(x^2-\frac{1}{4}x - \nicefrac{1}{4}\) & \(\int_{-1}^1 (x^2-\frac{1}{4}x - \nicefrac{1}{4})^2\) & \(\sqrt{\frac{30}{7}}(x^2-\frac{1}{4}x - \nicefrac{1}{4})\)\\
 &  & \(\frac{[\nicefrac{1}{3}x^3 + \nicefrac{1}{4}x^4 dx]_{-1}^1}{\nicefrac{8}{3}}(1+x) =\) &  & \(=\frac{7}{30}\) & \\
 &  & \(\frac{1}{4}(1+x)\) &  &  & \\
 &  &  &  &  & \\
\hline
\end{tabulary}
\end{adjustbox}

Note that \(1\) is not in the span of \(\beta\). Let \(v_3=1\).

Therefore, 
\begin{align}
\sum_{j=1}^{2} \frac{ \ipr{v_3}{u_j}}{\ipr{u_j}{u_j}}u_j &= \frac{ \ipr{1}{u_1}}{\ipr{u_1}{u_1}}u_1 + \frac{ \ipr{1}{u_2}}{\ipr{u_2}{u_2}}u_2\\
&= \frac{\int_{-1}^1 (1+x) dx}{\nicefrac{8}{3}}(1+x) + \frac{\int_{-1}^1 (x^2-\nicefrac{1}{4}\ x -\nicefrac{1}{4}) dx}{\nicefrac{7}{30}}(x^2-\nicefrac{1}{4}\ x - \nicefrac{1}{4})\\
&=\nicefrac{3}{4} + \nicefrac{3}{4}\ x + \nicefrac{5}{7}\ x^2 - \nicefrac{5}{28}\ x - \nicefrac{5}{28}\\
&= \nicefrac{5}{7}\ x^2 + \nicefrac{4}{7}\ x + \nicefrac{4}{7}
\end{align}

Thus, 
\begin{align}
u_3 = 1 - (\nicefrac{5}{7}\ x^2 + \nicefrac{4}{7}\ x + \nicefrac{4}{7}) =
 -\nicefrac{5}{7}\ x^2 - \nicefrac{4}{7}\ x + \nicefrac{3}{7}
\end{align}

Since \(\beta' = \set{u_1, u_2}\) is a basis for \(W\), while
\(\beta'\cup\set{u_3}\) is a basis for \(V\) by construction, since \(V =
W\oplus W^{\bot}\), then \(W^{\perp}\) is spanned by \(u_3\).

Note that \(\norm{u_3}^2 = \int_{-1}^1 ( -\nicefrac{5}{7}\ x^2 -
\nicefrac{4}{7}\ x + \nicefrac{3}{7})^2 dx = \nicefrac{8}{21}\), and
thus
\[
e_3 =  \frac{\sqrt{42}}{4}(-\nicefrac{5}{7}\ x^2 - \nicefrac{4}{7}\ x + \nicefrac{3}{7})
\]

\begin{problem*}
Fix $u \in \FF$ and suppose that $\theta_u : V \to F$ is the linear function given by $\theta_u(f(x)) = f(u)$. Find $g_u(x) \in V$ such that $\theta_u(f(x)) = \langle f(x),g_u(x)\rangle$ for all $f(x) \in V$. 
\end{problem*}

\textbf{Solution}

Let \(y = \sum_{i=1}^n \ol{\theta_u(e_i)}e_i\), where \(e_i\) are
orthonormal vectors given in the tables above. Then by Theorem 6.8 in
Friedberg \emph{et al}, we obtain that \(g_u(x) = y\).

The coefficients before each \(e_i\) in the expression for \(y\) are given
below:
\begin{itemize}
\item  $\ol{\theta_u(e_1)} = \frac{\sqrt{6}}{4}(1+u)$
\item  $\ol{\theta_u(e_2)} = \sqrt{\frac{30}{7}}(u^2-\frac{1}{4}u - \nicefrac{1}{4})$
\item  $\ol{\theta_u(e_3)} = \frac{\sqrt{42}}{4}(-\nicefrac{5}{7}\ u^2 - \nicefrac{4}{7}\ u + \nicefrac{3}{7})$
\end{itemize}

\begin{problem*}
Suppose that $T : V \to V$ is defined by $T(f(x)) = f'(x)$. Find $T^*(1+x)$.
\end{problem*}

\textbf{Solution}

Note that \(\beta = \set{1, x, x^2}\) is an orthonormal basis of \(\SP_2(\FF)\).

Note the following:
\begin{itemize}
\item $T(1) = 0$
\item $T(x) = 1$
\item $T(x^2) = 2x$
\end{itemize}

Therefore,
\[[T]_{\beta} = 
\begin{pmatrix}
0 & 1 & 0\\
0 & 0 & 2\\
0 & 0 & 0 
\end{pmatrix}\]

Since \([T^{*}]_{\beta} = [T_{\beta}]^{*}\), it follows that
\(T^{*}(1+x)\) can be represented in the matrix form as
\[\begin{pmatrix}
0 & 0 & 0\\
1 & 0 & 0\\
0 & 2 & 0 
\end{pmatrix}\cv{1;1;0}=\cv{0;1;2}\]

Therefore, \(T^{*} =x+2x^2\).
\end{document}