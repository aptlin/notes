%%% Local Variables:
%%% mode: latex
%%% TeX-master: t
%%% End:

\documentclass[11pt]{scrartcl}
\usepackage[beaue, pset, anon]{masty}
\pSet{Alexander Illarionov -- 1003590937}
\usepackage{lineno}
% ----------------------------------------------------------------------
% Page setup
% ----------------------------------------------------------------------

\pagenumbering{gobble}

% ----------------------------------------------------------------------
% Custom commands
% ----------------------------------------------------------------------

% alignment

\newcommand*{\LongestHence}{$\Rightarrow$}% function name
\newcommand*{\LongestName}{$f_o(-x)+f_e(-x)$}% function name
\newcommand*{\LongestValue}{$(-a)x +(-a)(-y)$}% function value
\newcommand*{\LongestText}{\defi}%

\newlength{\LargestHenceSize}%
\newlength{\LargestNameSize}%
\newlength{\LargestValueSize}%
\newlength{\LargestTextSize}%

\settowidth{\LargestHenceSize}{\LongestHence}%
\settowidth{\LargestNameSize}{\LongestName}%
\settowidth{\LargestValueSize}{\LongestValue}%
\settowidth{\LargestTextSize}{\LongestText}%

% Choose alignment of the various elements here: [r], [l] or [c]

\newcommand*{\mbh}[1]{{\makebox[\LargestHenceSize][r]{\ensuremath{#1}}}}%
\newcommand*{\mbn}[1]{{\makebox[\LargestNameSize][r]{\ensuremath{#1}}}}%
\newcommand*{\mbv}[1]{\ensuremath{\makebox[\LargestValueSize][r]{\ensuremath{#1}}}}%
\newcommand*{\mbt}[1]{\makebox[\LargestTextSize][l]{#1}}%

\newcommand{\R}[1]{\label{#1}\linelabel{#1}}
\newcommand{\lr}[1]{line~\lineref{#1}}

% ----------------------------------------------------------------------
% Launch!
% ----------------------------------------------------------------------

\begin{document}
\section*{CSC240 Problem Set II}

% ----------------------------------------------------------------------
% Body
% ----------------------------------------------------------------------
\linenumbers \textbf{Administrativia}: discussed problem 1 with Ming Feng Wan
and Darren Chan, no extra material consulted
\begin{enumerate}
\item\label{item:1} Given three variables, \textbf{P}, \textbf{Q}, and
  \textbf{R}, note that there are 8 possible truth value assignments:
  \begin{table}[h]
    \centering
    \begin{tabulary}{0.3\textwidth}{C|C|C}
      \textbf{P} & \textbf{Q} & \textbf{R} \\
      \hline
      T          & T          & T          \\
      T          & T          & F          \\
      T          & F          & T          \\
      T          & F          & F          \\
      F          & T          & T          \\
      F          & T          & F          \\
      F          & F          & T          \\
      F          & F          & F
    \end{tabulary}
  \end{table}

  Suppose a propositional formula $f$ contains at most 3 different
  variables. Let $[f]_{\min} = [f]$ be the representation of $f$ in the DNF.

  Note that the truth value of each minterm in $[f]$ is completely
  determined by some row in the table above (if some minterm contains
  less than three variables, it can be considered \textit{completable}
  by adding the value $T$ in conjunction until there are three terms
  in each minterm). Since there are $8$ rows, $8$ clauses in DNF, each
  containing at most 3 variables, will completely determine any
  propositional formula.

  DNF allows us to reduce the number of sufficient clauses by half,
  since, given the fact that 8 clauses represent all possible
  propositional formulas, in each disjunction only one of the minterms
  determines whether the disjunction is true or false (in the sense
  that removing one of the minterms would not change the equivalent
  value of the clause representing the disjunction of two
  minterms).

  Suppose the maxmimum number of sufficient minterms is lower than
  4. Since a propositional formula $g$ can be constructed in DNF with
  4 mutually exclusive minterms (see (\ref{item:2})), these terms
  cannot be sufficient.

  Therefore, 4 clauses are necessary and sufficient to represent all
  possible propositional formulas with at most three variables. Hence,
  the total frequency of variable occurence is at most 12.
\item
  Consider the propositional formula $P \IFF Q$.

    \label{item:2}
    \fitchprf{P\IFF Q}{

      \pline{\text{is logically equivalent to\ }}\\
      \pline{
        ((P \THEN Q) \AND (Q \THEN P))
      }\\

      \pline{\text{which is logically equivalent to\ }}\\
      \pline{
       ((\NOT P) \OR Q) \AND ((\NOT Q) \OR P)
      }\\

      \pline{\text{which is logically equivalent to\ }}\\
      \pline{
        ((\NOT P) \OR Q) \AND (\NOT Q)) \OR (((\NOT P) \OR Q) \AND P)
      }\\
      \pline{\text{which is logically equivalent to\ }}\\
      \pline{((\NOT P) \AND (\NOT Q) ) \OR}\\
      \pline{(Q \AND (\NOT Q)) \OR}\\
      \pline{ ((\NOT P) \AND P) \OR (Q \AND P)
      }\\
      \pline{\text{which is logically equivalent to\ }}\\
      \pline{
        ((\NOT P) \AND (\NOT Q)) \OR (P \AND Q)
      }
    }

  Let $A= P \IFF Q$ be the propositional formula in $P, Q$.

  From the second line of the derivation above, observe that

  \fitchprf{P\IFF Q}{
    \pline{ \text{\ is logically equivalent to\ }}\\
    \pline{ (\NOT( P \AND (\NOT Q)) \AND \NOT (Q \AND (\NOT P)))}\\
    \pline{\text{which is logically equivalent to\ }}\\
    \pline{ \NOT(( P \AND (\NOT Q)) \OR (Q \AND (\NOT P)))}
}

  and thus

  $\NOT A = ( P \AND (\NOT Q)) \OR ((\NOT P) \AND Q )$

  Consider now $A \IFF R$. By substitution, we obtain that it is
  logically equivalent to

  \fitchprf{((\NOT A)\AND(\NOT R))\OR(A\AND R)}{
      \pline{\text{which is logically equivalent to\ }}\\
      \pline{((( P \AND (\NOT Q)) \OR ((\NOT P) \AND Q )) \AND \NOT R) \OR }\\
      \pline{[([(\NOT P) \AND (\NOT Q)] \OR (P \AND Q)) \AND R]}\\
      \pline{\text{which is logically equivalent to\ }}\\
      \pline{( P \AND (\NOT Q) \AND \NOT R) \OR}\\
      \pline{ ((\NOT P) \AND Q \AND \NOT R ) \OR }\\
      \pline{((\NOT P) \AND (\NOT Q) \AND R) \OR}\\
      \pline{(P \AND Q \AND R)} }

  Since there are 4 mutually exclusive conjunctive clauses, the
  formula cannot be simplified further. There are 3 variables in each
  clause, and thus the number of occurences is 12.
\item
  \label{item:3}
  Note that CNF can be obtained from DNF by the following procedure:

  \begin{enumerate}[label=\alph*)]
  \item construct a truth table for all the variables and the corresponding values of $f$
  \item write down a column for the negation of $f$
  \item for each row where the negation of $f$ is true, construct its
    minterm representation to obtain the DNF of the negation of $f$
  \item negate the negation of $f$ to obtain CNF
  \end{enumerate}

  By De Morgan's Laws, the number of variable frequencies in each
  negated minterm does not change. From (\ref{item:1}), the number of
  variables for $f$ in DNF is at most 12. Therefore, CNF also contains at most 12
  variables in total.
  % From (\ref{item:1}), given three variables, \textbf{P}, \textbf{Q},
  % and \textbf{R}, there are 8 possible truth value assignments for any
  % combination of \textbf{P}, \textbf{Q}, and \textbf{R}. Therefore, 8
  % clauses containing \textbf{P}, \textbf{Q}, and \textbf{R} completely
  % determine any propositional formula in three variables.
  % Suppose a propositional formula $f$ contains at most 3 different
  % variables, \textbf{P}, \textbf{Q}, and \textbf{R}. Let
  % $[f]_{\max} = [f]$ be the representation of $f$ in the CNF. From the
  % previous paragraph, 8 maxterms written in \textbf{P}, \textbf{Q},
  % and \textbf{R} determine $[f]$ completely.
  % Each maxterm consists of literals or disjunction of
  % literals.
\item
  \label{item:4}
  Suppose a propositional formula $f$ containing $b$ binary
  connectives is given.

  Consider an algorithm:

  \begin{enumerate}[label=\alph*)]
  \item parenthesise the formula so that any pair of related
    subformulas is either a pair of literals in parentheses, a literal
    related to a parenthesised multiliteral subformula, itself partenthesisesd, a
    pair of parenthesised subformulas, also parenthesised
  \item write down all the pairs of related subformulas, first listing
    parenthesised literals, then literals with multilinear
    subformulas, then pairs of multilinear subformulas, and assign
    index to the each element in the resultant list
  \item for each parenthesised pair of literals, assign a new variable
    $X_i$ to the corresponding pair, where $i$ is the index of the
    pair.
  \item for each pairs of literals with multilinear subformulas assign
    new variables $X_{j}$ to the pair, where $j$ is the index of the
    corresponding pair, with multilinear subformulas substitituted by
    the variables given in the previous step. If there are no
    variables left for the pairs of literals, use variables for
    multilinears produced in the current step
  \item for each pair of multilinears, replace multilinears by the
    variables given in the previous step
  \end{enumerate}

  Note that the procedure terminates, since there is a finite number
  of variables in the propositional formula given.

  Note that the total number of new variables produced by the
  described algorithm is equal to the number of binary connectives,
  which itself corresponds to the number of literal-literal and
  literal-multiliteral pairs in $f$.

  Note that all new variables and the corresponding pairs make
  satisfiable clauses. Making a conjuction out of those clause, we
  obtain a formula which is satisfiable if and only if $f$ is
  satisfiable. Note that, as described in the procedure, there are at
  most three variables in each obtained clause.

\item By (\ref{item:4}), for any propositional formula $f$ containing
  $b$ binary connectives there is a conjuction $g$ of at most $b$
  propositional formulas with at most 3 different variables each such
  that $f$ is satisfiable if and only if $g$ is satisfiable.

  By (\ref{item:3}), each clause of such a representation can be
  written in CNF with the frequency of variable occurence of at most
  12. This shapes the whole representation into the CNF, giving
  $[g]$. Note that each clause of $[g]$ may contain at most 1 or 2
  variables originally present in $f$ (by the algorithm).  Observe
  that, since there are $b$ binary connectives in $f$, it contains at
  most $b+1$ different variables. Similarly, the number of different
  variables in $[g]$ which are also in $f$ is at most $b+1$.

  Because each clause in $[g]$ is in one-to-one correspondence with
  these connectives in $f$, and all the different variables in $f$ are
  also contained in $[g]$, while in each clause the frequency of
  variable occurence is at most 12, then it follows by the remark in
  the previous paragraph that the frequency of variable occurence in
  $[g]$ corresponding to the variables which are both in $[g]$ and $f$
  is at most $12(b+1)$.

  Therefore, if any propositional formula G is satisfiable if and only
  if a propositional formula H in conjunctive normal form, such that
  each clause of H contains at most 3 different variables, is
  satisfiable, then the number of occurrences of variables in H is at
  most 12 times the number of occurrences of variables in G.

\end{enumerate}
\end{document}