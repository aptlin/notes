% -*- coding: utf-8; -*-
%%% Local Variables:
%%% mode: latex
%%% TeX-engine: xetex
%%% TeX-master: t
%%% End:
\documentclass[11pt]{scrartcl}
\usepackage[fancy, beaue, pset, anon]{masty}
\pSet{\nt{CSC240}{}{Diagonalisation}}
\usepackage{lineno}
% ----------------------------------------------------------------------
% Page setup
% ----------------------------------------------------------------------

\pagenumbering{gobble}

% ----------------------------------------------------------------------
% Custom commands
% ----------------------------------------------------------------------

% alignment

\newcommand*{\LongestHence}{$\Rightarrow$}% function name
\newcommand*{\LongestName}{$f_o(-x)+f_e(-x)$}% function name
\newcommand*{\LongestValue}{$(-a)x +(-a)(-y)$}% function value
\newcommand*{\LongestText}{\defi}%

\newlength{\LargestHenceSize}%
\newlength{\LargestNameSize}%
\newlength{\LargestValueSize}%
\newlength{\LargestTextSize}%

\settowidth{\LargestHenceSize}{\LongestHence}%
\settowidth{\LargestNameSize}{\LongestName}%
\settowidth{\LargestValueSize}{\LongestValue}%
\settowidth{\LargestTextSize}{\LongestText}%

% Choose alignment of the various elements here: [r], [l] or [c]

\newcommand*{\mbh}[1]{{\makebox[\LargestHenceSize][r]{\ensuremath{#1}}}}%
\newcommand*{\mbn}[1]{{\makebox[\LargestNameSize][r]{\ensuremath{#1}}}}%
\newcommand*{\mbv}[1]{\ensuremath{\makebox[\LargestValueSize][r]{\ensuremath{#1}}}}%
\newcommand*{\mbt}[1]{\makebox[\LargestTextSize][l]{#1}}%

\newcommand{\R}[1]{\label{#1}\linelabel{#1}}
\newcommand{\lr}[1]{line~\lineref{#1}}

% ----------------------------------------------------------------------
% Launch!
% ----------------------------------------------------------------------

\begin{document}
\section{Diagonalisation}
\begin{theorem}
$P(\NN)$ is uncountable.
\end{theorem}

\begin{proof}
  \hfill

  Suppose $P(\NN)$ is countable. Thus, there is a surjective function
  $f:\NN\to P(\NN)$.

  Let $D = \set{i\in\NN; i\not\in f(i)}\in P(\NN)$.

  Since $f$ is surjective, there exists $j\in\NN$ such that $f(j) = D$.

  Then for all $i\in\NN$ $i\in f(j)$ if and only if $i\in D$, and thus
  iff $i\not\in f(i)$.

  Since $j\in\NN$, by specialisation $j\in f(j) \IFF j \not \in f(j)$ , which is a contradiction.

  Therefore, $P(\NN)$ is uncountable.
\end{proof}


Now, let $S\suq \set{1, 2, 3, 4}$. Note that $S$ can be represented by a binary sequence $S_1,S_2, S_3, S_4$, where $S_i= \begin{cases}
1, i\in S\\
0, i\not\in S.
\end{cases}$

Each $S_i$ is called a \textit{characteristic vector} of the set $S$.

In thus way, for example, $0, 1, 1, 0$ denotes $\set{2, 3}$ and $0,0,0,0$ denotes $\emptyset$.

Consider a list of all subsets of $\NN$, possibly with duplications, in the form of an infinite two-dimensional Boolean array $M$, where $M[i, j] = f(i)_j$:
\[
  \begin{pmatrix}
    f(0): f(0)_0 & f(0)_1 & \cdots & f(0)_j & \cdots\\
    f(1): f(1)_1 & f(1)_1 & \cdots & f(1)_j & \cdots\\
     & \vdots & \ddots & &
  \end{pmatrix}
\]

The characteristic vector of $D$ is the complement of the diagonal of the matrix $M$.

Note that the characteristic vector of $D$ does not agree with row $i$
of $M$ in column $i$. The characteristic vector is equal to $1$ if and only if $i\not\in f(i)$. Thus, $M$ does not contain $D$, which contrandicts that $f$ is surjective.

\begin{theorem}
The set of all functions from $\NN$ to $\NN$ is uncountable.
\end{theorem}

\begin{proof}
  \hfill

  Suppose $\SF$ is countable. Then there is a surjective function $f: \NN\to\SF$.

  $f_i$ is a function from $\NN$ to $\NN$. Construct an infinite two-dimensional matrix where row $i$ corresponds to $f_i\in f$. Define $g(n) = f(n) + 1$. Note that $g\in \SF$, since for all $n\in\NN$, $f_n(n) \in\NN$ and $\NN$ is closed under addition.

  Since $f$ is surjective, then $g = f_n$ for some $n\in\NN$. Then
  $g(n) = f_n(n)$. By definition, $g(n) = f_n(n)+1\neq f_n(n)$. This
  is a contradiction, and therefore $\SF$ is uncountable.
\end{proof}

If $\Sigma$ is a finite set of letters, let $\Sigma^{*}$ be a set of all finite strings of letters from $\Sigma$.

For example, $\set{0, 1}^{*}$ is the set of all finite length binary strings.

Note that there are $2^k$ binary strings of length $k\in\NN$.

\begin{theorem}
  $\set{0, 1}^{*}$ is countable.
\end{theorem}

\begin{proof}
  \hfill

Exercise. Construct a surjective function $f$ from $\NN$ to $\set{0, 1}^{*}$.
\end{proof}

In general, if $\abs{\Sigma} = s$ , then there are $s^k$ strings of length $k$ in $\Sigma^{*}$.

The set of all finite strings of ASCII characters is countable.
\begin{example}

The set of all syntactically correct C programmes is countable, since it is a subset of ASCII$^{*}$.

\end{example}

There is a function $G$ that determines whether a given ASCII string $P$ is a syntactically correct $C$ function:


\begin{equation}
  G: \text{ASCII}^{*} \to \set{0, 1},
\end{equation}

and thus

\begin{equation}
  G(P)= \begin{cases}
    1,\text{ if P is a syntactically correct C function}\\
    0, \text{otherwise}
  \end{cases}
\end{equation}

Consider the function $H:\text{ASCII}^{*}\times \text{ASCII}^{*} \to \set{0, 1}$ such that
\[
  H(P, x) =
\begin{cases}
  1 \text{if P is a syntactically correct C program that eventually returns an input $X$}\\
  0 \text{ otherwise}.
\end{cases}
\]

In this way, $H(P, x)$ returns $0$ if $P$is not syntactically correct or $P$ runs forever on input $X$.

The halting problem is solvable if such a C function $H$ exists.

\begin{theorem}
The halting problem is not solvable.
\end{theorem}

\begin{proof}
  \hfill

  We proceed by contradiction and diagonalisation.

  By way of contradiction, assume that such a function $H$ exists.

  Consider the syntactically correct C-function $D$ defined by the following program:

  D(x); \{

  if (H(x,x))

  while (1) \{
  \};
  
  else return 1;
  \}

  When $D$ runs on input $D$, if $H(D,D) = 0$, then $D$ returns on
  input $D$. If $H(D, D) = 1$, then $D$ goes into an infinite loop on
  input $D$.

  By the definition of $H$, if $D$ returns on input $D$, then $H(D, D) = 1$.

  If $D$ goes into an infinite loop on input $D$, then $H(D, D) = 0$.

  These are contradictory, and thus the halting problem is unsolvable
  and hence such a function $H$ does not exist.
  
\end{proof}

\end{document}