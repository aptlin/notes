% -*- coding: utf-8; -*-
%%% Local Variables:
%%% mode: latex
%%% TeX-engine: xetex
%%% TeX-master: t
%%% End:
\documentclass[11pt]{scrartcl}
\usepackage[fancy, beaue, pset, anon]{masty}
\pSet{\nt{CSC240}{VII}{Induction}}
\usepackage{lineno}
% ----------------------------------------------------------------------
% Page setup
% ----------------------------------------------------------------------

\pagenumbering{gobble}

% ----------------------------------------------------------------------
% Custom commands
% ----------------------------------------------------------------------

% alignment

\newcommand*{\LongestHence}{$\Rightarrow$}% function name
\newcommand*{\LongestName}{$f_o(-x)+f_e(-x)$}% function name
\newcommand*{\LongestValue}{$(-a)x +(-a)(-y)$}% function value
\newcommand*{\LongestText}{\defi}%

\newlength{\LargestHenceSize}%
\newlength{\LargestNameSize}%
\newlength{\LargestValueSize}%
\newlength{\LargestTextSize}%

\settowidth{\LargestHenceSize}{\LongestHence}%
\settowidth{\LargestNameSize}{\LongestName}%
\settowidth{\LargestValueSize}{\LongestValue}%
\settowidth{\LargestTextSize}{\LongestText}%

% Choose alignment of the various elements here: [r], [l] or [c]

\newcommand*{\mbh}[1]{{\makebox[\LargestHenceSize][r]{\ensuremath{#1}}}}%
\newcommand*{\mbn}[1]{{\makebox[\LargestNameSize][r]{\ensuremath{#1}}}}%
\newcommand*{\mbv}[1]{\ensuremath{\makebox[\LargestValueSize][r]{\ensuremath{#1}}}}%
\newcommand*{\mbt}[1]{\makebox[\LargestTextSize][l]{#1}}%

\newcommand{\R}[1]{\label{#1}\linelabel{#1}}
\newcommand{\lr}[1]{line~\lineref{#1}}

% ----------------------------------------------------------------------
% Launch!
% ----------------------------------------------------------------------

\begin{document}

\section{Induction}

\begin{example}

  Let $p:\NN\to\set{T, F}$ be a predicate.
  $\forall n \in \NN.p(n)$.

\begin{enumerate}[label=\alph*)]
\item\label{item:1} a. $p(0)$ is base case or basis step.
\item  Let $n\in\NN$ be arbitrary.
\item  Assume $p(n)$.
\item  p(n+1)
\item  $p(n) \THEN p(n+1)$ direct proof c-d
\item 
    $\forall n \in \NN.(p(n) \THEN p(n+1) \text{ generalization b-e }$

\item $\forall n \in \NN.(p(n) \text{ induction a-f }$

\end{enumerate}
\end{example}
\begin{theorem}
  Consider any square chessboard which sides have length which is a
  power of 2. If any one square is removed, then the resulting shape
  can be tiled using only 3-square L-shaped tiles.
\end{theorem}

\begin{proof}
  For all $n\in\NN$, let

  \[\shortstack[l]{ P(n) =\\ \ }
    \text{\shortstack[l]{ ``any $2^n\times 2^n$ with 1 square
        removed\\ can be tiled using 3-square L-shaped tiles. '' }}\]
  Let $C_n$ the set of all $2^n\times 2^n$ chessboards with 1 square removed.

  Let ``L-tile'' denote a 3-square L-shaped tile.

  $P(n) = \text{``$\forall c\in C_n.(c$ can be tiledusing only L-tiles.$)$''}$.

  $\forall n\in \NN. P(n)$.

  Basis: $P(0)$ is true.

  A $2^0\times 2^0$ chessboard with 1 square removed has no squares
  and, therefore, can be tiled with 0 tiles.

  Let $n\in\NN$ be arbitrary.

  Suppose $P(n)$ is true.

  Divide $c$ into $4$ equal $2^n\times 2^n$ chessboards. One of these has a square removed, so it is in $C_n$, and hence, by the induction hypothesis, it can be tiled with L-tiles.

  Consider the other 3 chessboards.

  Each has 1 square that is one of the 4 squares in the middle of $c$.

  With those squares removed, the remaining three squares are also in
  $C_n$, is the inductive hypothesis implies that they can be tiled by
  L-tiles.

  The 3 squares in the middle can be tiled with 1 L-tile.

    $c$ can be tiled using L-tiles.
  
  $\forall c\in C_{n+1}.$ $c$ can be tiled using L-tiles.

  $P(n+1)$ | generalisation
  
\end{proof}

Another theorem can be proved using the result above:

\begin{theorem}
  All square chessboards with sides of length a power of 2 and with 1
  square removed from the middle can be tiled using L-tiles.
\end{theorem}

\begin{theorem}
$\forall n \in (n\geq 3 \THEN 2n+1 \leq 2^n)$.
\end{theorem}

\begin{proof}
  For $n \in \NN$, let $q(n)=$``$2n+1 \leq 2^{n}$''.

  Let $n\in \NN$ be arbitrary. Assume $q(n)$.
  
  Base case: Let $p(n) = q(n+3)$ for all $n\in \NN$.

  Basis:

  $P(0)$ is true.

  Induction step:

  Let $n\in \NN$ be arbitrary.

  Assume $p(n)$.

  $\dots$

  $p(n+1)$

  $p(n)\THEN p(n+1)$, direct proof

  $\forall n\in \NN.(p(n)\THEN p(n+1)$, generalization
  
  $\forall n\in \NN. p(n)$, induction

\end{proof}


\end{document}

