% -*- coding: utf-8; -*-
%%% Local Variables:
%%% mode: latex
%%% TeX-engine: xetex
%%% TeX-master: t
%%% End:
\documentclass[11pt]{scrartcl}
\usepackage[fancy, beaue, pset, anon]{masty}
\pSet{\nt{CSC240}{}{Correctness of Algorithms}}
\usepackage{lineno}
% ----------------------------------------------------------------------
% Page setup
% ----------------------------------------------------------------------

\pagenumbering{gobble}

% ----------------------------------------------------------------------
% Custom commands
% ----------------------------------------------------------------------

% alignment

\newcommand*{\LongestHence}{$\Rightarrow$}% function name
\newcommand*{\LongestName}{$f_o(-x)+f_e(-x)$}% function name
\newcommand*{\LongestValue}{$(-a)x +(-a)(-y)$}% function value
\newcommand*{\LongestText}{\defi}%

\newlength{\LargestHenceSize}%
\newlength{\LargestNameSize}%
\newlength{\LargestValueSize}%
\newlength{\LargestTextSize}%

\settowidth{\LargestHenceSize}{\LongestHence}%
\settowidth{\LargestNameSize}{\LongestName}%
\settowidth{\LargestValueSize}{\LongestValue}%
\settowidth{\LargestTextSize}{\LongestText}%

% Choose alignment of the various elements here: [r], [l] or [c]

\newcommand*{\mbh}[1]{{\makebox[\LargestHenceSize][r]{\ensuremath{#1}}}}%
\newcommand*{\mbn}[1]{{\makebox[\LargestNameSize][r]{\ensuremath{#1}}}}%
\newcommand*{\mbv}[1]{\ensuremath{\makebox[\LargestValueSize][r]{\ensuremath{#1}}}}%
\newcommand*{\mbt}[1]{\makebox[\LargestTextSize][l]{#1}}%

\newcommand{\R}[1]{\label{#1}\linelabel{#1}}
\newcommand{\lr}[1]{line~\lineref{#1}}

% ----------------------------------------------------------------------
% Launch!
% ----------------------------------------------------------------------

\begin{document}

\section{Correctness of Algorithms}

An algorithm is said to be \textbf{correct} if it satisfies its
specifications.

A \textbf{precondition} is a statement which involves variables used
in the algorithm, specifying certain facts must hold before the
execution begins.

A \textbf{postcondition} is a statement about the variables used in
the algorithm, specifying the conditions which must be satisfied when
the execution ends without error. Postconditions often identify the
correct output for a given input.

\textbf{Termination} is the final state of the halting algorithm,
provided the preconditions are satisfied.

If the preconditions hold and the algorithm halts if executed, while also implying that the postconditions hold, the algorithm is said to be \textbf{partially correct}.

If an algorithm is partically correct and there is a termination
state, then the algorithm is \textbf{totally correct}.

\begin{problem*}
  Search an array $A$ for a key $k$.
\end{problem*}

\begin{description}

\item[Preconditions]\hfill
  
  \begin{itemize}
  \item $A$ is finite
  \item $k$ is of the same type as
    elements of $A$ or $k$ can be compared to the elements of $A$
  \end{itemize}
\item[Postconditions] \hfill
  
  \begin{itemize}
  \item return true if $k\in A$, false otherwise
  \item return an integer $i$ such that $1\leq i\leq \text{length} A$ and $A[i]=k$, or $-1$ if $k \not\in A$
  \item return a list of all locations in $A$ such that $A[i] = k$
  \item $A$ and $k$ are not changed by the programme
  \end{itemize}
\end{description}

We alswant to ensure that a programme acesses only elements of $A$ where $A$ is defined.

Consider now an algorithm of binary search

As our preconditions, we have that elements of $A$ and $k$ must belong
to a totally ordered set, and $A$ must be sorted in the nondecreasing
order.

For the postconditions, we must ensure that elements in the array are
permutations of the elements originally in the array, in the
nondecreasing order.

Suppose now we want to merge two arrays, $U$ and $V$, into $W$.

\begin{description}

\item[Preconditions] \hfill

  $U$ and $V$ are sorted in the nondecreasing order.

\item[Postconditions] \hfill

  $W$ is sorted in the nondecreasing and the multiset of elements in
  $W$ consists of the multiset union of the multiset of elements in
  $U$ and the multiset of elements in $V$. We must also make sure that
  $U$ and $V$ are not changed.

\end{description}

Consider now MERGESORT.

For $n\in\NN$, let $P(n) =$\textquote{for all arrays $A[1, n]$ with
  elements from a totally ordered domain, if MERGESORT$(A, n)$ is
  performed, then it eventually halts, while $A$ is sorted in the
  nondecreasing order and the multiset of elements in $A$ is
  unchanged}.

We prove that $\forall n\in\NN. P(n)$ by induction.

Let $n\in\NN$ be arbitrary.

Consider MERGESORT$(A, n)$ for any array $A[1, \dots, n]$.

\begin{description}
\item[Base Case] \hfill

  If $n=0$, the test on line 1 fails and $A$is
  unchanged. Vacuously,$A$ is sorted in the nondecreasing order.

\item[Inductive Step] \hfill

  If $n>1$, suppose that $P(n')$ is true for all $n'\in\NN$ such that $n' < n$.

  The test on the line 1 succeeds.

  By the Inductive Hypothesis, after lines 3 and 4 $A'$ and $A''$ are sorted in the nondecreasing order and the multiset of elements in $A'$ and $A''$ are unchanged.

  By the correctness of MERGE, the elements of $A$ are in
  nondecreasing order and the multiset of $A$, the union of multiset
  of $A'$ and $A'$, which has the same contents of $A$. 
\item[Conclusion]
  \hfill

  Hence, $P(n)$ and thus $\forall n\in\NN. P(n)$ by induction.
\end{description}
\end{document}
