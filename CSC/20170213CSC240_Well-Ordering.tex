% -*- coding: utf-8; -*-
%%% Local Variables:
%%% mode: latex
%%% TeX-engine: xetex
%%% TeX-master: t
%%% End:
\documentclass[11pt]{scrartcl}
\usepackage[fancy, beaue, pset, anon]{masty}
\pSet{\nt{CSC240}{XI}{Well-Ordering}}
\usepackage{lineno}
% ----------------------------------------------------------------------
% Page setup
% ----------------------------------------------------------------------

\pagenumbering{gobble}

% ----------------------------------------------------------------------
% Custom commands
% ----------------------------------------------------------------------

% alignment

\newcommand*{\LongestHence}{$\Rightarrow$}% function name
\newcommand*{\LongestName}{$f_o(-x)+f_e(-x)$}% function name
\newcommand*{\LongestValue}{$(-a)x +(-a)(-y)$}% function value
\newcommand*{\LongestText}{\defi}%

\newlength{\LargestHenceSize}%
\newlength{\LargestNameSize}%
\newlength{\LargestValueSize}%
\newlength{\LargestTextSize}%

\settowidth{\LargestHenceSize}{\LongestHence}%
\settowidth{\LargestNameSize}{\LongestName}%
\settowidth{\LargestValueSize}{\LongestValue}%
\settowidth{\LargestTextSize}{\LongestText}%

% Choose alignment of the various elements here: [r], [l] or [c]

\newcommand*{\mbh}[1]{{\makebox[\LargestHenceSize][r]{\ensuremath{#1}}}}%
\newcommand*{\mbn}[1]{{\makebox[\LargestNameSize][r]{\ensuremath{#1}}}}%
\newcommand*{\mbv}[1]{\ensuremath{\makebox[\LargestValueSize][r]{\ensuremath{#1}}}}%
\newcommand*{\mbt}[1]{\makebox[\LargestTextSize][l]{#1}}%

\newcommand{\R}[1]{\label{#1}\linelabel{#1}}
\newcommand{\lr}[1]{line~\lineref{#1}}

% ----------------------------------------------------------------------
% Launch!
% ----------------------------------------------------------------------

\begin{document}

\section{Well-Ordering}

\begin{definition}
An \textit{order set} (or \textit{partially ordered set}) $S$ is \textbf{well-ordered} if every nonempty subset of $S$ has a smallest element. 
\end{definition}
\begin{description}

\item[e.g.] If ordered by value, $\NN$ is well-ordered, while $\ZZ$ and $\QQ$ are not. However, if $\ZZ$ is ordered by absolute value and then by value, then $\ZZ$ is well-ordered. Similarly, if $\QQ^+$ is considered with all the elements in the reduced form, then ordering first by a denominator and then by a numerator induces well-ordering. Another well-ordering $\QQ^+$ is induced by taking $\max{\text{numerator},\text{denominator}}$ in reduced form and then ordering by value, which is a construction used in Cantor's diagonal argument.
\end{description}

The proposition $\forall e\in S.P(e)$ can be proved using the well-ordering principle for any well-ordered set $S$ with ordering $\prec$.

\begin{description}
\item[L1] To obtain a contradiction, suppose that $\forall e\in S.P(e)$ is false.
\item[L2] Let $C = \set{e\in S; P(e) \text{ is false}}$ be the set of counterexamples to $P$.
\item[L3] $C\neq \emptyset$ by L1 and L2
\item[L4] Let $e$ be the smallest element of $C$; well ordering principle, L3.
\item[L5] Let $e' = \dots$
\item[L6] $e'\in C$
\item[L7] $e' \prec e$
\item[L8] This is a contradiction to L4, L5, L6
\item[L9] $\forall e \in S. P(e)$; proof by contradiction, L1 - L7
\end{description}

\section{Diagonalization}

\begin{definition}
  A function $f: A\to B$ is said to be \textit{surjective} or \textit{onto} if $\forall y\in B.\exists x\in A.f(x) = y$. 
\end{definition}

From the existence of a surjective function, if $A$ and $B$ are finite sets, we can conclude that $\abs{B}\geq \abs{A}$.
\begin{definition}
A nonempty set $C$ is countable if there is a surjective function from $\NN$ to $\CC$. By convention, an empty set is countable.
\end{definition}

Note that, by definition, every finite set is countable.

Suppose $C = \set{c_0, \dots, c_{n-1}}$ is a nonempty finite set of $n$ elements. Define $f:\NN\to \CC$ such that

\begin{equation}
  f(i)=\begin{cases}
    c_i \text{, for $i = 0,\dots, n-1$}\\
    c_{n-1} \text{, for $i\geq n$}
   \end{cases}.
 \end{equation}

 Then $f$ is surjective and hence $C$ is countable.

 Moreover, $\ZZ$ is countable. Then

 \begin{equation}
   g(x) = \begin{cases}
     \frac{-x-1}{2} \text{, if $x$ is odd}\\
     \frac{x}{2} \text{if $x$ is even}
   \end{cases}
 \end{equation}

 is surjective, and hence $\ZZ$ is countable.

 Furthermore, if $A$ and $B$ are countable, then so is $A\cup B$. To prove this, we need the following lemma.

 \begin{lemma}
   If $A$ is countable and there is a surjective function $f: A\to B$,
   then $B$ is countable.
 \end{lemma}

 \begin{proof}
   \hfill
   Since $A$ is countable, then there exists a surjective function $g:\NN \to A$. Consider the function $h:\NN \to B$ defined by $h(i) = f(g(i))$ for all $i\in\NN$.

   To prove that $h$ is surjective, consider any $z\in B$. Since $f$
   is surjective, there exists $y\in A$ such that $f(y) = z$. Since
   $g$ is surjectiv, then there exists $x\in\NN$ such that $g(x) =
   y$. Hence, $f(g(x)) = z$, and by construction $h$ is surjective.
 \end{proof}

 
 Consider the function $f(m, n) = 2^m\* 3^{n}\in\NN$ for any
 $m, n\in\NN$.
 
 \begin{definition}
   For any set $A$, the power set of $A$ is the set of all subsets of
   $A$. Thus, $P(A) = \set{S; S\suq A}$. 
 \end{definition}

 Note that if $\abs{A}=n$, then $\abs{P(A)} = 2^n$.

 \begin{theorem}
   $P(\NN)$ is uncountable.
 \end{theorem}

 \begin{proof}
   \hfill

   Suppose $P(\NN)$ is countable. Then there is a surjective function $f:\NN \to P(\NN)$.

   Let $D = \set{i\in\NN; i\not\in f(i)} \in P(\NN)$.

   Since $f$ is surjective, there exists $j\in\NN$ such that $f(j) = D$. Then for all $i\in\NN$, $i\in f(j)\IFF i\in D$, since $f(j) = D$. This is true if $i\not\in f(i)$ by definition of $D$. Since $j\in\NN$, by specialisation $j\in f(j) \IFF j\not\in f(j)$. This is a contradiction, and therefore $P(\NN)$ is uncountable.
 \end{proof}





 









\end{document}