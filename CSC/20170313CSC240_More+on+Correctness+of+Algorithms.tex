% -*- coding: utf-8; -*-
%%% Local Variables:
%%% mode: latex
%%% TeX-engine: xetex
%%% TeX-master: t
%%% End:
\documentclass[11pt]{scrartcl}
\usepackage[fancy, beaue, pset, anon]{masty}
\pSet{\nt{CSC240}{}{Correctness of Algorithms}}
\usepackage{lineno}
% ----------------------------------------------------------------------
% Page setup
% ----------------------------------------------------------------------

\pagenumbering{gobble}

% ----------------------------------------------------------------------
% Custom commands
% ----------------------------------------------------------------------

% alignment

\newcommand*{\LongestHence}{$\Rightarrow$}% function name
\newcommand*{\LongestName}{$f_o(-x)+f_e(-x)$}% function name
\newcommand*{\LongestValue}{$(-a)x +(-a)(-y)$}% function value
\newcommand*{\LongestText}{\defi}%

\newlength{\LargestHenceSize}%
\newlength{\LargestNameSize}%
\newlength{\LargestValueSize}%
\newlength{\LargestTextSize}%

\settowidth{\LargestHenceSize}{\LongestHence}%
\settowidth{\LargestNameSize}{\LongestName}%
\settowidth{\LargestValueSize}{\LongestValue}%
\settowidth{\LargestTextSize}{\LongestText}%

% Choose alignment of the various elements here: [r], [l] or [c]

\newcommand*{\mbh}[1]{{\makebox[\LargestHenceSize][r]{\ensuremath{#1}}}}%
\newcommand*{\mbn}[1]{{\makebox[\LargestNameSize][r]{\ensuremath{#1}}}}%
\newcommand*{\mbv}[1]{\ensuremath{\makebox[\LargestValueSize][r]{\ensuremath{#1}}}}%
\newcommand*{\mbt}[1]{\makebox[\LargestTextSize][l]{#1}}%

\newcommand{\R}[1]{\label{#1}\linelabel{#1}}
\newcommand{\lr}[1]{line~\lineref{#1}}

% ----------------------------------------------------------------------
% Launch!
% ----------------------------------------------------------------------
\usepackage{program}
\begin{document}

Consider the algorithms of MERGESORT and QUICKSORT.

QUICKSORT($A$)
\begin{program}
  \IF \abs{A}>1 \THEN 

  |pivot|\gets A[i]

  |partition | A | into multisets| 

  L=\{\text{elements in } A < \text{ pivot}\}

  E=\{\text{elements in } A = \text{ pivot}\}

  G=\{\text{elements in } A > \text{ pivot}\}

  |QUICKSORT| (L)
  |QUICKSORT| (G)

  A \gets L, E, G 

  \FI 

\end{program}

Let $G(n)=$\textquote{For all arrays $A$ with $n$ elements from a totally ordered domain, if QUICKSORT($A$) is performed, $\dots$ unchanged.}

We proceed by induction.

Let $n\in\NN$ be arbitrary. 

Let $A$ be an arbirtrary array with $n$ elements from a totally ordered domain.

\begin{description}
\item[Base Case] \hfill

  If $n=0$ or $n=1$, the test on line 1 fails and thus $A$ is
  unchanged and vacuously sorted. 
\item[Inductive Step] \hfill

Suppose $G(n')$ is true for all $n'\in \NN$ with $n'< n$.

The test on line 1 succeeds.

By partitioning, all elements in $L$ are less than all elements in $E$, which are less than all elements in $G$. A multiset of elements in $A$ is the union of the multiset of elements in $L, E, G$.

$A[i]\in E$, so $\abs{L}$ and $\abs{G}$ are less than $\abs{A}$.

By the IH, after QUICKSORT($L$) and QUICKSORT($G$) are performed $L$ and $G$ are  sorted in nondecreasing order and the multiset of elements in $L$ and $G$ are unchanged.

After the assignment on line 6, $A$ is sorted in a non-decreasing order. All elements in $L$ are therefore less than all elements in $E$, which in turn are less than all elements in $G$, and thu multiset of elements in $A$ is unchanged. By generalisation, $P(n)$.

By induction, for any $n\in\NN.P(n)$.
\end{description}

\section{Divide and Conquer Algorithms}

\begin{itemize}
\item divide the problem into smaller parts, often of roughly equal size
\item solve each part independently
\item combine the solutions for the parts into a solution for the whole problem
\end{itemize}

\subsection{Correctness of Iterative Algorithms}

\begin{program}

z\gets 0
w \gets y
\WHILE w \neq 0 \DO
z \gets z+x
w\gets w-1
\OD
\end{program}

What are the values of the variables immediately after iteration $i$
of the loop? ($w=y-i$, $z = ix$)

Let $P(i)=$\textquote{if the loop is executed at least $i$ times, then
  immediately afer the iteration $i$ we have $w=y-i$ and $z = ix$}.

Note that, by convention, when we talk about the 0th iteration, we are
talking about the state immediately before the 1st iteration.

\begin{lemma}
\label{sec:corr-iter-algor}
Let $x, y \in\ZZ$. For all $i\in\NN$, we have $P(i)$.
\end{lemma}

\begin{proof}
  \hfill

Let $W_i$ and $z_i$ denote the values of $w$ and $z$ immediately after iteration $i$.

\begin{description}
\item[Base Case] \hfill

$w_0 = y =y-0$ by line 2.

$z_0 = 0 = 0\* x$ by line 1,

so $P(0)$ is true.

\item[Inductive Hypothesis] \hfill

Let $i\geq 0$ and assume $P(i)$ is true.

Then $w_i = y-i$ and $z_i = ix$.

From lines 4 and 5, we have that $z_{i+1}=z_i+x$ and $w_{i+1}=w_i-1$,
which by inductive hypothesis means that $z_{i+1}=(i+1) x$ and
$w_{i+1}= y-(i+1)$, and hence $P(i+1)$ holds.

By induction, we have that $\forall i\in\NN.P(i)$.
\end{description}
\end{proof}
\begin{corollary}
If the algorithm runs and halts, then at the end $z = xy$.
\end{corollary}
\begin{proof}
  \hfill

Suppose the loop halts immediately after the iteration $i$.

From the termination condition of the loop on line 3 we have that
$W_i= 0 $. By Lemma \ref{sec:corr-iter-algor},$w_i = y-i$ and
$z_i = ix$, so $i=y$ and $z_{i} = xy$ .
\end{proof}

A \textbf{loop invariant} is a predicate that is true each time a
particular place in the loop is reached (often the beginning or end).

\begin{lemma}
  $z = x(y-w)$ is a loop invariant which is true at the beginning and
  end of every iterations.
\end{lemma}

\begin{proof}
  \hfill

  Initially, yrom lines 1 and 2 we can see that $z = 0$ and $w =y$, so
  $x(y-w) = 0=z$.

  Consider an arbirtrary iteration of the loop.

  Let $w'$ and $z'$ denote the values of $w$ and $z$ before the
  iterations. Let $w''$ and $z''$ denote their variable at the end of the iteration. 

  Suppose the claim holds at the begining of the iteration so
  $z'=x(y-w')$.

  From lines 4 and 5, 

  $w'' =w' - 1$ and $z''=z'+x$, so
  $x(y-w'') = x(y-(w'-1)) = x(y-w') +x = z'+x = z''$, so the claim is
  true at the end of the iterations.

  By induction, $z = x(y-0)$ is true after every iteration.
\end{proof}

To prove that an algorithm terminates, we need to show that some nonnegative integral quantity decreases each time through the loop.

\begin{lemma}
If $x\in \ZZ$, $y\in\NN$, and the algorithm runs, it eventually halts.
\end{lemma}

\begin{proof}
  \hfill

  Before the loop is executed, $w\gets y \in \NN$.

  In each iteration of the loop, $w$ is decreased by 1, so it is a smaller natural number.

  Hence $w$must eventually reach 0, which is the exiting condition of
  the loop. Thus the loop terminates and the algorithm halts.
\end{proof}

\begin{proof}[\textit{More Formal}]
  \hfill

  Suppose the loop does not terminate.

  Let $w_{i}$ be the value of $w$immediately before the $i$th
  iteration of the loop.

  In each iteration of the loop $w$ is decreased by 1, so $w_{i+1}< w_i$.

  Since the loop does not terminate, we have $w_i \neq 0$.

  Thus, if $w_i\in\NN$, then $w_{i+1}\in\NN$.

  Also, we know that $w_1=y\in\NN$.

  By induction, $w_1, w_2, w_{3, \dots}$ is a decreasing sequence of
  natural numbers.

  By the Well-Ordering Principle, the set of elements in this sequence
  has a smallest element.

  Suppose that $w_{j}$ is the smallest element.

  But $w_{j+1}<w_{j}$, which contradicts the assumption that $w_j$ is
  the smallest element. Thus, the loop eventually terminates.
\end{proof}

\begin{proof}
  \hfill

  Prove by induction that $w_i=y -i$.

  Then $w_y = 0$, and so the algorithm terminates.
\end{proof}

Now we consider the algorithm of multiplication by shifting and addition.


MULT(m, n)
\begin{program}
  x \gets m
  y \gets n
  z \gets 0
  \WHILE x \neq 0 \DO
  \IF x \mod 2 = 1
  \THEN z \gets z+y
  x \gets x | div | 2
  y \gets y \* 2
  \FI
  \OD
  |return | z
\end{program}

Preconditions: $m\in \NN$, $n\in \RR$

Postconditions: $z=mn$

To derive loop invariants, it is worthwhile to look at examples.

Suppose $m=11 = 1011_2$, $n=20 = 10100_2$.

Let $x, y \neq z$ denote the values of $x, y\neq z$ immediately after the iteration $i$.

Therefore,

\begin{equation*}
\begin{matrix}
  i & x_i & (x_i)_2 & x_i\mod 2 & y_i & z_i \\
  0 & 11 & 1011 & 1 & 20 & 0\\
  1 & 5  & 101  & 1 & 40 & 20\\
  2 & 2  & 10   & 0 & 80 & 60\\
  3 & 1  & 1    & 1 & 160 & 60\\
  4 & 0  & 0    & 0 & 320 & 220
\end{matrix} 
\end{equation*}

Let $m[k-1], \dots, m[0]$ denote the binary representation of $m$:
\begin{equation*}
m = \sum_{j=0}^{k-1}m[j]2^j.
\end{equation*}

Thus, $m[j]\in \{0, 1\}$ for $j=0, \dots, k-1$.

Hence, $x_i = \sum_{i-i}^{k-1}m[j]2^{k-i}$ and $\floor{\nicefrac{m}{2}} = \floor{\sum_{j=0}^{k-1}m[j]2^{j-1}} = \sum_{i=1}^{k-1}m[j]2^{j-1}$.

Note that $z_0 = 0$.

Therefore, $z_{i+1} = z_i+m[i]\*n2^i$, and thus
\begin{align}
  z_{i+1} & = z_i+m[i]\*n2^i                                           \\
          & = z_{i-1} + m[i-1]\* n 2^{i-1} + m[i]n2^{i}                \\
          & = n\sum_{i=0}^{i}m[i]\*2^{i}                               \\
          & = n(\sum_{j=0}^{k-1} m[j]2^j- \sum_{j=i+1}^{k-1}m[j]2^{j}) \\
          & =n(m-x_{i+1}2^{i+1})                                           \\
          & =nm-x_{i+1}y_{i+1}
\end{align}

\begin{corollary}[Partial Correctness]

  Let $m\in\NN$ or $n\in\RR$.

  If the algorithm MULT($m, n$) is run and it halts, then, when it
  halts, $Z = nm$.
\end{corollary}
\begin{proof}
  \hfill

  From the termination condition of the loop $x=0$, so $z = nm -xy$, and by the result already obtained, $z = nm -xy = nm$.

\end{proof}
\end{document}