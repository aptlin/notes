% -*- coding: utf-8; -*-
%%% Local Variables:
%%% mode: latex
%%% TeX-engine: xetex
%%% TeX-master: t
%%% End:
\documentclass[11pt]{scrartcl}
\usepackage[fancy, beaue, pset, anon]{masty}
\pSet{\nt{CSC240}{}{Language Theory}}
\usepackage{lineno}
% ----------------------------------------------------------------------
% Page setup
% ----------------------------------------------------------------------

\pagenumbering{gobble}

% ----------------------------------------------------------------------
% Custom commands
% ----------------------------------------------------------------------

% alignment

\newcommand*{\LongestHence}{$\Rightarrow$}% function name
\newcommand*{\LongestName}{$f_o(-x)+f_e(-x)$}% function name
\newcommand*{\LongestValue}{$(-a)x +(-a)(-y)$}% function value
\newcommand*{\LongestText}{\defi}%

\newlength{\LargestHenceSize}%
\newlength{\LargestNameSize}%
\newlength{\LargestValueSize}%
\newlength{\LargestTextSize}%

\settowidth{\LargestHenceSize}{\LongestHence}%
\settowidth{\LargestNameSize}{\LongestName}%
\settowidth{\LargestValueSize}{\LongestValue}%
\settowidth{\LargestTextSize}{\LongestText}%

% Choose alignment of the various elements here: [r], [l] or [c]

\newcommand*{\mbh}[1]{{\makebox[\LargestHenceSize][r]{\ensuremath{#1}}}}%
\newcommand*{\mbn}[1]{{\makebox[\LargestNameSize][r]{\ensuremath{#1}}}}%
\newcommand*{\mbv}[1]{\ensuremath{\makebox[\LargestValueSize][r]{\ensuremath{#1}}}}%
\newcommand*{\mbt}[1]{\makebox[\LargestTextSize][l]{#1}}%

\newcommand{\R}[1]{\label{#1}\linelabel{#1}}
\newcommand{\lr}[1]{line~\lineref{#1}}

% ----------------------------------------------------------------------
% Launch!
% ----------------------------------------------------------------------

\begin{document}

\section{Language Theory}

Let $\Sigma$ be a finite set of letters in the alphablet.

We define $\Sigma^{*} = 2^{\Sigma}$.

A \textbf{language} is a subset of $\Sigma^{*}$ over $\Sigma$.

We define the following operations:

\begin{description}

\item[Concatenation:] 
\hfill

If $x,y \in \Sigma^{*}$, then $x:y \in \Sigma^{*}$. 

For $L, L'\in \Sigma^{*}$, we have $L\*L' = \set{x:y; x\in L, y\in L'}$.
\end{description}

We call $x\in\Sigma^{*}$ a \textbf{proper prefix} of $y\in\Sigma^{*}$ if there
exists a string $x'\in\Sigma^{*}$ such that $y = xx'$ and both $x$ and
$x'$ are not $\lambda$.

We call $x\in\Sigma^{*}$ a \textbf{proper suffix} of $y\in\Sigma^{*}$ if there
exists a string $x'\in\Sigma^{*}$ such that $y = x'x$ and both $x$ and
$x'$ are not $\lambda$.

We call $x\in\Sigma^{*}$ a \textbf{substring} of $y\in\Sigma^{*}$ if there
exist $x'\in \Sigma^{*}$ and $x''\in\Sigma^{*}$ such that
$y = x' x x''$.

\section{Regular Expressions}

Let $\Sigma$ be a finite alphabet. The \textbf{set of regular
  expressions} $R$ over $\Sigma$ is the inductively defined set of
strings $R$ such that $\emptyset, \lambda \in R$ and $\Sigma \in R$. If
$r, r' \in R$, then $r+r'$, $r\* r'$ and $r^{*}$ are in $R$.

A \textbf{generalised regular expression} allows complementation and
intersection, so that $r\cap r' \in R$, $r-r'$ and $\ol{r} \in R$.

The language denoted by a regular expression is $d(r)$, where
$\LL: R \to \set{L; L\suq \Sigma^{*}}$ is defined inductively as follows:
\begin{itemize}
\item $\LL(\emptyset) = \emptyset$
\item $\LL(\lambda)   = \set{\lambda}$
\item $\LL(a)         = \set{a}$ for each $a\in\Sigma$
\item $\LL(r+r')      = \LL(r)\cup \LL(r')$
\item $\LL(r-r')      = \LL(r)\*\LL(r')$
\item $\LL(r^{*})     = (\LL(r))^{*}$
\item $\LL(\ol{r}) = \Sigma^{*} - \LL(r)$
\item $\LL(r-r') = \LL(r)-\LL(r')$
\end{itemize} 

\begin{description}

\item[e.g.] $\LL(0^{*}(10^{*}10^{*})^{*})$ is the set of all strings over $\set{0, 1}$ with an even number of 1's.
\end{description}

\begin{claim*}
  Let $ L =\LL(0(00)^{*}(11)^{*} +(00)^{*}1(11)^{*}) = \set{0^m1^n;\text{$m+n$ is odd}}$.
\end{claim*}

\begin{proof}
  \hfill

  Let $x\in L$ be arbitrary.

  Then $x=0^m1^n$, where $m+n$ is odd.

  First, suppose that $m$ is odd. Say $m = 2k+1$. Then $n$ is even, so
  $n = 2 l$ for some $k, l \in \NN$.

  Then $x=0^{2k+1}1^{2l} = 0 (00)^k(11)^l \in \LL(0(00)^{*})\*\LL((11)^{*}) = \LL(0(00)^{*}\*(11)^{*}) = \LL(r)$.

  Suppose now that $m$ is even. The proof is similar.

  Hence $L \suq \LL(r)$.

  \textbf{Exercise:} Continue the proof.
\end{proof}

\section{Finite Automata}

A \textbf{finite automata} is a determenistic finite state automation
(abbreviated as DFA or DFSA) which has a finite set of states $Q$, a
finite alphabet $\Sigma$, an initial state $q_0\in Q$, a set of final
(also known as accepting) states and a state transition function
$\delta:Q\times \Sigma \to Q$.

An \textbf{extended transition function} $\delta^{*}: Q \times \Sigma^{*}\to Q$ is defined by $\delta^{*}(q, \lambda) =q$ for $q\in Q$ for all $a\in \Sigma$ and all $x\in \Sigma^{*}$ we have that $\delta^{*}(q, ax)= \delta^{*}(\delta(q, a), x)$.

Equivalently, for all $x\in \Sigma^{*}$ and $a\in \Sigma$ we have $\delta^{*}(q, ax) = \delta(\delta^{*}(q, a), x)$.

Let $A = (Q, \Sigma, \delta, q_0, f$, and denote
$\set{x\in\Sigma^{*};\delta^{*}(q_0, x)\in f}$ as $\LL(A)$.



\end{document}
