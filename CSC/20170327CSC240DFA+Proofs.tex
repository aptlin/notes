% -*- coding: utf-8; -*-
%%% Local Variables:
%%% mode: latex
%%% TeX-engine: xetex
%%% TeX-master: t
%%% End:
\documentclass[11pt]{scrartcl}
\usepackage[fancy, beaue, pset, anon]{masty}
\pSet{\nt{CSC240}{}{DFA Proofs}}
\usepackage{lineno}
% ----------------------------------------------------------------------
% Page setup
% ----------------------------------------------------------------------

\pagenumbering{gobble}

% ----------------------------------------------------------------------
% Custom commands
% ----------------------------------------------------------------------

% alignment

\newcommand*{\LongestHence}{$\Rightarrow$}% function name
\newcommand*{\LongestName}{$f_o(-x)+f_e(-x)$}% function name
\newcommand*{\LongestValue}{$(-a)x +(-a)(-y)$}% function value
\newcommand*{\LongestText}{\defi}%

\newlength{\LargestHenceSize}%
\newlength{\LargestNameSize}%
\newlength{\LargestValueSize}%
\newlength{\LargestTextSize}%

\settowidth{\LargestHenceSize}{\LongestHence}%
\settowidth{\LargestNameSize}{\LongestName}%
\settowidth{\LargestValueSize}{\LongestValue}%
\settowidth{\LargestTextSize}{\LongestText}%

% Choose alignment of the various elements here: [r], [l] or [c]

\newcommand*{\mbh}[1]{{\makebox[\LargestHenceSize][r]{\ensuremath{#1}}}}%
\newcommand*{\mbn}[1]{{\makebox[\LargestNameSize][r]{\ensuremath{#1}}}}%
\newcommand*{\mbv}[1]{\ensuremath{\makebox[\LargestValueSize][r]{\ensuremath{#1}}}}%
\newcommand*{\mbt}[1]{\makebox[\LargestTextSize][l]{#1}}%

\newcommand{\R}[1]{\label{#1}\linelabel{#1}}
\newcommand{\lr}[1]{line~\lineref{#1}}

% ----------------------------------------------------------------------
% Launch!
% ----------------------------------------------------------------------

\begin{document}

\section{DFA Proofs}

\begin{problem*}
  Give a DFA that accepts the language
  $\set{x\in\set{0, 1}^{*}; \text{ the second last letter of $x$ is
      1}} = \LL((0+1)^{*}1(0+1)$.
\end{problem*}

\begin{soln}
  \hfill

  For $\delta^{*}(q_{00}, \lambda) q_{00}\not\in F$,
  $\delta^{*}(q_{00}, 1) = q_{01}\not\in F$,
  $\delta^{*}(q_{00}, 0) = q_{00}\not\in F$, we can prove by induction
  that
  \begin{align}
&\forall n\in\NN.(n\geq 2 \THEN \\
&(\forall x\in \set{0, 1}^n.\\
    [(
& \delta^{*}(q_{00}, x) = q_{00}\THEN 00 \text{ is a suffix of x}\\
& \delta^{*}(q_{00}, x) = q_{01}\THEN 01 \text{ is a suffix of x}\\
& \delta^{*}(q_{00}, x) = q_{10}\THEN 10 \text{ is a suffix of x}\\
& \delta^{*}(q_{00}, x) = q_{11}\THEN 11 \text{ is a suffix of x}
]))
  \end{align}

We can start as follows.

Let $n \geq 2$.

$\forall x\in \set{0, 1}^n. \delta^{*}(q_{00}, x)\in F \IFF \text{ the second last letter of $x$ is 1}$. 


\end{soln}

\section{Nondetermenistic Finite Automata (NFA)}

An NFA is a 5-tuple $(Q, \Sigma, \delta, q_0, f)$, wher $Q$ is a finite set of states, $\Sigma$ is a finite alphabet, $q_0\in Q$ is an initial state, $F\suq Q$, and $\delta: Q\times \Sigma \to 2^{Q}$.

Then we define $\delta^{*}(q, \lambda) = \set{\lambda}$. 

For all $a\in \Sigma$, $x\in \Sigma^{*}$, $\delta^{*}(q, xa) = \bigcup \set{\delta(q', a); q'\in \delta^{*}(q, x)}$, or, equivalently, $\delta^{*}(q, ax) = \bigcup \set{\delta^{*}(q', x);q'\in \delta(q, a)}$.

A string is accepted by a DFA if the path labelled by $x$ starting from $q_0$ ends in a final stat.

$x$ is accepted by a NFA if there exists a path labelled by $x$
starting from $q_0$ that ends in a final state $L(M) = \set{x\in \Sigma^{*}; \delta^{*}(q_0, x)\cap F \neq \emptyset}$.

Every DFA $(Q, \Sigma, \delta, q_0, F$ can be easily transformed to an
NFA $(Q, \Sigma, \delta, q_0, F)$ by defining $\gamma(q, a) = \set{\delta(q, a)}$. 
\begin{ques*}

Are there some languages that can be accepted by an NFA but not by a DFA?

\end{ques*}
\section{Subset Construction}

\begin{theorem}
For every NFA $M = (Q, \Sigma, \delta, q_0, F)$ there is a DFA $\wh{M} = (\wh{Q}, \Sigma, \wh{\delta}, \wh{q_0}, \wh{F})$ such that $\LL(M) = \LL(\wh{M})$.
\end{theorem}

\begin{proof}
  \hfill

Let $M = (Q, \Sigma, \delta, q_0, F)$ be an arbitrary NFA. 

The DFA we construct keeps trakn of the possible states in which $M$ can be as it reads the input string.

Let $\wh{M} = (\wh{Q}, \Sigma, \wh{\delta}, \wh{q_0}, \wh{F})$, where $\wh{Q} = 2^{Q}$, $\wh{q_0} = \set{ q_0}$. Note that $\wh{q_0}\in \wh{Q}$.

For $Q'\in 2^Q$ and $a\in \Sigma$,
$\wh{\delta}(Q', a) = \bigcup \set{\delta(q, a);q\in Q'}$.

Let $\wh{F} = \set{Q' \in 2^Q; Q' \cap F = \emptyset}$.
\end{proof}
\begin{claim*}
$\LL(M) = \LL(\wh{M})$.
\end{claim*}
\begin{proof}
  \hfill

  For all $w\in \Sigma^{*}$, let $Q(w) = $
  \textquote{$\wh{\delta}^{*}(\set{q_0}, w) = \delta^{*}(q_0, w)$}.

  $\forall w\in ~S^{*}. P(w)$ by induction on the length of $w$ or
  structural induction on the string.
\end{proof}



\end{document}
