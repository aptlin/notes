% -*- coding: utf-8; -*-
%%% Local Variables:
%%% mode: latex
%%% TeX-engine: xetex
%%% TeX-master: t
%%% End:
\documentclass[11pt]{scrartcl}
\usepackage[fancy, beaue, pset, anon]{masty}
\pSet{\nt{CSC240}{X}{Structural Induction}}
\usepackage{lineno}
% ----------------------------------------------------------------------
% Page setup
% ----------------------------------------------------------------------

\pagenumbering{gobble}

% ----------------------------------------------------------------------
% Custom commands
% ----------------------------------------------------------------------

% alignment

\newcommand*{\LongestHence}{$\Rightarrow$}% function name
\newcommand*{\LongestName}{$f_o(-x)+f_e(-x)$}% function name
\newcommand*{\LongestValue}{$(-a)x +(-a)(-y)$}% function value
\newcommand*{\LongestText}{\defi}%

\newlength{\LargestHenceSize}%
\newlength{\LargestNameSize}%
\newlength{\LargestValueSize}%
\newlength{\LargestTextSize}%

\settowidth{\LargestHenceSize}{\LongestHence}%
\settowidth{\LargestNameSize}{\LongestName}%
\settowidth{\LargestValueSize}{\LongestValue}%
\settowidth{\LargestTextSize}{\LongestText}%

% Choose alignment of the various elements here: [r], [l] or [c]

\newcommand*{\mbh}[1]{{\makebox[\LargestHenceSize][r]{\ensuremath{#1}}}}%
\newcommand*{\mbn}[1]{{\makebox[\LargestNameSize][r]{\ensuremath{#1}}}}%
\newcommand*{\mbv}[1]{\ensuremath{\makebox[\LargestValueSize][r]{\ensuremath{#1}}}}%
\newcommand*{\mbt}[1]{\makebox[\LargestTextSize][l]{#1}}%

\newcommand{\R}[1]{\label{#1}\linelabel{#1}}
\newcommand{\lr}[1]{line~\lineref{#1}}

% ----------------------------------------------------------------------
% Launch!
% ----------------------------------------------------------------------

\begin{document}

\section{Structural Induction}

Let $p:S\to \set{T, F}$ be a recursively defined predicate.

To prove $\forall s \in S. p(s)$ by structural induction, prove:

\begin{itemize}
\item $p(s)$ for all base cases of the definition
\item $p(s)$ for the construct cases of the definition, assuming $s$ is true for the components.
\end{itemize}

We can define functions on recursively defined sets.
\begin{example}


For $f\in M$, let $n(f)=$ "the number of occurrences of propositional variables in $f$".

Then $n(P) = 1$ for any propositional variable $P$, and $n(f) = n(f') + n(f'')$ for $f = (f' \OR f'')$ and $f = (f'\AND f'')$.

Let $B$ be the set of all binary trees.
\begin{description}
\item[Base Case] \hfill

  The empty tree is in $B$.  
\item[Constructor Case] \hfill
  
  If $t_1, t_2 \in B$ and $r$ is a node, then we say $t_1 = \text{left}(t)$ and $t_2=\text{right}(t)$.
\end{description}


\end{example}

\begin{example}

  For $t\in B$, let $N(t)$ be the number of nodes in $t$

  \begin{description}
  \item[Base Case] \hfill

    $N(\text{empty tree}) =0$
  \item[Constructor Case] \hfill

    $N(t) = 1 + N(\text{left}(t)) + N(\text{right}(t))$

  \end{description}

\end{example}

\begin{example}
Let $L(t)$ be the number of leaves in $t$.
  \begin{description}
  \item[Base Case] \hfill

    Then $L(\text{empty tree}) = 0$ and $L(\text{one node tree }) = 1$.
    
  \item[Constructor Case] \hfill

    $L(t) = L(\text{left(t)} + L(\text{right}(t))$
  \end{description}
\end{example}

\begin{theorem}
  A binary three with $n$ nodes has at most $\ceiling{\nicefrac{n}{2}}$ leaves. Thus,
  \begin{equation*}
    \forall t \in B.L(t) \leq \ceiling{\nicefrac{N(t)}{2}}, 
  \end{equation*}
  or, equivalently,
  \begin{equation*}
    \forall n \in \NN. \forall t \in B.(N(t) = n \THEN L(t)\leq \ceiling{\nicefrac{n}{2}}
  \end{equation*}
\end{theorem}

\begin{example}

  For $t\in B$ and $n\in \NN$, let $S(t,n) =$ ``$t$ has $n$ nodes'' and $AL(t, n)=$ `` $t$ has at most $n$ leaves''.

  Then $\forall n\in N.\forall t\in B.(S(t, n) \THEN A(t, \ceiling{\nicefrac{n}{2}}))$.

  Let $p(n) = $ `` $\forall t\in B.(S(t, n) \THEN A(t,\ceiling{\nicefrac{n}{2}})$ ''.

  Let $n\in\NN$ be arbitrary.

  Suppose $\forall i\in\NN.(i < n \THEN p(i))$.
  Let $t\in B$   be arbitrary.

  Suppose $S(t, n)$.

  To prove $A(t, \ceiling{\nicefrac{n}{2}})$.

  \begin{description}

  \item[Case 1:]
    \hfill
    
    $n=0$.

    Then $t$ has 0 nodes and 0 leaves. Since $0 = \ceiling{\nicefrac{n}{2}}$, then $A(t, 0)$ is true.

    Thus, $A(t, 0)$ is true.
  \item[Case 2:]
    \hfill
    
    $n>0$.

    Then $t$ has a root, a left subtree $t'$ and a right subtree $t''$.

    Let $n' = N(t')$, $S(t',n')$, $n''=N(t''), S(t'', n'')$.

    Then $n=n' + n'' + 1$, so $n', n'' < n$.

    By specialisation of inductive hypothesis,
    $A(t', \ceiling{\nicefrac{n'}{2}}$, $A(t'', \ceiling{\nicefrac{n''}{2}})$ are true.

    Then
    \[
      L(t) = L(t') +L(t'') \leq \ceiling{\nicefrac{n'}{2}} + \ceiling{\nicefrac{n''}{2}} \leq \frac{n' + 1}{2} + \frac{n''+1}{2} = \frac{n+1}{2}
    \]

    If $n$ is odd, $\frac{n+1}{2} = \ceiling{\nicefrac{n}{2}}$.

    If $n$ is even, then $\frac{n+1}{2} = \frac{n}{2}+ \frac{1}{2}$ and thus $\ceiling{\nicefrac{n}{2}} = \frac{n}{2}$.

    Since $L(t) \leq \frac{n}{2} +\frac{1}{2}$ and $L(t) \in \NN$, then $L(t) \leq \frac{n}{2} =\ceiling{\nicefrac{n}{2}}$.

  \end{description}

  Then $A(t, \ceiling{\nicefrac{n}{2}})$.
  $p(n)$ by generalisation.
  $\forall n\in \NN.p(n)$ by strong induction.

  
  

\end{example}

\begin{theorem}\hfill
  
  Every integer greater than $1$ can be written as a product of primes.
\end{theorem}

\begin{proof}
  \hfill

  Suppose the claim is false.

  Let $n$ be the smallest integer that cannot be written as a product
  of primes. Then $n$ is not prime, since any prime can be written as
  a trivial product of itself.

  Therefore, $n$ is composite, so there exists $k, m \in\NN$ such
  that$k > 1$ and $m>1$ and $n = k\*m$, while $k$ and $m$ are smaller
  than $n$. Since they can be written as a product of prmies, $n$ is
  also a product of primes, which is a contradiction.
\end{proof}

\begin{definition}[Well Ordering Principle]
  Every nonempty subset of $\NN$ has the smallest element.
\end{definition}

\begin{example}

  Let $p(n) =$ "n cannot be written as aproduct of primes". Let
  $C = \set{n\in\NN; p(n) \text{ is false}}$, $C\neq \emptyset$.

  Assume $[\forall n\in \NN.p(n)]$ is false. Then by the Well-Ordering Principle there exists the smallest element in $C$.

\end{example}



\end{document}