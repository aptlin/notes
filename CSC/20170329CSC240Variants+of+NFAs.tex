% -*- coding: utf-8; -*-
%%% Local Variables:
%%% mode: latex
%%% TeX-engine: xetex
%%% TeX-master: t
%%% End:
\documentclass[11pt]{scrartcl}
\usepackage[fancy, beaue, pset, anon]{masty}
\pSet{\nt{CSC240}{}{Variants of NFAs}}
\usepackage{lineno}
% ----------------------------------------------------------------------
% Page setup
% ----------------------------------------------------------------------

\pagenumbering{gobble}

% ----------------------------------------------------------------------
% Custom commands
% ----------------------------------------------------------------------

% alignment

\newcommand*{\LongestHence}{$\Rightarrow$}% function name
\newcommand*{\LongestName}{$f_o(-x)+f_e(-x)$}% function name
\newcommand*{\LongestValue}{$(-a)x +(-a)(-y)$}% function value
\newcommand*{\LongestText}{\defi}%

\newlength{\LargestHenceSize}%
\newlength{\LargestNameSize}%
\newlength{\LargestValueSize}%
\newlength{\LargestTextSize}%

\settowidth{\LargestHenceSize}{\LongestHence}%
\settowidth{\LargestNameSize}{\LongestName}%
\settowidth{\LargestValueSize}{\LongestValue}%
\settowidth{\LargestTextSize}{\LongestText}%

% Choose alignment of the various elements here: [r], [l] or [c]

\newcommand*{\mbh}[1]{{\makebox[\LargestHenceSize][r]{\ensuremath{#1}}}}%
\newcommand*{\mbn}[1]{{\makebox[\LargestNameSize][r]{\ensuremath{#1}}}}%
\newcommand*{\mbv}[1]{\ensuremath{\makebox[\LargestValueSize][r]{\ensuremath{#1}}}}%
\newcommand*{\mbt}[1]{\makebox[\LargestTextSize][l]{#1}}%

\newcommand{\R}[1]{\label{#1}\linelabel{#1}}
\newcommand{\lr}[1]{line~\lineref{#1}}

% ----------------------------------------------------------------------
% Launch!
% ----------------------------------------------------------------------

\begin{document}

\section{Variants of NFAs}

\subsection{Multiple Start States}

Let $M = (Q, \Sigma, \delta, I, F)$, where $I\suq Q$ and $ I \neq \emptyset$.

\begin{theorem}
If $L = \LL(M)$, then there exists an NFA $N$ such that $\LL(N) = L$.
\end{theorem}

\begin{proof}
  \hfill

Suppose $M = (Q, \Sigma, \delta, I, F)$.

We construct $M' = (Q\cup \set{q_0}, \Sigma, \delta', q_0, F')$, where
$q_0\not \in Q$, and for each $a\in \Sigma$ we have $\delta'(q_0, a) = \bigcup \set{\delta(q, a); q\in I}$ and $\delta'(q, a) = \delta(q, a)$, and
\begin{equation*}
  F' = \begin{cases}
    F, \text{ if } I\cap F = \emptyset\\
    F\cup \set{q_0}, \text{ if } I\cap F \neq \emptyset.
  \end{cases}
\end{equation*}

We prove then that $\LL(M) = \LL(M')$.
\end{proof}

\subsection{NFA with $\lambda$-Transitions}

Let $M = (Q, \Sigma, \delta, q_0, F)$ and
$\delta: Q \times (\Sigma\cup {\lambda} \to 2^Q)$.

Let $\LL(M) = \set{x\in \Sigma^{*}; \text{ there is a path from $q_0$ to a final state such that $x$ is a concatenation of the labels on that path}}$.

If $L = \LL(M)$, then there exists an NFA $N$ such that $L = L(N)$.

For exery state $q\in Q$ let $E(q) = \set{q'; \text{ there is a path from $q$ to $q'$ labelled by $\lambda$}}$.

In particular, $q\in E(q)$.

Let $M' = (Q, \Sigma, \delta', E(q_0), F)$ and
$\delta'(q, a) = \bigcup \set{E(q'); q' \in \delta(q, a)}$.

\section{Closure Results}

Suppose that $L_1, L_2 \in \Sigma^{*}$ are accepted by finite state automata. 

Then $L_1\cup L_2$, $L_1\*L_2$, $\Sigma^{*}-L_1$, $L_1^{*}$, $L_1^{+}$, $L_1\cap L_2$ can also be accepted by finite automata.

We prove now that this statement holds for any languages $L_1$ and $L_2$.

Let $M_1=(Q, \Sigma, \delta, q_1, F_1)$ and $M_2=(Q, \Sigma, \delta, q_2, F_2)$, with $\LL(M_1) = L_1$ and $\LL(M_2) = L_2$.

Assume $Q_1\cap Q_2 = \emptyset$.

Let $M' = (Q,\Sigma, \delta_1, q, Q_1-F_1)$ (an NFA with interchanged accept and reject states). 

Then $\LL(M') = \Sigma^{*} - L_1$.

\begin{proof}
  \hfill

  Let $x\in \Sigma^{*}$. Then $x\in \LL(M')$ if and only if
  $\delta_1^{*}(q_1, x) \in Q - F_1$, which holds if and only
  if $\delta_1^{*}(q, x)\not \in F_1$, which is equivalent to
  $x\not\in \LL(M)= L_1$, and thus $x\in \Sigma^{*} - L_1$.

  If $M_1$ is an NFA, then it is possible that
  $\delta_1^{*}(q_1, x)\cap F \neq \emptyset$ and
  $\delta_1^{*}(q_1, x) \cap (Q_1-F)\neq \emptyset$.

  \begin{description}

  \item[Union] 
\hfill

Let $M' = (Q_1\cup Q_2, \Sigma, \delta, \set{q_1, q_2}, F_1\cup F_2)$
and $\delta: (Q_1\cup Q_2)\times \Sigma \to Q\cup Q_2$.

Then $\delta(q, a) = 
\begin{cases}
  \delta_1(q, a) \text{ if $q\in Q_1$ }\\
  \delta_2(q, a) \text{ if $q\in Q_2$ }
\end{cases}
$

Then $\LL(M') = \LL(M_1)\cup \LL(M_2)$.
\item[Intersection] 
\hfill

Let $M' = (Q_1\times Q_2, \Sigma, \delta, (q_1, q_2), F_1\times F_2)$,
thus creating an NFA consisting of two subNFA running in parallel.

Then $\delta((q_1, q_2), a) = (\delta_1(q_1, a), \delta(q_2, a))$ for
all $a\in \Sigma$ and $q_1\in Q_1$ and $q_2\in Q_2$.

\item[Concatenation] 

  Connect two NFAs by $\lambda$-transitions from the accepting
  states in one connected to the initial state in another.
\item[$L_1^{+}$]

Connect the accepting states to the initial states with $\lambda$-transitions.
\item[$L_1^*$] 

If $\lambda \in L_1$, then $L_1^{*} = L_1^+$ if $\lambda \not \in L_1$.

Connect the accepting states by $\lambda$-transitions to the old
initial state, adding a new accepting initial state connected to the
old initial state by the $\lambda$-transition.
  \end{description}
\end{proof}
\begin{corollary}
Every regular language can be accepted by a finite state automaton.
\end{corollary}

\begin{proof}
  \hfill

The proof proceeds by structural induction.
\end{proof}

\end{document}
