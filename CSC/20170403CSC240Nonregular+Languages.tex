% -*- coding: utf-8; -*-
%%% Local Variables:
%%% mode: latex
%%% TeX-engine: xetex
%%% TeX-master: t
%%% End:
\documentclass[11pt]{scrartcl}
\usepackage[fancy, beaue, pset, anon]{masty}
\pSet{\nt{CSC240}{}{Nonregular Languages}}
\usepackage{lineno}
% ----------------------------------------------------------------------
% Page setup
% ----------------------------------------------------------------------

\pagenumbering{gobble}

% ----------------------------------------------------------------------
% Custom commands
% ----------------------------------------------------------------------

% alignment

\newcommand*{\LongestHence}{$\Rightarrow$}% function name
\newcommand*{\LongestName}{$f_o(-x)+f_e(-x)$}% function name
\newcommand*{\LongestValue}{$(-a)x +(-a)(-y)$}% function value
\newcommand*{\LongestText}{\defi}%

\newlength{\LargestHenceSize}%
\newlength{\LargestNameSize}%
\newlength{\LargestValueSize}%
\newlength{\LargestTextSize}%

\settowidth{\LargestHenceSize}{\LongestHence}%
\settowidth{\LargestNameSize}{\LongestName}%
\settowidth{\LargestValueSize}{\LongestValue}%
\settowidth{\LargestTextSize}{\LongestText}%

% Choose alignment of the various elements here: [r], [l] or [c]

\newcommand*{\mbh}[1]{{\makebox[\LargestHenceSize][r]{\ensuremath{#1}}}}%
\newcommand*{\mbn}[1]{{\makebox[\LargestNameSize][r]{\ensuremath{#1}}}}%
\newcommand*{\mbv}[1]{\ensuremath{\makebox[\LargestValueSize][r]{\ensuremath{#1}}}}%
\newcommand*{\mbt}[1]{\makebox[\LargestTextSize][l]{#1}}%

\newcommand{\R}[1]{\label{#1}\linelabel{#1}}
\newcommand{\lr}[1]{line~\lineref{#1}}

% ----------------------------------------------------------------------
% Launch!
% ----------------------------------------------------------------------

\begin{document}

\section{Nonregular Languages}

Consider $L = \set{a^ib^i; a \geq 1}$.

Suppose that $L$ is regular. 

Therefore, there exists a DFA $(Q, \set{a, b}, \delta,q_0, F)$ which
accepts $L$.

Let $n = \abs{Q}$.

Consider the states $q_{i} = \delta^{*}(q_0, a^i)$ for $i = 0, \dots, n$.

By the pigeonhole principle, there exist $0 \leq i < j \leq n$ such
that $q_i = q_j$.

Then
$\delta^{*}(q_0, a^ib^j) = \delta^{*}(\delta^{*}(q_0, a^i), b^j)$, and
since $q_i = q_j$, then
$\delta^{*}(q_0, a^ib^j) = \delta^{*}(\delta^{*}(q_0, a^j), b^j) =
\delta^{*}(q_0, a^jb^j) \in F$, since $a^jb^j \in L$, which means that
$a^ib^j\in L$. But $a^ib^j\not \in L$, since $i\neq j$, which is a
contradiction. Thus, $L$ is not regular.

Now we prove that $A = \set{a^mb^r; m\neq r}$ is not regular.

Suppose $A$ is regular.

Then $(\set{a, b}^{*} - A)\cap \LL(aa^{*}bb^{*})$ is also regular,
because $\LL(aa^{*}bb^{*})$ is regular and $\set{a, b}^{*}-A$ is
regular by closure under complementation. 

But
$(\set{a, b}^{*} - A)\cap \LL(aa^{*}bb^{*}) = \set{a^ib^i; i \geq 1}$,
which is not regular, and thus we obtain a contradiction.

Consider now
$B = \set{a^ib^jc^h; i, j, h \geq 0 \AND i=1 \THEN j = h}$.

Suppose that $B$ is regular.

Then $B\cap \LL(abb^{*}cc^{*}) = \set{ab^jc^h; j =h \geq 1} = C$ is
regular.

Consider the homomorphism $T: \set{a, b, c} \to \set{a, b}^{*}$ such
that $T(a) = \epsilon$, $T(b) = a$ and $T(c) = b$.

Then $T(C) = \set{a^ib^i; i\geq 1}$ is regular since regular languages
are close under homomorphism. But $T(C) = L$ is not regular, which is
a contradiction. Thus, $B$ is not regular.

\subsection{Pumping Lemma}

\begin{lemma}
  For every regular language $L\suq \Sigma^{*}$ there exists
  $n\in\ZZ^+$ such that, for all $x\in L$, if $\abs{x}\geq n$, then
  there exist $u, v$ and $w$ in $\Sigma^{*}$ such that
  $v\neq \epsilon$, $\abs{uv}\leq n$, $x = uvw$ and for all $k\in \NN$
  we have $uv^kw\in L$.
\end{lemma}
\begin{proof}
  \hfill

  Since $L$ is regular, it is accepted by some DFA
  $M = (Q, \epsilon, \delta, q_0, F)$.

  Let $n = \abs{Q}$, and suppose that $x = x_1\cdots x_m\in L$, where
  $m \geq n$, is arbitrary.

  Now we construct the states $q_i = \delta^{*}(q_0, x_1\cdots x_i)$
  for $i\in [1, n]\cap \NN$.

  By the pigeonhole principle, there exist $0 \leq i < j \leq n$ such
  that $q_i = q_j$.

  Let $u= x_1\cdots x_i$, $v = x_{i+1} \cdots x_j$ and
  $w = x_{j+1}\cdots x_m$.

  Note that $\delta^{*}(q_0, uv^k) = \delta^{*}(q_0, uv)$ for all
  $k \geq 0$.

  So
  $\delta^{*}(q_0, uv^kw) = \delta^{*}(q_j, w) = \delta^{*}(q_0, uvw)
  = \delta^{*}(q_0, x)\in F$.

  Hence $uv^kw \in L = \LL(M)$.
\end{proof}

\begin{example}

  Let $P = \set{z\in \set{0,1}^{*}; z = z^R}$ be a language of palindromes.

  We prove that $P$ is not regular.

  Suppose $P$ is regular. 

  Then by the pumping lemma, there exists $n\in \ZZ^+$ such that for
  all $x\in P$, $\set{x} \geq n$ implies that there exist
  $u\in \set{0, 1}^{*}$, $v \in \set{0, 1}^{*}$ and
  $w\in \set{0, 1}^{*}$ such that $v\neq \epsilon$ and
  $\abs{uv}\leq n$, $x = uvw$ and for all $k\in \NN$ we have
  $uv^kw\in P$.

  Consider $x = 0^n 1 0^n\in P$.

  There exist $u, v, w \in \set{0, 1}^{*}$ such that $v\neq \epsilon$
  and $\abs{uv}\leq n$, $x = uvw$ and for all $k\in \NN$ we have
  $uv^kw\in P$.

  So $u = 0^i$, $v = 0^j$ for some $i\geq 0$ and $j\geq 0$ such that
  $i+j \leq n$. Thus, $w = 0^{n-i-j}10^n$.

  Consider $k=2$.

  Then $uv^2w = 0^{i+2j+n-i-j}1 0^{n} = 0^{n+j}10^n\not\in P$, which is a contradiction.
\end{example}

\begin{example}

  We prove that $L= \set{a^mb^r; m\neq r}$ is not regular.

  Suppose that $L$ is regular.

  Then it satisfies the pumping lemma.

  Let $n\in \ZZ^{+}$ be the constant rom the pumping Lemma.

  Consider $x = a^{n!}b^{(n+1)!}\in L$.

  Then there exist $u, v, w\in \set{a, b}^{*}$ that satisfy the
  conditions of the pumping lemma.

  Let $u= a^i$, $v = a^j$ for some $i \geq 0, j > 0$ such that
  $i+j \leq n$.

  Let $k = 1+\frac{n\*n!}{j}$.

  Then $uv^kw = a^mb^{(n+1)!}$, where $m = n!+(k-1)j = (n+1)!$.

  Hence,$uv^kw\not \in L$.

\end{example}

\end{document}
