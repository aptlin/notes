% -*- coding: utf-8; -*-
%%% Local Variables:
%%% mode: latex
%%% TeX-engine: xetex
%%% TeX-master: t
%%% End:
\documentclass[11pt]{scrartcl}
\usepackage[fancy, beaue, pset, anon]{masty}
\pSet{\nt{BIO130}{}{Membrane Proteins}}
\usepackage{lineno}
% ----------------------------------------------------------------------
% Page setup
% ----------------------------------------------------------------------

\pagenumbering{gobble}

% ----------------------------------------------------------------------
% Custom commands
% ----------------------------------------------------------------------

% alignment

\newcommand*{\LongestHence}{$\Rightarrow$}% function name
\newcommand*{\LongestName}{$f_o(-x)+f_e(-x)$}% function name
\newcommand*{\LongestValue}{$(-a)x +(-a)(-y)$}% function value
\newcommand*{\LongestText}{\defi}%

\newlength{\LargestHenceSize}%
\newlength{\LargestNameSize}%
\newlength{\LargestValueSize}%
\newlength{\LargestTextSize}%

\settowidth{\LargestHenceSize}{\LongestHence}%
\settowidth{\LargestNameSize}{\LongestName}%
\settowidth{\LargestValueSize}{\LongestValue}%
\settowidth{\LargestTextSize}{\LongestText}%

% Choose alignment of the various elements here: [r], [l] or [c]

\newcommand*{\mbh}[1]{{\makebox[\LargestHenceSize][r]{\ensuremath{#1}}}}%
\newcommand*{\mbn}[1]{{\makebox[\LargestNameSize][r]{\ensuremath{#1}}}}%
\newcommand*{\mbv}[1]{\ensuremath{\makebox[\LargestValueSize][r]{\ensuremath{#1}}}}%
\newcommand*{\mbt}[1]{\makebox[\LargestTextSize][l]{#1}}%

\newcommand{\R}[1]{\label{#1}\linelabel{#1}}
\newcommand{\lr}[1]{line~\lineref{#1}}

% ----------------------------------------------------------------------
% Launch!
% ----------------------------------------------------------------------

\begin{document}

\section{Membrane Proteins}


\begin{itemize}
\item Have specific functions
\item Associated with the lipid bilayer in different ways
\end{itemize}

\textbf{Integral membrane proteins} are located at the region where a protein penetrates or spans the lipid bilayer. If the lipid bilayer is spanned, a protein is called a transmembrane protein.

\textbf{Peripheral membrane proteins} associate with the phospholipid
bilayer and with the penetrating proteins noncovalently.

\textbf{Anchored proteins} can be anchored with a sugar or a lipid.

\subsection{Integral Membrane Proteins}

Integral membrane proteins are amphipathic, with AA side chains being
polar if the hydrophilic domains are aqueous. AA side chains are
non-polar if the membrane-spanning domain is hydrophobic.

Single-pass transmembrane proteins have the shape of a single
$\alpha$-helix, while multipass transmembrane proteins have multiple
$\alpha$-helices.

Other structures can span the membrane as well.

A membrane spanning $\alpha$-helix has 20 to 30 hydrophobic
aminoacids. Not all $\alpha$-helices, however, span the membrane.

\begin{description}

\item[e.g.] receptors (with hydrophilic tails participating in
  intracellurar signalling), ion channels (conformational changes
  regulate permeability), $\beta$-barrel (a rigid rolled beta-sheet,
  wich play a role of some channels in bacteria, mitochondria,
  chloroplasts).

\end{description}

\subsection{Techniques of Identification}

\begin{itemize}
\item X-ray crystallography
  
 Precise identification of the 3D structure

\item Hydrophobicity plots
  
  Segments of 20-30 hydrophobic amino acids spanning the lipid bilayer
  as an $\alpha$-helix show as high peakson the graph of
  hydrophobicity agians the amino acid residue number.

\end{itemize}

\subsection{Cell Membrane Proteins on One Sicde}

Proteins can be anchored on the ctosolic face by an amphipathic $\alpha$-helix.

\subsection{Lipid-Anchored Membrane Proteins}

GPI anchored proteins are synthesised in ER lumen and usually end up
on the cell surface. Proteins with other lipid anchors, which are
added by cytosolic enzymes, are directed to tho cytosolic face.

% \subsection{Peripheral Membrane Proteins}
% Peripheral membrane proteins are bound to other proteins or lipids

\subsection{Extraction of Membrane Proteins}

Peripheral membrane proteins use gentle extraction that does not destroy a lipid bilayer.

Integral membrane proteins destroy the membrane with detergents to extract the protein.
% \subsection{Studying the Properties of }

\subsection{Lateral Diffusion of Membrane Proteins}

\begin{itemize}
\item there is lateral diffusion within a leaflet
\item no flip-flop
\end{itemize}

The protein movement can be studies with GFP (green fluorescent proteins).

\subsection{FRAP (Fluorescence Recovery After Photobleaching)}

\begin{enumerate}
\item\label{item:1} First, proteins are labelled with a dye.
\item Then a spot is photobleached with a laser beam.
\item The bilayer is then recovered and the recovery time is then
  measured. If the recovery is faster, then the transmembrane proteins
  have lots of translative movement.
\end{enumerate}

Movement via simple diffusion througth the lipid bilayer is such that down the concentration gradient the concentration drops

\subsection{Impermeability of the Lipid Bilayer}

These require membrane proteins for transportation.

\subsection{Proteins Involved in Membrane Transport}

\begin{itemize}
\item Multipass transmembrane proteins
\begin{itemize}
\item create a protein-lined path across cell membrane
\item transport polar and charged molecules
\end{itemize}
\item Each transport protein is selective
\item Different cell membranes have a different complement of transport proteins
\end{itemize}

\subsection{Passive and Active Transport}

\textbf{Passive transport} is down the concentration gradient and hence does not require energy.

\textbf{Active transport} is the opposite.

Notice that there are two type of proteins: channel proteins (only passive transport) and transporter proteins (some do passive, some do active).

Channel proteins do not interact a lot with and do not bind strongly with the transported molecule. The conformation of channel proteins is also not changed a lot.

Transporter proteins are also called carrier proteins,

Both channel and transporter proteins are called transport proteins.

\subsection{Resting Membrane Potential}

A molecule with a positive charge has the greates motive force across
the membrane, even though the concentration difference is the same for
positive, negative and neutral molecules with the corresponding channel proteins.

\subsection{Concentration Gradient and Membrane Potential}

Electrochemical gradient is determined by the additive effect of the
concentration gradient and the membrane potential (electrical
gradient).

\section{Channel Proteins}

\begin{itemize}
\item hydrophilic pore across a membrane
\item most channel proteins are selective
\item passive transport

  \begin{itemize}
\item weak interactions with a solute
\item faster transport by channels than transporters, with several
  molecules passing when open
\end{itemize}
\end{itemize}

\subsection{Ion Channels}

There are two types of ion channels:

\begin{description}

\item[Non-gated] Always open

  \begin{description}

  \item[e.g.] K$^+$ leak channels, which are improtant in the plasma membrane of animal cells. They are also involved in determining the resting membrane potential.

  \end{description}
  
\item[Gated] Only open on certain circuimstances

  
\begin{itemize}
\item Voltage-gated ion channel  (change in the voltage across a membrane triggers the gate)
\item Mechanically-gated ion channel (kept closed by a mechanical stress, opens if the plasma membrane is stretched)
\item Ligand-gated (Extracellular ligand) (for example, neurotransmitters can trigger the gate)
\item Ligand-gated (Intracellular ligand) (ions, nucleotides can open a gate)
\end{itemize}
\end{description}

\section{Transporter Proteins}

\begin{itemize}
\item bind a specific solute
\item go through a conformational change that transports that solute
  across the membrane
\end{itemize}

One type of transporter proteins are uniporters:
\begin{itemize}
\item one molecule (with a passive transport down electrochemical gradient)
\item direction of transport is reversible
\end{itemize}

\begin{description}

\item[e.g.] GLUT uniporters (move glucose from a high concentration to low concentration down the electrochemical gradient, and can work in either directions, in and out of the cell)

\end{description}

Uniporters provide a passive transport with a transporter by facilitated diffusion.

\subsection{Active Transport}

\begin{itemize}
\item Active transport is against electrochemical gradient, which requires energy, but not necessarily in the form of ATP
\begin{itemize}
\item Contransporters
  One molecule down the gradient, the second molecule against the gradient
\item ATP-driven pumps
  Involved ATP hydrolysis, moving molecules against the gradient
\item Light-driven pumps (bacteria)
  Use light energy, moving molecules against the gradient
\end{itemize}
\end{itemize}

\subsection{Symporters and Antiporters}

\textbf{Symporters} move two molecules in the same direction.

\textbf{Antiporters} move two molecules in opposite directions.

Free energy from co-transported ion moving down the electrochemical
gradient is used to transport the 2nd molecule against its
electrochemical gradient or to drive a secondary active transport.

\begin{example}

   moves Na$^+$ down the electrochemical
  gradient, which provides the energy to move glucose agains the
  concentration gradient. Cooperative binding of Na$^{+}$ and glucose
  leads to a conformational change in the protein.

\end{example}

\begin{example}

  Most proteins require a spaciic pH (cytosol -- neutral pH, lysosomes
  -- acidic). Excess H$^{+}$ produced by acid forming reactions leaks
  into the cell, and then Na$^{+}$ direvn antiporters maintain
  cytosolic pH.

\end{example}

\begin{example}

  Na$^{+}$/H$^+$ exchanger (antiporter) uses the free energy stored in
  the Na$^{+}$ electrochemical gradient to move H$^{+}$ out of the
  cell. It responds to cytosolic pH so that if pH drops, the
  transporter activity increases.

\end{example}

Na$^{+}$/glucose symporter and Na$^{+}$/H$^{+}$ exchanger use the
energy stored in the Na$^{+}$ electrochemical gradient to move other
molecules against their electrochemical gradients. Continued action
would equalise the Na$^{+}$ gradient, if there is no external supply
of Na$^{+}$. The electrochemical gradient is hence maintained by the
Na$^{+}$/K$^{+}$ pump (a transport ATPase).

\subsection{ATP-driven pumps}

\begin{itemize}
\item P-type transport ATPases are phosphorylated
  
  \begin{itemize}
  \item Na$^{+}$ and K$^{+}$ are moved against electrochemical
    gradients
  \item Na$^{+}$ gradient is used to transport of nutrients (glucose)
    into cells and maintain the pH and cell volume.
  \end{itemize}
\end{itemize}

\subsubsection{The Pumping Cycle of the Na$^+$/K$^+$ Pump}

\begin{enumerate}
\item\label{item:2} ATP is bound to the pump, 3 Na$^{+}$ bind an open
  cytosolic pocket
\item The pocket closes preventing Na$^{+}$ escape
\item ATP hydrolysis occurs, the pump is phosphorylated, the release
  of ADP causes a conformational change to E2.
\item In E2 the binding pocket is exposed on the extracellular side,
  3Na$^{+}$ exit.
\item 2 K$^{+}$ bind
\item The pocket closes preventing K$^{+}$ release.
\item The pump is deposhporylated.
\item ATP binds the pump returning it to the E1 state.
\end{enumerate}
\end{document}
