%%% Local Variables:
%%% mode: latex
%%% TeX-engine: xetex
%%% TeX-master: t
%%% End:
\documentclass[11pt]{scrartcl}
\usepackage[fancy, beaue, pset, anon]{masty}
\pSet{\nt{BIO130}{}{Protein Sorting and Modification}}
\usepackage{lineno}
% ----------------------------------------------------------------------
% Page setup
% ----------------------------------------------------------------------

\pagenumbering{gobble}

% ----------------------------------------------------------------------
% Custom commands
% ----------------------------------------------------------------------

% alignment

\newcommand*{\LongestHence}{$\Rightarrow$}% function name
\newcommand*{\LongestName}{$f_o(-x)+f_e(-x)$}% function name
\newcommand*{\LongestValue}{$(-a)x +(-a)(-y)$}% function value
\newcommand*{\LongestText}{\defi}%

\newlength{\LargestHenceSize}%
\newlength{\LargestNameSize}%
\newlength{\LargestValueSize}%
\newlength{\LargestTextSize}%

\settowidth{\LargestHenceSize}{\LongestHence}%
\settowidth{\LargestNameSize}{\LongestName}%
\settowidth{\LargestValueSize}{\LongestValue}%
\settowidth{\LargestTextSize}{\LongestText}%

% Choose alignment of the various elements here: [r], [l] or [c]

\newcommand*{\mbh}[1]{{\makebox[\LargestHenceSize][r]{\ensuremath{#1}}}}%
\newcommand*{\mbn}[1]{{\makebox[\LargestNameSize][r]{\ensuremath{#1}}}}%
\newcommand*{\mbv}[1]{\ensuremath{\makebox[\LargestValueSize][r]{\ensuremath{#1}}}}%
\newcommand*{\mbt}[1]{\makebox[\LargestTextSize][l]{#1}}%

\newcommand{\R}[1]{\label{#1}\linelabel{#1}}
\newcommand{\lr}[1]{line~\lineref{#1}}

% ----------------------------------------------------------------------
% Launch!
% ----------------------------------------------------------------------

\begin{document}

\section{Protein Sorting and Modification}

Path of a transmembrane protein from translation to the plasma membrane (PM) proceeds as follows:
\begin{itemize}
\item cytosol $\to$ER
  \begin{itemize}
  \item translocation into the ER membrane (cotranslation translocation)
  \item requires a signal sequence
  \end{itemize}
\item ER $\to$ Golgi
  Glycolysation lipids and proteins go in the Golgi.
\item Golgi $to$ PM
\end{itemize}
\subsection{Integral Membrane Protein}

Synthesis of an integral membrane protein still involves SRP and SRP receptors.

Hydrophobic transmembrane segment makes up the internal signal
sequence.

Orientation of the protein depends on the distribution of charges on the molecule. The more negatively charged side of the transmembrane segment is directed towards the ER lumen. 

Translocator protein assumes a variety of configurations, and therefore allows for different ways of insertion.

\subsection{Golgi Apparatus}

The size of a Golgi apparatus

Vesicles transpot proteins/lipits from the ER. The side which is close
to the endoplasmic reticulum is called a \textit{cis side}.

Each protein is inserted into the membrane in the ER in a specific
manner. This protein assymetry is maintained through the vesical
point even beyond the insertion by a translocon protein.

\subsection{Lysosomes and Endosomes}

% On endocytosis, something can get into the endosome.

% Eventually, when the majority of what is endocytosed, they

Early endocytosed material is found in the early endosome. Lysosomal proteins delivered in vesicles from the Golgi are usually degested and eventully become called as late endosome

\subsection{Lysosomes are the main site of intracellular digestion}

Lysosomes, acidified by a proton pump, contain approx 40 types of hydrolytic enzimes. Note that ATP and ADP contentrations matter in determination whether the protein is of $f$- or $v$-type ointo th

Lysosomes are the main site of intracellular difeston. Low pH requirement of enzymes is necessary to protect the contents of cytosol itself.

\subsection{Formation of Lysosomes}

Lysosomal proteins from the ER and Golgi are incorporated into endosomes at different stages.

Plant vacuoles, occupying 30-90\% of the cell volume,  are typically acidic.

A large increase in the volume of a plant cell can happen without increasing the total volume.

The turgor pressure is an internal pressure of the cell that pushes it against  the cell walls.

\subsection{Peroxidsomes}

This agent uses molecular oxygen to oxidise chain reactions.

\section{Protein Sorting}

There are two types of sorting: post-translation and co-translational. When in mitochondria or plastids, post-translational protein

How does a nuclear protein get into the nucleus? Transport is gated, and the process proceeds via the nuclear pore complexes.

\subsection{Gated Transport}

Nuclear pore complexes provide selective transport of macromolecules,
as well as free diffusion of small molecules (< 50000 daltons).

% \subsection{Nuclear Import Signal Sequence}

% \item A protein with an intact nuclear import signal is found in the nucleus


\subsection{Estrogen Receptors}

The Estrogen Receptor is a ligand-modulated regulator of
transcription. When estradol is not present, then estrogen receptor is
in the cytosol.  When estradiol enters the cell, it binds to the
estrogen receptor. Ligand-bound ER moves into the nucleus thorugh
NPCs. Estradiol then binds to nehancer sites in the genome and
activates transcription of target genes.

\section{Vesicles}

\subsection{Cytoskeleton}

\begin{itemize}
\item Provides structural support
\item Postions organelles
\item Directs vesicular transport
\item Involved in locomotion
\item Required for cell divison
\end{itemize}

There are three types of filaments:

\begin{itemize}
\item microfilaments

  Actin, diam = 5-9 nm
\item intermediate filaments

  intermediate filament proteints, diam = 10 nm

  
\item microtubules (13 of them form a hollow cylinder)

  tubulin, diam =  25 nm
\end{itemize}
\subsubsection{Immunofluorescence}

A techinque used to determine the location of proteins with in the
cell. Cells are fixed, and a primary antibody is used binding
specifically to the protein of interest. A secondary antibody binds to
the first antibody and is covalently tagged with a fluorescent
molecule. A fluorescence microscope is used to excite the fluorescent
molecule and visualise the light emitted.

\subsubsection{Limits of Light Microscopy}

The light microscope has a resolution limit due to diffraction. With an electron microscope, where the wavelength of the beam is much shorter, the resolution is better by a factor of about 250.

\subsection{Dynamics}

For cell motility/crawling, the actin filaments must rapidly
disassemble and reassemble at the leading edge.

Most interphase microtubules radiate from one microtubule organising
centre, and they are recognised to form the bipolar mitotic spindles
in dividing cells.

\subsection{Microtubules}

Microtubules are involved in intracellular transport, sturctural
support, cell organisation, mitosis processes and cell motility
(flagella and cillia). Microtubules are made of tubulin, long hollow
tubes, which are stiff and inextensible.

The structure of microtubules is made of individual subunits of
$\alpha$-tubuling and $\beta$-tubulin. Two closely related globular
proteins form tubulin heterodimers. This regular arrangement of
$\alpha$ and $\beta$ subunits give the microtubule polarity. It has a
plus end $\beta$ and a minus end $\alpha$.

\subsection{Microtubule Protofilaments}

13 parallel protofilamets make up the hollow tubule. All the bonds
between the individual subunits are non-covalent. The bonds between
protofilaments are weaker than the bonds within each
protofilament. Growth and disassembly of microtubules can occur at the
ends.

After the heterodimers have been in this structure for a while, GTP is cut to GDP.

If there are GDP-bound heterodimers at the end, they will disassemble.

In vitro, microtubule growth is faster at the plus end.

\subsection{Tubulin Dimers}

Free dimers are bound to GT. Tubulin subunits are enzymes that hydrolyse GTP. When this occurs in the filament, GDP is trapped in the tubulin subunits.

The microtubule GTP cap stabilises the plus end, which is the faster growing end. The GTP cap stabilises the plus end and favours tubule growth. Dimers in the T form bind more strongly to other dimers in the tubule. Hydrolysis of bound GTP reduces the binding affinity of the subunit.

% Eventualy the $T$-form end wil turn the .
\subsection{Microtubules}
In cells microtubules are nucleated at the MTOC (microtubule
organising gentre). The centrosome is a MTOC which nucluates the
formation of microtubules, with the minus end stabilised and the plus
end dynamic.

\subsection{$\gamma$-Tubulin ath the MTOC}

$\gamma$-TuRCs is a complex of proteins forming a ring
structure. $\gamma$-Tubulin binds the ring structure and acts as an
attachment site for $\lambda/\beta$

Nerve cells in the spinal cord can exted to the finger tips. These
neurons can be more than a meter long.

Kinesin and dyening are motor proteins that walk along microtubules.

Dyenin movement is towards the cell body, while kinesin movement is towards the axon terminus.

Organelles are assocated with microtubules, which are walked to specific location.

\subsection{Movement of vesicles through the endomembrane system}

The Golgi apparatus is also a MTOC but it's different from the centrosome.


\end{document}
