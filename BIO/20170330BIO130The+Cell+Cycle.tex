% -*- coding: utf-8; -*-
%%% Local Variables:
%%% mode: latex
%%% TeX-engine: xetex
%%% TeX-master: t
%%% End:
\documentclass[11pt]{scrartcl}
\usepackage[fancy, beaue, pset, anon]{masty}
\pSet{\nt{BIO130}{}{The Cell Cycle}}
\usepackage{lineno}
% ----------------------------------------------------------------------
% Page setup
% ----------------------------------------------------------------------

\pagenumbering{gobble}

% ----------------------------------------------------------------------
% Custom commands
% ----------------------------------------------------------------------

% alignment

\newcommand*{\LongestHence}{$\Rightarrow$}% function name
\newcommand*{\LongestName}{$f_o(-x)+f_e(-x)$}% function name
\newcommand*{\LongestValue}{$(-a)x +(-a)(-y)$}% function value
\newcommand*{\LongestText}{\defi}%

\newlength{\LargestHenceSize}%
\newlength{\LargestNameSize}%
\newlength{\LargestValueSize}%
\newlength{\LargestTextSize}%

\settowidth{\LargestHenceSize}{\LongestHence}%
\settowidth{\LargestNameSize}{\LongestName}%
\settowidth{\LargestValueSize}{\LongestValue}%
\settowidth{\LargestTextSize}{\LongestText}%

% Choose alignment of the various elements here: [r], [l] or [c]

\newcommand*{\mbh}[1]{{\makebox[\LargestHenceSize][r]{\ensuremath{#1}}}}%
\newcommand*{\mbn}[1]{{\makebox[\LargestNameSize][r]{\ensuremath{#1}}}}%
\newcommand*{\mbv}[1]{\ensuremath{\makebox[\LargestValueSize][r]{\ensuremath{#1}}}}%
\newcommand*{\mbt}[1]{\makebox[\LargestTextSize][l]{#1}}%

\newcommand{\R}[1]{\label{#1}\linelabel{#1}}
\newcommand{\lr}[1]{line~\lineref{#1}}

% ----------------------------------------------------------------------
% Launch!
% ----------------------------------------------------------------------

\begin{document}

\section{The Cell Cycle}

\subsection{Tubulin Dimers}

Free dimers are bound to GTP (the ``T'' form).

Tubulin subunits are enzymes that hydrolyse GTP.

When the hydrolysis occurs in the filament GDP is trapped in the tubulin subunits (the ``D'' form)

\subsection{Cell Division}

Not all cells in the culture will be dividing at the same time.

In fact, very few cells can be observed in the process of cell
division at any given time.

\subsection{Cell Cycle}

\begin{description}

\item[M phase] 
The nucleus and cytoplasm divide:
\begin{itemize}
\item Mitosis (nuclear division)
\item Cytokynesis (cytoplasmic division)
\end{itemize}
\item[Interphase] 
The period between cell division:
\begin{itemize}
\item G1 phase
\item S phase
\item G2 phase
\end{itemize}
\end{description}
\subsection{Cells in Multicellular Organisms}
\begin{enumerate}
\item Many mature cells do not divide.
\begin{itemize}
\item nerve cells, muscle cells, RBC
\item As specialisation develops, the ability to divide is lost
\end{itemize}
\item Some cells only divide when given an appropriate stimulus
\begin{itemize}
\item liver cells
\item when part of the liver is surgically romevd the remaining liver
  cells start to divide to replace the lost tissue
\end{itemize}
\item Some cells normally divide on an ongoing basis.
\end{enumerate}

Cells that do not divide are in G0 (in the state of cell cycle exit).

The cell-cycle control system delays later events until earlier events
are complete (start checkponint, metaphase-to-anaphase
transition). Problems with the cell-cycle control system cause
chromosome segregation defects.

If the environment is not favourable at some checkpoint, the cell
cycle arrest occurs.

\subsection{Prophase}

\begin{itemize}
\item replicated chromosomes condese
\item Mitotic spindle assembly starts and requires centrosome
  duplication and bipolar spindle assembly.
\end{itemize}
When a mitotic cell is fused with another cell in G1, some of the
proteins will move over and condense G1 proteins.

When a mitotic cell is fused with another cell in G2, then the S2
chromosomes are also condensed.

When a mitotic cell is fused with another cell in S phase, chromosomes
are pulverized.

\subsection{Chromosome Condensation}

At the end of G2 the replicated chromosomes are dispersed and tangled.

At the beginning of mitosis choromosomes condense to a condensin protein complex.

The sister chromatids are resolved but remain associated due to the
cohesin at the centromere.

\subsection{Dynamic Microtubules}

Dynamic microtubules are required for mitosis.

In an interphase cell microtubules are arranged in a radial pattern. Minus ends are stabilised at the MTOC. 

In prophase, bipolar mitotic spindle starts, which requires
dissasembly and reassambly of microtubules.

\subsection{Centrosome Structure}

The centrosome MTC consists of a pair of centrioles organised at right
angles to each other and composed by nine fibrils of three mictrotubules each.

The centrosome MTOC is surrounded by $\gamma$-tubulin ring complexes.

\subsection{Centrosome Duplication and Mitotic Spindle Assembly}

Coentrosome duplication starts in $S$ phase, while the bipolar mitotic
spindle assembly starts in M phase.

Centrosome duplication occurs only once per cell cycle. Each centriole
serves as a template for a new centriole.

Complete mitotic spindle assembly requires nuclear envelope breakdown.

\subsection{Nuclear Envelope Breakdown}

Nuclear envelope breakdown occurs at the boundary between prophase and prometaphase.

\begin{description}

\item[Nuclear lamina] Meshwork of intreconnected nuclear lamin proteins
\item[Lamin] A special class of intermediatefilaments that form a
  two-dimensional lattice on the inner nuclear membrane.

\end{description}

Phosphorylation of lamins is thought to trigger nuclear envelope breakdown.

\subsection{Prometaphase}

Mitotic spindle assembly is completed. Chromosomes attach to spindle
microtubules. Chromosome movement begins and ancompasses the
microtubulene.

\subsection{Mitotic Spindle Assembly}

Requiremenets:
\begin{itemize}
\item Microtuble dynabics
\item Microtubule motor protein activity (kinesins, dyenins)
\end{itemize}

\subsection{Metaphase}

In metaphase, all chromosomes are aligned on the metaphase plate.

\subsection{Metaphase-Anaphase Transition}

Anaphase does not start until all the chromosomes are aligned on the
metaphase plate.

\subsection{Anaphase}

Sister chromatids separate to form the two daughter chromosomes, and
the cohesin is cleaved.

They are pulled towards opposite poles, and kinetochore microtubules
shorten.

When anaphase is triggered a protease called separase is activated
which cleaves the cohesin complex and allows sister chromatids to
separate.

\subsection{Cytokinesis}

Cytoplasm is divided in two by a contractile ring of actin and myosin
(in animal cells). The interphase microtubules reform in each daughter
cell, which marks the end of the M-phase.

Cytokinesis requires dynaic actin filaments. At the beginning of
mitosis actin amd myosin arrays disassemble. They assemble at the
contractile ring towards the ned of mitosis (starting in anaphase).

In plant cells vesicles line up and fuse together.


\end{document}
