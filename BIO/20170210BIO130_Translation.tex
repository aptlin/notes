% -*- coding: utf-8; -*-
%%% Local Variables:
%%% mode: latex
%%% TeX-engine: xetex
%%% TeX-master: t
%%% End:
\documentclass[11pt]{scrartcl}
\usepackage[fancy, beaue, pset, anon]{masty}
\pSet{\nt{BIO130}{V}{Translation}}
\usepackage{lineno}
% ----------------------------------------------------------------------
% Page setup
% ----------------------------------------------------------------------

\pagenumbering{gobble}

% ----------------------------------------------------------------------
% Custom commands
% ----------------------------------------------------------------------

% alignment

\newcommand*{\LongestHence}{$\Rightarrow$}% function name
\newcommand*{\LongestName}{$f_o(-x)+f_e(-x)$}% function name
\newcommand*{\LongestValue}{$(-a)x +(-a)(-y)$}% function value
\newcommand*{\LongestText}{\defi}%

\newlength{\LargestHenceSize}%
\newlength{\LargestNameSize}%
\newlength{\LargestValueSize}%
\newlength{\LargestTextSize}%

\settowidth{\LargestHenceSize}{\LongestHence}%
\settowidth{\LargestNameSize}{\LongestName}%
\settowidth{\LargestValueSize}{\LongestValue}%
\settowidth{\LargestTextSize}{\LongestText}%

% Choose alignment of the various elements here: [r], [l] or [c]

\newcommand*{\mbh}[1]{{\makebox[\LargestHenceSize][r]{\ensuremath{#1}}}}%
\newcommand*{\mbn}[1]{{\makebox[\LargestNameSize][r]{\ensuremath{#1}}}}%
\newcommand*{\mbv}[1]{\ensuremath{\makebox[\LargestValueSize][r]{\ensuremath{#1}}}}%
\newcommand*{\mbt}[1]{\makebox[\LargestTextSize][l]{#1}}%

\newcommand{\R}[1]{\label{#1}\linelabel{#1}}
\newcommand{\lr}[1]{line~\lineref{#1}}

% ----------------------------------------------------------------------
% Launch!
% ----------------------------------------------------------------------

\begin{document}

\section{Translation}

Codons are read as mRNA triplets, encoding all 20 amino acids, so
there is a redundancy with multiple codons for most amino acids.

Reaing frames define the amino acid sequence, which are in turn
determined by the position of the starting codon AUG.

\subsection{TmRNA and protein sequence}

The 3' end has an acceptor arm, where it will stick to the
acid. Anticodons will base-pair complementary and antiparellel to the
codon.

There are modifications to traditional bases, such as D =
dihydroeuridine.

\subsection{Codon Redundancy}


Some strategies of managing the redundancy for translation include the
correspondence of more than 1 tRNA to many amino acids, with some
tRNAs recognizing and base pairing with more than 1 codon.
% There is some flexibility in codon bonding between .
There are two sequential steps in ensuring fidelity: utilising
aminoacyl-tRNA synthetases and base pairing. Hydrolytic editing, in
turn, is conducted by aminoacyl-RNA-synthetase.

Amino acid addition proceeds by condensation reactions.

\subsection{Prokaryotic vs Eukaryotic Ribosomes}

Both prokaryotic and eukaryotic ribosomes have large and small subunits.

Eukaryotic ribosomes are typically larger and more complex.

In eukaryotes, ribosomes are located on endoplasimc reticulum (which
secretes proteins in some organelles) and in cytosol. In prokaryotes, ribosomes are only in cytosol.




\end{document}