% -*- coding: utf-8; -*-
%%% Local Variables:
%%% mode: latex
%%% TeX-engine: xetex
%%% TeX-master: t
%%% End:
\documentclass[11pt]{scrartcl}
\usepackage[fancy, beaue, pset, anon]{masty}
\pSet{\nt{BIO130}{}{Transport Proteins}}
\usepackage{lineno}
% ----------------------------------------------------------------------
% Page setup
% ----------------------------------------------------------------------

\pagenumbering{gobble}

% ----------------------------------------------------------------------
% Custom commands
% ----------------------------------------------------------------------

% alignment

\newcommand*{\LongestHence}{$\Rightarrow$}% function name
\newcommand*{\LongestName}{$f_o(-x)+f_e(-x)$}% function name
\newcommand*{\LongestValue}{$(-a)x +(-a)(-y)$}% function value
\newcommand*{\LongestText}{\defi}%

\newlength{\LargestHenceSize}%
\newlength{\LargestNameSize}%
\newlength{\LargestValueSize}%
\newlength{\LargestTextSize}%

\settowidth{\LargestHenceSize}{\LongestHence}%
\settowidth{\LargestNameSize}{\LongestName}%
\settowidth{\LargestValueSize}{\LongestValue}%
\settowidth{\LargestTextSize}{\LongestText}%

% Choose alignment of the various elements here: [r], [l] or [c]

\newcommand*{\mbh}[1]{{\makebox[\LargestHenceSize][r]{\ensuremath{#1}}}}%
\newcommand*{\mbn}[1]{{\makebox[\LargestNameSize][r]{\ensuremath{#1}}}}%
\newcommand*{\mbv}[1]{\ensuremath{\makebox[\LargestValueSize][r]{\ensuremath{#1}}}}%
\newcommand*{\mbt}[1]{\makebox[\LargestTextSize][l]{#1}}%

\newcommand{\R}[1]{\label{#1}\linelabel{#1}}
\newcommand{\lr}[1]{line~\lineref{#1}}

% ----------------------------------------------------------------------
% Launch!
% ----------------------------------------------------------------------

\begin{document}

\section{Transport Proteins}

Transport proteins work together to transfer glucose from the
intestine to the blood stream.

The internal side of the lumen is covered by vili, increasing the
surface area, within which epithelial cells are located. On each vili
there are microvili, increasing the surface area even more.

The basal and lateral sites of a vilus are very similar. Outside of
vili blood or extracellular fluids are flowing.

Glucose is thus goes from low concetration inside the lumen to the
high concentration in the cytosol of the epithelial cell and then tho
the low concentration in the extracellular fluid on the basolateral
side of the epithelial cell, where GLUT uniporters and Na$^+$ and
K$^+$ pumps are located.

Transcellular transport of clucose requires the asymmetric
distribution of membrane proteins. There are specific areas called
tight junctions which restrict the location of transport proteins,
keeping Na$^+$/glucose symporter on the apical membrane and GLUT2
uniporter and Na$^+$/K$^+$ pum on the basolateral plasma membrane.

Tight junctions also stop intercellular molecular movement.

\section{ATP-Driven Pumps}

There are three kinds of transport ATPases: P-type pumps, F-type(and
V-type) proton pumps and ABC transporters.

\subsection{F-type and V-type ATPase}

F-type and V-type ATPases are structurally related, but have opposite
modes of action.

F-type ATPases (present in mitochondria, chloroplasts and bacteria)
use the H$^+$ gradient to drive the synthesis of ATP. On the other
hand, V-type ATPases (present in lysosomes and plant vacuoles) use ATP and pump H$^+$

For example, ATP synthases use the H$^+$ electrochemical gradient to
produce ATP (F-type), while H$^+$ pumps use ATP to pump H$^+$ against
the electrochemical gradient (V-type) and acidify the lumen.

\section{Membrane Potential}

Membrane potential is the difference in electrical charge on two sides
of the membrane, which is used by symporters and antiporters to carry
out secondary active transport (in animals and plants). Moreover,
membrane potential provides action potential in nerve cells.

\subsection{Generation of Membrane Potential}

K$^+$ leak channels play a makor role in membrane potential by
securing an outward flow of K$^+$.

Na$^{+}$/K$^{+}$ pump, on the other hand, has the following characteristics:
\begin{itemize}
\item Na$^+$ gradient with low cytosolic [Na$^+$]
\item K$^+$ gradient with high cytosolic [K$^+$]
\item Elecrogenic

  \begin{itemize}
  \item Net 1+ ion pumped out
  \item Responsible for ~10\% of the membrane potential
  \end{itemize}

\end{itemize}

In animal cells, ions in solution are present in pairs, with more (+)
charge on the outside because of Na$^+$ and K$^+$ and more (-) charge
on the inside due to Cl$^-$ and fixed anions. At equilibrium, the
resting membrane potential can be measured, and is equal to -70 mV on
average.

\section{Intracellular Compartments}

Cytosol usually assumes half the cell volume. Cytosol, where protein
synthesis and degradation occur, is also the location of intermediary
metabolism.

\begin{definition}
  An organelle is a subcellular compartment or large macromolecular
  complex, often membrane-enclosed, that has a distinct structure,
  composition, and function.
\end{definition}

\begin{description}

\item[e.g.] nucleus, endoplasmic reticulum, Golgi apparatus, nucleolus

\end{description}

Rough endoplasmic reticulum is the location of synthesis of
transmembrane, organellar and secreted proteins, while smooth
endoplasmic reticulum is where fats and steroid hormones are metabolised.

There are a lot more membranes in the cell than around the cell.

We can compare liver hepatocytes and pancreatic exocrine cells.  A
standard liver cell, liver hepatocyte, is involved with
detoxification. Pancreatic exocrine cells secrete digestive
enzymes. ERs constitute approximately 50\% of both cells. In liver
hepatocyte there are significantly more SER, however.

Percentage composition of cells varies from one cell to another.

\subsection{Dynamics of Intracellular Compartments}

Intracellular compartments exchange lipids and proteins. Together
organelles form the endomembrane system.

The endomembrane system is involved in biosynthetic/secretory pathways (biosynthesis (proteins and lipids are produced and shared with other organelles) and secretion (proteins, exocytosis)) and endocytic pathway (\textit{eg}. endocytosis).

During exocytosis vesicle contents are delivered to extracellular space, while vesicle membrane becomes a part of the plasma membrane. During endocytosis, the plasma membrane forms the vesicle membrane, with vesical luminal contents coming from the extracellular space.

\section{Vesicular Transport}

\begin{definition}
  A \textbf{vesicle} is a small, membrane-enclosed organelle in the
  cytoplasm of a eukaryotic cell. Vesicles shuttle components back and fioth in the endomembrane system.
\end{definition}

Transmembrane proteins can select specific proteins to go inside a
vesicle from the donor compartment to the recipient compartment.

Specific proteins are targeted to different organells in the following
steps. First, mRNA arrives in the cytoplasm and translation starts on
ribosomes in the cytosol. A cytosolic protein is then translated in
the cytosol and has no sorting signal.

Mitochondria and chloroplasts have their own genome and ribosomes, but
most proteins are nuclear-encloded.

Nuclear-encoded proteins are translated in the cytosol and then
targeted by a signal sequence, which imports them into the organelle post-translationally.

Proteins remain unfolded in the cytosol by association with hsp70
chaperones.

The path of a secreted protein from translation to secretion can be described as follows.

First, mRNA arrives in the cytoplasm and translation starts on
ribosomes in the cytosol. While translation is still occuring,
insertion of the protein into the endoplasmic reticulum starts, as
directed by the signal sequence. In this way, \textbf{co-translational
  translocation} occurs.

SRP stops the translation on detection of the signal sequence, and
taking the ribosome to the translocator by binding with the SRP
receptor, which then associates with the plug in a translocus. In this
way, the protein gets made while it moves into the ER lumen, and then
a signal peptidase cuts off the signal after it has been used.

Pancreatic cells make lots of secreted proteins. These cells can be
used to follow the path of newly synthesised proteins. First, the
cells are provided with a short pulse of radioactive amino acids. The
path of these aminoacids can then be followed as they are incorporated
into proteins. This technique is called a \textbf{pulse-chase
  experiment}.

Proteins move from the rough endoplasmic reticulum, proceed to the
Golgi apparatus and then get to the secretory vesicles.

The secretory pathway can be of two types:
\begin{itemize}
\item constitutive (continual production of secreted proteins)
  \begin{description}

  \item[e.g.] collagen

  \end{description}
  
\item regulated (proteins are stored in secretory granules ready for export in response to a stimulus)
  \begin{description}

  \item[e.g.] neurotransmitters

  \end{description}
\end{itemize}


\section{Protein Sorting Mechanisms}
\begin{itemize}
\item Gated (proteins move between the cytosol and nucleus through nuclear pore complexes)
\item Transmembrane (a translocon protein is needed to transport specific proteins across a membrane)
\item Vesicular (membrane-enclosed transport vesicles ferry proteins from one compartment to another)
\end{itemize}

\subsection{Signal Sequences}

A signal sequence is a stretch of the amino acid sequence of a protein
that directs the protein to the correct destination. Each signal
sequence specifies a specific destination in the cell (to a nucleus,
mitochondria, ER, peroxisomes, etc). Signal sequences are recognized
by sorting receptors that take proteins to their destination.

Signal sequences are often found at the N-terminus of the
protein. However, some signal sequences are internal stretches of AA
which remain part of the protein. For proteins with N-terminal signal
sequences there are signal peptidases removing the signal sequence
from the protein.

Secreted proteins have N-terminal signal sequences.

\section{Review}

Sorting of a secreted protein proceeds as follows:

\begin{itemize}
\item translation starts on cytosolic ribosomes
\item signal sequence of the amino-terminal end directs the protein to the ER (the signal sequence is hydrofobic)
\item protein inserted throuch the membrane by the translocon lumen
\end{itemize}

\end{document}