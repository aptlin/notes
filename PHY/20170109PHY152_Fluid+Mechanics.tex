% -*- coding: utf-8; -*-
%%% Local Variables:
%%% mode: latex
%%% TeX-engine: xetex
%%% TeX-master: t
%%% End:
\documentclass[11pt]{scrartcl}
\usepackage[fancy, beaue, pset, anon]{sdll}
\pSet{\nt{PHY152}{II}{Fluid Mechanics}}
\newcommand{\R}[1]{\label{#1}\linelabel{#1}}
\newcommand{\lr}[1]{line~\lineref{#1}}

% ----------------------------------------------------------------------
% Launch!
% ----------------------------------------------------------------------

\begin{document}

09 - 16 Jan Office Hours: Mon \& Fri 2-3pm

\section{Fluid Mechanics}
\label{sec:fmech}

Motivation: applications in astrophysics, atmospheric physics,
geophysics, oceanography, chemical and mechanical engineering, etc.

\begin{note*}
  Macroscopic properties are used in describing fluids. How can we
  make a leap from describing points to describing fields?
\end{note*}

\begin{example}

  UV light frow white dwarves causes gases to flow -- fluid mechanics
  allows us to model their movement.

\end{example}

\begin{definition}
  \textbf{Fluids} -- liquids or gases (with pressure and density as
  main descriptive parameters)

  \begin{itemize}
  \item flow when acted upon by external force
  \item liquid \begin{itemize}
    \item almost incompressible
    \item packed -- both order and movement are present
    \end{itemize} 
  \item gas \begin{itemize}
    \item compressible
    \item disordered -- random motion dominates
    \end{itemize}
  \end{itemize}
\end{definition}
\begin{note*}
More interesting definitions exist -- cf. nonlinear fluid dynamics
\end{note*}

\begin{example}[Liquid Behaviour in Zero Gravity]

Scott Kelly has shown that you can play ping pong in space.

\end{example}

For static fluids, the following formulas apply:
\begin{description}

\item[density] \(\rho = \frac{\text{mass}}{\text{volume}}\) unit: \(\frac{kg}{m^{3}}\)
  
\item[pressure] \(P = \frac{\text{force}}{\text{area}}\) unit: \(Pa = \frac{N}{m^{2}}\)
  \begin{itemize}
  \item pressure acts normally
  \end{itemize}

\end{description}

\begin{example}

  approximately every \(16\ km\) up the atmospheric pressure goes down
  by an order of magnitude. 

\end{example}
\begin{ques*}
  How do we check that \(P\) varies continuously with height?
\end{ques*}

\begin{answer*}

  Determine a unit volume small enough for \(P\) to be constant and
  large enough to behave as a fluid. In this way,

  \[\rho = \lim_{\Delta V \to 0}\frac{\Delta m}{\Delta V}\]

\end{answer*}

Consider pressure of a fluid at rest.

Note that since the fluid is at rest, all force components acting on a
parallellopiped cancel each other out.

Thus,

\begin{align}
  \rho_{l}A' &= \rho_{r}A'\\
  \rho_{f}A' &= \rho_{n}A'\\
  \rho_{t}A + mg &= \rho_{b}A\\
  \ra P_{b} &= P_{t} + \frac{mg}{A}\\
  m = \rho_{f}Ad &\ra P(d) = P(t) + \rho_{f}gd
\end{align}

\subsection{Measurement Methods}
\label{subsec:meas}

\begin{itemize}
\item gauge pressure = absolute pressure - atmospheric pressure
\item used for tires, sports equipment
\end{itemize}
\end{document}